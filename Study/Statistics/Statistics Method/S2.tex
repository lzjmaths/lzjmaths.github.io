\section{Basic Concepts}
\begin{definition}[variable]
    \begin{itemize}
        \item \name{Quantitive variable}:Variables that take numerical values for which arithmetic operations, such as adding and averaging, work.(连续量)
        \item \name{Categorical variables}:Variables that fall into one of several categories. What can be counted is the count or proportion of individuals in each category.(离散量,用于分类)


    \end{itemize}
\end{definition}
\paragraph{Ways to chart  quantitative data}

\begin{itemize}
    \item Histograms: \begin{itemize}
        \item The range of values that a variable can take is divided into equal size.
        \item The histogram shows the number of individual data points that fall into each interval.
        \item It is  called \subname{unimodel}{histogram} if it has a single peak and \name{bimodel} if it has two peaks.
        \item \subname{outliers}{histogram}: Outliers are observations that lie outside the overall pattern of a distribution. Always look for outliers and try to explain them.(奇异点)

    \end{itemize}
    \item Line graphs: time plots.
    \begin{itemize}
        \item A \name{trend} is a rise or fall that persists over time, despite small irregularities.
        \item A pattern that repeats itself at regular intervals of time is called \name{seasonal variation}.
    \end{itemize}
\end{itemize}