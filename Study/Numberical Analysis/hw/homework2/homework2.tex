% Sample tex file for usage of iidef.sty
% Homework template for Inference and Information
% UPDATE: October 12, 2017 by Xiangxiang
% UPDATE: 22/03/2018 by zhaofeng-shu33
\documentclass[a4paper]{article}
\usepackage[T1]{fontenc}
\usepackage{amsmath, amssymb, amsthm}
% amsmath: equation*, amssymb: mathbb, amsthm: proof
\usepackage{moreenum}
\usepackage{mathtools}
\usepackage{url}
\usepackage{graphicx}
\usepackage{subcaption}
\usepackage{booktabs} % toprule
\usepackage[mathcal]{eucal}
\usepackage{dsfont}

\usepackage{setspace}  
\setstretch{1.6}








\theoremstyle{definition}
\newtheorem{definition}{Definition}[section]
\newtheorem{example}[definition]{Example}
\newtheorem{exercise}[definition]{Exercise}
\newtheorem{remark}[definition]{Remark}
\newtheorem{observation}[definition]{Observation}
\newtheorem{assumption}[definition]{Assumption}
\newtheorem{convention}[definition]{Convention}
\newtheorem{priniple}[definition]{Principle}
\newtheorem{notation}[definition]{Notation}
\newtheorem*{axiom}{Axiom}
\newtheorem{coa}[definition]{Theorem}
\newtheorem{srem}[definition]{$\star$ Remark}
\newtheorem{seg}[definition]{$\star$ Example}
\newtheorem{sexe}[definition]{$\star$ Exercise}
\newtheorem{sdf}[definition]{$\star$ Definition}
\newtheorem{question}{Question}




\newtheorem{problem}{Problem}
%\renewcommand*{\theprob}{{\color{red}\arabic{section}.\arabic{prob}}}
\newtheorem{rprob}[problem]{\color{red} Problem}
%\renewcommand*{\thesprob}{{\color{red}\arabic{section}.\arabic{sprob}}}
% \newtheorem{ssprob}[prob]{$\star\star$ Problem}



\theoremstyle{plain}
\newtheorem{theorem}[definition]{Theorem}
\newtheorem{Conclusion}[definition]{Conclusion}
\newtheorem{thd}[definition]{Theorem-Definition}
\newtheorem{proposition}[definition]{Proposition}
\newtheorem{corollary}[definition]{Corollary}
\newtheorem{lemma}[definition]{Lemma}
\newtheorem{sthm}[definition]{$\star$ Theorem}
\newtheorem{slm}[definition]{$\star$ Lemma}
\newtheorem{claim}[definition]{Claim}
\newtheorem{spp}[definition]{$\star$ Proposition}
\newtheorem{scorollary}[definition]{$\star$ Corollary}


\newtheorem{condition}{Condition}
\newtheorem{Mthm}{Main Theorem}
\renewcommand{\thecondition}{\Alph{condition}} % "letter-numbered" theorems
\renewcommand{\theMthm}{\Alph{Mthm}} % "letter-numbered" theorems


%\substack   multiple lines under sum
%\underset{b}{a}   b is under a


% Remind: \overline{L_0}



\usepackage{calligra}
\DeclareMathOperator{\shom}{\mathscr{H}\text{\kern -3pt {\calligra\large om}}\,}
\DeclareMathOperator{\sext}{\mathscr{E}\text{\kern -3pt {\calligra\large xt}}\,}
\DeclareMathOperator{\Rel}{\mathscr{R}\text{\kern -3pt {\calligra\large el}~}\,}
\DeclareMathOperator{\sann}{\mathscr{A}\text{\kern -3pt {\calligra\large nn}}\,}
\DeclareMathOperator{\send}{\mathscr{E}\text{\kern -3pt {\calligra\large nd}}\,}
\DeclareMathOperator{\stor}{\mathscr{T}\text{\kern -3pt {\calligra\large or}}\,}
%write mathscr Hom (and so on) 

\usepackage{aurical}
\DeclareMathOperator{\VVir}{\text{\Fontlukas V}\text{\kern -0pt {\Fontlukas\large ir}}\,}

\newcommand{\vol}{\text{\Fontlukas V}}
\newcommand{\dvol}{d~\text{\Fontlukas V}}
% perfect Vol symbol

\usepackage{aurical}








\newcommand{\fk}{\mathfrak}
\newcommand{\mc}{\mathcal}
\newcommand{\wtd}{\widetilde}
\newcommand{\wht}{\widehat}
\newcommand{\wch}{\widecheck}
\newcommand{\ovl}{\overline}
\newcommand{\udl}{\underline}
\newcommand{\tr}{\mathrm{t}} %transpose
\newcommand{\Tr}{\mathrm{Tr}}
\newcommand{\End}{\mathrm{End}} %endomorphism
\newcommand{\idt}{\mathbf{1}}
\newcommand{\id}{\mathrm{id}}
\newcommand{\Hom}{\mathrm{Hom}}
\newcommand{\cond}[1]{\mathrm{cond}_{#1}}
\newcommand{\Conf}{\mathrm{Conf}}
\newcommand{\Res}{\mathrm{Res}}
\newcommand{\res}{\mathrm{res}}
\newcommand{\KZ}{\mathrm{KZ}}
\newcommand{\ev}{\mathrm{ev}}
\newcommand{\coev}{\mathrm{coev}}
\newcommand{\opp}{\mathrm{opp}}
\newcommand{\Rep}{\mathrm{Rep}}
\newcommand{\Dom}{\mathrm{Dom}}
\newcommand{\loc}{\mathrm{loc}}
\newcommand{\con}{\mathrm{c}}
\newcommand{\uni}{\mathrm{u}}
\newcommand{\ssp}{\mathrm{ss}}
\newcommand{\di}{\slashed d}
\newcommand{\Diffp}{\mathrm{Diff}^+}
\newcommand{\Diff}{\mathrm{Diff}}
\newcommand{\PSU}{\mathrm{PSU}(1,1)}
\newcommand{\Vir}{\mathrm{Vir}}
\newcommand{\Witt}{\mathscr W}
\newcommand{\Span}{\mathrm{Span}}
\newcommand{\pri}{\mathrm{p}}
\newcommand{\ER}{E^1(V)_{\mathbb R}}
\newcommand{\prth}[1]{( {#1})}
\newcommand{\bk}[1]{\langle {#1}\rangle}
\newcommand{\bigbk}[1]{\big\langle {#1}\big\rangle}
\newcommand{\Bigbk}[1]{\Big\langle {#1}\Big\rangle}
\newcommand{\biggbk}[1]{\bigg\langle {#1}\bigg\rangle}
\newcommand{\Biggbk}[1]{\Bigg\langle {#1}\Bigg\rangle}
\newcommand{\GA}{\mathscr G_{\mathcal A}}
\newcommand{\vs}{\varsigma}
\newcommand{\Vect}{\mathrm{Vec}}
\newcommand{\Vectc}{\mathrm{Vec}^{\mathbb C}}
\newcommand{\scr}{\mathscr}
\newcommand{\sjs}{\subset\joinrel\subset}
\newcommand{\Jtd}{\widetilde{\mathcal J}}
\newcommand{\gk}{\mathfrak g}
\newcommand{\hk}{\mathfrak h}
\newcommand{\xk}{\mathfrak x}
\newcommand{\yk}{\mathfrak y}
\newcommand{\zk}{\mathfrak z}
\newcommand{\pk}{\mathfrak p}
\newcommand{\hr}{\mathfrak h_{\mathbb R}}
\newcommand{\Ad}{\mathrm{Ad}}
\newcommand{\DHR}{\mathrm{DHR}_{I_0}}
\newcommand{\Repi}{\mathrm{Rep}_{\wtd I_0}}
\newcommand{\im}{\mathbf{i}}
\newcommand{\Co}{\complement}
%\newcommand{\Cu}{\mathcal C^{\mathrm u}}
\newcommand{\RepV}{\mathrm{Rep}^\uni(V)}
\newcommand{\RepA}{\mathrm{Rep}(\mathcal A)}
\newcommand{\RepN}{\mathrm{Rep}(\mathcal N)}
\newcommand{\RepfA}{\mathrm{Rep}^{\mathrm f}(\mathcal A)}
\newcommand{\RepAU}{\mathrm{Rep}^\uni(A_U)}
\newcommand{\RepU}{\mathrm{Rep}^\uni(U)}
\newcommand{\RepL}{\mathrm{Rep}^{\mathrm{L}}}
\newcommand{\HomL}{\mathrm{Hom}^{\mathrm{L}}}
\newcommand{\EndL}{\mathrm{End}^{\mathrm{L}}}
\newcommand{\Bim}{\mathrm{Bim}}
\newcommand{\BimA}{\mathrm{Bim}^\uni(A)}
%\newcommand{\shom}{\scr Hom}
\newcommand{\divi}{\mathrm{div}}
\newcommand{\sgm}{\varsigma}
\newcommand{\SX}{{S_{\fk X}}}
\newcommand{\DX}{D_{\fk X}}
\newcommand{\mbb}{\mathbb}
\newcommand{\mbf}{\mathbf}
\newcommand{\bsb}{\boldsymbol}
\newcommand{\blt}{\bullet}
\newcommand{\Vbb}{\mathbb V}
\newcommand{\Ubb}{\mathbb U}
\newcommand{\Xbb}{\mathbb X}
\newcommand{\Kbb}{\mathbb K}
\newcommand{\Abb}{\mathbb A}
\newcommand{\Wbb}{\mathbb W}
\newcommand{\Mbb}{\mathbb M}
\newcommand{\Gbb}{\mathbb G}
\newcommand{\Cbb}{\mathbb C}
\newcommand{\Nbb}{\mathbb N}
\newcommand{\Zbb}{\mathbb Z}
\newcommand{\Qbb}{\mathbb Q}
\newcommand{\Pbb}{\mathbb P}
\newcommand{\Rbb}{\mathbb R}
\newcommand{\Ebb}{\mathbb E}
\newcommand{\Dbb}{\mathbb D}
\newcommand{\Hbb}{\mathbb H}
\newcommand{\cbf}{\mathbf c}
\newcommand{\Rbf}{\mathbf R}
\newcommand{\wt}{\mathrm{wt}}
\newcommand{\Lie}{\mathrm{Lie}}
\newcommand{\btl}{\blacktriangleleft}
\newcommand{\btr}{\blacktriangleright}
\newcommand{\svir}{\mathcal V\!\mathit{ir}}
\newcommand{\Ker}{\mathrm{Ker}}
\newcommand{\Cok}{\mathrm{Coker}}
\newcommand{\Sbf}{\mathbf{S}}
\newcommand{\low}{\mathrm{low}}
\newcommand{\Sp}{\mathrm{Sp}}
\newcommand{\Rng}{\mathrm{Rng}}
\newcommand{\vN}{\mathrm{vN}}
\newcommand{\Ebf}{\mathbf E}
\newcommand{\Nbf}{\mathbf N}
\newcommand{\Stb}{\mathrm {Stb}}
\newcommand{\SXb}{{S_{\fk X_b}}}
\newcommand{\pr}{\mathrm {pr}}
\newcommand{\SXtd}{S_{\wtd{\fk X}}}
\newcommand{\univ}{\mathrm {univ}}
\newcommand{\vbf}{\mathbf v}
\newcommand{\ubf}{\mathbf u}
\newcommand{\wbf}{\mathbf w}
\newcommand{\CB}{\mathrm{CB}}
\newcommand{\Perm}{\mathrm{Perm}}
\newcommand{\Orb}{\mathrm{Orb}}
\newcommand{\Lss}{{L_{0,\mathrm{s}}}}
\newcommand{\Lni}{{L_{0,\mathrm{n}}}}
\newcommand{\UPSU}{\widetilde{\mathrm{PSU}}(1,1)}
\newcommand{\Sbb}{{\mathbb S}}
\newcommand{\Gc}{\mathscr G_c}
\newcommand{\Obj}{\mathrm{Obj}}
\newcommand{\bpr}{{}^\backprime}
\newcommand{\fin}{\mathrm{fin}}
\newcommand{\Ann}{\mathrm{Ann}}
\newcommand{\Real}{\mathrm{Re}}
\newcommand{\Imag}{\mathrm{Im}}
%\newcommand{\cl}{\mathrm{cl}}
\newcommand{\Ind}{\mathrm{Ind}}
\newcommand{\Supp}{\mathrm{Supp}}
\newcommand{\Specan}{\mathrm{Specan}}
\newcommand{\red}{\mathrm{red}}
\newcommand{\uph}{\upharpoonright}
\newcommand{\Mor}{\mathrm{Mor}}
\newcommand{\pre}{\mathrm{pre}}
\newcommand{\rank}{\mathrm{rank}}
\newcommand{\Jac}{\mathrm{Jac}}
\newcommand{\emb}{\mathrm{emb}}
\newcommand{\Sg}{\mathrm{Sg}}
\newcommand{\Nzd}{\mathrm{Nzd}}
\newcommand{\Owht}{\widehat{\scr O}}
\newcommand{\Ext}{\mathrm{Ext}}
\newcommand{\Tor}{\mathrm{Tor}}
\newcommand{\Com}{\mathrm{Com}}
\newcommand{\Mod}{\mathrm{Mod}}
\newcommand{\nk}{\mathfrak n}
\newcommand{\mk}{\mathfrak m}
\newcommand{\Ass}{\mathrm{Ass}}
\newcommand{\depth}{\mathrm{depth}}
\newcommand{\Coh}{\mathrm{Coh}}
\newcommand{\Gode}{\mathrm{Gode}}
\newcommand{\Fbb}{\mathbb F}
\newcommand{\sgn}{\mathrm{sgn}}
\newcommand{\Aut}{\mathrm{Aut}}
\newcommand{\Modf}{\mathrm{Mod}^{\mathrm f}}
\newcommand{\codim}{\mathrm{codim}}
\newcommand{\card}{\mathrm{card}}
\newcommand{\dps}{\displaystyle}
\newcommand{\Int}{\mathrm{Int}}
\newcommand{\Nbh}{\mathrm{Nbh}}
\newcommand{\Pnbh}{\mathrm{PNbh}}
\newcommand{\Cl}{\mathrm{Cl}}
\newcommand{\diam}{\mathrm{diam}}
\newcommand{\eps}{\varepsilon}
\newcommand{\Vol}{\mathrm{Vol}}
\newcommand{\LSC}{\mathrm{LSC}}
\newcommand{\USC}{\mathrm{USC}}
\newcommand{\Ess}{\mathrm{Rng}^{\mathrm{ess}}}
\newcommand{\Jbf}{\mathbf{J}}
\newcommand{\SL}{\mathrm{SL}}
\newcommand{\GL}{\mathrm{GL}}
\newcommand{\Lin}{\mathrm{Lin}}
\newcommand{\ALin}{\mathrm{ALin}}
\newcommand{\bwn}{\bigwedge\nolimits}
\newcommand{\nbf}{\mathbf n}
\newcommand{\dive}{\mathrm{div}}




\usepackage{algorithm}
\usepackage{algorithmic}

\newcommand{\<}{\left<}
\renewcommand{\>}{\right>}


\numberwithin{equation}{problem}
% count the eqation by section countation


\DeclareMathOperator{\sign}{sign}
\DeclareMathOperator{\dom}{dom}
\DeclareMathOperator{\ran}{ran}
\DeclareMathOperator{\ord}{ord}
\DeclareMathOperator{\img}{Im}
\DeclareMathOperator{\dd}{d\!}
\newcommand{\ie}{ \textit{ i.e. } }
\newcommand{\st}{ \textit{ s.t. }}


\usepackage[numbered,framed]{matlab-prettifier}
\lstset{
  style              = Matlab-editor,
  captionpos         =b,
  basicstyle         = \mlttfamily,
  escapechar         = ",
  mlshowsectionrules = true,
}

\usepackage[thehwcnt = 1]{iidef}
\thecourseinstitute{Tsinghua University}
\thecoursename{Numerical Analysis}
\theterm{Fall 2024}
\hwname{Homework}
\usepackage{geometry}
\geometry{left=1.5cm,right=1.5cm,top=2.5cm,bottom=2.5cm}
\begin{document}
\courseheader
\name{Lin Zejin}
\rule{\textwidth}{1pt}
\begin{itemize}
\item {\bf Collaborators: \/}
  I finish this homework by myself. 
%   If you finish your homework all by yourself, make a similar statement. If you get help from others in finishing your homework, state like this:
%   \begin{itemize}
%   \item 1.2 (b) was solved with the help from \underline{\hspace{3em}}.
%   \item Discussion with \underline{\hspace{3em}} helped me finishing 1.3.
%   \end{itemize}
\end{itemize}
\rule{\textwidth}{1pt}

\vspace{2em}

\sloppy
\pagenumbering{arabic}

\begin{problem}
    \begin{proof}[Proof of Lemma 2.5.3]
         $ B_1=\begin{cases}
            1&x \leq 0\\
            0&x \geq 1\\
            1-x&0<x<1
         \end{cases} $.
         
         Then  $ B_2=\begin{cases}
            1&x \leq -\frac{1}{2}\\
            0&x \geq \frac{3}{2}\\
            -\frac{x^2}{2}-\frac{1}{2}x+\frac{7}{8}&-\frac{1}{2}<x<\frac{1}{2}\\
            \frac{1}{2}(\frac{3}{2}-x)^2&\frac{1}{2}<x<\frac{3}{2}
         \end{cases} $ 

         So  $  B_3=\begin{cases}
            0&x \leq -1\\
            -\dfrac{{t}^{3}+3\,{t}^{2}-3\,t-5}{6}&-1<x<0\\
            \dfrac{t^3}{3}-\dfrac{1}{2}t^2-\dfrac{1}{2}t+\dfrac{5}{6}&0<x<1\\
            \dfrac{\left({2-t}\right)^{3}}{6}&1<x<2\\
            1&x \geq 2\\
            3&-1<x<2
         \end{cases} $ with  $ B_2 $ 
        
        So  $ B_3(x),B_3(x-1),B_3(x-2) $ linearly independent.
    \end{proof}
    \begin{proof}[Proof of Lemma 2.5.4]
        If 
        \[\sum_{k=-1}^{n+1}\lambda_kB_3(\frac{x-a-kh}{h})=0\]
        Noticed that 
        \[B_3(\frac{x-a-kh}{h})=B_3(\frac{x-a}{h}-k)\]
        So on each interval  $ (a+jh,a+(j+1)h) $, it will be 
        \[\sum_{k=-1}^{n+1}\lambda_kB_3(j-k+\frac{x-jh-a}{h})=\lambda_{j-1}B_3(-1+\frac{x-jh-a}{h})+\lambda_{j}B_3(0+\frac{x-jh-a}{h})+\lambda_{j+1}B_3(1+\frac{x-jh-a}{h})=0\]
        where  $ \dps\frac{x-jh-a}{h}\in (0,1) $ 


        So we have matrix equation 
        \[\begin{pmatrix}
            B_3(-1+t)&B_3(t)&B_3(1+t)\cdots&\cdots&\cdots&0\\
            0&B_3(-1+t)&B_3(t)&B_3(1+t)\cdots&\cdots&0\\
            0&0&B_3(-1+t)&B_3(t)&B_3(1+t)&\cdots&0\\
            \vdots&\vdots&\vdots&\ddots&\ddots&\vdots\\
            0&0&0&\cdots&B_3(-1+t)&B_3(t)&B_3(1+t)\\
            0&0&0&0&\cdots&B_3(-1+t)&B_3(t)\\
            0&0&0&0&\cdots&0&B_3(-1+t)
        \end{pmatrix}\begin{pmatrix}
            \lambda_{-1}\\
            \lambda_{0}\\
            \lambda_{1}\\
            \vdots\\
            \lambda_{n-1}\\
            \lambda_{n}\\
        \end{pmatrix}=0\]
        on  $ (0,1) $ 
        By lemma 2.5.3, we have  $ \lambda_{-1}=\lambda_0=\lambda_1=\lambda_2=\cdots=\lambda_n=0 $.
        So they are linearly independent.
    \end{proof}
    \begin{proof}[Proof of lemma 2.6.2]
        We have proved in lemma 2.6.1 that 
        \[\sum{|c_i|} \leq \frac{\|P\|_Y}{\theta-\Omega(\delta)}\]
        Then 
        \begin{align*}
            \|f-P\|_X &\leq \max_{y\in Y,x\in X}|f(x)-f(y)|+|f(y)-P(y)|+|P(y)-P(x)|\\
            & \leq \omega(\delta,f)+\|f-P\|_Y+\sum|c_i|\Omega(\delta)\\
            & \leq \omega(\delta,f)+\|f-P\|_Y+\frac{\|P\|_Y\Omega(\delta)}{\theta-\Omega(\delta)}\\
        \end{align*}
        As  $ \Omega(\delta)\rightarrow 0 $,  $ \dps\frac{2}{\theta} $ is what we need.  $ \theta>0 $ since  $ g_i $ linearly independent.    
    \end{proof}


\end{problem}

\begin{problem}
    \begin{align*}
        N_{i,0}(u)&=\begin{cases}
            1&\text{ if  $ u_i \leq u<u_{i+1} $}\\
            0&\text{otherwise}
       \end{cases}\\
       N_{i,p}(u)&=\frac{u-u_i}{u_{i+p}-u_i}N_{i,p-1}(u)+\frac{u_{i+p+1}-u}{u_{i+p+1}-u_{i+1}}N_{i+1,p-1}(u)
    \end{align*}
    Easy to check the facts that 
    \begin{itemize}
        \item  $ N_{i,p}(u) $ is non-zero on  $ [u_i,u_{i+p+1}) $.
        \item On each interval  $ [u_i,u_{i+1}) $, there is at most  $ p+1 $  $ N_{i,p}(u) $ for some  $ i $.    
    \end{itemize}
    Therefore,  $ N_{i,3} $ is the basis of the B-spine space. 
\end{problem}

\begin{problem}
    For  $ f $ odd.

    Consider the optimal approximation polynomial  $ p $ on  $ [-1,1] $. Since  $ f(x)=-f(-x) $, so  $ -p(-x) $ is also  the optimal approximation polynomial of  $ -f(-x)=f(x) $ on  $ [-1,0] $ $ \Rightarrow $  By the uniqueness,  $ p(x)=-p(-x) $ \ie  $ p(x) $ is odd.
    
    For  $ f $ is even 
    
    Consider the optimal approximation polynomial  $ p $ on  $ [-1,1] $. Since  $ f(x)=f(-x) $, so  $ p(-x) $ is also  the optimal approximation polynomial of  $ f(-x)=f(x) $ on  $ [-1,0] $ $ \Rightarrow $  By the uniqueness,  $ p(x)=p(-x) $ \ie  $ p(x) $ is even.
\end{problem}
\begin{problem}
    For  $ s(x_j)=m_j $, 
     $ s(x) $ can be expressed as 
     \[s(x)=m_j\alpha_j(x)+m_{j+1}\alpha_{j+1}(x)+y_j\beta_j(x)+y_{j+1}\beta_{j+1}(x)\]
     on  $ (x_j,x_{j+1}) $, where  $ \alpha_i,\beta_j $ is the Hermite polynomial of degree 3.
     
     Use  $ s''(x_j^+)=s''(x_j^-) $ and we have 
     \[\frac{\mu_j}{h_j}m_{j+1}+(\frac{\lambda_j}{h_{j-1}}-\frac{\mu_j}{h_j})m_j-\frac{\lambda_j}{h_{j-1}}m_{j-1}=d_j\]
     where 
     \[d_j=\frac{1}{3}\left(\lambda_jy_{j-1}+2y_j+\mu_jy_{j+1}\right)\,,\lambda_j=\frac{h_j}{h_{j-1}+h_j}\,,\mu_j=\frac{h_{j-1}}{h_{j-1}+h_j}\,,h_j=x_{j+1}-x_j\]

     The boundary condition  $ s(x_0)=m_0=a,s(x_n)=m_{n}=b $ contributes to 
     \[\begin{pmatrix}
        \frac{\lambda_1}{h_0}-\frac{\mu_1}{h_1}&\frac{\mu_1}{h_1}&0&\cdots&0&0\\
        -\frac{\lambda_2}{h_1}&\frac{\lambda_2}{h_1}-\frac{\mu_2}{h_2}&\frac{\mu_2}{h_2}&\cdots&0&0\\
        0&-\frac{\lambda_3}{h2}&\frac{\lambda_3}{h_2}-\frac{\mu_3}{h_3}&\frac{\mu_3}{h_3}&\cdots&0\\
        \vdots&\vdots&\vdots&\vdots&\vdots&\vdots\\
        0&0&\cdots&0&-\frac{\lambda_{n-1}}{h_{n-2}}&\frac{\lambda_{n-1}}{h_{n-2}}-\frac{\mu_{n-1}}{h_{n-1}}
     \end{pmatrix}\cdot\begin{pmatrix}
        m_1\\
        m_2\\
        \vdots\\
        m_{n-1}
     \end{pmatrix}=\begin{pmatrix}
        d_1+\frac{\lambda_1}{h_0}a\\
        d_2\\
        \vdots\\
        d_{n-1}-\frac{\mu_{n-1}}{h_{n-1}}b
     \end{pmatrix}\]

     The boundary condition  $ s''(x_0)=y_0',s''(x_n)=y_n' $ contributes to 
     \[-2y_0-y_1+\frac{3}{h_0}m_1-\frac{3}{h_0}m_0=\frac{1}{2}y_0'h_0\]
     So we have 
     \[
        \begin{pmatrix}
            -\frac{3}{h_0}&\frac{3}{h_0}&0&\cdots&0&0&0\\
            -\frac{\lambda_1}{h_0}&\frac{\lambda_1}{h_0}-\frac{\mu_1}{h_1}&\frac{\mu_1}{h_1}&0&\cdots&0&0\\
            0&-\frac{\lambda_2}{h_1}&\frac{\lambda_2}{h_1}-\frac{\mu_2}{h_2}&\frac{\mu_2}{h_2}&\cdots&0&0\\
            0&0&-\frac{\lambda_3}{h_2}&\frac{\lambda_3}{h_2}-\frac{\mu_3}{h_2}&\frac{\mu_3}{h_3}&\cdots&0\\
            \vdots&\vdots&\vdots&\vdots&\vdots&\vdots&\vdots\\
            0&0&\cdots&0&0&-\frac{\lambda_{n-1}}{h_{n-2}}&\frac{\lambda_{n-1}}{h_{n-2}}-\frac{\mu_{n-1}}{h_{n-1}}\\
            0&0&\cdots&0&0&-\frac{3}{h_{n-1}}&\frac{3}{h_{n-1}}
         \end{pmatrix}\cdot\begin{pmatrix}
            m_0\\
            m_1\\
            m_2\\
            \vdots\\
            m_{n-1}\\
            m_n
         \end{pmatrix}=\begin{pmatrix}
            2y_0+y_1+\frac{1}{2}y_0'h_0\\
            d_1\\
            d_2\\
            \vdots\\
            d_{n-1}\\
            2y_n+y_{n-1}+\frac{1}{2}y_n'h_{n-1}
         \end{pmatrix}
     \] 

     And the period boudary condtion  $ s(x_0)=s(x_n),s'(x_0)=s'(x_n),s''(x_0)=s''(x_n) $ contributes to
     \[m_0=m_n,\frac{\mu_j}{h_j}m_{j+1}+(\frac{\lambda_j}{h_{j-1}}-\frac{\mu_j}{h_j})m_j-\frac{\lambda_j}{h_{j-1}}m_{j-1}=d_j\] 
     for  $ j=0 $,  $ \mu_{-1}=\mu_n $.
     
     So 
     \[\begin{pmatrix}
        \frac{\lambda_1}{h_0}-\frac{\mu_1}{h_1}&\frac{\mu_1}{h_1}&0&\cdots&0&-\frac{\lambda_1}{h_0}\\
        -\frac{\lambda_2}{h_1}&\frac{\lambda_2}{h_1}-\frac{\mu_2}{h_2}&\frac{\mu_2}{h_2}&\cdots&0&0\\
        0&-\frac{\lambda_3}{h2}&\frac{\lambda_3}{h_2}-\frac{\mu_3}{h_3}&\frac{\mu_3}{h_3}&\cdots&0\\
        \vdots&\vdots&\vdots&\vdots&\vdots&\vdots\\
        \frac{\mu_{n-1}}{h_{n-1}}&0&\cdots&0&-\frac{\lambda_{n-1}}{h_{n-2}}&\frac{\lambda_{n-1}}{h_{n-2}}-\frac{\mu_{n-1}}{h_{n-1}}
     \end{pmatrix}\cdot\begin{pmatrix}
        m_1\\
        m_2\\
        \vdots\\
        m_{n-1}
     \end{pmatrix}=\begin{pmatrix}
        d_1\\
        d_2\\
        \vdots\\
        d_{n-1}
     \end{pmatrix}\]

     In short, we discuss about three different boundary condition in this problem.


\end{problem}

\begin{problem}
    The boundary condition is 
    \[2m_0+m_1=3f[x_0,x_1]-\frac{1}{2}f''(x_0)h\]
    So combined with the equation in Lemma 2.5.2, we have 
    \[Am=d\]
    where 
    \[A=\begin{pmatrix}
        2&1\\
        \lambda_1&2&\mu_1\\
        0&\lambda_2&2&\mu_2\\
        \vdots&\vdots&\vdots&\vdots\\
        0&0&0&\lambda_{n-1}&2&\mu_{n-1}\\
        0&0&0&0&1&2
    \end{pmatrix}\]
    \[d=\begin{pmatrix}
        d_0\\
        d_1\\
        \vdots\\
        d_{n}
    \end{pmatrix}\]
     $ d_j=3\lambda_jf[x_{j-1},x_j]+3\mu_jf[x_j,x_{j+1}] $,  $ 1 \leq j \leq n-1 $.  $ d_0=3f[x_0,x_1]-\frac{1}{2}f''(x_0)h $,  $ d_n=3f[x_{n-1},x_n]-\frac{1}{2}f''(x_n)h $.
     
    Let  $ q=[m_0-f_0',m_1-f_1',\cdots,m_{n-1}-f'_{n-1}]^T $, then 
    \[Aq=c\]
    where  $ c=[c_j]^T $, 
    \[c_j=d_j-\lambda_{j}f'(x_{j-1})-2f'(x_j)-\mu_jf'(x_{j+1}),1 \leq j \leq n-1\]
    \[c_0=d_0-2f'(x_0)-f'(x_1),c_n=d_n-f'(x_{n-1})-2f'(x_n)\]
    Similar to Lemma 2.5.2, we can prove that 
    \[\|q\|_\infty \leq \|A^{-1}\|\cdot\|c\| \leq \|c\|_\infty\]
    And we have proved in Lemma 2.5.2 that 
    \[|c_j| \leq \frac{1}{24}h^3\|f^{(4)}\|_\infty,1 \leq j \leq n-1\]
    Now suffices to prove it for  $ c_0,c_n $.

    By Talor's equation 
    \begin{align*}
        c_0&=3\cdot\frac{f(x_1)-f(x_0)}{h_0}-2f'(x_0)-f'(x_1)-\frac{1}{2}f''(x_0)h_0\\
        &=3\left(f'(x_0)+\frac{1}{2}f''(x_0)h_0+\frac{1}{6}f'''(x_0)h_0^2+\frac{1}{6h_0}\int_{x_0}^{x_1}(x_1-v)^3f^{(4)}(v)\dd v\right)\\
        &-2f'(x_0)-\left(f'(x_0)+f''(x_0)h_0+\frac{1}{2}f'''(x_0)h_0^2+\frac{1}{2}\int_{x_0}^{x_1}(x_1-v)^2f^{(4)}(v)\dd v\right)-\frac{1}{2}f''(x_0)h_0\\
        &=\frac{1}{2}\int_{x_0}^{x_1}\left[\frac{1}{h_0}(x_1-v)^3-(x_1-v)^2\right]f^{(4)}(v)\dd v\\
        & \leq \frac{1}{2}\|f^{(4)}(v)\|_\infty\cdot \frac{1}{h_0}|\int_{0}^{h_0}\tau^3-h_0\tau^2\dd\tau|\\
        & =\frac{1}{24}\|f^{(4)}(v)\|_\infty
    \end{align*}

    By symmetry,  $ c_n $ also satisfy the same bound.

    So Lemma 2.5.2 still holds. Hence Theorem 2.5.1 still holds. \ie 
    \[\|s-f\|_\infty \leq \frac{5}{384}h^4\|f^{(4)}\|_\infty\] 
\end{problem}

\begin{problem}
    
    Let  $ \varphi(x) $ be the Hermite polynomial of degree 3 w.r.t  $ f $, \ie 
    \[\varphi(x)=f(x_j)\alpha_j(x)+f(x_{j+1})\alpha_{j+1}(x)+f'(x_j)\beta_j(x)+f'(x_{j+1})\beta_{j+1}(x)\]
    Then we have  $ \varphi(x)-f(x)=\dps\frac{f^{(4)}(\xi )}{24}\omega^2(x) $, where   $ \omega(x)=(x-x_j)(x-x_{j+1}) $.

    Note that Lemma 2.5.2 tells us  $ |m_j-f'_j| \leq \dps\frac{1}{24}h^3\|f^{(4)}\|_\infty $ 
    Since 
    \[s(x)-\varphi(x)=(m_j-f_j')\beta_j(x)+(m_{j+1}-f'_{j+1})\beta_{j+1}(x)\]
    we have 
    \begin{align*}
        |s'(x)-f'(x)|& \leq |s'(x)-\varphi'(x)|+|\varphi'(x)-f'(x)|\\
        & \leq \frac{1}{24}h^3\|f^{(4)}\|_\infty A+\frac{1}{24}\|f^{(4)}\|_\infty B\\
        & \leq \frac{1}{6}h^3\|f^{(4)}\|_\infty
    \end{align*}
    \begin{align*}
        |s''(x)-f''(x)|& \leq |s''(x)-\varphi''(x)|+|\varphi''(x)-f''(x)|\\
        & \leq \frac{1}{24}h^3\|f^{(4)}\|_\infty A'+\frac{1}{24}\|f^{(4)}\|_\infty B'\\
        & \leq \frac{2}{3}h^3\|f^{(4)}\|_\infty
    \end{align*}
    where \[A=\max|\beta_j'(x)|+|\beta_{j+1}'(x)| \leq 1,\, B=\max|\omega(x)\omega'(x)| \leq 3h^3\]
    \[A'=\max|\beta_j''(x)|+|\beta_{j+1}''(x)| \leq \frac{4}{h}\, B'=\max|(\omega^2(x))''| \leq 12h^2\]

    So  $ \|s'-f'\| \leq \frac{1}{6}h^3\|f^{(4)}\|_\infty $,  $ \|s''-f''\| \leq \frac{2}{3}h^3\|f^{(4)}\|_\infty $ 

\end{problem}
\end{document}