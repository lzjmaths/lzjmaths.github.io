% Sample tex file for usage of iidef.sty
% Homework template for Inference and Information
% UPDATE: October 12, 2017 by Xiangxiang
% UPDATE: 22/03/2018 by zhaofeng-shu33
\documentclass[a4paper]{article}
\usepackage[T1]{fontenc}
\usepackage{amsmath, amssymb, amsthm}
% amsmath: equation*, amssymb: mathbb, amsthm: proof
\usepackage{moreenum}
\usepackage{mathtools}
\usepackage{url}
\usepackage{graphicx}
\usepackage{subcaption}
\usepackage{booktabs} % toprule
\usepackage[mathcal]{eucal}
\usepackage{dsfont}

\usepackage{setspace}  
\setstretch{1.6}








\theoremstyle{definition}
\newtheorem{definition}{Definition}[section]
\newtheorem{example}[definition]{Example}
\newtheorem{exercise}[definition]{Exercise}
\newtheorem{remark}[definition]{Remark}
\newtheorem{observation}[definition]{Observation}
\newtheorem{assumption}[definition]{Assumption}
\newtheorem{convention}[definition]{Convention}
\newtheorem{priniple}[definition]{Principle}
\newtheorem{notation}[definition]{Notation}
\newtheorem*{axiom}{Axiom}
\newtheorem{coa}[definition]{Theorem}
\newtheorem{srem}[definition]{$\star$ Remark}
\newtheorem{seg}[definition]{$\star$ Example}
\newtheorem{sexe}[definition]{$\star$ Exercise}
\newtheorem{sdf}[definition]{$\star$ Definition}
\newtheorem{question}{Question}




\newtheorem{problem}{Problem}
%\renewcommand*{\theprob}{{\color{red}\arabic{section}.\arabic{prob}}}
\newtheorem{rprob}[problem]{\color{red} Problem}
%\renewcommand*{\thesprob}{{\color{red}\arabic{section}.\arabic{sprob}}}
% \newtheorem{ssprob}[prob]{$\star\star$ Problem}



\theoremstyle{plain}
\newtheorem{theorem}[definition]{Theorem}
\newtheorem{Conclusion}[definition]{Conclusion}
\newtheorem{thd}[definition]{Theorem-Definition}
\newtheorem{proposition}[definition]{Proposition}
\newtheorem{corollary}[definition]{Corollary}
\newtheorem{lemma}[definition]{Lemma}
\newtheorem{sthm}[definition]{$\star$ Theorem}
\newtheorem{slm}[definition]{$\star$ Lemma}
\newtheorem{claim}[definition]{Claim}
\newtheorem{spp}[definition]{$\star$ Proposition}
\newtheorem{scorollary}[definition]{$\star$ Corollary}


\newtheorem{condition}{Condition}
\newtheorem{Mthm}{Main Theorem}
\renewcommand{\thecondition}{\Alph{condition}} % "letter-numbered" theorems
\renewcommand{\theMthm}{\Alph{Mthm}} % "letter-numbered" theorems


%\substack   multiple lines under sum
%\underset{b}{a}   b is under a


% Remind: \overline{L_0}



\usepackage{calligra}
\DeclareMathOperator{\shom}{\mathscr{H}\text{\kern -3pt {\calligra\large om}}\,}
\DeclareMathOperator{\sext}{\mathscr{E}\text{\kern -3pt {\calligra\large xt}}\,}
\DeclareMathOperator{\Rel}{\mathscr{R}\text{\kern -3pt {\calligra\large el}~}\,}
\DeclareMathOperator{\sann}{\mathscr{A}\text{\kern -3pt {\calligra\large nn}}\,}
\DeclareMathOperator{\send}{\mathscr{E}\text{\kern -3pt {\calligra\large nd}}\,}
\DeclareMathOperator{\stor}{\mathscr{T}\text{\kern -3pt {\calligra\large or}}\,}
%write mathscr Hom (and so on) 

\usepackage{aurical}
\DeclareMathOperator{\VVir}{\text{\Fontlukas V}\text{\kern -0pt {\Fontlukas\large ir}}\,}

\newcommand{\vol}{\text{\Fontlukas V}}
\newcommand{\dvol}{d~\text{\Fontlukas V}}
% perfect Vol symbol

\usepackage{aurical}








\newcommand{\fk}{\mathfrak}
\newcommand{\mc}{\mathcal}
\newcommand{\wtd}{\widetilde}
\newcommand{\wht}{\widehat}
\newcommand{\wch}{\widecheck}
\newcommand{\ovl}{\overline}
\newcommand{\udl}{\underline}
\newcommand{\tr}{\mathrm{t}} %transpose
\newcommand{\Tr}{\mathrm{Tr}}
\newcommand{\End}{\mathrm{End}} %endomorphism
\newcommand{\idt}{\mathbf{1}}
\newcommand{\id}{\mathrm{id}}
\newcommand{\Hom}{\mathrm{Hom}}
\newcommand{\cond}[1]{\mathrm{cond}_{#1}}
\newcommand{\Conf}{\mathrm{Conf}}
\newcommand{\Res}{\mathrm{Res}}
\newcommand{\res}{\mathrm{res}}
\newcommand{\KZ}{\mathrm{KZ}}
\newcommand{\ev}{\mathrm{ev}}
\newcommand{\coev}{\mathrm{coev}}
\newcommand{\opp}{\mathrm{opp}}
\newcommand{\Rep}{\mathrm{Rep}}
\newcommand{\Dom}{\mathrm{Dom}}
\newcommand{\loc}{\mathrm{loc}}
\newcommand{\con}{\mathrm{c}}
\newcommand{\uni}{\mathrm{u}}
\newcommand{\ssp}{\mathrm{ss}}
\newcommand{\di}{\slashed d}
\newcommand{\Diffp}{\mathrm{Diff}^+}
\newcommand{\Diff}{\mathrm{Diff}}
\newcommand{\PSU}{\mathrm{PSU}(1,1)}
\newcommand{\Vir}{\mathrm{Vir}}
\newcommand{\Witt}{\mathscr W}
\newcommand{\Span}{\mathrm{Span}}
\newcommand{\pri}{\mathrm{p}}
\newcommand{\ER}{E^1(V)_{\mathbb R}}
\newcommand{\prth}[1]{( {#1})}
\newcommand{\bk}[1]{\langle {#1}\rangle}
\newcommand{\bigbk}[1]{\big\langle {#1}\big\rangle}
\newcommand{\Bigbk}[1]{\Big\langle {#1}\Big\rangle}
\newcommand{\biggbk}[1]{\bigg\langle {#1}\bigg\rangle}
\newcommand{\Biggbk}[1]{\Bigg\langle {#1}\Bigg\rangle}
\newcommand{\GA}{\mathscr G_{\mathcal A}}
\newcommand{\vs}{\varsigma}
\newcommand{\Vect}{\mathrm{Vec}}
\newcommand{\Vectc}{\mathrm{Vec}^{\mathbb C}}
\newcommand{\scr}{\mathscr}
\newcommand{\sjs}{\subset\joinrel\subset}
\newcommand{\Jtd}{\widetilde{\mathcal J}}
\newcommand{\gk}{\mathfrak g}
\newcommand{\hk}{\mathfrak h}
\newcommand{\xk}{\mathfrak x}
\newcommand{\yk}{\mathfrak y}
\newcommand{\zk}{\mathfrak z}
\newcommand{\pk}{\mathfrak p}
\newcommand{\hr}{\mathfrak h_{\mathbb R}}
\newcommand{\Ad}{\mathrm{Ad}}
\newcommand{\DHR}{\mathrm{DHR}_{I_0}}
\newcommand{\Repi}{\mathrm{Rep}_{\wtd I_0}}
\newcommand{\im}{\mathbf{i}}
\newcommand{\Co}{\complement}
%\newcommand{\Cu}{\mathcal C^{\mathrm u}}
\newcommand{\RepV}{\mathrm{Rep}^\uni(V)}
\newcommand{\RepA}{\mathrm{Rep}(\mathcal A)}
\newcommand{\RepN}{\mathrm{Rep}(\mathcal N)}
\newcommand{\RepfA}{\mathrm{Rep}^{\mathrm f}(\mathcal A)}
\newcommand{\RepAU}{\mathrm{Rep}^\uni(A_U)}
\newcommand{\RepU}{\mathrm{Rep}^\uni(U)}
\newcommand{\RepL}{\mathrm{Rep}^{\mathrm{L}}}
\newcommand{\HomL}{\mathrm{Hom}^{\mathrm{L}}}
\newcommand{\EndL}{\mathrm{End}^{\mathrm{L}}}
\newcommand{\Bim}{\mathrm{Bim}}
\newcommand{\BimA}{\mathrm{Bim}^\uni(A)}
%\newcommand{\shom}{\scr Hom}
\newcommand{\divi}{\mathrm{div}}
\newcommand{\sgm}{\varsigma}
\newcommand{\SX}{{S_{\fk X}}}
\newcommand{\DX}{D_{\fk X}}
\newcommand{\mbb}{\mathbb}
\newcommand{\mbf}{\mathbf}
\newcommand{\bsb}{\boldsymbol}
\newcommand{\blt}{\bullet}
\newcommand{\Vbb}{\mathbb V}
\newcommand{\Ubb}{\mathbb U}
\newcommand{\Xbb}{\mathbb X}
\newcommand{\Kbb}{\mathbb K}
\newcommand{\Abb}{\mathbb A}
\newcommand{\Wbb}{\mathbb W}
\newcommand{\Mbb}{\mathbb M}
\newcommand{\Gbb}{\mathbb G}
\newcommand{\Cbb}{\mathbb C}
\newcommand{\Nbb}{\mathbb N}
\newcommand{\Zbb}{\mathbb Z}
\newcommand{\Qbb}{\mathbb Q}
\newcommand{\Pbb}{\mathbb P}
\newcommand{\Rbb}{\mathbb R}
\newcommand{\Ebb}{\mathbb E}
\newcommand{\Dbb}{\mathbb D}
\newcommand{\Hbb}{\mathbb H}
\newcommand{\cbf}{\mathbf c}
\newcommand{\Rbf}{\mathbf R}
\newcommand{\wt}{\mathrm{wt}}
\newcommand{\Lie}{\mathrm{Lie}}
\newcommand{\btl}{\blacktriangleleft}
\newcommand{\btr}{\blacktriangleright}
\newcommand{\svir}{\mathcal V\!\mathit{ir}}
\newcommand{\Ker}{\mathrm{Ker}}
\newcommand{\Cok}{\mathrm{Coker}}
\newcommand{\Sbf}{\mathbf{S}}
\newcommand{\low}{\mathrm{low}}
\newcommand{\Sp}{\mathrm{Sp}}
\newcommand{\Rng}{\mathrm{Rng}}
\newcommand{\vN}{\mathrm{vN}}
\newcommand{\Ebf}{\mathbf E}
\newcommand{\Nbf}{\mathbf N}
\newcommand{\Stb}{\mathrm {Stb}}
\newcommand{\SXb}{{S_{\fk X_b}}}
\newcommand{\pr}{\mathrm {pr}}
\newcommand{\SXtd}{S_{\wtd{\fk X}}}
\newcommand{\univ}{\mathrm {univ}}
\newcommand{\vbf}{\mathbf v}
\newcommand{\ubf}{\mathbf u}
\newcommand{\wbf}{\mathbf w}
\newcommand{\CB}{\mathrm{CB}}
\newcommand{\Perm}{\mathrm{Perm}}
\newcommand{\Orb}{\mathrm{Orb}}
\newcommand{\Lss}{{L_{0,\mathrm{s}}}}
\newcommand{\Lni}{{L_{0,\mathrm{n}}}}
\newcommand{\UPSU}{\widetilde{\mathrm{PSU}}(1,1)}
\newcommand{\Sbb}{{\mathbb S}}
\newcommand{\Gc}{\mathscr G_c}
\newcommand{\Obj}{\mathrm{Obj}}
\newcommand{\bpr}{{}^\backprime}
\newcommand{\fin}{\mathrm{fin}}
\newcommand{\Ann}{\mathrm{Ann}}
\newcommand{\Real}{\mathrm{Re}}
\newcommand{\Imag}{\mathrm{Im}}
%\newcommand{\cl}{\mathrm{cl}}
\newcommand{\Ind}{\mathrm{Ind}}
\newcommand{\Supp}{\mathrm{Supp}}
\newcommand{\Specan}{\mathrm{Specan}}
\newcommand{\red}{\mathrm{red}}
\newcommand{\uph}{\upharpoonright}
\newcommand{\Mor}{\mathrm{Mor}}
\newcommand{\pre}{\mathrm{pre}}
\newcommand{\rank}{\mathrm{rank}}
\newcommand{\Jac}{\mathrm{Jac}}
\newcommand{\emb}{\mathrm{emb}}
\newcommand{\Sg}{\mathrm{Sg}}
\newcommand{\Nzd}{\mathrm{Nzd}}
\newcommand{\Owht}{\widehat{\scr O}}
\newcommand{\Ext}{\mathrm{Ext}}
\newcommand{\Tor}{\mathrm{Tor}}
\newcommand{\Com}{\mathrm{Com}}
\newcommand{\Mod}{\mathrm{Mod}}
\newcommand{\nk}{\mathfrak n}
\newcommand{\mk}{\mathfrak m}
\newcommand{\Ass}{\mathrm{Ass}}
\newcommand{\depth}{\mathrm{depth}}
\newcommand{\Coh}{\mathrm{Coh}}
\newcommand{\Gode}{\mathrm{Gode}}
\newcommand{\Fbb}{\mathbb F}
\newcommand{\sgn}{\mathrm{sgn}}
\newcommand{\Aut}{\mathrm{Aut}}
\newcommand{\Modf}{\mathrm{Mod}^{\mathrm f}}
\newcommand{\codim}{\mathrm{codim}}
\newcommand{\card}{\mathrm{card}}
\newcommand{\dps}{\displaystyle}
\newcommand{\Int}{\mathrm{Int}}
\newcommand{\Nbh}{\mathrm{Nbh}}
\newcommand{\Pnbh}{\mathrm{PNbh}}
\newcommand{\Cl}{\mathrm{Cl}}
\newcommand{\diam}{\mathrm{diam}}
\newcommand{\eps}{\varepsilon}
\newcommand{\Vol}{\mathrm{Vol}}
\newcommand{\LSC}{\mathrm{LSC}}
\newcommand{\USC}{\mathrm{USC}}
\newcommand{\Ess}{\mathrm{Rng}^{\mathrm{ess}}}
\newcommand{\Jbf}{\mathbf{J}}
\newcommand{\SL}{\mathrm{SL}}
\newcommand{\GL}{\mathrm{GL}}
\newcommand{\Lin}{\mathrm{Lin}}
\newcommand{\ALin}{\mathrm{ALin}}
\newcommand{\bwn}{\bigwedge\nolimits}
\newcommand{\nbf}{\mathbf n}
\newcommand{\dive}{\mathrm{div}}




\usepackage{algorithm}
\usepackage{algorithmic}

\newcommand{\<}{\left<}
\renewcommand{\>}{\right>}


\numberwithin{equation}{problem}
% count the eqation by section countation


\DeclareMathOperator{\sign}{sign}
\DeclareMathOperator{\dom}{dom}
\DeclareMathOperator{\ran}{ran}
\DeclareMathOperator{\ord}{ord}
\DeclareMathOperator{\img}{Im}
\DeclareMathOperator{\dd}{d\!}
\newcommand{\ie}{ \textit{ i.e. } }
\newcommand{\st}{ \textit{ s.t. }}


\usepackage[numbered,framed]{matlab-prettifier}
\lstset{
  style              = Matlab-editor,
  captionpos         =b,
  basicstyle         = \mlttfamily,
  escapechar         = ",
  mlshowsectionrules = true,
}

\usepackage[thehwcnt = 1]{iidef}
\thecourseinstitute{Tsinghua University}
\thecoursename{Numerical Analysis}
\theterm{Fall 2024}
\hwname{Homework}
\usepackage{geometry}
\geometry{left=1.5cm,right=1.5cm,top=2.5cm,bottom=2.5cm}
\begin{document}
\courseheader
\name{Lin Zejin}
\rule{\textwidth}{1pt}
\begin{itemize}
\item {\bf Collaborators: \/}
  I finish this homework by myself. 
%   If you finish your homework all by yourself, make a similar statement. If you get help from others in finishing your homework, state like this:
%   \begin{itemize}
%   \item 1.2 (b) was solved with the help from \underline{\hspace{3em}}.
%   \item Discussion with \underline{\hspace{3em}} helped me finishing 1.3.
%   \end{itemize}
\end{itemize}
\rule{\textwidth}{1pt}

\vspace{2em}

\sloppy
\pagenumbering{arabic}

\begin{problem}
    Assume there exists  $ x_1,x_2,\cdots,x_{2n+1}\in [a,a+2\pi) $ \st    
    \[\begin{pmatrix}
        1\\
        \cos x_i\\
        \sin x_i\\
        \vdots\\
        \cos nx_i\\
        \sin nx_i
    \end{pmatrix}\]
    linearly dependent.

    \ie  $ \exists  a_1,\cdots,a_{2n+1}\in \Rbb $, such that 
    \[\sum_{i=1}^{2n+1}a_i \begin{pmatrix}
        1\\
        \cos x_i\\
        \sin x_i\\
        \vdots\\
        \cos nx_i\\
        \sin nx_i
    \end{pmatrix}=0\]

    Since  $ e^{i x}=\cos x+i\sin x $, we have 
    \[\sum_{j=1}^{n+1}(a_{2j-1}+a_{2j})\begin{pmatrix}
        1\\
        e^{ix_1}\\
        \vdots\\
        e^{ix_n}
    \end{pmatrix}\]
    which is impossible since we know that the Vandermonde determinant is invertible. (In this equation,  $ a_{2n+2}=0 $)

\end{problem}

\begin{problem}
    Assume  $ \exists a=x_1<x_2<\cdots<x_N \leq b $ such that  $ |\epsilon(x_i)|=\Delta(P),\epsilon(x_j)=(-1)^{j-1}\epsilon (x_1) $,  $ j=0,1,\cdots, $.n Then  $ \forall Q\in \Span\{g_1,\cdots,g_N\} $, if  $ \Delta(Q)<\Delta(P) $, 
    let 
    \[\eta(x)=P(x)-Q(x)=(P(x)-f(x))-(Q(x)-f(x))\]
    Then 
    \[\sgn(\eta(x_j))=\eta(P(x_j)-f(x_j))=\eta(\epsilon(x_j))=(-1)^{j-1},j=0,1,\cdots,n\]
    So  $ Q $  has    at least  $ n $ roots on  $ [a,b] $. Since  $ \{g_1,\cdots,g_n\} $ satisfies the Haar condition,  $ Q\equiv 0 $.
    
    So  $ P $ is the best approximation of  $ f $.
    
    Conversely, if  $ P $  is the best approximation. If the result is not true, then we can divide  $ [a,b] $  into 
    \[[a,\zeta_1],[\zeta_1,\zeta_2],\cdots,[\zeta_{N},b]\]
    such that on each interval  $ \Delta(P) $ satisfies  $ N \leq n-1 $ and  
    \[-\Delta(P) \leq \epsilon(x)<\Delta(P)-\alpha\]
    or 
    \[-\Delta(P)+\alpha \leq \epsilon(x)<\Delta(P)\] 
    Denote  $ \Phi(x) $ as an element with roots  $ \zeta_1,\cdots,\zeta_N $. (The existence because of Haar condition)
    
    Then  $ Q(x):=P(x)+\omega \Phi(x) $ with difference 
    \[Q(x)-f(x)=P(x)-f(x)+\omega\Phi(x)\]
    On  $ [a,b] $,  $ \Phi(x) $ is bounded. Take  $ |\omega| $ sufficiently small, and choose the signature of  $ \omega $     properly, we have 
    \[\Delta(Q)<\Delta(P)\]
    which causes contradiction.

    Here we end the proof.
\end{problem}

\begin{problem}
    Replace  $ f $ with  $ f-p_n $. WLOG we assume the best approximation polynomial is 0.
    
    If  $ \exists q_n $ such that 
    \[\|f-q_n\|<\|f\|+\lambda\|q_n\|\]
    where  $ \lambda<\frac{1}{2} $.
    
    Now if  $ \forall \lambda_m=\frac{1}{m},m \geq 2 $,   $ \exists  $ $ q_m $ such that 
    \[\|f-q_m\|<\|f\|+\lambda_m\|q_m\|\]
    Since  $ \|f-q_m\| \geq \|q_m\|-\|f\| $, we have  $ \|q_m\|<\frac{2}{1-\lambda_m}\|f\|<4\|f\| $.
    
    So  $ \|q_m\| $ are uniformly bounded. Hence,  $ \{q_m\} $ is precompact in the polynomial space, or equivalently, there exists  $ q\in \Pbb_n $ such that some subsequence  $ \{q_{m_i}\} $ converges to  $ q $.
    
    As  $ m\rightarrow0,\,\lambda_m\rightarrow0 $, then 
    \[\|f-q\| \leq \|f\|\]
    So  $ q\equiv 0 $. 
    
    So  $ \exists N>0 $ such that  $ \forall i \geq N $,  $ \|q_{m_i}\|<\|f\| $. Choose  $ q $ be some  $ q_{m_i} $,  $ i \geq N $.   
    
    Now for  $ x^i=\arg\max|f(x)-q_{m_i}(x)| $, since  $ |f(x^i)-q_{m_i}(x^i)| \geq \|f\| $,  $ q_{m_i}(x^i) $ and  $ f(x^i) $  have  different signature. So  $ |f(x^i)| \geq \|f\|-|q_{m_i}(x^i)| $ 
    
    Since there exists  $ a \leq x_0<x_1<\cdots<x_{n+1} \leq b $  such that  $ f(x_i)=\delta(-1)^i\|f\| $ where  $ \delta=\pm 1 $, in the finite dimentional polynomial space, the norm is all equivalent. Therefore  $ \exists \lambda>0 $ \st 
    \[\max_{0 \leq i \leq n+1}|q(x_i)|>\lambda\|q\|\]
    But if   $ q(x_i)f(x_i)>0 $, then  $ q $ has root on each interval  $ (x_i,x_{i+1}) $,  $ i=0,1,\cdots,n $, which contradicts with the fact that  $ q  $ has degree of  $ n $.
    
    So  $ \exists i $ \st  $ q(x_i)f(x_i) \leq 0 $. Then  $ |f(x_i)-q(x_i)|=|f(x_i)|+|q(x_i)| \geq \|f\|+\lambda\|q\| $.

    
\end{problem}

\begin{problem}
    For  $ x\in [a,b], $ WLOG assume  $ x\neq x_i $. ($ x=x_i $ is trivial)
    Define 
    \[G(t)=R_{2n+1}(t)-\frac{\omega_{n+1}^2(t)}{\omega^2_{n+1}(x)}R_{2n+1}(x)\]
    Then 
    \[G(x_i)=0,G(x)=0\]
    So there are   $ n+2 $ roots on  $ [a,b] $.
    
    By Rolle's theorem, there are  $ n+1 $ roots on  $ [a,b]\setminus\{x_0,\cdots,x_n,x\} $.
    
    Since  $ G'(t)=R_{2n+1}'(t)-\frac{\omega_{n+1}(t)\omega'_{n+1}(t)}{\omega^2_{n+1}}R_{2n+1}(x) $,  $ G'(x_i)=0 $.
    
    So there are at least $ 2n+2 $ roots on $ [a,b] $ of  $ G' $.
    
    Apply  $ 2n+1 $ times of Rolle's theorem to  $ G' $, we obtain there is at least one root on  $ [a,b] $  of  $ G^{(2n+2)} $.

    So  $ \exists \zeta\in [a,b] $,  $ 0=G^{(2n+2)}(\zeta)=f^{(2n+2)}(\zeta)-\frac{(2n+2)!}{\omega_{n+1}^2(x)}R_{2n+1}(x) $.
    
    So  $ \exists \zeta\in [a,b] $,   $ R_{2n+1}(x)=\dps\frac{f^{(2n+2)}(zeta)}{(2n+2)!}\omega_{n+1}(x) $ 
\end{problem}

\begin{problem}
    By partition of unity, we could find  $ \|f\|=1 $ such that  $ f(t)\cdot D_n(t)=|D_n(t)| $ except for a small enough set  $ E $ with  $ m(E)<\epsilon $.
    Then   
    \[\|s_n\| \geq |\frac{1}{\pi}\int_{-\pi}^{\pi}f(t)D_n(t)\dd t| \geq \frac{1}{\pi}\int_{-\pi}^\pi |D_n(t)|\dd t -\epsilon\cdot \|D_n\| =\lambda_n-\epsilon \cdot \|D_n\|\]
    Hence if let  $ \epsilon\rightarrow 0 $,  $ \|s_n\| \geq \lambda_n $. Therefore,  $ \|s_n\|=\lambda_n $.   
\end{problem}
\begin{problem}
    Define the inner product in  $ C^2[a,b] $  as 
    \[\<f,g\>=\int_a^b\<f'',g''\>\dd x\]
    Easy to check that it is  a inner product.

    Let  $ g=f-s $.
    
    Then  $ \dps\int_a^b g''(x)\dd x=\left.(f'(x)-s'(x))\right|_a^b=0 $.
    
     $ \dps\int_a^b xg''(x)\dd x=\left.(xg'(x))\right|_a^b-\int_a^b g'(x)\dd x=0 $.
     
     Therefore  $ \dps\int_a^b p(x) g''(x)\dd x=0$ for all  $ p $ polynomial of degree 1.
     
     Since  $ s''\in \Pbb_1 $,  $ \<s,g\>=0 $. Therefore, 
     \[\|f\|=\|s\|+\|f-s\| \geq \|s\|\]
     It is what we need.  
\end{problem}

\begin{problem}
    Noticed that  $ f(x)=-\frac{3}{4}(x-1)(x-2)(x+\frac{2}{3})+1 $ satisfies 
    \[f(1)=f(2)=1,\,f(0)=0\]
    and  
    \[f'(0)=0\]
    So  $ f(x) $ is what we need.
\end{problem}

\end{document}