\section{函数逼近}
\begin{theorem}[Weierstrass]\label{Weierstrass}
     $ f(x)\in C([a,b]) $, 则对于 $ \forall \epsilon>0 $,存在一个 $ n $ 次多项式 $ P_n (x)  $ 使得 $ \|f(x)-P_n(x)\|_{\infty}<\epsilon $.    
\end{theorem}
\begin{proof}
    令
    \begin{equation}
        B_n^f(x)=\sum_{k=0}^nf(\frac{k}{n})C_n^kx^k(1-x)^{n-k}
    \end{equation}
    不妨设 $ f(x)\in C([0,1]) $。
    注意到 
    \[\sum_{k=0}^nC_n^kx^k(1-x)^{n-k}=1\]
    \[\sum_{k=0}^n(nx-k)^2C_n^kx^k(1-x)^{n-k}=nx(1-x)\] 
    由于 $ f  $ 在 $ [0,1] $ 上一致连续,容易验证成立。 
\end{proof}
\begin{theorem}[Weierstrass第二定理]\label{Second Weierstrass}
    设 $ f\in C_{2\pi} $, $ \forall \epsilon>0  $,存在三角多项式 $ T(x)  $ 使得 $ \forall x\in \Rbb $
    \[|T(x)-f(x)|<\epsilon,\,\forall x\in [0,2\pi]\]   
\end{theorem}
\begin{lemma}
    设 $ f\in C[0,\pi] $, $ \forall \epsilon>0  $,存在偶的三角多项式 $ T(x)  $ 使得 $ \forall x\in \Rbb $
    \[|T(x)-f(x)|<\epsilon,\,\forall x\in [0,\pi]\]   
\end{lemma}
\begin{proof}
    考虑 $ f(\arccos x)\in C[-1,1] $,则由Theorem \ref{Weierstrass} 即得。
\end{proof}
因此由Theorem \ref{Weierstrass}和引理很容易得到Theorem \ref{Second Weierstrass}是正确的。

第二定理推第一定理:设 $ f(x)\in C([-\pi,\pi]) $,令 
\[g(x)=f(x)+\dps\frac{f(-\pi)-f(\pi)}{2\pi}x,\,x\in [-\pi ,\pi]\]
可延拓至 $ \Rbb  $ 上的周期函数。

设 $ X  $ 为紧距离空间, $ A\subset C(X) $。称 $ A  $ \name{分离} $ X  $ 中的点,如果 $ \forall x,y\in X $, $ x\neq y $, 存在 $ f\in A  $ 使得
\[f(x)\neq f(y)\]    
\begin{theorem}[Stone]
    设 $ X  $ 是紧距离空间,  $ A  $ 是 $ C(X)  $ 的子代数。若 $ 1\in A $,且 $ A  $ 分离 $ X  $ 中的点,则 $ A  $ 在 $ C(X)  $ 中稠密
\end{theorem}

设 $ p\in\Pbb_n=\{\text{小于等于 $ n $ 次多项式全体}\} $。令 
\[\Delta(p)=\max_{x\in [a,b]}|f(x)-p(x)|\]\index{$ \Delta(p) $}
\[E_n=\inf_{p\in \Pbb_n}\Delta(p)\]\index{$ E_n $}
\begin{theorem}[Borel]
    对于 $ \forall f(x)\in C[a,b]$,存在 $ p^*\in \Pbb_n $ 使得
    \[\Delta(p^*)=E_n\]  
\end{theorem}
记 $ \epsilon(x)=p(x)-f(x) $,若\index{$ \epsilon(x) $}
\[|\epsilon(x_0)|=\Delta(p)\]
则称 $ x_0 $ 为\name{偏离点}。

若 $ \epsilon(x_0)>0 $,则称为正偏离点,反之为负偏离点。

\begin{lemma}
    若 $ p(x)  $ 为 $ f(x) $ 的最佳逼近多项式,则正负偏离点必都存在 
\end{lemma}
只需作一定的微调。

\begin{theorem}[Vall{\'e}e-Poussin]
    设 $ p(x)\in \Pbb_n $, $ \epsilon(x) $ 在 $ x_1<x_2<\cdots<x_N $ 上取值为非零的正负相间值 $ \lambda_1,-\lambda_2,\cdots,(-1)^{N-1}\lambda_N $, $ \lambda_j>0 $, $ j=1,\cdots,N $ 且 $ N \geq n+2 $。则 $ \forall  $  $ Q(x)\in \Pbb_n $
    \[\Delta(Q) \geq \min_{1 \leq i \leq N}\lambda_i\]        
\end{theorem}
证明考虑 $ P(x)-Q(x) $ 的正负性和零点的关系.
\begin{theorem}[Chebyshev]
    对于任意 $ f(x)\in C[a,b]  $, $ \Pbb_n  $ 中的最佳逼近多项式存在且唯一,且 $ p(x)  $ 为最佳逼近多项式当且仅当存在 $ a \leq x_1<x_2<\cdots<x_N \leq b $,  $ N \geq n+2 $ 使得
    \[|\epsilon(x_j)|=\Delta(p),\,\epsilon(x_j)=(-1)^{j-1}\epsilon(x_1),\,j=1,\cdots,N\]   
\end{theorem}
\begin{definition}[连续模]
    设 $ f(x)  $ 定义于 $ [a,b] $ 上,则 
    \[\omega(t)=\omega(t,f)=\sup_{|x-y| \leq t}|f(x)-f(y)|\] 
    称为 $ f(x) $ 在 $ [a,b] $ 上的\name{连续模}  
\end{definition}
\begin{proposition}
    \,\\
    \begin{enumerate}
        \item 若 $ f(x) \in C[a,b] $,则 $ \omega(t) $ 是 $ t $ 的连续非减函数且 $ \dps\lim_{t\to0}\omega(t)=0 $.
        \item (半可加性) $ \forall t_1,t_2 \geq 0 $, $ \omega(t_1+t_2) \leq \omega(t_1)+\omega(t_2) $。
        \item 若 $ \omega(t)=o(t) $, $ t\rightarrow 0 $,则 $ f(x)\equiv  $ constant.         
    \end{enumerate}
\end{proposition}
\begin{definition}
    若 $ \omega(t,f) \leq Mt^\alpha $, $ 0<\alpha \leq 1 $,则称 $ f(x) $ 在 $ [a,b] $ 上满足 $ \alpha $ 阶Lipschitz条件,记作 $ f(x)\in \mathrm{Lip}\alpha $     
\end{definition}
\begin{theorem}[Jackson]
    设 $ f(x)\in C_{2\pi} $,则
    \[E_n(f) \leq 12\omega(\frac{1}{n},f)\] 
\end{theorem}
\begin{proof}
    证明考虑 $ \dps\Phi_n(t)=\frac{1}{A_n}\left(\frac{\sin\frac{nt}{2}}{\sin\frac{t}{2}}\right)^4 $,其在弱的意义下逼近 $ \Delta $ 函数 
\end{proof}
\begin{corollary}
     $ f\in C_{2\pi} $ 且 $ f'\in C_{2\pi} $,则 
    \[E_n(f) \leq \dps\frac{12}{n}\|f'\|_{\infty}\]  
\end{corollary}