 % !TeX spellcheck = en_US
% !TEX program = pdflatex
\documentclass[12pt,b5paper,notitlepage]{article}
\usepackage[b5paper, margin={0.5in,0.65in}]{geometry}
%\usepackage{fullpage}
\usepackage{amsmath,amscd,amssymb,amsthm,mathrsfs,amsfonts,layout,indentfirst,graphicx,caption,mathabx, stmaryrd,appendix,calc,imakeidx,upgreek,amsbsy,thmtools} % mathabx for \wtidecheck
\usepackage{mathtools}
\let \underbrace\overbrace \relax

\usepackage{ulem} %wave underline
\usepackage[dvipsnames]{xcolor}
\usepackage{palatino}  %template

\usepackage{stmaryrd}

\usepackage{slashed} % Dirac operator
\usepackage{mathrsfs} % Enable using \mathscr
%\usepackage{eufrak}  another template/font
\usepackage{extarrows} % long equal sign, \xlongequal{blablabla}
\usepackage{enumitem} % enumerate label change e.g. [label=(\alph*)]  shows (a) (b) 


%%%%%%%%%%%%%%%%%%%%%%%%%%%%%%

%\usepackage{fontspec}
%\setmainfont{Palatino Linotype}
%\usepackage{emoji}


% emoji, use lualatex  remove \usepackage{palatino}

%%%%%%%%%%%%%


\usepackage{CJK}   % Chinese package





\usepackage{csquotes} % \begin{displayquote}   \begin{displaycquote}  for quotation
\usepackage{epigraph}   %\epigraph{}{}  for quotation
%\pmb  mandatory math bold 

\usepackage{fancyhdr} % date in footer

%\usepackage{soul}  %\ul underline break line automatically

\usepackage{ulem}  % \uline  underline break line   also    \uwave

\usepackage{relsize} % use \mathlarger \larger \text{\larger[2]$...$} to enlarge the size of math symbols

\usepackage{verbatim}  % comment environment


\usepackage{halloweenmath} % Interesting halloween math symbols

%%%%%%%%%%%%%%%%%%%%%%%%%%%%%%
\usepackage{tcolorbox}
\tcbuselibrary{theorems}
% box around equations   \tcboxmath
%%%%%%%%%%%%%%%%%%%%%%%%%%%%%%%%%%





%%%%%%%%%%%%%%%%%%%%%%%%%%%%%
% circled colon and thick colon \hcolondel and \colondel

\usepackage{pdfrender}

\newcommand*{\hollowcolon}{%
	\textpdfrender{
		TextRenderingMode=Stroke,
		LineWidth=.1bp,
	}{:}%
}

\newcommand{\hcolondel}[1]{%
	\mathopen{\hollowcolon}#1\mathclose{\hollowcolon}%
}
\newcommand{\colondel}[1]{%
	\mathopen{:}#1\mathclose{:}%
}

%%%%%%%%%%%%%%%%%%%%%%%%%%%%%%%%


\usepackage{setspace}  
\setstretch{1.6}



\usepackage{tikz}
\usetikzlibrary{fadings}
\usetikzlibrary{patterns}
\usetikzlibrary{shadows.blur}
\usetikzlibrary{shapes}

\usepackage{tikz-cd}
\usepackage[nottoc]{tocbibind}   % Add  reference to ToC


\makeindex


% The following set up the line spaces between items in \thebibliography
\usepackage{lipsum}  
\let\OLDthebibliography\thebibliography
\renewcommand\thebibliography[1]{
	\OLDthebibliography{#1}
	\setlength{\parskip}{0pt}
	\setlength{\itemsep}{2pt} 
}


%\hyperref{page.10}{...}

\allowdisplaybreaks  %allow aligns to break between pages
\usepackage{latexsym}
\usepackage{chngcntr}
\usepackage[colorlinks,linkcolor=blue,anchorcolor=blue, linktocpage,
%pagebackref
]{hyperref}
\hypersetup{ urlcolor=cyan,
	citecolor=[rgb]{0,0.5,0}}


\setcounter{tocdepth}{2}	 %hide subsections in the content


\counterwithin{figure}{section}

\counterwithin*{footnote}{section}   % Footnote numbering is recounted from the beginning of each subsection

\renewcommand\theenumi{(\arabic{enumi})}
\renewcommand\theenumii{\alph(enumii)}
\renewcommand\theenumiii{\roman{enumii})}

%Greek letters counter
\newcounter{greek}
\renewcommand{\thegreek}{
  \ifcase\value{gr}
     $ \alpha $ 
  \or
     $ \beta $ 
  \or
     $ \gamma $ 
  \or
    $\delta  $ 
  \or
    $\epsilon  $ 
  \or
    $\zeta  $ 
  \or
    $\eta  $ 
  \or
    $\theta  $ 
  \or
    $\iota  $ 
  \or
    $\kappa  $ 
  \or
    $\lambda  $ 
  \or
    $\mu  $ 
  \or
    $\nu  $ 
  \or
    $\xi  $ 
  \or
    $\omicron  $ 
  \or
    $\pi  $ 
  \or
    $\rho  $ 
  \or
    $\sigma  $ 
  \or
    $\tau  $ 
  \or
    $\upsilon  $ 
  \or
    $\phi  $ 
  \or
    $\chi  $ 
  \or
    $\psi  $ 
  \or
    $\omega  $ 
  \else
    % 超出范围时的默认输出
    \text{None}
  \fi
}


\pagestyle{plain}

% \captionsetup[figure]
% {
% 	labelsep=none	
% }
% 控制图表结构












\theoremstyle{definition}
\newtheorem{definition}{Definition}[section]
\newtheorem{example}[definition]{Example}
\newtheorem{exercise}[definition]{Exercise}
\newtheorem{remark}[definition]{Remark}
\newtheorem{observation}[definition]{Observation}
\newtheorem{assumption}[definition]{Assumption}
\newtheorem{convention}[definition]{Convention}
\newtheorem{priniple}[definition]{Principle}
\newtheorem{notation}[definition]{Notation}
\newtheorem*{axiom}{Axiom}
\newtheorem{coa}[definition]{Theorem}
\newtheorem{srem}[definition]{$\star$ Remark}
\newtheorem{seg}[definition]{$\star$ Example}
\newtheorem{sexe}[definition]{$\star$ Exercise}
\newtheorem{sdf}[definition]{$\star$ Definition}
\newtheorem{question}{Question}
\theoremstyle{remark}
\newtheorem*{note}{Note}
\newtheorem*{claim}{Claim}


\newtheorem{problem}{\color{red}Problem}[section]
%\renewcommand*{\theprob}{{\color{red}\arabic{section}.\arabic{prob}}}
\newtheorem{sprob}[problem]{\color{red}$\star$ Problem}
%\renewcommand*{\thesprob}{{\color{red}\arabic{section}.\arabic{sprob}}}
% \newtheorem{ssprob}[prob]{$\star\star$ Problem}



\theoremstyle{plain}
\newtheorem{theorem}[definition]{Theorem}
\newtheorem{Conclusion}[definition]{Conclusion}
\newtheorem{thd}[definition]{Theorem-Definition}
\newtheorem{proposition}[definition]{Proposition}
\newtheorem{corollary}[definition]{Corollary}
\newtheorem{lemma}[definition]{Lemma}
\newtheorem{sthm}[definition]{$\star$ Theorem}
\newtheorem{slm}[definition]{$\star$ Lemma}

\newtheorem{spp}[definition]{$\star$ Proposition}
\newtheorem{scorollary}[definition]{$\star$ Corollary}
\newtheorem{fact}[definition]{Fact}

\newcommand{\cond}{\mathrm{cond}}
\newtheorem{Mthm}{Main Theorem}
\renewcommand{\theMthm}{\Alph{Mthm}} % "letter-numbered" theorems


%\substack   multiple lines under sum
%\underset{b}{a}   b is under a


% Remind: \overline{L_0}



\usepackage{calligra}
\DeclareMathOperator{\shom}{\mathscr{H}\text{\kern -3pt {\calligra\large om}}\,}
\DeclareMathOperator{\sext}{\mathscr{E}\text{\kern -3pt {\calligra\large xt}}\,}
\DeclareMathOperator{\Rel}{\mathscr{R}\text{\kern -3pt {\calligra\large el}~}\,}
\DeclareMathOperator{\sann}{\mathscr{A}\text{\kern -3pt {\calligra\large nn}}\,}
\DeclareMathOperator{\send}{\mathscr{E}\text{\kern -3pt {\calligra\large nd}}\,}
\DeclareMathOperator{\stor}{\mathscr{T}\text{\kern -3pt {\calligra\large or}}\,}
%write mathscr Hom (and so on) 

\usepackage{aurical}
\DeclareMathOperator{\VVir}{\text{\Fontlukas V}\text{\kern -0pt {\Fontlukas\large ir}}\,}

\newcommand{\vol}{\text{\Fontlukas V}}
\newcommand{\dvol}{d~\text{\Fontlukas V}}
% perfect Vol symbol

\usepackage{aurical}
\usepackage[T1]{fontenc}








\newcommand{\fk}{\mathfrak}
\newcommand{\mc}{\mathcal}
\newcommand{\wtd}{\widetilde}
\newcommand{\wht}{\widehat}
\newcommand{\wch}{\widecheck}
\newcommand{\ovl}{\overline}
\newcommand{\udl}{\underline}
\newcommand{\tr}{\mathrm{t}} %transpose
\newcommand{\Tr}{\mathrm{Tr}}
\newcommand{\End}{\mathrm{End}} %endomorphism
\newcommand{\idt}{\mathbf{1}}
\newcommand{\id}{\mathrm{id}}
\newcommand{\Hom}{\mathrm{Hom}}
\newcommand{\Conf}{\mathrm{Conf}}
\newcommand{\Res}{\mathrm{Res}}
\newcommand{\res}{\mathrm{res}}
\newcommand{\KZ}{\mathrm{KZ}}
\newcommand{\ev}{\mathrm{ev}}
\newcommand{\coev}{\mathrm{coev}}
\newcommand{\opp}{\mathrm{opp}}
\newcommand{\Rep}{\mathrm{Rep}}
\newcommand{\diag}{\mathrm{diag}}
\newcommand{\Dom}{\mathrm{Dom}}
\newcommand{\loc}{\mathrm{loc}}
\newcommand{\con}{\mathrm{c}}
\newcommand{\uni}{\mathrm{u}}
\newcommand{\ssp}{\mathrm{ss}}
\newcommand{\di}{\slashed d}
\newcommand{\Diffp}{\mathrm{Diff}^+}
\newcommand{\Diff}{\mathrm{Diff}}
\newcommand{\PSU}{\mathrm{PSU}(1,1)}
\newcommand{\Vir}{\mathrm{Vir}}
\newcommand{\Witt}{\mathscr W}
\newcommand{\Span}{\mathrm{Span}}
\newcommand{\pri}{\mathrm{p}}
\newcommand{\ER}{E^1(V)_{\mathbb R}}
\newcommand{\prth}[1]{( {#1})}
\newcommand{\bk}[1]{\langle {#1}\rangle}
\newcommand{\bigbk}[1]{\big\langle {#1}\big\rangle}
\newcommand{\Bigbk}[1]{\Big\langle {#1}\Big\rangle}
\newcommand{\biggbk}[1]{\bigg\langle {#1}\bigg\rangle}
\newcommand{\Biggbk}[1]{\Bigg\langle {#1}\Bigg\rangle}
\newcommand{\GA}{\mathscr G_{\mathcal A}}
\newcommand{\vs}{\varsigma}
\newcommand{\Vect}{\mathrm{Vec}}
\newcommand{\Vectc}{\mathrm{Vec}^{\mathbb C}}
\newcommand{\scr}{\mathscr}
\newcommand{\sjs}{\subset\joinrel\subset}
\newcommand{\Jtd}{\widetilde{\mathcal J}}
\newcommand{\gk}{\mathfrak g}
\newcommand{\hk}{\mathfrak h}
\newcommand{\xk}{\mathfrak x}
\newcommand{\yk}{\mathfrak y}
\newcommand{\zk}{\mathfrak z}
\newcommand{\pk}{\mathfrak p}
\newcommand{\hr}{\mathfrak h_{\mathbb R}}
\newcommand{\Ad}{\mathrm{Ad}}
\newcommand{\DHR}{\mathrm{DHR}_{I_0}}
\newcommand{\Repi}{\mathrm{Rep}_{\wtd I_0}}
\newcommand{\im}{\mathbf{i}}
\newcommand{\Co}{\complement}
%\newcommand{\Cu}{\mathcal C^{\mathrm u}}
\newcommand{\RepV}{\mathrm{Rep}^\uni(V)}
\newcommand{\RepA}{\mathrm{Rep}(\mathcal A)}
\newcommand{\RepN}{\mathrm{Rep}(\mathcal N)}
\newcommand{\RepfA}{\mathrm{Rep}^{\mathrm f}(\mathcal A)}
\newcommand{\RepAU}{\mathrm{Rep}^\uni(A_U)}
\newcommand{\RepU}{\mathrm{Rep}^\uni(U)}
\newcommand{\RepL}{\mathrm{Rep}^{\mathrm{L}}}
\newcommand{\HomL}{\mathrm{Hom}^{\mathrm{L}}}
\newcommand{\EndL}{\mathrm{End}^{\mathrm{L}}}
\newcommand{\Bim}{\mathrm{Bim}}
\newcommand{\BimA}{\mathrm{Bim}^\uni(A)}
%\newcommand{\shom}{\scr Hom}
\newcommand{\divi}{\mathrm{div}}
\newcommand{\sgm}{\varsigma}
\newcommand{\SX}{{S_{\fk X}}}
\newcommand{\DX}{D_{\fk X}}
\newcommand{\mbb}{\mathbb}
\newcommand{\mbf}{\mathbf}
\newcommand{\bsb}{\boldsymbol}
\newcommand{\blt}{\bullet}
\newcommand{\Vbb}{\mathbb V}
\newcommand{\Ubb}{\mathbb U}
\newcommand{\Xbb}{\mathbb X}
\newcommand{\Kbb}{\mathbb K}
\newcommand{\Abb}{\mathbb A}
\newcommand{\Wbb}{\mathbb W}
\newcommand{\Mbb}{\mathbb M}
\newcommand{\Gbb}{\mathbb G}
\newcommand{\Cbb}{\mathbb C}
\newcommand{\Nbb}{\mathbb N}
\newcommand{\Zbb}{\mathbb Z}
\newcommand{\Qbb}{\mathbb Q}
\newcommand{\Pbb}{\mathbb P}
\newcommand{\Rbb}{\mathbb R}
\newcommand{\Ebb}{\mathbb E}
\newcommand{\Dbb}{\mathbb D}
\newcommand{\Hbb}{\mathbb H}
\newcommand{\cbf}{\mathbf c}
\newcommand{\Rbf}{\mathbf R}
\newcommand{\wt}{\mathrm{wt}}
\newcommand{\Lie}{\mathrm{Lie}}
\newcommand{\btl}{\blacktriangleleft}
\newcommand{\btr}{\blacktriangleright}
\newcommand{\svir}{\mathcal V\!\mathit{ir}}
\newcommand{\Ker}{\mathrm{Ker}}
\newcommand{\Cok}{\mathrm{Coker}}
\newcommand{\Sbf}{\mathbf{S}}
\newcommand{\low}{\mathrm{low}}
\newcommand{\Sp}{\mathrm{Sp}}
\newcommand{\Rng}{\mathrm{Rng}}
\newcommand{\vN}{\mathrm{vN}}
\newcommand{\Ebf}{\mathbf E}
\newcommand{\Nbf}{\mathbf N}
\newcommand{\Stb}{\mathrm {Stb}}
\newcommand{\SXb}{{S_{\fk X_b}}}
\newcommand{\pr}{\mathrm {pr}}
\newcommand{\SXtd}{S_{\wtd{\fk X}}}
\newcommand{\univ}{\mathrm {univ}}
\newcommand{\vbf}{\mathbf v}
\newcommand{\ubf}{\mathbf u}
\newcommand{\wbf}{\mathbf w}
\newcommand{\CB}{\mathrm{CB}}
\newcommand{\Perm}{\mathrm{Perm}}
\newcommand{\Orb}{\mathrm{Orb}}
\newcommand{\Lss}{{L_{0,\mathrm{s}}}}
\newcommand{\Lni}{{L_{0,\mathrm{n}}}}
\newcommand{\UPSU}{\widetilde{\mathrm{PSU}}(1,1)}
\newcommand{\Sbb}{{\mathbb S}}
\newcommand{\Gc}{\mathscr G_c}
\newcommand{\Obj}{\mathrm{Obj}}
\newcommand{\bpr}{{}^\backprime}
\newcommand{\fin}{\mathrm{fin}}
\newcommand{\Ann}{\mathrm{Ann}}
\newcommand{\Real}{\mathrm{Re}}
\newcommand{\Imag}{\mathrm{Im}}
%\newcommand{\cl}{\mathrm{cl}}
\newcommand{\Ind}{\mathrm{Ind}}
\newcommand{\Supp}{\mathrm{Supp}}
\newcommand{\Specan}{\mathrm{Specan}}
\newcommand{\red}{\mathrm{red}}
\newcommand{\uph}{\upharpoonright}
\newcommand{\Mor}{\mathrm{Mor}}
\newcommand{\pre}{\mathrm{pre}}
\newcommand{\rank}{\mathrm{rank}}
\newcommand{\Jac}{\mathrm{Jac}}
\newcommand{\emb}{\mathrm{emb}}
\newcommand{\Sg}{\mathrm{Sg}}
\newcommand{\Nzd}{\mathrm{Nzd}}
\newcommand{\Owht}{\widehat{\scr O}}
\newcommand{\Ext}{\mathrm{Ext}}
\newcommand{\Tor}{\mathrm{Tor}}
\newcommand{\Com}{\mathrm{Com}}
\newcommand{\Mod}{\mathrm{Mod}}
\newcommand{\nk}{\mathfrak n}
\newcommand{\mk}{\mathfrak m}
\newcommand{\Ass}{\mathrm{Ass}}
\newcommand{\depth}{\mathrm{depth}}
\newcommand{\Coh}{\mathrm{Coh}}
\newcommand{\Gode}{\mathrm{Gode}}
\newcommand{\Fbb}{\mathbb F}
\newcommand{\sgn}{\mathrm{sgn}}
\newcommand{\Aut}{\mathrm{Aut}}
\newcommand{\Modf}{\mathrm{Mod}^{\mathrm f}}
\newcommand{\codim}{\mathrm{codim}}
\newcommand{\card}{\mathrm{card}}
\newcommand{\dps}{\displaystyle}
\newcommand{\Int}{\mathrm{Int}}
\newcommand{\Nbh}{\mathrm{Nbh}}
\newcommand{\Pnbh}{\mathrm{PNbh}}
\newcommand{\Cl}{\mathrm{Cl}}
\newcommand{\diam}{\mathrm{diam}}
\newcommand{\eps}{\varepsilon}
\newcommand{\Vol}{\mathrm{Vol}}
\newcommand{\LSC}{\mathrm{LSC}}
\newcommand{\USC}{\mathrm{USC}}
\newcommand{\Ess}{\mathrm{Rng}^{\mathrm{ess}}}
\newcommand{\Jbf}{\mathbf{J}}
\newcommand{\SL}{\mathrm{SL}}
\newcommand{\GL}{\mathrm{GL}}
\newcommand{\Lin}{\mathrm{Lin}}
\newcommand{\ALin}{\mathrm{ALin}}
\newcommand{\bwn}{\bigwedge\nolimits}
\newcommand{\nbf}{\mathbf n}
\newcommand{\dive}{\mathrm{div}}
\newcommand{\Alt}{\mathrm{Alt}}
\newcommand{\Sym}{\mathrm{Sym}}


\renewcommand{\epsilon}{\varepsilon}






\usepackage{tipa} % wierd symboles e.g. \textturnh
\newcommand{\tipar}{\text{\textrtailr}}
\newcommand{\tipaz}{\text{\textctyogh}}
\newcommand{\tipaomega}{\text{\textcloseomega}}
\newcommand{\tipae}{\text{\textrhookschwa}}
\newcommand{\tipaee}{\text{\textreve}}
\newcommand{\tipak}{\text{\texthtk}}
\newcommand{\mol}{\upmu}
\newcommand{\dmol}{d\upmu}




\usepackage{tipx}
\newcommand{\tipxgamma}{\text{\textfrtailgamma}}
\newcommand{\tipxcc}{\text{\textctstretchc}}
\newcommand{\tipxphi}{\text{\textqplig}}



% make some new environment such that we can put line_in math mode in the center.

\newenvironment{Then}
{Then%
\hspace{\stretch{1}}}
{\hspace{\stretch{1}}%
\phantom{Then}}









\numberwithin{equation}{section}
% count the eqation by section countation



\title{*******}
\author{{\sc Lin150117}
	\\
	{\small \sc Tsinghua University.}\\
	{\small linzj23@mails.tsinghua.edu.cn}
}

\DeclareMathOperator{\sign}{sign}
\DeclareMathOperator{\dom}{dom}
\DeclareMathOperator{\ran}{ran}
\DeclareMathOperator{\ord}{ord}
\DeclareMathOperator{\img}{Im}
\DeclareMathOperator{\dd}{d\!}
\newcommand{\ie}{ \textit{ i.e. } }
\newcommand{\st}{ \textit{ s.t. }}
\newcommand{\name}[1]{\textbf{#1}\index{#1}}
\newcommand{\subname}[2]{\textbf{#1}\index{#2!#1}}


\renewcommand{\|}{\Vert}



\begin{document}
\sloppy
\pagenumbering{arabic}
\maketitle
\tableofcontents
\newpage

\begin{theorem}
    If  $ B\in\mathbb{R}^{n\times n} $ satisfying  $ ||B||<1 $, then  $ I+B  $ invertible and 
    \[||(I+B)^{-1}||_2 \leq \frac{1}{1-||B||}\]   
\end{theorem}
Cholesky transformation

Doolottle Decomposition

Condition number

\[\mathcal{K}_2(A)^2=\cond(A^TA)\text{ if }A\text{ full of column rank}\]
\[\cond(A) \geq \frac{|\lambda_1|}{|\lambda_n|}\text{ equality holds if  $ A $ is symmetric matrix}\]
\[||A||_2^2=\rho(A^TA)=||A^TA||_2\]
\begin{theorem}
    If  $ \det A\not=0 $, then 
    \[\min_{|A+\delta A|=0}\frac{||\delta A||_2}{||A||_2}=\frac{1}{\cond(A)_2}\] 
\end{theorem}
Moore-Penrose pseudoinverse

\begin{theorem}
    For the least squrare equation of  $ Ax=b $, 
    \[\frac{||\delta x||_2}{||x||_2} \leq \mathcal{K}_x(A)\cdot\frac{||A\delta x||_2}{||Ax||_2}\] 
\end{theorem}
Hauseholder transiformation

Givens transoformation

\[R_k(B)=-\ln ||B^k||^{\frac{1}{k}}\]
\[R(B)=-\ln \rho(B)\]
Jacobian iteration

Gauss-Seidel iteration

\begin{theorem}
    
\end{theorem}
\begin{theorem}
    By Steepest Descent Algorithm, 
    \[||x^k-x^*||_A \leq \left(\frac{\lambda_n-\lambda_1}{\lambda_n+\lambda_1}\right)^k||x^0-x^*||_A\]
\end{theorem}
\begin{theorem}
    In conjugated gradient method,
    \[P^{(k)}=r^{(k)}+\beta_{k-1}P^{(k-1)}\]
    with 
    \[\beta_{k-1}=-\frac{(r^{(k)},AP^{(k-1)})}{(P^{(k-1)},AP^{(k-1)})},r^{(k)}=b-Ax^{(k)}\]
    And the iteration
    \[x^{(k+1)}=x^{(k)}+\alpha_kP^{(k)}\]
    where 
    \[\alpha_k=\arg\min_{\alpha\in\mathbb R}\varphi(x^{(k)}+\alpha P^{(k)})=\frac{(r^{(k)},P^{(k)})}{(P^{(k)},AP^{(k)})}\]
    This iteration satsifies
    \[x^{(k)}=\arg\min_{x-x^{(0)}\in\Span\{P^{(0)},\cdots,P^{(k-1)}\}}\varphi(x)\]
\end{theorem}
\begin{theorem}
     $ (r^{(i)},r^{(j)})=0,i\not=j $
     
      $ (AP^{(i)},P^{(j)})=0 ,i\not=j$ 

       $ (r^{(j)},P^{(i)})=0,i<j $ 

        $ \Span\{r^{0},\cdots,r^{(k)}\}=\Span\{P^{(0)},\cdots,P^{(k)}\}=\Span\{r^{(0)},Ar^{(0)},\cdots,A^kr^{(0)}\} $ 
\end{theorem}
\begin{theorem}[Arnoldi Elimination]
    \begin{align*}
        Aq_k&=Q_kh_k+\beta_kq_{k+1}\\
        h_k&=Q_k^TAq_k\\
        \beta_k&=\|Aq_k-Q_kh_k\|\\
        q_{k+1}&=\frac{Aq_k-Q_kh_k}{\beta_k}
    \end{align*}

    Then 
    \[AQ_k=Q_kH_k+h_{k+1,k}q_{k+1}e_k^T\]
    and 
    \[AQ_n=Q_nH_n,\,Q_n^TAQ_n=H_n\]
\end{theorem}
\begin{theorem}[Lanczos Elimination]
    For  $ A  $ symmetric, 
    \[AQ_k=Q_kT_k+\beta_kq_{k+1}e_k^T\]
    where  $ T_k $ symmetric three triangular matrix. 
\end{theorem}
\begin{theorem}[Non-symmetric Lanczos Elimination]
     $ \omega_iv_j=\begin{cases}
        0,&i\neq j\\
        1,&i=j
     \end{cases} $.
    \begin{align*}
        \beta_jv_{j+1}&=Av_j-\alpha_jv_j-\gamma_{j-1}v_{j-1}\\
        \gamma_{j}\omega_{j+!}&=A^T\omega_j-\alpha_j\omega_j-\beta_{j-1}\omega_{j-1}
    \end{align*} 
    Take inner product we obain  $ \alpha_j=\omega_j^TAv_j $.
    And  $ \beta_j=\sqrt{|\eta_j|} $ for  $ \eta_j=\tilde{v}_{j+1}^T\tilde{\omega}_{j+1} $.   

    It can be derived from 
    \begin{align*}
        AV_k&=V_kT_k+\beta_kv_{k+1}e_k^T\\
        A^TW_k&=W_kT_k^T+\gamma_k\omega_{k+1}e_k^T\\
        V_k^TW_k=T_k,v_{k+1}^TW_k=0&,\omega_{k+1}^TV_k=0,\omega_{k+1}^Tv_{k+1}=1\\
        \mathcal{K}_k(A,v_1)&=\Span\{v_1,\cdots,v_k\}\\
        \mathcal{K}_k(A^T,\omega_1)&=\Span\{\omega_1,\cdots,\omega_k\}
    \end{align*}
\end{theorem}
\begin{theorem}[Conjugate Gradient Method]
    It follows from the fact that  $ (b-Ax_k)\perp \mathcal{K}_k(A,b) $ if and only if  $ x_k=\arg \min\{\|x-x_*\|_A:x\in \mathcal{K}_k(A,b)\} $.
    
    Take  $ q_1=\dps\frac{b}{\|b\|_2} $. Then Lanzos method implies 
    \[AQ_k=Q_kT_k+\beta_kq_{k+1}e_k^T\] 
    For  $ x_k=Q_ky_k $, it is equivalent to calculate
    \[T_ky_k=\beta_1\]
     $ T_k $ symmetric positive definite, then 
     \[T_k=L_kD_kL_k^T\]
     where 
     \[L_k=\begin{pmatrix}
        1&0&\cdots&0\\
        \gamma_1&1&\cdots&0\\
        \vdots&\vdots&\ddots&\vdots\\
        0&\cdots&\gamma_{k-1}&1
     \end{pmatrix},\,D_k=\diag\{\delta_1,\cdots,\delta_k\}\] 
     Then  $ x_{k+1}=\tilde{P}_{k+1}z_{k+1}=x_k+\zeta_{k+1}\tilde{p}_{k+1} $ where  $ \tilde{P}_k=Q_kL_k^{-T} $.   
\end{theorem}
\begin{theorem}[MINRES method]
     $ A\in \Rbb^{n\times n} $ symmetric, by Lanzos elimination, 
     \[AQ_k=Q_{k+1}\hat{T}_k\]
     where  
    \[T_k=\begin{pmatrix}
        \alpha_1&\beta_1\\
        \beta_1&\alpha_2&\ddots\\
        &\ddots&\ddots&\beta_{k-1}\\
        &&\beta_{k-1}&\alpha_k\\
        &&&\beta_k
    \end{pmatrix}\] 
     For Givens transformation  $ G_kG_{k-1}\cdots G_1\hat{T}_k=\begin{pmatrix}
        R_k\\
        0
     \end{pmatrix} $.
     
     Then suffices to find  $ y $ \st  
     \[y=\arg\min\|R_ky-t_k\|\]
\end{theorem}
\begin{theorem}
    If  $ \varphi\in C^p $,  $ \varphi(x^*)=\varphi'(x^*)=\varphi''(x^*)=\cdots=\varphi^{(p-1)}(x^*)=0 $, then 
    \[\dps\lim_{k\to\infty}\frac{e_{k+1}}{e_k^p}=\frac{\varphi^{(p)}(x^*)}{p!}\]  
\end{theorem}
\begin{theorem}[Steflensen iteration method]
     $ \psi(x)=x-\dps\frac{(\varphi(x)-x)^2}{\varphi(\varphi(x))-2\varphi(x)+x} $. 
     Then fixed points of $ \psi(x) $ are that of  $ \varphi $. If  $ \varphi $ converges of order 1, then  $ \psi $ converges for order 2. If $ \varphi $ converges of order  $ p>1 $, then  $ \psi $ converges of order  $ 2p-1 $. All under the condition that  $ \varphi\in C^{p+1} $.       
\end{theorem}
\begin{theorem}
     $ f(x^*)=0,f'(x^*)\neq 0 $.  $ f\in C^2 $ locally. Then Newton's iteration  $ \varphi(x)=x-\dps\frac{f(x)}{f'(x)} $
     converges of order 2 locally. And 
     \[\lim_{k\to\infty}\frac{x_{k+1}-x^*}{(x_k-x^*)^2}=\frac{f''(x^*)}{2f'(x^*)}\]   

     For  $ f(x)=(x-x^*)^mg(x) $, 
     \[\varphi(x)=x-\frac{f(x)}{f'(x)}=x-\frac{(x-x^*)g(x)}{mg(x)+(x-x^*)g'(x)}\]
     \[\varphi'(x)=1-\frac{1}{m}\] 
     So Newton's method converges linearly.

     For  $ \varphi(x)=x-\dps\frac{mf(x)}{f'(x)} $,  $ \varphi'(x^*)=0 $. It converges of at least two.
     
     Or  $ \mu(x)=\dps\frac{f(x)}{f'(x)} $. It ha simple root  $ x^* $. Use Newton's method.  
\end{theorem}
\begin{theorem}[GeXian]
     $ f'(x_k)\simeq \dps\frac{f(x_k)-f(x_{k-1})}{x_k-x_{k-1}} $. 
    
     If on interval  $ \triangle=[x^*-\delta,x^*+\delta] $  $ f'(x)\neq 0 $ and  $ f\in C^2(\triangle) $.
     
      $ M\delta<1 $ where  $ M=\dps\frac{\max_{x\in\triangle}|f''(x)|}{2\min_{x\in \triangle}|f'(x)|} $.
      
      Then if  $ x_0,x_1\in \triangle $, it converges of order  $ \dps\frac{1}{2}(1+\sqrt{5}) $.  
\end{theorem}
\begin{theorem}
    Iteration converges locally if  $ \rho(\Psi'(x^*))=\sigma<1 $. 
\end{theorem}
\begin{theorem}
  Newton's method  $ x^{k+1}=x^k-(F'(x^k))^{-1}F(x^k) $ converges of order  $ >1 $  if  $ F'(x) $ invertible and continuous. If  $ \|F'(x)-F'(x^*)\| \leq \gamma\|x-x^*\| $,  $ \forall x\in S $, then  $ \{x^k\} $ converges of at least order 2.   
\end{theorem}
\begin{theorem}
   $ x^{k+1}=x^k-A_k^{-1}F(x^k) $ where 
   \[\triangle A_k=\arg \min_{A\in Q}\|A\|_F\]
   under the Frobenius norm and  $ Q=\{A\in \Rbb^{n\times n}:L(A_k+A)p^k=q^k,p^k=x^{k+1}-x^k,q^k=F(x^{k+1})-F(x^k)\} $.
\end{theorem}
\begin{lemma}
   $ A\in \Cbb^{n\times n},B\in \Cbb^{p\times p} $,  $ X\in \Cbb^{n\times p} $ satisfies 
   \[AX=XB,\,\rank(X)=p\]
   Then there exists unitary matrix  $ Q\in \Cbb^{n\times n} $ 
   \[Q^HAQ=T=\begin{pmatrix}
   T_{11} & T_{12} \\
   0&T_{22}
   \end{pmatrix}\]
   and  $ \sigma(T_{11})=\sigma(A)\cap \sigma(B) $.    
\end{lemma}
\begin{theorem}[Schur decomposition]
     $ A\in \Cbb^{n\times n} $, then there exists unitary matrix  $ Q\in Cbb^{n\times n} $ \st   $ Q^HAQ=U $.  
\end{theorem}
\begin{theorem}[Real Schur decomposition]
   $ A\in \Rbb^{n\times n} $. Then there exists  $ Q\in \Rbb^{n\times n} $ \st 
   \[Q^TAQ=\begin{pmatrix}
    R_{11}&R_{12}&\cdots&R_{1m}\\
    &R_{22}&\cdots&R_{2m}\\
    &&\ddots&\vdots\\
    & & &R_{mm}
   \end{pmatrix}\]  
\end{theorem}
\begin{theorem}
   $ \mu  $ is an eigenvalue of  $ A+E\in \Cbb^{n\times n} $. If  $ X^{-1}AX $ diagonal, then 
   \[\min_{\lambda\in \sigma(A)}|\lambda-\mu| \leq \|X^{-1}\|_p\cdot\|X\|_p\cdot\|E\|_p\] 
\end{theorem}
Condition number w.r.t.  $\lambda $ of  $ A $: $ \dps\frac{1}{|y^Hx|} $  
\begin{theorem}
  If  $ A,E $ symmetric in  $ \Rbb^{n\times n} $. Eigenvalues of  $ A,E,A+E $ are 
  \begin{align*}
    \lambda_1 \geq \lambda_2 \geq \cdots \geq \lambda_n\\
    \nu_1 \geq \nu_2 \geq \cdots \geq \nu_n\\
    \mu_1 \geq \mu_2 \geq \cdots \geq \mu_n
  \end{align*}   
  Then  $  \lambda_i+\nu_n\leq \mu_i \leq \lambda_i+\nu_1 $.
\end{theorem} 
Proof use Rayeligh quotient.
\printindex
\newpage
\listoftheorems[ignoreall, show={theorem,proposition}]
\end{document}