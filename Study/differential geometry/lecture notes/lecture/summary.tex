 % !TeX spellcheck = en_US
% !TEX program = pdflatex
\documentclass[12pt,b5paper,notitlepage]{article}
\usepackage[b5paper, margin={0.5in,0.65in}]{geometry}
%\usepackage{fullpage}
\usepackage{amsmath,amscd,amssymb,amsthm,mathrsfs,amsfonts,layout,indentfirst,graphicx,caption,mathabx, stmaryrd,appendix,calc,imakeidx,upgreek,amsbsy,thmtools} % mathabx for \wtidecheck
%\usepackage{ulem} %wave underline
\usepackage[dvipsnames]{xcolor}
\usepackage{palatino}  %template

\usepackage{slashed} % Dirac operator
\usepackage{mathrsfs} % Enable using \mathscr
%\usepackage{eufrak}  another template/font
\usepackage{extarrows} % long equal sign, \xlongequal{blablabla}
\usepackage{enumitem} % enumerate label change e.g. [label=(\alph*)]  shows (a) (b) 


%%%%%%%%%%%%%%%%%%%%%%%%%%%%%%

%\usepackage{fontspec}
%\setmainfont{Palatino Linotype}
%\usepackage{emoji}


% emoji, use lualatex  remove \usepackage{palatino}

%%%%%%%%%%%%%


\usepackage{CJK}   % Chinese package





\usepackage{csquotes} % \begin{displayquote}   \begin{displaycquote}  for quotation
\usepackage{epigraph}   %\epigraph{}{}  for quotation
%\pmb  mandatory math bold 

\usepackage{fancyhdr} % date in footer

%\usepackage{soul}  %\ul underline break line automatically

\usepackage{ulem}  % \uline  underline break line   also    \uwave

\usepackage{relsize} % use \mathlarger \larger \text{\larger[2]$...$} to enlarge the size of math symbols

\usepackage{verbatim}  % comment environment


\usepackage{halloweenmath} % Interesting halloween math symbols

%%%%%%%%%%%%%%%%%%%%%%%%%%%%%%
\usepackage{tcolorbox}
\tcbuselibrary{theorems}
% box around equations   \tcboxmath
%%%%%%%%%%%%%%%%%%%%%%%%%%%%%%%%%%





%%%%%%%%%%%%%%%%%%%%%%%%%%%%%
% circled colon and thick colon \hcolondel and \colondel

\usepackage{pdfrender}

\newcommand*{\hollowcolon}{%
	\textpdfrender{
		TextRenderingMode=Stroke,
		LineWidth=.1bp,
	}{:}%
}

\newcommand{\hcolondel}[1]{%
	\mathopen{\hollowcolon}#1\mathclose{\hollowcolon}%
}
\newcommand{\colondel}[1]{%
	\mathopen{:}#1\mathclose{:}%
}

%%%%%%%%%%%%%%%%%%%%%%%%%%%%%%%%


\usepackage{setspace}  
\setstretch{1.6}



\usepackage{tikz}
\usetikzlibrary{fadings}
\usetikzlibrary{patterns}
\usetikzlibrary{shadows.blur}
\usetikzlibrary{shapes}

\usepackage{tikz-cd}
\usepackage[nottoc]{tocbibind}   % Add  reference to ToC


\makeindex


% The following set up the line spaces between items in \thebibliography
\usepackage{lipsum}  
\let\OLDthebibliography\thebibliography
\renewcommand\thebibliography[1]{
	\OLDthebibliography{#1}
	\setlength{\parskip}{0pt}
	\setlength{\itemsep}{2pt} 
}


%\hyperref{page.10}{...}

\allowdisplaybreaks  %allow aligns to break between pages
\usepackage{latexsym}
\usepackage{chngcntr}
\usepackage[colorlinks,linkcolor=blue,anchorcolor=blue, linktocpage,
%pagebackref
]{hyperref}
\hypersetup{ urlcolor=cyan,
	citecolor=[rgb]{0,0.5,0}}


\setcounter{tocdepth}{2}	 %hide subsections in the content


\counterwithin{figure}{section}

\counterwithin*{footnote}{section}   % Footnote numbering is recounted from the beginning of each subsection



\pagestyle{plain}

% \captionsetup[figure]
% {
% 	labelsep=none	
% }
% 控制图表结构












\theoremstyle{definition}
\newtheorem{definition}{Definition}[section]
\newtheorem{example}[definition]{Example}
\newtheorem{exercise}[definition]{Exercise}
\newtheorem{remark}[definition]{Remark}
\newtheorem{observation}[definition]{Observation}
\newtheorem{assumption}[definition]{Assumption}
\newtheorem{convention}[definition]{Convention}
\newtheorem{priniple}[definition]{Principle}
\newtheorem{notation}[definition]{Notation}
\newtheorem*{axiom}{Axiom}
\newtheorem{coa}[definition]{Theorem}
\newtheorem{srem}[definition]{$\star$ Remark}
\newtheorem{seg}[definition]{$\star$ Example}
\newtheorem{sexe}[definition]{$\star$ Exercise}
\newtheorem{sdf}[definition]{$\star$ Definition}
\newtheorem{question}{Question}
\theoremstyle{remark}
\newtheorem*{note}{Note}
\newtheorem*{claim}{Claim}


\newtheorem{problem}{\color{red}Problem}[section]
%\renewcommand*{\theprob}{{\color{red}\arabic{section}.\arabic{prob}}}
\newtheorem{sprob}[problem]{\color{red}$\star$ Problem}
%\renewcommand*{\thesprob}{{\color{red}\arabic{section}.\arabic{sprob}}}
% \newtheorem{ssprob}[prob]{$\star\star$ Problem}



\theoremstyle{plain}
\newtheorem{theorem}[definition]{Theorem}
\newtheorem{Conclusion}[definition]{Conclusion}
\newtheorem{thd}[definition]{Theorem-Definition}
\newtheorem{proposition}[definition]{Proposition}
\newtheorem{corollary}[definition]{Corollary}
\newtheorem{lemma}[definition]{Lemma}
\newtheorem{sthm}[definition]{$\star$ Theorem}
\newtheorem{slm}[definition]{$\star$ Lemma}

\newtheorem{spp}[definition]{$\star$ Proposition}
\newtheorem{scorollary}[definition]{$\star$ Corollary}
\newtheorem{fact}[definition]{Fact}

\newtheorem{cond}{Condition}
\newtheorem{Mthm}{Main Theorem}
\renewcommand{\thecond}{\Alph{cond}} % "letter-numbered" theorems
\renewcommand{\theMthm}{\Alph{Mthm}} % "letter-numbered" theorems


%\substack   multiple lines under sum
%\underset{b}{a}   b is under a


% Remind: \overline{L_0}



\usepackage{calligra}
\DeclareMathOperator{\shom}{\mathscr{H}\text{\kern -3pt {\calligra\large om}}\,}
\DeclareMathOperator{\sext}{\mathscr{E}\text{\kern -3pt {\calligra\large xt}}\,}
\DeclareMathOperator{\Rel}{\mathscr{R}\text{\kern -3pt {\calligra\large el}~}\,}
\DeclareMathOperator{\sann}{\mathscr{A}\text{\kern -3pt {\calligra\large nn}}\,}
\DeclareMathOperator{\send}{\mathscr{E}\text{\kern -3pt {\calligra\large nd}}\,}
\DeclareMathOperator{\stor}{\mathscr{T}\text{\kern -3pt {\calligra\large or}}\,}
%write mathscr Hom (and so on) 

\usepackage{aurical}
\DeclareMathOperator{\VVir}{\text{\Fontlukas V}\text{\kern -0pt {\Fontlukas\large ir}}\,}

\newcommand{\vol}{\text{\Fontlukas V}}
\newcommand{\dvol}{d~\text{\Fontlukas V}}
% perfect Vol symbol

\usepackage{aurical}
\usepackage[T1]{fontenc}








\newcommand{\fk}{\mathfrak}
\newcommand{\mc}{\mathcal}
\newcommand{\wtd}{\widetilde}
\newcommand{\wht}{\widehat}
\newcommand{\wch}{\widecheck}
\newcommand{\ovl}{\overline}
\newcommand{\udl}{\underline}
\newcommand{\tr}{\mathrm{t}} %transpose
\newcommand{\Tr}{\mathrm{Tr}}
\newcommand{\End}{\mathrm{End}} %endomorphism
\newcommand{\idt}{\mathbf{1}}
\newcommand{\id}{\mathrm{id}}
\newcommand{\Hom}{\mathrm{Hom}}
\newcommand{\Conf}{\mathrm{Conf}}
\newcommand{\Res}{\mathrm{Res}}
\newcommand{\res}{\mathrm{res}}
\newcommand{\KZ}{\mathrm{KZ}}
\newcommand{\ev}{\mathrm{ev}}
\newcommand{\coev}{\mathrm{coev}}
\newcommand{\opp}{\mathrm{opp}}
\newcommand{\Rep}{\mathrm{Rep}}
\newcommand{\diag}{\mathrm{diag}}
\newcommand{\Dom}{\mathrm{Dom}}
\newcommand{\loc}{\mathrm{loc}}
\newcommand{\con}{\mathrm{c}}
\newcommand{\uni}{\mathrm{u}}
\newcommand{\ssp}{\mathrm{ss}}
\newcommand{\di}{\slashed d}
\newcommand{\Diffp}{\mathrm{Diff}^+}
\newcommand{\Diff}{\mathrm{Diff}}
\newcommand{\PSU}{\mathrm{PSU}(1,1)}
\newcommand{\Vir}{\mathrm{Vir}}
\newcommand{\Witt}{\mathscr W}
\newcommand{\Span}{\mathrm{Span}}
\newcommand{\pri}{\mathrm{p}}
\newcommand{\ER}{E^1(V)_{\mathbb R}}
\newcommand{\prth}[1]{( {#1})}
\newcommand{\bk}[1]{\langle {#1}\rangle}
\newcommand{\bigbk}[1]{\big\langle {#1}\big\rangle}
\newcommand{\Bigbk}[1]{\Big\langle {#1}\Big\rangle}
\newcommand{\biggbk}[1]{\bigg\langle {#1}\bigg\rangle}
\newcommand{\Biggbk}[1]{\Bigg\langle {#1}\Bigg\rangle}
\newcommand{\GA}{\mathscr G_{\mathcal A}}
\newcommand{\vs}{\varsigma}
\newcommand{\Vect}{\mathrm{Vec}}
\newcommand{\Vectc}{\mathrm{Vec}^{\mathbb C}}
\newcommand{\scr}{\mathscr}
\newcommand{\sjs}{\subset\joinrel\subset}
\newcommand{\Jtd}{\widetilde{\mathcal J}}
\newcommand{\gk}{\mathfrak g}
\newcommand{\hk}{\mathfrak h}
\newcommand{\xk}{\mathfrak x}
\newcommand{\yk}{\mathfrak y}
\newcommand{\zk}{\mathfrak z}
\newcommand{\pk}{\mathfrak p}
\newcommand{\hr}{\mathfrak h_{\mathbb R}}
\newcommand{\Ad}{\mathrm{Ad}}
\newcommand{\DHR}{\mathrm{DHR}_{I_0}}
\newcommand{\Repi}{\mathrm{Rep}_{\wtd I_0}}
\newcommand{\im}{\mathbf{i}}
\newcommand{\Co}{\complement}
%\newcommand{\Cu}{\mathcal C^{\mathrm u}}
\newcommand{\RepV}{\mathrm{Rep}^\uni(V)}
\newcommand{\RepA}{\mathrm{Rep}(\mathcal A)}
\newcommand{\RepN}{\mathrm{Rep}(\mathcal N)}
\newcommand{\RepfA}{\mathrm{Rep}^{\mathrm f}(\mathcal A)}
\newcommand{\RepAU}{\mathrm{Rep}^\uni(A_U)}
\newcommand{\RepU}{\mathrm{Rep}^\uni(U)}
\newcommand{\RepL}{\mathrm{Rep}^{\mathrm{L}}}
\newcommand{\HomL}{\mathrm{Hom}^{\mathrm{L}}}
\newcommand{\EndL}{\mathrm{End}^{\mathrm{L}}}
\newcommand{\Bim}{\mathrm{Bim}}
\newcommand{\BimA}{\mathrm{Bim}^\uni(A)}
%\newcommand{\shom}{\scr Hom}
\newcommand{\divi}{\mathrm{div}}
\newcommand{\sgm}{\varsigma}
\newcommand{\SX}{{S_{\fk X}}}
\newcommand{\DX}{D_{\fk X}}
\newcommand{\mbb}{\mathbb}
\newcommand{\mbf}{\mathbf}
\newcommand{\bsb}{\boldsymbol}
\newcommand{\blt}{\bullet}
\newcommand{\Vbb}{\mathbb V}
\newcommand{\Ubb}{\mathbb U}
\newcommand{\Xbb}{\mathbb X}
\newcommand{\Kbb}{\mathbb K}
\newcommand{\Abb}{\mathbb A}
\newcommand{\Wbb}{\mathbb W}
\newcommand{\Mbb}{\mathbb M}
\newcommand{\Gbb}{\mathbb G}
\newcommand{\Cbb}{\mathbb C}
\newcommand{\Nbb}{\mathbb N}
\newcommand{\Zbb}{\mathbb Z}
\newcommand{\Qbb}{\mathbb Q}
\newcommand{\Pbb}{\mathbb P}
\newcommand{\Rbb}{\mathbb R}
\newcommand{\Ebb}{\mathbb E}
\newcommand{\Dbb}{\mathbb D}
\newcommand{\Hbb}{\mathbb H}
\newcommand{\cbf}{\mathbf c}
\newcommand{\Rbf}{\mathbf R}
\newcommand{\wt}{\mathrm{wt}}
\newcommand{\Lie}{\mathrm{Lie}}
\newcommand{\btl}{\blacktriangleleft}
\newcommand{\btr}{\blacktriangleright}
\newcommand{\svir}{\mathcal V\!\mathit{ir}}
\newcommand{\Ker}{\mathrm{Ker}}
\newcommand{\Cok}{\mathrm{Coker}}
\newcommand{\Sbf}{\mathbf{S}}
\newcommand{\low}{\mathrm{low}}
\newcommand{\Sp}{\mathrm{Sp}}
\newcommand{\Rng}{\mathrm{Rng}}
\newcommand{\vN}{\mathrm{vN}}
\newcommand{\Ebf}{\mathbf E}
\newcommand{\Nbf}{\mathbf N}
\newcommand{\Stb}{\mathrm {Stb}}
\newcommand{\SXb}{{S_{\fk X_b}}}
\newcommand{\pr}{\mathrm {pr}}
\newcommand{\SXtd}{S_{\wtd{\fk X}}}
\newcommand{\univ}{\mathrm {univ}}
\newcommand{\vbf}{\mathbf v}
\newcommand{\ubf}{\mathbf u}
\newcommand{\wbf}{\mathbf w}
\newcommand{\CB}{\mathrm{CB}}
\newcommand{\Perm}{\mathrm{Perm}}
\newcommand{\Orb}{\mathrm{Orb}}
\newcommand{\Lss}{{L_{0,\mathrm{s}}}}
\newcommand{\Lni}{{L_{0,\mathrm{n}}}}
\newcommand{\UPSU}{\widetilde{\mathrm{PSU}}(1,1)}
\newcommand{\Sbb}{{\mathbb S}}
\newcommand{\Gc}{\mathscr G_c}
\newcommand{\Obj}{\mathrm{Obj}}
\newcommand{\bpr}{{}^\backprime}
\newcommand{\fin}{\mathrm{fin}}
\newcommand{\Ann}{\mathrm{Ann}}
\newcommand{\Real}{\mathrm{Re}}
\newcommand{\Imag}{\mathrm{Im}}
%\newcommand{\cl}{\mathrm{cl}}
\newcommand{\Ind}{\mathrm{Ind}}
\newcommand{\Supp}{\mathrm{Supp}}
\newcommand{\Specan}{\mathrm{Specan}}
\newcommand{\red}{\mathrm{red}}
\newcommand{\uph}{\upharpoonright}
\newcommand{\Mor}{\mathrm{Mor}}
\newcommand{\pre}{\mathrm{pre}}
\newcommand{\rank}{\mathrm{rank}}
\newcommand{\Jac}{\mathrm{Jac}}
\newcommand{\emb}{\mathrm{emb}}
\newcommand{\Sg}{\mathrm{Sg}}
\newcommand{\Nzd}{\mathrm{Nzd}}
\newcommand{\Owht}{\widehat{\scr O}}
\newcommand{\Ext}{\mathrm{Ext}}
\newcommand{\Tor}{\mathrm{Tor}}
\newcommand{\Com}{\mathrm{Com}}
\newcommand{\Mod}{\mathrm{Mod}}
\newcommand{\nk}{\mathfrak n}
\newcommand{\mk}{\mathfrak m}
\newcommand{\Ass}{\mathrm{Ass}}
\newcommand{\depth}{\mathrm{depth}}
\newcommand{\Coh}{\mathrm{Coh}}
\newcommand{\Gode}{\mathrm{Gode}}
\newcommand{\Fbb}{\mathbb F}
\newcommand{\sgn}{\mathrm{sgn}}
\newcommand{\Aut}{\mathrm{Aut}}
\newcommand{\Modf}{\mathrm{Mod}^{\mathrm f}}
\newcommand{\codim}{\mathrm{codim}}
\newcommand{\card}{\mathrm{card}}
\newcommand{\dps}{\displaystyle}
\newcommand{\Int}{\mathrm{Int}}
\newcommand{\Nbh}{\mathrm{Nbh}}
\newcommand{\Pnbh}{\mathrm{PNbh}}
\newcommand{\Cl}{\mathrm{Cl}}
\newcommand{\diam}{\mathrm{diam}}
\newcommand{\eps}{\varepsilon}
\newcommand{\Vol}{\mathrm{Vol}}
\newcommand{\LSC}{\mathrm{LSC}}
\newcommand{\USC}{\mathrm{USC}}
\newcommand{\Ess}{\mathrm{Rng}^{\mathrm{ess}}}
\newcommand{\Jbf}{\mathbf{J}}
\newcommand{\SL}{\mathrm{SL}}
\newcommand{\GL}{\mathrm{GL}}
\newcommand{\Lin}{\mathrm{Lin}}
\newcommand{\ALin}{\mathrm{ALin}}
\newcommand{\bwn}{\bigwedge\nolimits}
\newcommand{\nbf}{\mathbf n}
\newcommand{\dive}{\mathrm{div}}









\usepackage{tipa} % wierd symboles e.g. \textturnh
\newcommand{\tipar}{\text{\textrtailr}}
\newcommand{\tipaz}{\text{\textctyogh}}
\newcommand{\tipaomega}{\text{\textcloseomega}}
\newcommand{\tipae}{\text{\textrhookschwa}}
\newcommand{\tipaee}{\text{\textreve}}
\newcommand{\tipak}{\text{\texthtk}}
\newcommand{\mol}{\upmu}
\newcommand{\dmol}{d\upmu}




\usepackage{tipx}
\newcommand{\tipxgamma}{\text{\textfrtailgamma}}
\newcommand{\tipxcc}{\text{\textctstretchc}}
\newcommand{\tipxphi}{\text{\textqplig}}















\numberwithin{equation}{section}
% count the eqation by section countation



\title{Differential Geometry}
\author{{\sc Lin150117}
	\\
	{\small \sc Tsinghua University.}\\
	{\small linzj23@mails.tsinghua.edu.cn}
}

\DeclareMathOperator{\sign}{sign}
\DeclareMathOperator{\dom}{dom}
\DeclareMathOperator{\ran}{ran}
\DeclareMathOperator{\ord}{ord}
\DeclareMathOperator{\img}{Im}
\DeclareMathOperator{\dd}{d\!}
\newcommand{\ie}{ \textit{ i.e. } }
\newcommand{\st}{ \textit{ s.t. }}
\newcommand{\name}[1]{\textbf{#1}\index{#1}}
\newcommand{\subname}[2]{\textbf{#1}\index{#2!#1}}
%\makeindex[columns=2,title=Index, options=-s example_style.ist]

\begin{document}
\sloppy
\pagenumbering{arabic}
\maketitle
\tableofcontents
\newpage
\section{Smooth Manifold}
\begin{definition}[Topological manifold]
    A space  $ M  $ is called a topological manifold if 
    \begin{enumerate}
        \item locally Euclidean
        \item Hausdorff
        \item second countable
    \end{enumerate}
\end{definition}
\begin{definition}[Smooth Manifold]
     A smooth structure is given by an equivalence class of smooth atlas  $ \{(U_\alpha,\varphi_\alpha)\} $ \st  $ \varphi_{\alpha\beta}:\varphi_\alpha(U_\alpha\cap U_\beta)\rightarrow \varphi_\beta(U_\alpha\cap U_\beta) $  is smooth  $ \forall \alpha,\beta $.  $ M=\cup U_\alpha $.\\
     A \name{smooth manifold} is a topological manifold with a smooth structure.\\
     Define when a continuous map  $ f:M_1\rightarrow M_2 $ is smooth if  $ \forall (U_1,\varphi_1)\in\mathcal{A}_1,(U_2,\varphi_2)\in\mathcal{A}_2 $, we have  $ \varphi_2\circ f\circ \varphi_1^{-1}:\varphi_1(U_1\cap U_2)\rightarrow \varphi_2(U_1\cap U_2) $ is smooth. 
\end{definition}
\begin{definition}
    Given  $ (M_1,\mathcal{A}_1),(M_2,\mathcal{A}_2) $. A homeomorphism  $ f:M_1\rightarrow M_2 $ is called a diffeomorphism if   $ f $, $ f^{-1} $  is smooth. \\
    In this case we say  $ (M_1,\mathcal{A}_1),(M_2,\mathcal{A}_2) $ are diffeomorphism. 
\end{definition}
\begin{theorem}[Kervaire]
     $ \exists  $ 1 10-dimensional topological manifold without smooth manifold.
\end{theorem}
\begin{theorem}[Milnor]
     $ \exists  $ a smooth manifold  $ M  $ \st  $ M\cong S^7 $ but not in diffeomorphism meaning.  
\end{theorem}
\begin{theorem}[Kervaire-Milnor]
     $ \exists  $ 28 smooth structures (up to orientation preserving diffeomorphism) on  $ S^7 $ 
\end{theorem}
\begin{theorem}[Morse-Birg]
    On  $ S^7  $. If  $ n \leq 3  $, then any  $ n  $-dimensional topological manifold  $ M  $ has a unique smooth structure up to diffeomorphism.
\end{theorem}
\begin{theorem}[Stallings]
    If  $ n\not=4  $, then  $ \exists  $ a unique smooth structure on  $ \mathbb{R}^n  $ up to diffeomorphism.
\end{theorem}
\begin{theorem}[Donaldson-Freedom-Gompf-Faubes]
     $ \exists  $ uncountable smooth structures on  $ \mathbb{R}^4 $ up to diffeomorphism. 
\end{theorem}
\begin{definition}[topological manifold with boundary]
    A space  $ M  $ is called a topological manifold with boundary if 
    \begin{enumerate}
        \item  $ M  $ is Hausdorff
        \item  $ M  $ is second countable 
        \item  $ \forall  p\in M $,  $ \exists   $ a neighbourhood  $ U  $ of  $ p  $ and a homeomorphism  $ \varphi:U\rightarrow V    $  where  $ V  $ is open in  $ \mathbb{H}^n $ 
    \end{enumerate}
    We say a manifold  $ M  $ is closed if  $ M  $ is compact and  $ \partial M  $ is empty.
\end{definition}
Our motivation for studying manifold is to study the space of solution for equations.
\begin{question}
    Given  $ f:\mathbb{R}^n\rightarrow \mathbb{R} $ smooth,  $ q\in \mathbb{R}^n $, when is  $ f^{-1}(q)  $ is a smooth manifold?
\end{question}

For  $ f:U\rightarrow \mathbb{R}^n $ smooth,  $ U  $ open in  $ \mathbb{R}^m $,  the differential of  $ f  $ at  $ p\in U  $ denoted as  $ \mathrm{d}f(p) $.  
\begin{definition}
    We say  $ p\in U  $ is a \textbf{regular point}\index{regular point} of  $ f  $ if  $ \mathrm{d}f(p)  $ is surjective. Otherwise we say  $ p\in U  $ is a \textbf{critical point}\index{critical point}.\\
    A point  $ q\in \mathbb{R}^n  $ is called a \textbf{regular value}\index{regular value} of  $ f  $ if  $ \forall  p\in f^{-1}(q)  $ ,  $ p  $ is a regular point of  $ f $.\\
    A point  $ q\in \mathbb{R}^n  $ is called a \textbf{critical value}\index{critical value} of  $ f  $ if  $ \forall  p\in f^{-1}(q)  $ ,  $ p  $ is a critical point of  $ f $.
\end{definition}
\begin{theorem}[Implicit function theorem]
    If  $ p\in U  $ is a regular point of  $ f:U\rightarrow \mathbb{R}^n  $. Then there exists 
    \begin{itemize}
        \item An open neighbourhood  $ V  $ of  $ p  $ in  $ U  $
        \item An open subset  $ V'  $ of  $ \mathbb{R}^m $
        \item  A diffeomorphism  $ \varphi:V\rightarrow V'  $ such that  $ P\circ \varphi=f $ where  $ P  $ is the projection from  $ \mathbb{R}^m $ to  $ \mathbb{R}^n $. 
    \end{itemize}
    In other words, near a regular point, we can do local coordinate change to turn  $ f  $ into the projection.
\end{theorem}
\begin{remark}
    In particular, we have a homeomorphism
    \[ f^{-1}(f(p))\cap V \xrightarrow[\text{restriction of  $ \varphi $ }]{\cong }\{(x_1,\dots,x_m)\in V'|(x_1,\cdots.x_n)=f(p)\}\]
    \ie if we set  $ M=f^{-1}(f(p)) $, then  $ (M\cap V,\varphi_p) $ is a chart that contains  $ p  $.  
\end{remark}
\begin{corollary}
    If  $ q  $ is a regular value of  $ f:U\rightarrow \mathbb{R}^n $ then  $ f^{-1}(q) $ is a smooth manifold.
\end{corollary}
\begin{remark}
    It suffices to show that the corresponding charts are compatible.
\end{remark}
\begin{theorem}[Sard]
    If  $ f:U\rightarrow \mathbb{R}^n $ is a smooth map, then the set of critical values of  $ f $ has measure $  0 $.
\end{theorem}
\begin{remark}
    For a "generic"  $ q  $,  $ f^{-1}(q)  $ is a manifold of dimension  $ m-n $. 
\end{remark}
\begin{corollary}
    If  $ f:U\rightarrow\mathbb{R}^n $ is smooth and  $ m<n  $ then  $ f(U ) $ has measure $  0  $. 
\end{corollary}

\subsection{Lie groups and homogeneous spaces}

\begin{definition}
    We say  $ G  $ is a \name{Lie group} if it is a topological group with a smooth structure such that the multiplication map $ \cdot : G \times G \to G $ and the inverse map  $ G\rightsquigarrow G  $  is smooth. 
\end{definition}
\begin{example}
     $ GL(n,\mathbb{R})=\{n\times n \text{ matrices with non-zero determinant}\}\subset \mathbb{R}^{n\times n } $\\
      $ O(n)=\{A\in GL(n,\mathbb{R})|AA^T=I\} $\\
       $ SO(n)=\{A\in O(n)|\det A=1\} $\\
        $ U(n)=\{A\in GL(n,\mathbb{C})|A\overline{A}^T=1\} $\\
         $ SU(n)=\{A\in U(n)|\det A=1\} $     
\end{example}
\begin{exercise}
    \begin{align}
        O(1)&\cong S^2 &SO(1)&\cong *\\
            SO(2)&\cong S^1 &SO(3)&\cong \mathbb{R}\mathbb{P}^3 \\
            SU(2)&\cong S^3 &U(n)&\cong S^1\times SU(n)
    \end{align}
    The last one is a diffeomorphism but do not preserve the multiplicatioin, \ie not an isomorphism of Lie group.
\end{exercise}
\begin{theorem}[Carton]
    Let  $ H  $ be a closed subgroup of Lie group  $ G  $. Then  $ H  $ is a Lie group. More precisely,  $ H  $ is topological manifold, carries a canonical smooth structure that makes  the multiplication and inverse smooth. Also,  $ G/H  $ is a smooth manifold
\end{theorem}
\begin{definition}
    Let  $ M  $ be a smooth manifold. We say   $ M  $ is a \name{homogeneous space} if  $ \exists  $ a Lie group 
     $ G  $ with a smooth transitive action $ \rho: G\times M \rightarrow M  $.
\end{definition}
\begin{definition}
    For  $ M  $ be a homogeneous space.
    The \name{isotropy} group of  $ x\in M  $ is defined as 
    \[Iso(x)=\{g\in G|gx=x\}\]
    closed subgroup of  $ G $\\
    Given any  $ x,x'\in M  $,  $ Iso(x)\cong Iso(x') $ because the group  action is transitive. \\
    Hence, we have a well-defined map 
    \begin{align}
        p:G/_{Iso(x)}&\rightarrow M \\
        g\mapsto gx
    \end{align}
\end{definition}
\begin{theorem}
     $ p  $ is always a diffeomorphism.
\end{theorem}
Therefore, we have this proposition
\begin{proposition}
    $ M  $ is a homogeneous space  $ \Leftrightarrow  $  $ M=G/H  $ for some closed subgroup  $ H  $.
\end{proposition}
\begin{example}
    If  $ M=S^n  $, let  $ G=SO(n+1)  $.\\
    Then  $ Iso(1,0,\cdots,0)\cong SO(n) $.\\
    So  $ S^n\cong SO(n+1)/(SO(n)) $. \\
    Similarly, we can prove  $ \mathbb{RP}^n\cong SO(n+1)/(O(n)) $,  $ \mathbb{CP}^n\cong SO(n+1)/(U(n))$\\
    The isotropy k dimensional linear subspaces of  $ \mathbb{R}^n  $ can be  $ O(k)\times O(n-k) $ if  $ G=O
    (n) $   
\end{example}
A connected closed surface is a homogeneous space if and only if it is diffeomorphic to  $ \mathbb{RP}^2,S^2,T^2 $ and Klein bottle. 
\begin{theorem}[Whithead]
    Any smooth manifold has a triangulation.
\end{theorem}
\begin{theorem}[Poincare-Hopf]
     $ G  $ is compact Lie group  $ \Rightarrow  $  $ \chi(G)=0 $. 
\end{theorem}
\begin{theorem}[Mostow2005]
     $ M  $ is a compact homogeneous space  $ \Rightarrow  $  $ \chi(M) \geq 0 $. 
\end{theorem}
\subsection{Bump Function and Partition of Unity}
\begin{theorem}[Urysohn smooth version]
    Given  $ M  $, closed disjoint  $ A,B  $,  $ \exists  $ smooth  $ f:M\rightarrow[0,1] $ \st  $ f|_A=0,f|_B=1 $.  
\end{theorem}
\begin{theorem}[Tietze]
    Given  $ M  $, closed  $ A  $, smooth  $ f:A\rightarrow \mathbb{R}^n  $, there exists smooth  $ \hat{f }:M\rightarrow \mathbb{R}^n $ \st  $ \hat{f}|_A=f $  
\end{theorem}
To prove these and much more result we need partition of unity theorem.

First we define bump function.
\begin{lemma}
    Let  $ U  $ be a neighbourhood of  $ p\in M $. Then  $ \exists   $ smooth  $ \sigma:M\rightarrow [0,1]  $ \st 
    \begin{enumerate}
        \item  $ \sigma \equiv 1  $ near  $ p $
        \item Supp $ \sigma \subset U  $  
    \end{enumerate}
    Such  $ \sigma  $ is called a \name{bump function } at  $ p  $, supported in  $ U $. 
\end{lemma}
\begin{definition}
    An open cover of a space  $ X  $ is \name{locally finite} if any point has a neighbourhood that intersects only finite many open sets of this cover.
\end{definition}
\begin{proposition}
    Given compact  $ K\subset U $ and open neighbourhood  $ U  $ of  $ K  $,  $ \exists  $ a smooth  $ g:M\rightarrow [0,+\infty )$ \st  $ g|_K\equiv 1 $ and  $ Supp \,g\subset U $.  
\end{proposition}
\begin{definition}
    An \name{exhaust} of a space  $ X  $ is a sequence of open sets  $ \{U_i \} $ \st 
    \begin{enumerate}
        \item  $ X=\bigcup\limits_{i=1}^\infty  U_i $ 
        \item  $ \overline{U_i }  $ is compact and contained in  $ U_{i+1}  $
    \end{enumerate}
\end{definition}
\begin{theorem}
    Any topological manifold has an exhaust.
\end{theorem}
Given two open covers  $ \mathcal{ U}, \mathcal{V} $, we say  $ \mathcal{V } $ is a \name{refinement} of  $ \mathcal{U }  $ if  $ \forall U_\alpha\in U  $,  $ \exists V_\beta\in \mathcal{V} $ \st $ V_\beta\subset U_\alpha $.

We say a space  $ X  $ is paracompact if any open cover has a locally finite refinement.

Actually, any metric space is paracompact.(The proof is hard)
\begin{proposition}
    Let  $ \mathcal{U}=\{U_\alpha \}  $ be an open cover of a topological manifold  $ M  $. Then there exists countable open covers  $ \mathcal{W}=\{W_i \} $,  $ \mathcal{V}=\{V_i\}    $\st
    \begin{itemize}
        \item For any  $ i  $,  $ \overline{V_i } $ is compact and  $ \overline{V_i}\subset W_i $
        \item  $ \mathcal{W } $ is locally finite.
        \item  $ \mathcal{W }  $ is a refinement of  $ \mathcal{U} $.  
    \end{itemize} 
\end{proposition}
As a corollary, we have any topological manifold  is paracompact.
\begin{definition}
    Given open cover  $ \mathcal{U } $ of a smooth  $ M  $, a partition of unity subordinate\index{partition of unity subordinat(P.O.U)} to  $ \mathcal{U } $ is a collection of smooth functions  $ \{\rho_\alpha:M\rightarrow [0,1] \}_{\alpha\in \mathcal{A}} $ \st
    \begin{enumerate}
        \item  $ \forall p\in M  $,  $ \exists  $ only finitely many  $ \alpha\in \mathcal{A} $ \st  $ p\in Supp\,\rho_\alpha $  
        \item  $ \sum\limits_{\alpha\in \mathcal{A}} \rho_\alpha(p) =1 $ 
        \item  $ Supp\,\rho_\alpha\subset U_\alpha $ 
    \end{enumerate} 
\end{definition}
\begin{theorem}[Existence of P.O.U]
    For any open cover  $ \mathcal{U } $ of smooth  $ M  $,  $ \exists  $ a P.O.U subordinate to  $ \mathcal{U} $  
\end{theorem}
\begin{theorem}[Whitney approximation theorem]
    Given any smooth  $ M  $, any closed  $ A  $ and any continuous  $ f:M\rightarrow \mathbb{R } $,  $ \delta:M\rightarrow (0,+\infty ) $. Suppose  $ f  $ is smooth on  $ A  $. Then  $ \exists g:M\rightarrow \mathbb{R} $ smooth \st
    \begin{itemize}
        \item  $ g|_A=f|_A $ 
        \item  $ \forall p\in M $, $ |g(p)-f(p)|<\delta(p) $.  
    \end{itemize}
\end{theorem}
\section{Tangent space and tangent vectors}
\subsection{Tangent Space}
Given  $ p\in M  $, consider the set  $ C_p^\infty(M)=\{\text{smooth function } V\rightarrow \mathbb{R}\}/_\sim $ where  $ f_1\sim f_2 $ if and only if  $ \exists $ neighbourhood  $ U $ of  $ p $,  $ f_1|_U=f_2|_U $.

 $ C_p^\infty(M) $ is the space of \name{genus of smooth function} near  $ p $. 

 
A \name{partial-derivative} of  $ p $ is a  $ \mathbb{R} $-linear map  $ D:C_p^\infty(M)\rightarrow \mathbb{R} $ that satisfies the Leibniz rule:
\[D(fg)=D(f)g(p)+f(p)D(g)\] 
\begin{definition}
    A \name{tangent vector} of  $ M  $ at  $ p  $ is a partial-derivative at  $ p $.
    
    Define the \name{tangent space}  $ T_pM=\{\text{all partial-derivative at  $ p $ }\} $, which is a  $ \mathbb{R} $-vector space.   
\end{definition}
\begin{proposition}
    For  $ M=U\subset \mathbb{R}^n $ open. We have  $ \{\dfrac{\partial }{\partial x_i}\} $ is a basis for  $ T_pU $.   
\end{proposition}
\begin{proposition}
    \[\frac{\partial }{\partial x^i}|_p=\sum\limits_{1 \leq i \leq n}\frac{\partial y^i}{\partial x^i}\cdot \frac{\partial }{\partial y^i}|_p\]
\end{proposition}

Now we try to define differential of a smooth map.

 $ M,N  $ smooth manifolds,  $ C^\infty(N,M)=\{\text{smooth } F:N\rightarrow M\} $.
 
 Given  $ F\in C^\infty(N,M) $,  $ F  $ induces  $ F^*:C^\infty_{F(p)}(M)\rightarrow C^\infty_p(N) $ $ f\rightarrow f\circ F $.
 
 By taking dual, we get 
 \[F_*:T_pN\rightarrow T_{F(p)}M\]    
 we also write  $ F_*  $ as  $ F_{*,p} $, call it the \name{differential} of  $ F $ at  $ p $. 
 
 where 
 \[F_*(\frac{\partial }{\partial x^i}|_p)=\sum\limits_{k}\frac{\partial F^k}{\partial x^i}\cdot \frac{\partial }{\partial y^k}|_{F(p)}\]
 \begin{proposition}
    The differential satisfies the composition law.
    \[(G\circ F)_*=G_*\circ F_*:T_pN\rightarrow T_{G\circ F(p)}W\]
 \end{proposition}
 \begin{definition}
    A smooth \name{curve} is a smooth map  $ \gamma:(a,b)\rightarrow M $. We say  $ \gamma  $ starts at  $ p  $ if  $ \gamma(0)=p $. We define the \subname{velocity}{curve} of  $ \gamma  $ at  $ \gamma(0) $ as  $ \gamma_*(\frac{\partial}{\partial t}|_0)\in T_{\gamma(0)}M $
    
    Take charts  $ (U,x^1,\cdots,x^n) $ about  $ p  $, let  $ \gamma^i=x^i\circ \gamma $.
    
    We say  $ \gamma, \delta $ are \subname{tangent}{curve} to each other at  $ p  $ if  $ (\gamma^i)'(0)=(\delta^i)'(0) $.       
 \end{definition}
 Now we can define 
 \[(T_pM)_{curve}:=\{\text{smooth curves  $ \gamma  $ starting at  $ p  $ }\}/_\sim\]
 where  $ \gamma\sim \delta  $ iff they are tangent to each other.
 
 Then these definition is more geometric.
 
 \begin{lemma}
    Given  $ F\in C^\infty(M,M) $,  $ p\in N $, the diagram commutes:
    \begin{center}
        \begin{tikzcd}
            \gamma\in (T_pN)_{curve} \arrow[r,"\cong"]\arrow[d] & T_pN\arrow[d]\\
            F\circ \gamma\in (T_{F(p)}M)_{curve}\arrow[r,"\cong"] & T_{F(p)}M
        \end{tikzcd} 
    \end{center}
 
 \end{lemma}
 \subsection{Tangent Bundle}
 Let  $ (M,\mathcal{A}) $ be a smooth manifold,  $ TM=\dps\bigcup_{p\in M}T_pM $, called the \name{tangent bundle}
 
 Now we want to define a natural topology and smooth structure on  $ TM $. Take any chart  $ (U,\varphi)=(U,x^1,\cdots,x^n)\in \mathcal{A} $.
 
 We have a map 
 \begin{align}
    \hat{\varphi}:TU&\xrightarrow{\cong}\varphi(U)\times\mathbb{R}^n\subset \mathbb{R}^n\times\mathbb{R}^n\\
    X\in T_pU&\mapsto (\varphi(p),X^1,\cdots,X^n)
 \end{align}  
 where  $ X=\sum X^i\frac{\partial}{\partial x^i}|_p $.
 
 Then pull back standard topology on  $ \varphi(U)\times\mathbb{R}^n $ to a topology on  $ TU $.
 \[\mathcal{B}=\{\hat{\varphi}^{-1}(V)|(\varphi,U)\in \mathcal{A},  V \text{ open in  $ \varphi(U)\times\mathbb{R}^n $ }\}\]
 There is some fact in topology:
 \begin{itemize}
    \item  $ \mathcal{B} $ is a basis
    \item  $ \mathcal{B} $ generates a Hausdorff, second countable topology on  $ TM $. 
 \end{itemize} 
 So  $ TM  $ is a topological manifold covered by charts  $ \hat{\mathcal{A}}=\{(TU,\hat{\varphi})|(U,\varphi)\in\mathcal{A}\} $.
 
Given  $ (TU,\hat{\varphi}),(TV,\hat{\psi})\in \hat{\mathcal{A}} $, the transition function is 
\begin{align}
    \varphi(U\cap V)\times\mathbb{R}^n&\xrightarrow{\hat{\psi}\circ\hat{\varphi}^{-1}}\psi(U\cap V)\times \mathbb{R}^n\\
    (p,x)&\mapsto (\psi\circ\varphi^{-1},J(\psi\circ \varphi^{-1})|_p(X))
\end{align}   
So  $ \hat{\mathcal{A}} $ is a smooth atlas on  $ TM  $, making  $ TM  $ into a smooth manifold.
\begin{definition}[vector bundle]
    Given a continuous map  $ f:E\rightarrow B $, we say  $ f $ is a  $ n $-dimensional \name{vector bundle} if: $ \exists $ an open cover  $ \mathcal{U}=\{U_\alpha\}_{\alpha\in I} $ of  $ B $ and homeomorphisms  $ \{f^{-1}(U_\alpha)\xrightarrow[\cong ]{\rho_\alpha}U_\alpha\times \mathbb{R}\} $ \st   
    \begin{itemize}
        \item  
        \begin{tikzcd}
            f^{-1}(U_\alpha)\arrow[d,"f"]\arrow[r,"\rho_\alpha"]&U_\alpha\times \mathbb{R}^n\arrow[ld,"projection"]\\
            U_\alpha
        \end{tikzcd}
        commutes for  $ \alpha\in I $ 
        \item  $ \forall  $  $ p\in U_\alpha\cap U_\beta $, the map 
        \[\mathbb{R}^n=\{p\}\times\mathbb{R}^n\xrightarrow{\rho_\alpha}f^{-1}(p)\xrightarrow{\rho_\beta}\{p\}\times \mathbb{R}^n=\mathbb{R}^n\] 
        is linear.
    \end{itemize}
    Call  $ f^{-1}(p)  $  the \name{fiber} over  $ p $. 
\end{definition} 
\begin{proposition}
    Given vector bundle  $ f:E\rightarrow B $, the fiber  $ f^{-1}(p) $ has a structure of a vector space.  
\end{proposition}
\begin{example}[Product bundle]
    $ E=\mathbb{R}^n\times B $ 
\end{example}
\begin{example}[Tautological bundle]
    \[B=\mathbb{CP}^n= \{\text{1-dim complex subspace of }\mathbb{C}^{n+1}\},E= \{(L,v)\in\mathbb{CP}^n\times \mathbb{C}^{n+1}\}  \]
    
    And we map  $ (L,v)\mapsto L $  
\end{example}
Given vector bundles  $ E_1\xrightarrow{\pi_1} B_1,E_2\xrightarrow{\pi_2} B_2 $, a bundle map consists of  $ (\hat{f},f) $ \st
\begin{itemize}
   \item \begin{tikzcd}
       E_1\arrow[r,"\hat{f}"]\arrow[d,"\pi"] & E_2\arrow[d,"\pi"] \\
       B_1\arrow[r,"f"] & B_2
   \end{tikzcd} commutes.
   \item  $ \forall b\in  B  $,  $ \hat{f}:\pi_1^{-1}(b)\rightarrow \pi_2^{-1}(f(b)) $ is linear.
\end{itemize}
If  $ \hat{f},f $ are diffeomorphisms, then we call  $ (\hat{f},f) $ an \subname{isomorphism}{vector bundle} of vector bundle.

An isomorphism to a product bundle is called a \name{trivialization}. An bundle is \textbf{trivial} if it has a trivialization.

\begin{example}
    $ TS^1,TS^2 $ are both trivial.
    
     $ S^1\cong O(1)\cong SO(2),S^3\cong SU(2) $ 
\end{example}
\begin{theorem}
   If  $ G  $ is a Lie group, then  $ TG  $ is trivial.
\end{theorem}
\begin{proof}
    For $(x^1,x^2,\cdots,x^n)$ is a basis of $T_eG$
    The bundle isomorphism is 
    \[G\times \mathbb{R}^n\xrightarrow{\varphi}TG,\, (g,c^1,\cdots,c^n)\mapsto (g,(l_g)_{*,e}(\sum_ic^ix^i))\] 
    where 
    \[l_g:G\rightarrow G, h\mapsto gh\]
    is a diffeomorphism. Hence, it induces the isomorphism $(l_g)_*$\\
\end{proof}
\begin{proposition}[Adams, 1960s]
    $ TS^n  $ is trivial if and only if  $ n=0,1,3,7 $. 
\end{proposition}
\begin{proposition}
   \begin{enumerate}
       \item Given any  $ F\in C^\infty(M,N) $,  $ F_*:TM\rightarrow TN $ is a bundle map.
       \item  $ TS^n $ is isomorphic to the following bundle:
        \[B=s^n\qquad E=\{(p,v)\in S^n\times \mathbb{R}^{n+1}|v\perp p\}\]   
   \end{enumerate}
\end{proposition}
\begin{definition}[smooth section]
   Given a smooth vector bundle  $ \pi:E\rightarrow B $, a \name{smooth section} is a smooth map  $ S:B\rightarrow E $ \st  $ \pi\circ S=id_b $.
   
    $ s_0:B\rightarrow E, b\mapsto 0\in\text{0-vector in  $ \pi^{-1}b $ } $. 
\end{definition}
\subsection{Vector Field, Curves and Flows}
\begin{definition}
   A (tangent) \name{vector field} is a smooth section of  $ TM $. \ie a smooth map  $ M\xrightarrow{X} TM $ \st  $ X(p)\in T_pM $,$ \forall p\in M $  
\end{definition}
Given any  $ f:\mathbb{R}^n\rightarrow \mathbb{R} $, define the \name{gradient vector field} 
\[\dps\triangledown f_p:=\sum\limits_{1 \leq i \leq n}\frac{\partial f}{\partial x^i}(p)\frac{\partial}{\partial x^i}\] 
\begin{example}
    $ X=f^1\partial x^1+f^2\partial x^2 $ is a gradient field if and only if  $ \dfrac{\partial f^1}{\partial x^2}=\dfrac{\partial f^2}{\partial x^1} $ 
\end{example}
\begin{theorem}[Poincare-Hopf]
   For closed  $ M  $,  $ M  $ has a nowhere vanishing vector field if and only if  $ \chi(M)=0 $. 
\end{theorem}
So  $ S^n $ has a  nowhere vanishing vector field if and only if  $ n  $ is odd.
\begin{theorem}[MaoQiu]
    $ S^2 $ has no no-where vanishing vector field.
    
   
\end{theorem}
So We cannot smooth out all the hairs on a ball.

   Given  a vector field  $ X=\{X_p\}_{p\in M} $, a curve  $ \gamma:(a,b)\rightarrow M $ is called an \name{integral curve} of  $ X $ if  $ \gamma'(t)=X_{\gamma(t)} $, $ \forall t\in (a,b) $, 
   where  $ \gamma'(t)=\gamma_*(\dfrac{\partial }{\partial t})\in T_{\gamma(t)}M $.
   
   We say  $ \gamma  $ is maximal if the domain cannot be extended to a larger interval.     
Denote the set of all smooth vector fields on  $ M $ by \name{$ \mathfrak{T}M $} 

Recall that   $ \gamma  $ is maximal if it's domain can not be extended to a large open interval.

In a local chart  $ (U,x^1,\cdots,x^n) $,  $ \dps X|_U=\sum_{i=1 }^{n}a^i\partial x^i $. Then  $ \gamma $ is an integral curve if and only if  $ (\gamma^i)'(t)=a^i(\gamma(t)) $,  $ \forall 1 \leq i \leq n $, where  $ \gamma^i=x^i\circ\gamma:(a,b)\rightarrow \mathbb{R} $.

And in this case the initial value condition:  $ \gamma(0)=p $ $ \Leftrightarrow $ $ \gamma^i(0)=p^i $.

Locally, solving integral curve starting at  $ p  $ is equivalent to solving ODE with initial value  $ p^1,\cdots,p^n $.
 By existence and uniqueness of solutions of ODE, we have
 \begin{theorem}[Fundamental theorem of integral curve]
    Let  $ X\in \mathfrak{T}M $,  $ p\in M $, then:
    \begin{enumerate}
        \item [(1)](Uniqueness) Given any two integral curves  $ \gamma_1,\gamma_2 :(a,b)\rightarrow M$, then we have:
        \[\gamma_1(c)=\gamma_2(c)\text{ for some  $ c\in (a,b) $ }\Rightarrow \gamma_1=\gamma_2\] 
        \item[(2)] there exists a unique max integral curve  $ \gamma:(a(p),b(p))\rightarrow M $ starting at  $ p $.  
        \item[(3)](integral curve smoothly depend on initial values)  $ \exists  $ Nbh  $ U  $ of  $ p $, $ \epsilon>0 $, and smooth  $ \varphi:(-\epsilon,\epsilon)\times U\rightarrow M $ \st  $ \forall q\in U $,  $ \varphi_\epsilon:=\varphi(-,q):(-\epsilon,\epsilon)\rightarrow M $ is an integral curve starting at  $ p $.      
    \end{enumerate} 
    we call such  $ \varphi $ a local \name{flow} generated by  $ X $. 
 \end{theorem}
 \begin{definition}
    Given  $ X\in \mathfrak{T}M $, a global \name{flow} generated by  $ X $ is a smooth map  $ \varphi:\mathbb{R}\times M\rightarrow M $ \st  $ \forall q\in M $,  $ \varphi_q:=\varphi(-,q) $ is the maximal integral curve of  $ X  $ starting at  $ q $.\\
     $ \Leftrightarrow $  $ \dps \frac{\partial \varphi}{\partial t}(s,p)=X_{\varphi(s,p)} $,  $ \forall s\in\mathbb{R}, p\in M $ and  $ \varphi(0,p)=p,\forall p\in M $.   
 \end{definition}
 If such global flow exists, then we say  $ X  $ is \name{complete}.
 \begin{example}
    \,
    \begin{itemize}
        \item  $ X=x\cdot\partial x\in\mathfrak{T}\mathbb{R} $ is complete, where global flow  $ \varphi:\mathbb{R}\times M\rightarrow M $,  $ \varphi(t,p)=p\cdot e^t $.
        \item  $ X=x^2\partial x $ is not complete. Max integral curve starting at  $ 1 $ is given by  $ \gamma(t)=\dps\frac{1}{1-t},t\in (-\infty,1)\not=\mathbb{R} $.  
    \end{itemize}
 \end{example}
 Given  $ X\in \mathfrak{T}M $, we define \name{$ \Supp X $}$=\overline{\{p\in M:X_p\not=0\}}  $.
 \begin{theorem}
    If a vector field  $ X  $ is compactly supported, then  $ X  $ is complete.\label{completeness of compact supported vector field}
 \end{theorem}  
\begin{corollary}
    Any vector field on closed  manifold is complete.
\end{corollary}
\begin{lemma}[Escaping lemma]
    Suppose  $ \gamma:(a,b)\rightarrow M $ is a max integral curve, with  $ (a,b)\not=\mathbb{R} $. Then  $ \not\exists  $ compact  $ K\subset M $ \st  $ \gamma(a,b)\subset K $    
\end{lemma}
\begin{proof}
    Otherwise, suppose  $ \gamma(a,b)\subset K $. WLOG, we may assume  $ b<+\infty $.

    Take  $ (t_i)\rightarrow b $ from left. Then  $ \gamma(t_i)\in K $. After passing to subsequence, we may assume   $ (\gamma(t_i))\rightarrow p\in K $.

    Then  $ \exists $  $ U $ Nbh of  $ p $, local flow $ \varphi:(-\epsilon,\epsilon)\times U\rightarrow M $. Take  $ n $ large enough \st  $ b-t_n<\epsilon $,  $ \gamma(t_n)\in U $. Then  $ \gamma(-+t_n):(a-t_n,b-t_n)\rightarrow M $,  $ \varphi(-,\gamma(t_n)):(-\epsilon,\epsilon)\rightarrow M $ are both integral curves for  $ X  $ starting at  $ \gamma(t_n) $. By uniqueness, they coincide.

    Let  $ \hat{\gamma}:(a,t_n+\epsilon)\rightarrow M $ be defined by  $ \dps\hat{\gamma}(t)=
        \begin{cases}
            \gamma(t),t\in (a,b)\\
            \varphi(t-t_n,\gamma(t_n)), t\in [b,t_n+\epsilon)
        \end{cases} $  

    Then  $ \hat{\gamma} $ is an integral curve with larger domain, then  $ \gamma $ contradiction with the maxity of  $ \gamma $.   
\end{proof}
\begin{proof}[Proof of \ref{completeness of compact supported vector field}]
    Take any max integral curve  $ \gamma:(a,b)\rightarrow M $. Suppose  $ (a,b)\not=\mathbb{R} $. Then  $ X_{\gamma(s)}\not=0 $, $ \forall s $. Otherwise, the constant map  $ \mathbb{R}\rightarrow M,t\mapsto \gamma(s) $ is an integral curve with lager domain.
    
    So  $ \forall s $,  $ \gamma(s)\in \Supp X $ $ \Rightarrow  $ $ \gamma(a,b)\subset \Supp X $ which is compact  $ \Rightarrow $  $ (a,b)=\mathbb{R} $ by the lemma. This causes contradiction!      
\end{proof}
A smooth  $ \varphi:\mathbb{R}\times M\rightarrow M $ is called an \name{one-parameter transformation group} if 
\begin{enumerate}
    \item[(1)]  $ \varphi_0:=\varphi(0,-)=\id_M $ 
    \item[(2)]  $ \varphi_s\circ \varphi_t=\varphi_{s+t} $ for all  $ s,t\in \mathbb{R} $. In particular,  $ \varphi_s^{-1}=\varphi_{-s} $.   
\end{enumerate} 
\begin{theorem}\label{equivalence of one-parameter transformation group}
     $ \varphi\in C^\infty(\mathbb{R}\times M,M) $, then  $ \varphi $ is an one-parameter transformation group if and only if  $ \varphi $ is the global flow generated by some  $ X\in \mathfrak{T}M $   
\end{theorem}
\begin{lemma}[Translation lemma]
    If  $ \gamma:(a,b)\rightarrow M $ is an integral curve for some  $ X\in \mathfrak{T}M $, then  $ \forall s\in \mathbb{R} $,  $ \gamma(-+s) :(a-s,b-s)\rightarrow M$ is also an integral curve for  $ X$.   
\end{lemma}
\begin{proof}
    Let $ \iota=\gamma(-+s) $. Then  $ \iota'(t)=X_{\gamma'(t+s)}=X_{\iota(t)} $  
\end{proof}
\begin{lemma}
    Let  $ \varphi:(-\epsilon,\epsilon)\times U\rightarrow M $ be a local flow for some  $ X\in \mathfrak{T}M $. Then  $ \varphi_s\circ \varphi_r(p)=\varphi_{s+r}(p) $ provided that  $ s,t,s+t\in (-\epsilon,\epsilon), p,\varphi_r(p)\in U $.    
\end{lemma}
\begin{proof}
     $ \gamma_p=\varphi(-,p) $ is an integral curve for  $ X $.
     
      $ \Rightarrow  $  $ \gamma_p(-+s) $ is an integral curve for  $ X  $ starting at  $ \gamma_p(s)=\varphi_s(p) $. But  $ \gamma_{\varphi_s(p)} $ is also an integral curve starting at  $ \varphi_s(p) $. Thus  $ \gamma_{\varphi_s(p)}=\gamma_p(-+s) $ $ \Rightarrow $  $ \varphi_r\circ \varphi_s(p)=\gamma_{r+s}(p) $     
\end{proof}
\begin{lemma}
    Let  $ \varphi:(-\epsilon,\epsilon)\times U\rightarrow M $ be a local flow for some  $ X\in \mathfrak{T}M $. Then  $ \varphi_{s,*}(X_p)=X_{\varphi_s(p)}\in T_{\varphi_s(p)}M $ \ie any vector field is invariant under its flow.  
\end{lemma}
\begin{proof}
    Take  $ f\in C^\infty_{\varphi(p)}(M) $.  
    \begin{align}
        \varphi(s,*)(X_p)(f)&=X_p(f\circ \varphi_s)\\
        &=\frac{\mathrm{d}}{\mathrm{d}t}(f\circ\varphi_s(\varphi_t(p)))|_{t=0}\\
        &=\frac{\mathrm{d}}{\mathrm{d}t}(f\circ\varphi_t(\varphi_s(p)))|_{t=0}\\
        &=X_{\varphi_s(p)}(f)
    \end{align} 
\end{proof}
\begin{proof}[Proof of \ref{equivalence of one-parameter transformation group}]
    "$ \Leftarrow $" is because the lemma  $ \varphi_s\circ \varphi_r=\varphi_{s+r} $ 

     "$ \Rightarrow $" Let  $ X=\{X_p\} $  where  $ X_p=\dps\frac{\partial\varphi}{\partial t}|_{(0,p)} $.
     
     Leave it as an exercise.
\end{proof}

\subname{Time dependent}{vector field} vector field is a smooth map  $ X:\mathbb{R}\times M\rightarrow TM $ \st  $ X_{(t,p)}\in T_pM $.

A smooth curve  $ \gamma(a,b)\rightarrow M $ is the \name{integral curve} for  $ X $ if  $ \gamma'(t)=X_{(t,\gamma(t))} $.

In local chart, solving  $ \gamma $ is still solving ODE, so most results still holds for time dependent vector field. Those are some properties:
\begin{itemize}
    \item Uniqueness:  $ \gamma_1,\gamma_2 $ are both integral curves for  $ X  $,  $ \gamma_1(c)=\gamma_2(c)\Rightarrow \gamma_1\equiv\gamma_2 $  
    \item Max integral curve exists and is unique.
    \item Local flow exists.
\end{itemize}

Now define \name{$ \Supp X $}$ =\overline{\{p\in M: X_{t,p}\not=0\text{ for some }t\}} $.

Then  $ X  $ is compactly supported, then  $ X $ is complete(\ie a global flow  $ \varphi:\mathbb{R}\times M \rightarrow M$)

But something is not true for time dependent vector field:
\begin{itemize}
    \item translation lemma is not true.
    \item vector field change under its flow.
    \item global flow can not implies one-parameter transformation group.
\end{itemize}
\subsection{Another definition of vector field}
A derivation on   $ M  $ is a  $ \mathbb{R} $-linear map  $ C^\infty(M)\xrightarrow{D}C^\infty(M) $ that satisfies the Leibniz rule:
\[D(f\cdot g)=Df\cdot g+f\cdot Dg\] 
\begin{theorem}

    We have a bijection:
    \begin{align*}
        \rho:\mathfrak{T}M&\xrightarrow{1:1}\{\text{derivation on }M\}\\
        X&\mapsto D_X:f\mapsto X(f)
    \end{align*}
\end{theorem} 

\begin{lemma}\label{Lemma A}
    $D_p:\mathfrak{T}_p{M} \rightarrow \mathbb{R}$-linear map $\mathbb{C}^\infty(M)\rightarrow\mathbb{R} \st D_p(f\cdot g)=D_p(f)\cdot g(p)+f(p)\cdot D_p(g)$ is an isomorphism of vector spaces.
\end{lemma}
\begin{proof}

     Leave it as an exercise.
\end{proof}
\begin{lemma}\label{Lemma B}
    Given a vector field(not necessarily smooth) $X=\left\{X_p\right\}_{p\in M}$ , X is smooth $\Leftrightarrow$ $\forall f \in C^\infty(M),X(f)$ is smooth.
\end{lemma}
\begin{proof}
    "$ \Leftarrow $" $\forall p\in M$, take chart $(U,x^1,x^2,\cdots,x^n)$ around $p$. $X|_U=\sum_{i=1}^{n}{f^i\frac{\partial}{\partial x^i}}$  $f^i:U\rightarrow \mathbb{R}$, where $f^i=X|_U(x^i)$.
    Take $\varphi:M\rightarrow [0,1] \st \varphi \equiv 1 $ near $p$, $\Supp \varphi\subset U$,$\varphi \cdot x^i\in C^\infty(M)$.
    
    Then $X(\varphi \cdot x^i)=f^i$ near $p$. By assumption, $f^i$ is smooth near $p$. So $f^i$ is smooth, so $X$ is smooth.
    
     "$ \Rightarrow $" Similar.
\end{proof}
\begin{theorem}
    The map $\rho:\mathfrak{T}M\rightarrow \{\text{derivation on }M\},X\mapsto (D_x:f\mapsto X(f))$ is well-defined and bijective.
\end{theorem}
\begin{proof}
     $ \rho $ is well-defined: $ X(f)\in C^\infty(M) $ by Lemma \ref{Lemma B}, and  $ D_x(fg)=D_x(f)g+fD_x(g) $ since  $ X $ is a point-derivation.
     
      $ \rho $ is injective: $ D_x=D_y\Rightarrow $ $ D_{X_p}=D_{Y_p} $ as maps  $ C^\infty(M) $ to  $ \mathbb{R} $. By Lemma \ref{Lemma A}, we have  $ X_p=Y_p,\,\forall p $. So  $ X=Y $.  
      
       $ \rho $ is surjective: Given  $ D:C^\infty(M)\rightarrow C^\infty(M) $. Define  $ D_p:C^\infty(M)\rightarrow \mathbb{R} $ by  $ D_p(f):=D(f)(p) $ satisfies the Leibniz rule. By Lemma \ref{Lemma A},  $ D_p=D_{X_p} $ for some  $ X_p\in T_pM $. Define  $ X=\{X_p\}_{p\in M} $. Then  $ X(f)=D(f) ,\,\forall f\in C^\infty(M)$. By Lemma\ref{B},  $ X $ is a smooth vector field.    
\end{proof}
\subsection{Lie bracket}
In this section, we can actually find those identification:
\begin{align*}
    \{\text{Tangent vector at $ p $}\}&=\{\text{point derivation at $ p $}\}\\
    &=\{\text{$ \mathbb{R} $-linear maps  $ C^\infty_p(M)\xrightarrow{D_p} \mathbb{R} $ \st}\\&D_p(fg)=D_p(f)g(p)+f(p)D_p(g)\}
\end{align*}
\begin{align*}
    \{\text{smooth vector fields}\}&=\{\text{smooth sections of  $ TM $}\}\\
    &=\{\text{derivation on  $ M $}\}
\end{align*}
\begin{notation}
    We will identify  $ X\in \mathfrak{T}M$  with its derivation  $ D_x:C^\infty(M)\rightarrow C^\infty(M) $. So a vector field is just a  $ \mathbb{R} $-linear map  $ X:C^\infty(M)\rightarrow C^\infty(M) $ \st  $ X(fg)=fX(g)+X(f)g $.    
\end{notation}
\begin{definition}[Lie bracket]
    Given two (smooth) vector field  $ X,Y:C^\infty(M)\rightarrow C^\infty(M) $, we define the \name{Lie bracket}
    \[[X,Y]=X\circ Y-Y\circ X:C^\infty(M)\rightarrow C^\infty(M)\] 
\end{definition}
\begin{theorem}
    For any  $ X,y\in\mathfrak{T}M $,  $ [X,Y]\in \mathfrak{T}M $  
\end{theorem}
\begin{proof}
    Easy to check that  $ [X,Y] $ is linear.
    
    By Leibuniz rule,
    \begin{align*}
        [X,Y](fg)&=X\circ Y(fg)-Y\circ X(fg)\\
        &=X(Yf\cdot g+f\cdot Yg)-Y(Xf\cdot g+f\cdot Xg)\\
        &=(X\cdot Y)(f)\cdot g+f\cdot(X\circ Y)(g)-(Y\cdot X)(g)-f\cdot ((Y\circ X)(g))\\
        &=[X,Y](f)\cdot f\cdot [X,Y](g)
    \end{align*}
\end{proof}
So What is the geometric meaning of  $ [X,Y] $? Non commutatiy of flows.
\begin{fact}
    Given  $ X,Y\in \mathfrak{T}M $, we say  $ X,Y $ are commutative vector field if  $ [X,Y]=0 $
    
    $ X,Y $ are commuative iff  for any local flows  $ \varphi^X:(-\epsilon,\epsilon)\times U\rightarrow M $, $ \varphi^Y:(-\epsilon,\epsilon)\times U\rightarrow M $  we have $ \varphi_s^X\circ \varphi_t^T=\varphi_t^Y\circ\varphi_s^X $   
\end{fact} 


\begin{proposition}[Calculation of    { $ \left[V,W\right] $ }    using local charts]
    Chart  $ (U,x^1,\cdots,x^n) $,  $ V,W\in \mathfrak{T}M $,  $ V|_U=\dps\sum_{i=1}^n V^i\frac{\partial }{\partial  x^i} $,  $ W|_U=\dps\sum_{i=1}^nW^i\frac{\partial}{\partial x^i} $. Then 
    \begin{align*}
        [V,W]|_U&=\sum_{i=1}^n(V(W^i-W(V^i)))\frac{\partial}{\partial x^i}\\
        &=\sum_{i=1}^n(\sum_{j=1}^nV^j\frac{\partial W^i}{\partial x^j}-W^j\frac{\partial V^i}{\partial X^j})\frac{\partial }{\partial x^j}\\
        &=\sum_{1 \leq i,j \leq n}(V^j\frac{\partial W^i}{\partial x^j}-W^j\frac{\partial V^i}{\partial x^j})\frac{\partial}{\partial x^j}
    \end{align*}
\end{proposition}
\begin{example}
     $ V=x\partial x+y\partial y $,  $ W=-y\partial x+x\partial y $ commutes.  
\end{example}
\begin{proposition}[Properties of Lie bracket]\label{properties of Lie bracket}
    \,
    \begin{enumerate}
        \item[(a)] Natuality under push-forword.
        
        Given any  $ F\in \Diff(M,N) $,  $ V\in \mathfrak{T}M, W\in \mathfrak{T}M  $, we have  $ [F_*V,F_*W]=F_*[V,W] $.
        \item[(b)]  $ \mathbb{R} $-linearity  $ \forall a,b\in \mathbb{R} $
        \begin{align*}
            [aX+bV,W]&=a[X,W]+b[V,W]\\
            [W,aX+bV]&=b[W,X]+a[W,V]
        \end{align*}  
        
        \item[(c)] anti-symmetric  $ [V,W]=-[W,V] $
        \item[(d)] Jacobi identity 
        \[[V,[W,X]]+[W,[X,V]]+[X,[V,W]]=0\]     
        \item[(f)] Leibuniz rule 
        \[[fV,gW]=fg[V,W]+(f\cdot Vg)W-(g\cdot Wf)V\]
    \end{enumerate}
\end{proposition}
\begin{definition}
    Given  $ F\in C^\infty(M,N) $,  $ V\in \mathfrak{T}M $,  $ W\in \mathfrak{T}N $. We say  $ W $ is  \name{$ F $-related} to  $ V $ if  $ \forall p\in M $,  $ F_{p,*}(V_p)=W_{F(p)}, F_{p,X}:T_pM\rightarrow T_{f(p)}N $      
\end{definition}
\begin{example}
     $ F:S^1\rightarrow \mathbb{R}^2,\theta\mapsto (\cos \theta,\sin \theta) $,  $ V=\partial \theta,W=-y\partial x+x\partial y $. 
\end{example}
\begin{note}
    In general, given  $ V\in\mathfrak{T}M $ and  $ F\in C^\infty(M,N) $. There may not exist  $ W\in \mathfrak{T}M $ \st  $ V,W $ are  $ F $-related. Even such  $ W  $ exists, it may not be unique.
    
    However, if  $ F $ is a diffeomorphism, given any  $ V $,  $ \exists $ unique  $ W $ \st  $ V $ and  $ W $ are  $ F $-related. Actually,  $ W_p=F_*V_{F^{-1}(p)} $.     
    
    Such  $ W $ is called \name{push forward} of  $ V $ along  $ F $, denoted by  $ F_*V $, only defined when  $ F $ is a diffeomorphism.     
\end{note}
\begin{lemma}
     $ \forall V\in\mathfrak{T}M, W\in \mathfrak{T}N $,  $ F\in C^\infty (M,N) $. Then  $ W $ is  $ F $-related to  $ V $ iff  $ \forall  f\in C^\infty (N) $,  $ V(f\circ F)=W(f)\circ F\in C^\infty (M) $     
\end{lemma}
\begin{proof}
    Check that  $ F_{p,*}(V_p)(f)=W_{F(p)}(f) $,  $ \forall f\in C^\infty(N) $
\end{proof}
\begin{proposition}
    Given  $ V_0,V_1\in \mathfrak{T}M $,  $ W_0,W_1\in \mathfrak{T}N $,  $ F\in C^\infty(M,N) $,  $ W_i $ is  $ F $-related to  $ V_i $, $ i=0,1 $  $ \Rightarrow  $ $ [W_0,W_1] $  is  $ F $-related to  $ [V_0,V_1] $         
\end{proposition}
\begin{corollary}[Natuality of Lie bracket]
    Given any  $ F\in \Diff(M,N) $,  $ V\in \mathfrak{T}M, W\in \mathfrak{T}M  $, we have  $ [F_*V,F_*W]=F_*[V,W] $   
\end{corollary}


The rest of Proposition \ref{properties of Lie bracket} is easy to check if it is viewed as a mapping  $ C^\infty(M)\rightarrow C^\infty(M) $.

\subsection{Lie algebra of a Lie group}
\begin{definition}
    A \name{Lie algebra} $ g $ is  $ \mathbb{R} $-linear space  $ g $ with map   $ [-,-]:g\times g\rightarrow g $ \st it is bilinear, anti-symmetric and satisfies the Jacobian identity.
    
    Then  $ (\mathfrak{T}M,[-,-]) $ is an infinite dimensional Lie algebra.
\end{definition}
For  $ G $ Lie group,  $ \forall g\in G $ we have diffeomorphism  
\begin{align*}
    l^g:G\rightarrow G&,h\mapsto gh\\
    r^g:G\rightarrow G&,h\mapsto hg   
\end{align*}
We say  $ X\in \mathfrak{T}G $ is \name{left invariant} if  $ l_*^g(X)=X $,  $ \forall g\in G $. Similarly,  $ X $ is \name{right invariant} if  $ r_*^g(X)=X $.   
\begin{proposition}
     $ X,Y $ are left/right invariant  $ \Rightarrow  $  $ [X,Y] $ is left/right invariant.  
\end{proposition}
\begin{proof}
     $ l_*^g[X,Y]=[l_*^gX,l_*^gY]=[X,Y] $ 
\end{proof}
So we can find a natural Lie algebra of  $ G $: 
\[\name{$ \Lie(G) $}:=\{\text{left invariant vector fields on  $ G $}\},\text{with  $ [-,-] $ restricted from  $ \mathfrak{T}G $}\]
\begin{theorem}\label{left invariant vector field is determined at e}
    Given any  $ V\in T_eG $,  $ \exists $ unique left invariant  $ \hat{V}\in \mathfrak{T}G $ \st  $ \hat{V_e}=V $.    
\end{theorem} 
\begin{corollary}
     $ \dps \Lie(G)\cong T_eG $ as vector spaces. 
\end{corollary}
\begin{proof}[Proof of Theorem \ref{left invariant vector field is determined at e}]\,

    \textbf{Uniqueness of  $ \hat{V} $:}  $ \hat{V_g}=l_{e,*}^g(\hat{V_e})=l_{e,*}^g(V) $. So  $ \hat{V} $ is determined by  $ V $.
    
    \textbf{Existence of  $ \hat{V} $:} Let  $ \hat{V}=\{\hat{V_g}\}_{g\in G} $ where  $ \hat{V_g}=l_{e,*}^g(\hat{V_e}) $.
    
     $ \hat{V} $ is left-invariant because 
     \[\dps(l_*^h(\hat{V}))_g=l_{h^{-1}g,*}^h(\hat{V}_{h^{-1}g})=l_{h^{-1}g,*}^h(l_{e,*}^{h^{-1}g}(V))=l_{e,*}^g(V)=\hat{V_g}\] 

      $ \hat{V} $ is smooth: Take any  $ f\in C^\infty(G) $ suffices to show  $ \hat{V}(f)\in C^\infty(G) $.
      
      Take any smooth  $ \gamma:\mathbb{R}\rightarrow G $ \st  $ \gamma(0)=e,\gamma'(0)=V $. Then  $ l^g\circ \gamma :\mathbb{R}\rightarrow G $ satisfies  $ l^g\circ\gamma(0)=g,(l^g\circ\gamma)(0)=g,(l^g\circ \gamma)'(0)=l_{e,*}^g(V)=\hat{V_g} $   

      So  \begin{equation}
        \hat{V}(f)(g)=\hat{V_g}(f)=\dps\frac{\mathrm{d}}{\mathrm{d}t}f(l^g\circ \gamma(t))|_{t=0}=\frac{\mathrm{d}}{\mathrm{d}t}f(g\cdot \gamma(t))|_{t=0} \label{eq:lie derivative}
      \end{equation}
      
      Consider the map 
      \begin{align*}
        \hat{f}:G\times \mathbb{R}&\xrightarrow{\id\times \gamma}G\times G&\xrightarrow{\cdot}G&\xrightarrow{f}\mathbb{R}\\
        (g,t)&\mapsto (g,\gamma(t))&\mapsto g\cdot\gamma(t)&\mapsto f(g\cdot\gamma(t))
      \end{align*}
      Then  $ \hat{f} $ s smooth,  $ \dps\frac{\partial \hat{f}}{\partial t}|_{t=0}:G\rightarrow \mathbb{R} $ is smooth, but  $ \dps\frac{\partial f}{\partial t}|_{t=0}(g)=\hat{V}(f)(g) $ by \ref{eq:lie derivative}. So  $ \hat{V}(f)\in C^\infty(G) $.    
\end{proof}
\begin{example}
     $ G=\GL(n,\mathbb{R})=\{A\in M_n(\mathbb R)|\det A\not=0\}\subset M_n(\mathbb{R})\cong \mathbb{R}^2 $.
     
      \name{$ \mathrm{gl}(n,\mathbb{R}) $}$ =\Lie(\GL(n,\mathbb{R}))=T_I\GL(n,\mathbb{R})=M_n(\mathbb{R}) $ 
\end{example}
\begin{theorem}\label{Theorem:Lie bracket of GLn}
     $ \forall A,B\in\mathrm{gl}(n,\mathbb{R})=M_n(\mathbb{R}) $,  $ [A,B]=AB-BA $.  
\end{theorem}
\begin{remark}
    This theorem shows that the Lie bracket viewed as the Lie algebra and matrix are the same. In some sense, it means the Lie bracket defined in three sets  $ \mathrm{gl}(n,\mathbb{R})=T_I\GL(n,\mathbb
    R)=M_n(\mathbb{R}) $ can commute with those corresponding, or equivalently, are just the same. 
\end{remark}
\begin{lemma}
     $ \forall A\in \mathrm{gl}(n,\mathbb{R}) $, the left invariant vector field  $ \hat{A} $ is complete and generated the flow  $ \dps\varphi_t:\GL(n,\mathbb{R})\rightarrow \GL(n,\mathbb{R}), \varphi_t(g)=ge^{At}=g(I+At+\frac{A^2t^2}{2!}+\cdots) $   
\end{lemma}
\begin{proof}
     \[\hat{A_g}=g\cdot A\in T_gG=M_n(\mathbb{R})\]
      \[\frac{\partial}{\partial t}\varphi_t(g)=\frac{\partial}{\partial t}(g(e^{At}))=ge^{At}A=\hat{A_{g\cdot e^{At}}}=\hat{A}_{\varphi_t(g)}\]
\end{proof}
\begin{proof}[Proof of Theorem \ref{Theorem:Lie bracket of GLn}]
    Take  $ A,B\in \mathrm{gl}(n,\mathbb{R}) $. Want to show  $ [\hat{A},\hat{B}]_I=AB-BA $.
    
    Pick  $ f\in C^\infty_I(G) $, need to show  $ A(\hat{B}(f))-B(\hat{A}(f))=(AB-BA)(f) $
    
    Further Simplification: Just need to focus on  $ f=x^{ij} $, where  $ x^{ij}:\GL(n,\mathbb{R})\rightarrow \mathbb{R}, E\mapsto (E-I)_{ij} $.
    
    Such  $ f $ satisfies  $ f(I+-) $ is  $ \mathbb{R} $-linear.
    
    Recall that Given  $ W\in \mathfrak{T}M $,  $ W(f)(p)=\dps\frac{\mathrm{d}}{\mathrm{d}t}f(\varphi_t^W(p))|_{t=0} $.
    
    So  $ \hat{B}(f)(g)=\dps\frac{\mathrm{d}}{\mathrm{d}t}f(ge^{tB})|_{t=0} $.
    
    So  \[\dps A(\hat{B}(f)) =\frac{\mathrm{d}}{\mathrm{d}t}(\hat{B}(f)(e^{As}))|_{s=0}=\frac{\mathrm{d}^2}{\mathrm{d}s\mathrm{d}t}f(I+sA+tB+\frac{s^2}{2}A^2+stAB+\frac{t^2}{2}B^2+\cdots)|_{s=t=0}\]
    Similarly,
    \[\dps B(\hat{A}(f))=\frac{\mathrm{d}^2}{\mathrm{d}s\mathrm{d}t}f(I+sA+tB+\frac{s^2}{2}A^2+stBA+\frac{t^2}{2}B^2+\cdots)|_{s=t=0}\]
    So $ A(\hat{B}(f))-B(\hat{A}(f))=f(I+(AB-BA))=(AB-BA)(f) $ since  $ f $ is  $ \mathbb{R} $-linear.  
\end{proof}
Similarly, for  $ G=\GL(n,\mathbb{C}), \Lie(G)=\mathrm{gl}(n,\mathbb{C})=M_n(\mathbb{C}) $, we have  $ [A,B]=AB-BA $.
  
Actually, we have those properties of Lie group and Lie algebra.
\begin{itemize}
    \item Any simply connected Lie group are determined by its Lie algebra.
    \item Given any connected Lie group  $ G $, its universal cover  $ \hat{G} $  is simply-connected with  $ \pi^{-1}(G)\subset Z(\hat{G}) $. 
\end{itemize}
What is the meaning of Lie bracket. There is a fact about it:
\begin{fact}
     $ G $ is connected Lie group.  $ G $ is abelian iff  $ [-,-]=0 $ on  $ \Lie(G) $    
\end{fact}
\subsection{Morphisms between Lie group and Lie algebras}
A smooth map  $ F:G\rightarrow H $ between two Lie group is called a \name{morphism} if  $ F(gh)=F(g)F(h) $.

A linear map  $ L:g\rightarrow h$ between Lie algebra is called a \name{morphism} if  $ L[u,v]=[Lu,Lv] $. 
\begin{proposition}
     Let  $ F:G\rightarrow H $ be a morphism of Lie groups. Then  $ F_{e,*}:\Lie(G)\rightarrow \Lie(H) $ is a morphism of Lie algebra.  
\end{proposition}
\begin{proof}
      $ V_0,V_1\in \Lie(G)=T_eG $,  $ W_i=F_{e,*}(V_i)\in \Lie(H)=T_eH $. Let  $ \hat{V},\hat{W} $ be left-invariant vector fields.   
     \begin{claim}
          $ \hat{W_i} $ is  $ F $-compatible with  $ \hat{V_i} $ for  $ i=0,1 $.    
     \end{claim}
     \begin{proof}[Proof of Claim]
          $ \forall g\in G $,  $ F_*(\hat{V_g})=F_*(l_*^g(V))=(F\circ l^g)_*(V)=(l^{F(g)}\circ F)_*(V)=l^{F(g)}(W)=\hat{W}_{F(g)} $  
     \end{proof}
     So  $ [\hat{W_0},\hat{W_1}] $ is  $ F $-compatible with  $ [\hat{V_0},\hat{V_1}] $. In particular,  $ [W_0,W_1]=F_*([V_0,V_1]) $.  
\end{proof}


 
% !TEX root = lecture/Differential_Geometry.tex

\section{Vector Field}

\subsection{Canonical form of a Field}
Recall that  $ V\in \mathfrak{T}M $,  $ p\in M $ is called a \name{regular point} if  $ V_p\neq 0 $, and is called a \name{singular point} if  $ V_p=0 $.
\begin{theorem}[Canonical Form Theorem]\label{Canonical Form Theorem}
    Let  $ p $ be a regular point of  $ V $. Then  $ \exists $ local chart  $ (U,x^1,\cdots,x^n) $ around  $ p $ \st  $ V|_U=\partial x^1 $      
\end{theorem}    
\begin{proof}
    This is a local problem. We may assume  $ M\subset \mathbb{R}^n $ open. We may also assume  $ p=0,V_0=\partial r^1|_0$ where  $ r^i $ coordinate function.
    
    Let  $ \varphi:(-\epsilon,\epsilon)\times (-\epsilon,\epsilon)^n\rightarrow M $ be the local flow of  $ V $.
    
    Define  $ \psi:(-\epsilon,\epsilon)^n\rightarrow M $ by  $ \psi(t,r^2,\cdots,r^n)=\varphi(t,(0,r^2,\cdots,r^n)) $. Then  $ \psi(-,r^2,\cdots, r^n) $ is an integral curve for  $ V $. Therefore,  $ \psi_*(\partial t)=V $.   
    
    At  $ \vec{0} $, we have  $ \psi_{\vec{0},*}(\partial t)=V_{\vec{0}}=\partial r^1 $,  $ \psi_{\vec{0},*}(\partial r^i)=\partial r^i $.
    
    So  $ \psi_{*,\vec{0}}:T_{\vec{0}}(-\epsilon,\epsilon)^n\rightarrow T_{\vec{0}}M $ is an isomorphism.

    By the inverse function theorem,  $ \exists U'\subset (-\epsilon,\epsilon)^n $,  $ U\subset M $ \st  $ \psi|_{U'}:U'\rightarrow U $ is a diffeomorphism.
    
    Then  $ (U,(\psi|_{U'})^{-1}) $ is the local chart what we need.
\end{proof}
\begin{remark}
    Regular point in a vector field is simple, as we can view it in the standard chart locally. However, behavior of  $ V $ art a singular point can be complicated. For example,
    for  $ f(x,y)=x^2-y^2 $,  $ \nabla f=2x\partial x-2y\partial y$,  $ g:\mathbb{C}\rightarrow C,z\mapsto z^n $, they behave differently at  $ \vec{0} $.     
\end{remark}
\subsection{Lie Derivative of Vector Field}
 $ V,W\in \mathfrak{T}M $,  \name{$ \mathcal{L}_VW $} is the directional derivative of  $ W $ in the direction of  $ V $.
 \begin{definition}
    The \name{Lie derivative}  $ \mathcal{L}_VW\in \mathfrak{T}M $ is defined as follows: $ \forall p\in M $, let  $ \{\theta_t:U\rightarrow M\}_{t\in (-\epsilon,\epsilon)} $ be the local flow for  $ V $. Then  \[\dps (\mathcal{L}_VW)_p=\lim\limits_{t\to 0}\frac{(\theta_{-t})_*W_{\theta_t(p)}-W_p}{t} \]    
 \end{definition}
\begin{remark}
    This defintion is actually a difference between  $ T_{\theta_t(p)} $ and  $ T_p $, which need pullback.  
\end{remark}
 \begin{lemma}
     $ \mathcal{L}_VW $ is well-defined  and smooth.
 \end{lemma}    
\begin{proof}
    For  $ p\in M $, take local chart  $ (U,x^1,\cdots,x^n) $. Let  $ \theta :(-\epsilon,\epsilon)\times U\rightarrow M $ be the flow of  $ V $. Take  $ J_0\subset(-\epsilon,\epsilon) $,  $ U_0\subset U $. Let  $ \theta ^i=x^i\circ \theta:J_0\times U_0\rightarrow \mathbb{R} ,  $   $ \dps W|_U=\sum\limits_{i=1}^nW^i\partial x^i $.    
    
    Under the basis  $ \{\partial x^i\} $,  $ (\theta_{-t})_*:T_{\theta_t(p)}M\rightarrow T_pM $ is represented by 
    \[ \left(\dps\frac{\partial \theta^i(-t,\theta(t,x))}{\partial x^j}\right)_{i,j}\]
    
    So  $ \dps(\theta_{-t})_*W_{\theta_t(x)}=\sum\limits_{i,j}\frac{\partial \theta^i(-t,\theta(t,x))}{\partial x^j}W^j(\theta(t,x))\cdot \partial x^i $ is smooth in  $ t,x $.
    So 
    \[(\mathcal{L}_VW)_x=\dps\frac{\partial ((\theta_{-t})_*(W_{\theta_t(x)}))}{\partial t}|_{t=0}\] is well-defined and smooth.  
\end{proof}
 \begin{theorem}
    For all  $ V,W\in \mathfrak{T}M $,  $ \mathcal{L}_VW=[V,W] $.  
 \end{theorem}
\begin{proof}
    For  $ p $ is a regular point of  $ V $.  By canonical form theorem \ref{Canonical Form Theorem},  $ \exists $ local chart  $ (U,x^1,\cdots,x^n) $ around  $ p $\st  $ V|_U=\partial x^1 $. Let  $ W|_U=\dps\sum\limits_{i=1}^nW^i\partial x^i $.
    
    Then  $ \theta_t(x_1,\cdots,x_n)=(x_1+t,x_2,\cdots,x_n) $. So  \[ \mathcal{L}_VW|_U=\dps\sum\limits_i\frac{\partial W^i}{\partial x^1}\cdot \partial x^i \].  \[[V,W]|_U=\dps\sum\limits_iV(W^i)\partial x^i-\sum\limits_iW(V^i)\partial x^i=\sum\limits_i\frac{\partial W^i}{\partial x^1}\cdot\partial x^i \]
    Then  $ [V,W]|_U=\mathcal{L}_VW $.
    
    For  $ p $ is a singular point but  $ p\in \Supp(V) $. Then  $ \exists  $ $ p_i\rightarrow p $ \st  $ V_p\neq 0 $. By the previous case  $ (\mathcal{L}_VW)_{p_i}=[V,W]|_{p_i} $. By continuity,  We have  $ (\mathcal{L}_VW)_p=[V,W]_p $.
    
    For  $ p \not\in \Supp(V)$,  $ \exists $ Nbd  $ U $ of  $ p $ \st  $ V|_U=0 $. Then  $ \theta_t(q)=q $. So 
    \[(\mathcal{L}_VW)|_U=0=[V,W]|_U\]     
\end{proof}
\begin{corollary}
    \,
    \begin{itemize}
        \item  $ \mathcal{L}_VW $ is  $ \mathbb{R} $-linear with respect to  $ V,W $.
        \item  $ \mathcal{L}_VW=-\mathcal{L}_WV $.
        \item (Jacobian identity)  $ \mathcal{L}_V[W,X]=[\mathcal{L}_VW,X]+[W,\mathcal{L}_VX] $.
        \item (Jacobian identity) $ \mathcal{L}_{[V,W]}X=\mathcal{L}_V\mathcal{L}_WX-\mathcal{L}_W\mathcal{L}_VX $.
        \item  $ \mathcal{L}_V(fW)=(Vf)\cdot W+f\mathcal{L}_VW $
        \item Let  $ F:M\rightarrow N $ be a diffeomorphism. Then  $ F_*(\mathcal{L}_VW)=\mathcal{L}_{F_*(V)}F_*(W) $.           
    \end{itemize}
\end{corollary}
 \subsection{Commuting Vector Fields}
\begin{definition}
    We say  $ V,W\in\mathfrak{T}M $ \name{commutes} if  $ [V,W]=0 $.  
\end{definition}
 \begin{theorem}\label{Equivalent condition of Commuting vector fields}
    TFAE:
    \begin{enumerate}[label=\arabic*]
        \item  $ V,W $ commutes.
        \item   $ W $ is invariant under the flow generated by  $ V $,\ie  $ \theta_{t,*}(W_p)=W_{\theta_t(p)} $
        \item The flow for  $ V,W $ commutes, \ie  $ \theta_t\circ \eta_s=\eta_s\circ \theta_t $ whenever either side is defined or equivalently, whose the domain is compatible.     
    \end{enumerate}
 \end{theorem}
\begin{lemma}
     Given  $ F\in C^\infty(M,N) $,  $ V\in\mathfrak{T}M ,W\in \mathfrak{T}N$. Then  $ W $ is  $ F $-related to  $ V $ if and only  if  $ \forall t\in \mathbb{R} $,  $ \eta_t\circ F=F\circ \theta_t $ on the domain of  $ \theta_t $, which means 
    \begin{center}
        % \tikzset{external/export=false}
        \begin{tikzcd}
            M\arrow[r,"F"]\arrow[d,"\theta_t"]&N\arrow[d,"\eta_t"]\\
            M\arrow[r,"F"]&N
        \end{tikzcd}
        % \tikzset{external/export=true}
        commutes.
    \end{center}
    
      
\end{lemma}
\begin{proof}
    "$ \Rightarrow $" Let  $ \gamma=F\circ \theta^p:J\rightarrow N $ satisfies 
     \[\gamma'(t)=(F\circ \theta^p)'(t)=F_*((\theta^p)'(t))=F_*(V_{\theta^p(t)})=W_{F(\theta^p(t))}=W_{\gamma(t)}\]
     So  $ \gamma $ is an inetgral curve of  $ W $ starting at  $ \gamma(0)=F(p) $\ie  $ F\circ \theta^p =\gamma(t)=\eta^{F(p)}(t)$ \ie  $ F\circ \theta_t=\eta\circ F $.
     
     "$ \Leftarrow $" Suppose  $ F\circ \theta_t =\eta\circ F $. Then   $ (F\circ \theta^p)(t)=\eta^{F(p)}(t) $.
     
     Then  $ F_*V_p=F_*((\theta^p)'(0))=(F\circ \theta^p)'(0)=(\eta^{F(p)})'(0)=W_{F(p)} $. So  $ W $ is  $ F $-related to  $ V $.   
\end{proof}
\begin{proof}[Proof of Theorem \ref{Equivalent condition of Commuting vector fields}]
     $ 2\Rightarrow 1 $:  $ (\theta_{-t})_*(W_{\theta_t(p)})=W_p $. So  \[ \mathcal L_VW=\dps\lim\limits_{t\to 0}\frac{(\theta_{-t})_*(W_{\theta_t(p)})-W_p}{t}=0 \]
     
     
     $ 1\Rightarrow 2 $: Let  $ X(t)=(\theta_{-t})_*(W_{\theta_t(p)}) $,  $ p\in M $.
     
     Want to show that  $ X(t)=X_p $ for all  $ t $. Suffices to show  $ \dps\frac{\mathrm{d}}{\mathrm{d}t}|_{t=t_0}X(t)=0 $. 
     
     For  $ t_0=0 $,  $ \dps\frac{\mathrm{d}}{\mathrm{d}t}|_{t=0}X(t)=(\mathcal{L}_VW)_p=0 $.
     
     In general, set  $ s=t-t_0 $,  $ X(t)=(\theta_{-t_0})_*\circ (\theta_{-s})_*(W_{\theta_s(\theta_{t_0}(p))}) $.
     Then 
     \begin{align*}
        \frac{\mathrm{d}}{\mathrm{d}t}|_{t=t_0}X(t)&=\frac{\mathrm{d}}{\mathrm{d}s}|_sX(s+t_0)\\
        &=\frac{\mathrm{d}}{\mathrm{d}s}|_s(\theta_{-t_0})_*\circ (\theta_{-s})_*(W_{\theta_s(\theta_{t_0}(p))})\\
        &=(\theta_{t_0})_*\frac{\mathrm{d}}{\mathrm{d}s}|_{s=0}(\theta_{-s})_*(W_{\theta_s(\theta_{t_0}(p))})\\
        &=(\theta_{t_0})_*(\mathcal{L}_VW)_{\theta_{t_0}(p)}\\
        &=0
     \end{align*}
     
      $ 2\Rightarrow3 $. For simplicity, assume  $ V,W $ are complete.  $ F=\theta_s:M\rightarrow M $. By 2,  $ W $ is  $ F $-related to  $ W $. So by the lemma,
      \begin{center}
        % \tikzset{external/export=false}
        \begin{tikzcd}
            M\arrow[r,"F"]\arrow[d,"\theta_t"]&M\arrow[d,"\eta_t"]\\
            M\arrow[r,"F"]&M
        \end{tikzcd}
        % \tikzset{external/export=true}
        commutes.
    \end{center}  
    
    $ \eta_t $ is flow for  $ W $. \ie  $ \theta_s\circ\eta_t=\eta\circ \theta_s $ 
      
       $ 3\Rightarrow 2 $  is similar. The diagram commutes, so  $ W $ is  $ F $-related to  $ W $.   
\end{proof}
\printindex
\newpage
\listoftheorems[ignoreall, show={theorem,proposition}]
\end{document}