% !TEX root = lecture/Differential_Geometry.tex

Now we try to define differential of a smooth map.

 $ M,N  $ smooth manifolds,  $ C^\infty(N,M)=\{\text{smooth } F:N\rightarrow M\} $.
 
 Given  $ F\in C^\infty(N,M) $,  $ F  $ induces  $ F^*:C^\infty_{F(p)}(M)\rightarrow C^\infty_p(N) $, $ f\mapsto f\circ F $.
 
 By taking dual, we get 
 \[F_*:T_pN\rightarrow T_{F(p)}M\]    
 we also write  $ F_*  $ as  $ F_{*,p} $, call it the \name{differential} of  $ F $ at  $ p $. 
 
 where 
\[F_*(\frac{\partial }{\partial x^i}|_p)=\sum\limits_{k}\frac{\partial F^k}{\partial x^i}\cdot \frac{\partial }{\partial y^k}|_{F(p)}\]
\begin{proposition}
    The differential satisfies the composition law.
    \[(G\circ F)_*=G_*\circ F_*:T_pN\rightarrow T_{G\circ F(p)}W\]
\end{proposition}
\begin{definition}
    A smooth \name{curve} is a smooth map  $ \gamma:(a,b)\rightarrow M $. We say  $ \gamma  $ starts at  $ p  $ if  $ \gamma(0)=p $. We define the \subname{velocity}{curve} of  $ \gamma  $ at  $ \gamma(0) $ as  $ \gamma_*(\frac{\partial}{\partial t}|_0)\in T_{\gamma(0)}M $
    
    Take charts  $ (U,x^1,\cdots,x^n) $ about  $ p  $, let  $ \gamma^i=x^i\circ \gamma $.
    
    We say  $ \gamma, \delta $ are \subname{tangent}{curve} to each other at  $ p  $ if  $ (\gamma^i)'(0)=(\delta^i)'(0) $.       
\end{definition}
Now we can define 
\[(T_pM)_{curve}:=\{\text{smooth curves  $ \gamma  $ starting at  $ p  $ }\}/_\sim\]
where  $ \gamma\sim \delta  $ iff they are tangent to each other.
 
Then these definition is more geometric.
 
\begin{lemma}
    Given  $ F\in C^\infty(M,M) $,  $ p\in N $, the diagram commutes:
    \begin{center}
        % \tikzset{external/export=false}
        \begin{tikzcd}
            \gamma\in (T_pN)_{curve} \arrow[r,"\cong"]\arrow[d] & T_pN\arrow[d]\\
            F\circ \gamma\in (T_{F(p)}M)_{curve}\arrow[r,"\cong"] & T_{F(p)}M
        \end{tikzcd} 
        % \tikzset{external/export=true}
    \end{center}
 
\end{lemma}
\subsection{Tangent Bundle}
 Let  $ (M,\mathcal{A}) $ be a smooth manifold,  $ TM=\dps\bigcup_{p\in M}T_pM $, called the \name{tangent bundle}
 
 Now we want to define a natural topology and smooth structure on  $ TM $. Take any chart  $ (U,\varphi)=(U,x^1,\cdots,x^n)\in \mathcal{A} $.
 
 We have a map 
\begin{align*}
    \hat{\varphi}:TU&\xrightarrow{\cong}\varphi(U)\times\mathbb{R}^n\subset \mathbb{R}^n\times\mathbb{R}^n\\
    X\in T_pU&\mapsto (\varphi(p),X^1,\cdots,X^n)
\end{align*}  
 where  $ X=\sum X^i\frac{\partial}{\partial x^i}|_p $.
 
 Then pull back standard topology on  $ \varphi(U)\times\mathbb{R}^n $ to a topology on  $ TU $.
 \[\mathcal{B}=\{\hat{\varphi}^{-1}(V)|(\varphi,U)\in \mathcal{A},  V \text{ open in  $ \varphi(U)\times\mathbb{R}^n $ }\}\]
 There is some fact in topology:
\begin{itemize}
    \item  $ \mathcal{B} $ is a basis
    \item  $ \mathcal{B} $ generates a Hausdorff, second countable topology on  $ TM $. 
\end{itemize} 
 So  $ TM  $ is a topological manifold covered by charts  $ \hat{\mathcal{A}}=\{(TU,\hat{\varphi})|(U,\varphi)\in\mathcal{A}\} $.
 
Given  $ (TU,\hat{\varphi}),(TV,\hat{\psi})\in \hat{\mathcal{A}} $, the transition function is 
\begin{align}
    \varphi(U\cap V)\times\mathbb{R}^n&\xrightarrow{\hat{\psi}\circ\hat{\varphi}^{-1}}\psi(U\cap V)\times \mathbb{R}^n\\
    (p,x)&\mapsto (\psi\circ\varphi^{-1},J(\psi\circ \varphi^{-1})|_p(X))
\end{align}   
So  $ \hat{\mathcal{A}} $ is a smooth atlas on  $ TM  $, making  $ TM  $ into a smooth manifold.
\begin{definition}[vector bundle]
    Given a continuous map  $ f:E\rightarrow B $, we say  $ f $ is a  $ n $-dimensional \name{vector bundle} if: $ \exists $ an open cover  $ \mathcal{U}=\{U_\alpha\}_{\alpha\in I} $ of  $ B $ and homeomorphisms  $ \{f^{-1}(U_\alpha)\xrightarrow[\cong ]{\rho_\alpha}U_\alpha\times \mathbb{R}\} $ \st   
    % \tikzset{external/export=false}
    \begin{itemize}
        \item \begin{tikzcd}
            f^{-1}(U_\alpha)\arrow[d,"f"]\arrow[r,"\rho_\alpha"]&U_\alpha\times \mathbb{R}^n\arrow[ld,"projection"]\\
            U_\alpha
        \end{tikzcd}
        commutes for  $ \alpha\in I $.

        \item  $ \forall  $  $ p\in U_\alpha\cap U_\beta $, the map 
        \[\mathbb{R}^n=\{p\}\times\mathbb{R}^n\xrightarrow{\rho_\alpha}f^{-1}(p)\xrightarrow{\rho_\beta}\{p\}\times \mathbb{R}^n=\mathbb{R}^n\] 
        is linear.
    \end{itemize}
    % \tikzset{external/export=true}
    Call  $ f^{-1}(p)  $  the \name{fiber} over  $ p $. 
\end{definition} 
\begin{proposition}
    Given vector bundle  $ f:E\rightarrow B $, the fiber  $ f^{-1}(p) $ has a structure of a vector space.  
\end{proposition}