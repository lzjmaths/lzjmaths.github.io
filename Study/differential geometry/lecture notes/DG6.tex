% !TEX root = lecture/Differential_Geometry.tex

\begin{example}[Product bundle]
    $ E=\mathbb{R}^n\times B $ 
\end{example}
\begin{example}[Tautological bundle]
    \[B=\mathbb{CP}^n= \{\text{1-dim complex subspace of }\mathbb{C}^{n+1}\},E= \{(L,v)\in\mathbb{CP}^n\times \mathbb{C}^{n+1}\}  \]

    And we map  $ (L,v)\mapsto L $  
\end{example}
Given vector bundles  $ E_1\xrightarrow{\pi_1} B_1,E_2\xrightarrow{\pi_2} B_2 $, a bundle map consists of  $ (\hat{f},f) $ \st
% \tikzset{external/export=false}
\begin{itemize}
   \item 
   \begin{tikzcd}
       E_1\arrow[r,"\hat{f}"]\arrow[d,"\pi"] & E_2\arrow[d,"\pi"] \\
       B_1\arrow[r,"f"] & B_2
   \end{tikzcd} 
   commutes.
   \item  $ \forall b\in  B  $,  $ \hat{f}:\pi_1^{-1}(b)\rightarrow \pi_2^{-1}(f(b)) $ is linear.
\end{itemize}
% \tikzset{external/export=true}
If  $ \hat{f},f $ are diffeomorphisms, then we call  $ (\hat{f},f) $ an \subname{isomorphism}{vector bundle} of vector bundle.

An isomorphism to a product bundle is called a \name{trivialization}. An bundle is \textbf{trivial} if it has a trivialization.

\begin{example}
    $ TS^1,TS^2 $ are both trivial.
    
     $ S^1\cong O(1)\cong SO(2),S^3\cong SU(2) $ 
\end{example}
\begin{theorem}
   If  $ G  $ is a Lie group, then  $ TG  $ is trivial.
\end{theorem}
\begin{proof}
    For $(x^1,x^2,\cdots,x^n)$ is a basis of $T_eG$
    The bundle isomorphism is 
    \[G\times \mathbb{R}^n\xrightarrow{\varphi}TG,\, (g,c^1,\cdots,c^n)\mapsto (g,(l_g)_{*,e}(\sum_ic^ix^i))\] 
    where 
    \[l_g:G\rightarrow G, h\mapsto gh\]
    is a diffeomorphism. Hence, it induces the isomorphism $(l_g)_*$\\
\end{proof}
\begin{proposition}[Adams, 1960s]
    $ TS^n  $ is trivial if and only if  $ n=0,1,3,7 $. 
\end{proposition}
\begin{proposition}
    \,
   \begin{enumerate}
       \item Given any  $ F\in C^\infty(M,N) $,  $ F_*:TM\rightarrow TN $ is a bundle map.
       \item  $ TS^n $ is isomorphic to the following bundle:
        \[B=s^n\qquad E=\{(p,v)\in S^n\times \mathbb{R}^{n+1}|v\perp p\}\]   
   \end{enumerate}
\end{proposition}
\begin{definition}[smooth section]
   Given a smooth vector bundle  $ \pi:E\rightarrow B $, a \name{smooth section} is a smooth map  $ S:B\rightarrow E $ \st  $ \pi\circ S=id_b $.
   
    $ s_0:B\rightarrow E, b\mapsto 0\in\text{0-vector in  $ \pi^{-1}b $ } $. 
\end{definition}
\subsection{Vector Field, Curves and Flows}
\begin{definition}
   A (tangent) \name{vector field} is a smooth section of  $ TM $. \ie a smooth map  $ M\xrightarrow{X} TM $ \st  $ X(p)\in T_pM $,$ \forall p\in M $  
\end{definition}
Given any  $ f:\mathbb{R}^n\rightarrow \mathbb{R} $, define the \name{gradient vector field} 
\[\dps\triangledown f_p:=\sum\limits_{1 \leq i \leq n}\frac{\partial f}{\partial x^i}(p)\frac{\partial}{\partial x^i}\] 
\begin{example}
    $ X=f^1\partial x^1+f^2\partial x^2 $ is a gradient field if and only if  $ \dfrac{\partial f^1}{\partial x^2}=\dfrac{\partial f^2}{\partial x^1} $ 
\end{example}
\begin{theorem}[Poincare-Hopf]
   For closed  $ M  $,  $ M  $ has a nowhere vanishing vector field if and only if  $ \chi(M)=0 $. 
\end{theorem}
So  $ S^n $ has a  nowhere vanishing vector field if and only if  $ n  $ is odd.
\begin{theorem}[MaoQiu]
    $ S^2 $ has no no-where vanishing vector field.
    
   
\end{theorem}
So We cannot smooth out all the hairs on a ball.

   Given  a vector field  $ X=\{X_p\}_{p\in M} $, a curve  $ \gamma:(a,b)\rightarrow M $ is called an \name{integral curve} of  $ X $ if  $ \gamma'(t)=X_{\gamma(t)} $, $ \forall t\in (a,b) $, 
   where  $ \gamma'(t)=\gamma_*(\dfrac{\partial }{\partial t})\in T_{\gamma(t)}M $.
   
   We say  $ \gamma  $ is maximal if the domain cannot be extended to a larger interval.     