Denote the set of all smooth vector fields on  $ M $ by \name{$ \mathfrak{T}M $} 

Recall that   $ \gamma  $ is maximal if it's domain can not be extended to a large open interval.

In a local chart  $ (U,x^1,\cdots,x^n) $,  $ \dps X|_U=\sum_{i=1 }^{n}a^i\partial x^i $. Then  $ \gamma $ is an integral curve if and only if  $ (\gamma^i)'(t)=a^i(\gamma(t)) $,  $ \forall 1 \leq i \leq n $, where  $ \gamma^i=x^i\circ\gamma:(a,b)\rightarrow \mathbb{R} $.

And in this case the initial value condition:  $ \gamma(0)=p $ $ \Leftrightarrow $ $ \gamma^i(0)=p^i $.

Locally, solving integral curve starting at  $ p  $ is equivalent to solving ODE with initial value  $ p^1,\cdots,p^n $.
 By existence and uniqueness of solutions of ODE, we have
 \begin{theorem}[Fundamental theorem of integral curve]
    Let  $ X\in \mathfrak{T}M $,  $ p\in M $, then:
    \begin{enumerate}
        \item [(1)](Uniqueness) Given any two integral curves  $ \gamma_1,\gamma_2 :(a,b)\rightarrow M$, then we have:
        \[\gamma_1(c)=\gamma_2(c)\text{ for some  $ c\in (a,b) $ }\Rightarrow \gamma_1=\gamma_2\] 
        \item[(2)] there exists a unique max integral curve  $ \gamma:(a(p),b(p))\rightarrow M $ starting at  $ p $.  
        \item[(3)](integral curve smoothly depend on initial values)  $ \exists  $ Nbh  $ U  $ of  $ p $, $ \epsilon>0 $, and smooth  $ \varphi:(-\epsilon,\epsilon)\times U\rightarrow M $ \st  $ \forall q\in U $,  $ \varphi_\epsilon:=\varphi(-,q):(-\epsilon,\epsilon)\rightarrow M $ is an integral curve starting at  $ p $.      
    \end{enumerate} 
    we call such  $ \varphi $ a local \name{flow} generated by  $ X $. 
 \end{theorem}
 \begin{definition}
    Given  $ X\in \mathfrak{T}M $, a global \name{flow} generated by  $ X $ is a smooth map  $ \varphi:\mathbb{R}\times M\rightarrow M $ \st  $ \forall q\in M $,  $ \varphi_q:=\varphi(-,q) $ is the maximal integral curve of  $ X  $ starting at  $ q $.\\
     $ \Leftrightarrow $  $ \dps \frac{\partial \varphi}{\partial t}(s,p)=X_{\varphi(s,p)} $,  $ \forall s\in\mathbb{R}, p\in M $ and  $ \varphi(0,p)=p,\forall p\in M $.   
 \end{definition}
 If such global flow exists, then we say  $ X  $ is complete.
 \begin{example}
    \,
    \begin{itemize}
        \item  $ X=x\cdot\partial x\in\mathfrak{T}\mathbb{R} $ is complete, where global flow  $ \varphi:\mathbb{R}\times M\rightarrow M $,  $ \varphi(t,p)=p\cdot e^t $.
        \item  $ X=x^2\partial x $ is not complete. Max integral curve starting at  $ 1 $ is given by  $ \gamma(t)=\dps\frac{1}{1-t},t\in (-\infty,1)\not=\mathbb{R} $.  
    \end{itemize}
 \end{example}
 Given  $ X\in \mathfrak{T}M $, we define \name{$ \Supp X $}$=\overline{\{p\in M:X_p\not=0\}}  $.
 \begin{theorem}
    If a vector field  $ X  $ is compactly supported, then  $ X  $ is complete.\label{completeness of compact supported vector field}
 \end{theorem}  
\begin{corollary}
    Any vector field on closed  manifold is complete.
\end{corollary}
\begin{lemma}[Escaping lemma]
    Suppose  $ \gamma:(a,b)\rightarrow M $ is a max integral curve, with  $ (a,b)\not=\mathbb{R} $. Then  $ \not\exists  $ compact  $ K\subset M $ \st  $ \gamma(a,b)\subset K $    
\end{lemma}
\begin{proof}
    Otherwise, suppose  $ \gamma(a,b)\subset K $. WLOG, we may assume  $ b<+\infty $.

    Take  $ (t_i)\rightarrow b $ from left. Then  $ \gamma(t_i)\in K $. After passing to subsequence, we may assume   $ (\gamma(t_i))\rightarrow p\in K $.

    Then  $ \exists $  $ U $ Nbh of  $ p $, local flow $ \varphi:(-\epsilon,\epsilon)\times U\rightarrow M $. Take  $ n $ large enough \st  $ b-t_n<\epsilon $,  $ \gamma(t_n)\in U $. Then  $ \gamma(-+t_n):(a-t_n,b-t_n)\rightarrow M $,  $ \varphi(-,\gamma(t_n)):(-\epsilon,\epsilon)\rightarrow M $ are both integral curves for  $ X  $ starting at  $ \gamma(t_n) $. By uniqueness, they coincide.

    Let  $ \hat{\gamma}:(a,t_n+\epsilon)\rightarrow M $ be defined by  $ \dps\hat{\gamma}(t)=
        \begin{cases}
            \gamma(t),t\in (a,b)\\
            \varphi(t-t_n,\gamma(t_n)), t\in [b,t_n+\epsilon)
        \end{cases} $  

    Then  $ \hat{\gamma} $ is an integral curve with larger domain, then  $ \gamma $ contradiction with the maxity of  $ \gamma $.   
\end{proof}
\begin{proof}[Proof of \ref{completeness of compact supported vector field}]
    Take any max integral curve  $ \gamma:(a,b)\rightarrow M $. Suppose  $ (a,b)\not=\mathbb{R} $. Then  $ X_{\gamma(s)}\not=0 $, $ \forall s $. Otherwise, the constant map  $ \mathbb{R}\rightarrow M,t\mapsto \gamma(s) $ is an integral curve with lager domain.
    
    So  $ \forall s $,  $ \gamma(s)\in \Supp X $ $ \Rightarrow  $ $ \gamma(a,b)\subset \Supp X $ which is compact  $ \Rightarrow $  $ (a,b)=\mathbb{R} $ by the lemma. This causes contradiction!      
\end{proof}
A smooth  $ \varphi:\mathbb{R}\times M\rightarrow M $ is called an \name{one-parameter transformation group} if 
\begin{enumerate}
    \item[(1)]  $ \varphi_0:=\varphi(0,-)=\id_M $ 
    \item[(2)]  $ \varphi_s\circ \varphi_t=\varphi_{s+t} $ for all  $ s,t\in \mathbb{R} $. In particular,  $ \varphi_s^{-1}=\varphi_{-s} $.   
\end{enumerate} 
\begin{theorem}\label{equivalence of one-parameter transformation group}
     $ \varphi\in C^\infty(\mathbb{R}\times M,M) $, then  $ \varphi $ is an one-parameter transformation group if and only if  $ \varphi $ is the global flow generated by some  $ X\in \mathfrak{T}M $   
\end{theorem}
\begin{lemma}[Translation lemma]
    If  $ \gamma:(a,b)\rightarrow M $ is an integral curve for some  $ X\in \mathfrak{T}M $, then  $ \forall s\in \mathbb{R} $,  $ \gamma(-+s) :(a-s,b-s)\rightarrow M$ is also an integral curve for  $ X$.   
\end{lemma}
\begin{proof}
    Let $ \iota=\gamma(-+s) $. Then  $ \iota'(t)=X_{\gamma'(t+s)}=X_{\iota(t)} $  
\end{proof}
\begin{lemma}
    Let  $ \varphi:(-\epsilon,\epsilon)\times U\rightarrow M $ be a local flow for some  $ X\in \mathfrak{T}M $. Then  $ \varphi_s\circ \varphi_r(p)=\varphi_{s+r}(p) $ provided that  $ s,t,s+t\in (-\epsilon,\epsilon), p,\varphi_r(p)\in U $.    
\end{lemma}
\begin{proof}
     $ \gamma_p=\varphi(-,p) $ is an integral curve for  $ X $.
     
      $ \Rightarrow  $  $ \gamma_p(-+s) $ is an integral curve for  $ X  $ starting at  $ \gamma_p(s)=\varphi_s(p) $. But  $ \gamma_{\varphi_s(p)} $ is also an integral curve starting at  $ \varphi_s(p) $. Thus  $ \gamma_{\varphi_s(p)}=\gamma_p(-+s) $ $ \Rightarrow $  $ \varphi_r\circ \varphi_s(p)=\gamma_{r+s}(p) $     
\end{proof}
\begin{lemma}
    Let  $ \varphi:(-\epsilon,\epsilon)\times U\rightarrow M $ be a local flow for some  $ X\in \mathfrak{T}M $. Then  $ \varphi_{s,*}(X_p)=X_{\varphi_s(p)}\in T_{\varphi_s(p)}M $ \ie any vector field is invariant under its flow.  
\end{lemma}
\begin{proof}
    Take  $ f\in C^\infty_{\varphi(p)}(M) $.  
    \begin{align}
        \varphi(s,*)(X_p)(f)&=X_p(f\circ \varphi_s)\\
        &=\frac{\mathrm{d}}{\mathrm{d}t}(f\circ\varphi_s(\varphi_t(p)))|_{t=0}\\
        &=\frac{\mathrm{d}}{\mathrm{d}t}(f\circ\varphi_t(\varphi_s(p)))|_{t=0}\\
        &=X_{\varphi_s(p)}(f)
    \end{align} 
\end{proof}
\begin{proof}[Proof of \ref{equivalence of one-parameter transformation group}]
    "$ \Leftarrow $" is because the lemma  $ \varphi_s\circ \varphi_r=\varphi_{s+r} $ 

     "$ \Rightarrow $" Let  $ X=\{X_p\} $  where  $ X_p=\dps\frac{\partial\varphi}{\partial t}|_{(0,p)} $.
     
     Leave it as an exercise.
\end{proof}

\subname{Time dependent}{vector field} vector field is a smooth map  $ X:\mathbb{R}\times M\rightarrow TM $ \st  $ X_{(t,p)}\in T_pM $.

A smooth curve  $ \gamma(a,b)\rightarrow M $ is the \name{integral curve} for  $ X $ if  $ \gamma'(t)=X_{(t,\gamma(t))} $.

In local chart, solving  $ \gamma $ is still solving ODE, so most results still holds for time dependent vector field. Those are some properties:
\begin{itemize}
    \item Uniqueness:  $ \gamma_1,\gamma_2 $ are both integral curves for  $ X  $,  $ \gamma_1(c)=\gamma_2(c)\Rightarrow \gamma_1\equiv\gamma_2 $  
    \item Max integral curve exists and is unique.
    \item Local flow exists.
\end{itemize}

Now define \name{$ \Supp X $}$ =\overline{\{p\in M: X_{t,p}\not=0\text{ for some }t\}} $.

Then  $ X  $ is compactly supported, then  $ X $ is complete(\ie a global flow  $ \varphi:\mathbb{R}\times M \rightarrow M$)

But something is not true for time dependent vector field:
\begin{itemize}
    \item translation lemma is not true.
    \item vector field change under its flow.
    \item global flow can not implies one-parameter transformation group.
\end{itemize}
\subsection{Another definition of vector field}
A derivation on   $ M  $ is a  $ \mathbb{R} $-linear map  $ C^\infty(M)\xrightarrow{D}C^\infty(M) $ that satisfies the Leibniz rule:
\[D(f\cdot g)=Df\cdot g+f\cdot Dg\] 
\begin{theorem}

    We have a bijection:
    \begin{align*}
        \rho:\mathfrak{T}M&\xrightarrow{1:1}\{\text{derivation on }M\}\\
        X&\mapsto D_X:f\mapsto X(f)
    \end{align*}
\end{theorem} 
