\begin{lemma}\label{Lemma A}
    $D_p:\mathfrak{T}_p{M} \rightarrow \mathbb{R}$-linear map $\mathbb{C}^\infty(M)\rightarrow\mathbb{R} \st D_p(f\cdot g)=D_p(f)\cdot g(p)+f(p)\cdot D_p(g)$ is an isomorphism of vector spaces.
\end{lemma}
\begin{proof}

     Leave it as an exercise.
\end{proof}
\begin{lemma}\label{Lemma B}
    Given a vector field(not necessarily smooth) $X=\left\{X_p\right\}_{p\in M}$ , X is smooth $\Leftrightarrow$ $\forall f \in C^\infty(M),X(f)$ is smooth.
\end{lemma}
\begin{proof}
    "$ \Leftarrow $" $\forall p\in M$, take chart $(U,x^1,x^2,\cdots,x^n)$ around $p$. $X|_U=\sum_{i=1}^{n}{f^i\frac{\partial}{\partial x^i}}$  $f^i:U\rightarrow \mathbb{R}$, where $f^i=X|_U(x^i)$.
    Take $\varphi:M\rightarrow [0,1] \st \varphi \equiv 1 $ near $p$, $\Supp \varphi\subset U$,$\varphi \cdot x^i\in C^\infty(M)$.
    
    Then $X(\varphi \cdot x^i)=f^i$ near $p$. By assumption, $f^i$ is smooth near $p$. So $f^i$ is smooth, so $X$ is smooth.
    
     "$ \Rightarrow $" Similar.
\end{proof}
\begin{theorem}
    The map $\rho:\mathfrak{T}M\rightarrow \{\text{derivation on }M\},X\mapsto (D_x:f\mapsto X(f))$ is well-defined and bijective.
\end{theorem}
\begin{proof}
     $ \rho $ is well-defined: $ X(f)\in C^\infty(M) $ by Lemma \ref{Lemma B}, and  $ D_x(fg)=D_x(f)g+fD_x(g) $ since  $ X $ is a point-derivation.
     
      $ \rho $ is injective: $ D_x=D_y\Rightarrow $ $ D_{X_p}=D_{Y_p} $ as maps  $ C^\infty(M) $ to  $ \mathbb{R} $. By Lemma \ref{Lemma A}, we have  $ X_p=Y_p,\,\forall p $. So  $ X=Y $.  
      
       $ \rho $ is surjective: Given  $ D:C^\infty(M)\rightarrow C^\infty(M) $. Define  $ D_p:C^\infty(M)\rightarrow \mathbb{R} $ by  $ D_p(f):=D(f)(p) $ satisfies the Leibniz rule. By Lemma \ref{Lemma A},  $ D_p=D_{X_p} $ for some  $ X_p\in T_pM $. Define  $ X=\{X_p\}_{p\in M} $. Then  $ X(f)=D(f) ,\,\forall f\in C^\infty(M)$. By Lemma\ref{B},  $ X $ is a smooth vector field.    
\end{proof}
\subsection{Lie bracket}
In this section, we can actually find those identification:
\begin{align*}
    \{\text{Tangent vector at $ p $}\}&=\{\text{point derivation at $ p $}\}\\
    &=\{\text{$ \mathbb{R} $-linear maps  $ C^\infty_p(M)\xrightarrow{D_p} \mathbb{R} $ \st}\\&D_p(fg)=D_p(f)g(p)+f(p)D_p(g)\}
\end{align*}
\begin{align*}
    \{\text{smooth vector fields}\}&=\{\text{smooth sections of  $ TM $}\}\\
    &=\{\text{derivation on  $ M $}\}
\end{align*}
\begin{notation}
    We will identify  $ X\in \mathfrak{T}M$  with its derivation  $ D_x:C^\infty(M)\rightarrow C^\infty(M) $. So a vector field is just a  $ \mathbb{R} $-linear map  $ X:C^\infty(M)\rightarrow C^\infty(M) $ \st  $ X(fg)=fX(g)+X(f)g $.    
\end{notation}
\begin{definition}[Lie bracket]
    Given two (smooth) vector field  $ X,Y:C^\infty(M)\rightarrow C^\infty(M) $, we define the \name{Lie bracket}
    \[[X,Y]=X\circ Y-Y\circ X:C^\infty(M)\rightarrow C^\infty(M)\] 
\end{definition}
\begin{theorem}
    For any  $ X,y\in\mathfrak{T}M $,  $ [X,Y]\in \mathfrak{T}M $  
\end{theorem}
\begin{proof}
    Easy to check that  $ [X,Y] $ is linear.
    
    By Leibuniz rule,
    \begin{align*}
        [X,Y](fg)&=X\circ Y(fg)-Y\circ X(fg)\\
        &=X(Yf\cdot g+f\cdot Yg)-Y(Xf\cdot g+f\cdot Xg)\\
        &=(X\cdot Y)(f)\cdot g+f\cdot(X\circ Y)(g)-(Y\cdot X)(g)-f\cdot ((Y\circ X)(g))\\
        &=[X,Y](f)\cdot f\cdot [X,Y](g)
    \end{align*}
\end{proof}
So What is the geometric meaning of  $ [X,Y] $? Non commutatiy of flows.
\begin{fact}
    Given  $ X,Y\in \mathfrak{T}M $, we say  $ X,Y $ are commutative vector field if  $ [X,Y]=0 $
    
    $ X,Y $ are commuative iff  for any local flows  $ \varphi^X:(-\epsilon,\epsilon)\times U\rightarrow M $, $ \varphi^Y:(-\epsilon,\epsilon)\times U\rightarrow M $  we have $ \varphi_s^X\circ \varphi_t^T=\varphi_t^Y\circ\varphi_s^X $   
\end{fact} 


\begin{proposition}[Calculation of    { $ \left[V,W\right] $ }    using local charts]
    Chart  $ (U,x^1,\cdots,x^n) $,  $ V,W\in \mathfrak{T}M $,  $ V|_U=\dps\sum_{i=1}^n V^i\frac{\partial }{\partial  x^i} $,  $ W|_U=\dps\sum_{i=1}^nW^i\frac{\partial}{\partial x^i} $. Then 
    \begin{align*}
        [V,W]|_U&=\sum_{i=1}^n(V(W^i-W(V^i)))\frac{\partial}{\partial x^i}\\
        &=\sum_{i=1}^n(\sum_{j=1}^nV^j\frac{\partial W^i}{\partial x^j}-W^j\frac{\partial V^i}{\partial X^j})\frac{\partial }{\partial x^j}\\
        &=\sum_{1 \leq i,j \leq n}(V^j\frac{\partial W^i}{\partial x^j}-W^j\frac{\partial V^i}{\partial x^j})\frac{\partial}{\partial x^j}
    \end{align*}
\end{proposition}
\begin{example}
     $ V=x\partial x+y\partial y $,  $ W=-y\partial x+x\partial y $ commutes.  
\end{example}
\begin{proposition}[Properties of Lie bracket]
    \,
    \begin{itemize}
        \item Natuality under push-forword.
        
        Given any  $ F\in \Diff(M,N) $,  $ V\in \mathfrak{T}M, W\in \mathfrak{T}M  $, we have  $ [F_*V,F_*W]=F_*[V,W] $.
        \item  $ \mathbb{R} $-linearity  $ \forall a,b\in \mathbb{R} $
        \begin{align*}
            [aX+bV,W]&=a[X,W]+b[V,W]\\
            [W,aX+bV]&=b[W,X]+a[W,V]
        \end{align*}  
        
        \item anti-symmetric  $ [V,W]=-[W,V] $
        \item Jacobi identity 
        \[[V,[W,X]]+[W,[X,V]]+[X,[V,W]]=0\]     
        \item Leibuniz rule 
        \[[fV,gW]=fg[V,W]+(f\cdot Vg)W-(g\cdot Wf)V\]
    \end{itemize}
\end{proposition}
\begin{definition}
    Given  $ F\in C^\infty(M,N) $,  $ V\in \mathfrak{T}M $,  $ W\in \mathfrak{T}N $. We say  $ W $ is  \name{$ F $-related} to  $ V $ if  $ \forall p\in M $,  $ F_{p,*}(V_p)=W_{F(p)}, F_{p,X}:T_pM\rightarrow T_{f(p)}N $      
\end{definition}
\begin{example}
     $ F:S^1\rightarrow \mathbb{R}^2,\theta\mapsto (\cos \theta,\sin \theta) $,  $ V=\partial \theta,W=-y\partial x+x\partial y $. 
\end{example}
\begin{note}
    In general, given  $ V\in\mathfrak{T}M $ and  $ F\in C^\infty(M,N) $. There may not exist  $ W\in \mathfrak{T}M $ \st  $ V,W $ are  $ F $-related. Even such  $ W  $ exists, it may not be unique.
    
    However, if  $ F $ is a diffeomorphism, given any  $ V $,  $ \exists $ unique  $ W $ \st  $ V $ and  $ W $ are  $ F $-related. Actually,  $ W_p=F_*V_{F^{-1}(p)} $.     
    
    Such  $ W $ is called \name{push forward} of  $ V $ along  $ F $, denoted by  $ F_*V $, only defined when  $ F $ is a diffeomorphism.     
\end{note}
\begin{lemma}
     $ \forall V\in\mathfrak{T}M, W\in \mathfrak{T}N $,  $ F\in C^\infty (M,N) $. Then  $ W $ is  $ F $-related to  $ V $ iff  $ \forall  f\in C^\infty N $,  $ V(f\circ F)=W(f)\circ F\in C^\infty (M) $     
\end{lemma}
\begin{proof}
    Check that  $ F_{p,*}(V_p)(f)=W_{F(p)}(f) $,  $ \forall f\in C^\infty(N) $
\end{proof}
\begin{proposition}
    Given  $ V_0,V_1\in \mathfrak{T}M $,  $ W_0,W_1\in \mathfrak{T}N $,  $ F\in C^\infty(M,N) $,  $ W_i $ is  $ F $-related to  $ V_i $, $ i=0,1 $  $ \Rightarrow  $ $ [W_0,W_1] $  is  $ F $-related to  $ [V_0,V_1] $         
\end{proposition}
\begin{proof}
    
\end{proof}
\begin{corollary}[Natuality of Lie bracket]
    Given any  $ F\in \Diff(M,N) $,  $ V\in \mathfrak{T}M, W\in \mathfrak{T}M  $, we have  $ [F_*V,F_*W]=F_*[V,W] $   
\end{corollary}