Let  $ d_m=c_m  $ in  $ (\square) $,we get
\begin{equation}
    \int_{a }^{b } |S_n|^2\, \mathrm{d}x=\sum\limits_{m=1 }^{n } |c_m|^2= \int_{a}^{b } |f|^2\, \mathrm{d}x -\int_{a }^{b } |f-S_n|^2 \, \mathrm{d}x  \leqslant \int_{a }^{b } |f|^2\, \mathrm{d}x\tag{ $ \square\square $ }     
\end{equation} 
Setting  $ n\rightarrow \infty  $ in the last inequality, we obtain
\begin{theorem}[Bessel's inequality]
    If  $ \{\phi_n \} $ is orthonormal on  $ [a,b] $ and  $ f\in \mathscr{R } $ on $ [a,b]  $, and if 
    \[f(x)\sim \sum\limits_{n=1 }^{\infty} c_n\phi_n(x)\]
    Then \[\sum\limits_{n=1}^{\infty}|c_n|^2  \leqslant \int_{a }^{b } |f(x)|^2\, \mathrm{d}x  \] 
    In particular,  $ \lim\limits_{n\to\infty } c_n=0  $. 
\end{theorem}
For the rest of the section, we only consider the trigonometric system. For  $ f\in\mathscr{R } $ on  $ [-\pi,\pi] $ and has period  $ 2\pi  $. 
Then the orthonormal system is  $ \{\dfrac{1}{\sqrt{2\pi}}e^{inx}\}_{n\in\mathbb{Z}} $.\\
  \begin{equation}
     (\square\square)\Rightarrow \dfrac{1 }{2\pi }\int_{-\pi }^{\pi } |S_n|^2\, \mathrm{d}x=\sum\limits_{m=-n }^{n } |c_n|^2 \leqslant \dfrac{1 }{2\pi }\int_{-\pi }^{\pi } |f|^2\, \mathrm{d}x\tag{$ \triangle $}
  \end{equation} 
  We define the \underline{Dirichlet Kernel}
  \begin{equation}
    D_N(x):=\sum\limits_{n=-N }^{N } e^{inx }=\frac{e^{-iNx }-e^{i(N+1)x}}{1-e^ix}=\frac{\sin{(N+\frac{1 }{2 })x}}{\sin{\frac{x }{2}}} \notag
  \end{equation}
  Then 
  \begin{align*}
    S_N(x)&=\sum\limits_{n=-N }^{N } \dfrac{1 }{2\pi }\int_{-\pi }^{\pi } f(t)e^{-int}\, \mathrm{d}t e^{inx} \\
    &=\dfrac{1 }{2\pi }\int_{-\pi }^{\pi } f(t)D_N(x-t)\, \mathrm{d}t \\
    &=\dfrac{1 }{2\pi }\int_{-\pi }^{\pi } f(x-t)D_N(t)\, \mathrm{d}t \tag{$ \square\square\square $}   
  \end{align*}
  \begin{theorem}
    If,  for some x,  $ \exists\delta>0  $ and  $ M<\infty  $ s.t.
    \[|f(x+t)-f(x)| \leqslant M|t|,\,\forall t \in(-\delta,\delta)\]
    then  $ \lim\limits_{N\to\infty } S_N(x)=f(x)  $. 
  \end{theorem}
  \begin{proof}
    Define 
    \[g(t):=\left\{
        \begin{aligned}
            \frac{f(x-t)-f(t)}{\sin{\frac{t }{2 }}},&\quad0<|t| \leqslant \pi\\
            0\qquad\qquad\qquad,&\quad t=0
        \end{aligned}
    \right.\]
    By the definition of  $ D_N $,  $ \dfrac{1 }{2\pi}\int_{-\pi }^{\pi } D_N(t)\, \mathrm{d}t=1   $  
    \begin{align*}
        (\square\square\square)\Rightarrow 2\pi(S_N(x)-f(x))&=\int_{-\pi }^{\pi } [f(x-t)-f(x)]D_N(t)\, \mathrm{d}t\\
        &=\int_{-\pi}^{\pi }g(t)\sin[(N+\frac{1 }{2})t]\, \mathrm{d}t\\
        &=\int_{-\pi }^{\pi  }[g(t)\cos{\frac{t}{2}}]\sin{Nt} \, \mathrm{d}t+\int_{-\pi }^{\pi }[g(t)\sin{\frac{t}{2}}]\cos{Nt} \, \mathrm{d}t        
    \end{align*}
    \[|f(x+t)-f(x)| \leqslant M|t|\Rightarrow \limsup\limits_{t\to0}|g(t)| \leqslant \limsup\limits_{t\to0}\frac{M|t|}{|\sin{\frac{t }{2 }}|}=2M\]
    \[\Rightarrow g(t)\cos(\frac{t }{2})\quad and \quad g(t)\sin(\frac{ t}{2})\in \mathscr{R }\quad
    on\quad [-\pi, \pi]\]
    Bessel'inequality \[ \Rightarrow \lim\limits_{N\in\infty} \int_{-\pi }^{\pi } h(t)\sin(Nt)\, \mathrm{d}t=\lim\limits_{N\to\infty } \int_{-\pi}^{\pi } h(t)\cos(N t)\, \mathrm{d}t=0,\,\forall h\in\mathscr{R }\,on\,[-\pi,\pi]       \]
     $ \Rightarrow \lim\limits_{N\to\infty } S_N(x)=f(x)  $  
  \end{proof}
  \begin{corollary}
    \begin{enumerate}
        \item[$ (1) $] If $ f(x)=0  $ for  $ \forall x\in(a,b) $, then  $ \lim\limits_{N\to\infty } S_N(x)=0   $ for  $ \forall x\in(a,b) $.
        \item[$ (2) $] If  $ f(t)=g(t) $ for  $ \forall t $ in some NBHD of x, then  $ S_N(f;x)-S_N(g;x)=S_N(f-g;x)\rightarrow0  $ as  $ N\rightarrow\infty $     
    \end{enumerate}
  \end{corollary}
  \begin{theorem}
    If  $ f  $ is continuous (with period  $ 2\pi $) and if  $ \epsilon>0 $, then there is a trigonometric polynomial P s.t. $ |P(x)-f(x)|<\epsilon  $ for $ \forall x\in\mathbb{R } $.  
  \end{theorem}
  The proof is given by homework.
  \begin{theorem}[Parseval's theorem]
    Suppose  $ f  $ and  $ g  $ are Riemann-integrable on  $ [-\pi,\pi] $ with period  $ 2\pi  $, and 
    \[f(x)\sim \sum\limits_{n=-\infty}^{\infty}c_ne^{inx},\qquad g(x)\sim \sum\limits_{n=-\infty }^{\infty} d_ne^{inx} \]
    Then 
    \begin{enumerate}[a)]
        \item  $ \lim\limits_{N\to\infty} \dfrac{1}{2\pi }\int_{-\pi }^{\pi } |f(x)-S_N(f;x)|\, \mathrm{d}x=0    $,
        \item  $ \dfrac{1 }{2\pi }\int_{-\pi }^{\pi } f(x)\overline{g(x)}\, \mathrm{d}x=\sum\limits_{n=-\infty}^{\infty} c_n\overline{d_n}   , $ 
        \item  $ \dfrac{1 }{2\pi }\int_{-\pi }^{\pi } |f(x)|^2\, \mathrm{d}x =\sum\limits_{n=-\infty}^{\infty } |c_n|^2  $ 
    \end{enumerate}
\end{theorem}
    \begin{proof}
        Fix  $ \epsilon>0 $, EX5 of HW7 $ \Rightarrow\exists  $ a continuous  $ 2\pi $-periodic function  $ h  $ s.t.
        \[||f-h||_2:=[\int_{-\pi }^{\pi } |f(x)-h(x)|^2\, \mathrm{d}x  ]<\epsilon\] 
        The previous theorem  $ \Rightarrow \exists  $ trigonometric polynomial  $ P  $ s.t. $ |h(x)-P(x)|<\frac{\epsilon}{\sqrt{2\pi }},\,\forall x\in\mathbb{R } $\\
         $ \Rightarrow||h-P||_2<\epsilon $\\
         Suppose  $ P  $ has degree  $ N_0  $, the 1st theorem in this section  $ \Rightarrow ||h-S_N(h)||_2 \leqslant ||h-P||_2<\epsilon,\,\forall N\geqslant N_0 $.\\
          \begin{align*}
            (\triangle)&\Rightarrow ||S_N(h)-S_N(f)||_2=||S_N(h-f)||_2 \leqslant ||h-f||_2<\epsilon\\
             &\Rightarrow ||f-S_N(f)||_2 \leqslant ||f-h||_2+||h-S_N(h)||_2+||S_N(h)-S_N(f)||_2<3\epsilon,\,\forall N\geqslant N_0\\
             &\Rightarrow \lim\limits_{N\to\infty} \dfrac{1 }{2\pi}\int_{-\pi }^{\pi } |f-S_N(f)|^2\, \mathrm{d}x=0 \tag{ $ \star  $ }   
          \end{align*} 
          \[
            \dfrac{1 }{2\pi }\int_{-\pi }^{\pi } S_N(f)\overline{g }\, \mathrm{d}x=\sum\limits_{n=-N }^{N } c_n\dfrac{1 }{2\pi }\int_{-\pi }^{\pi } e^{inx }\overline{g(x)} \, \mathrm{d}x=\sum\limits_{n=-N }^{N } c_n\overline{d_n}   
          \]   
          \begin{align*}
            \Rightarrow \left|\int_{-\pi }^{\pi } f\overline{g }\, \mathrm{d}x- \int_{-\pi }^{\pi } S_N(f)\overline{g }\, \mathrm{d}x\right| & \leqslant \int_{-\pi }^{\pi } |f-S_N(f)|\cdot|g|\, \mathrm{d}x \\
            &\overset{C-S}{ \leqslant } [\int_{-\pi }^{\pi } |f-S_N(f)|^2\, \mathrm{d}x \cdot \int_{-\pi }^{\pi } |g|^2\, \mathrm{d}x  ]^{\frac{1 }{2}}\\
            &\rightarrow0\,\text{ as  $ N\to\infty  $  by ($ \star $)}
          \end{align*}
          \[
            \Rightarrow \dfrac{1 }{2\pi }\int_{-\pi }^{\pi } f(x)\overline{g(x)}\, \mathrm{d}x=\dfrac{1 }{2\pi }\lim\limits_{N\to\infty} \int_{-\pi }^{\pi } S_N(f)\overline{g }\, \mathrm{d}x =\sum\limits_{n=-\infty }^{\infty} c_n\overline{d_n }.\]
            Setting  $ f=g $, we get  $ \dfrac{1}{2\pi }\int_{-\pi }^{\pi } |f|^2\, \mathrm{d}x=\sum\limits_{n=-\infty }^{\infty} |c_n|^2   $       
    \end{proof}
  \subsection{The Gamma Function}
  \[\Gamma(x):=\int_{0}^{\infty} t^{x-1}e^{-t}\, \mathrm{d}t,\,x\in(0,\infty)  \]
  Note that the integral converges for  $ x\in(0,\infty) $.
  \begin{theorem}
    \begin{enumerate}
        \item[$ (a) $]  $ \Gamma(x+1)=x\Gamma(x),\,x\in(0,\infty) $
        \item[$ (b) $]  $ \Gamma(n+1)=n!,\,n\in\mathbb{N } $
        \item[$ (c) $]  $ \ln\Gamma $ is convex on  $ (0,\infty) $.    
    \end{enumerate}
  \end{theorem} 
  \begin{proof}
    We only prove (c) \\
    Let  $ p\in(1,\infty ) $ and  $ \frac{1 }{p }+\frac{1 }{q }=1 $.
    Then
    \begin{align*}
        \Gamma(\frac{x }{p }+\frac{y }{q })&=\int_{0}^{\infty} t^{\frac{x }{p }+\frac{y }{q }-1}e^{-t}\, \mathrm{d}t \\
        &=\int_{0}^{\infty} (t^{\frac{x }{p }-\frac{1 }{p }}e^{-\frac{t }{p }})(t^{\frac{y }{q }-\frac{1 }{q }}e^{-\frac{t }{q }})\, \mathrm{d}x \\
         &\leqslant [\Gamma(x)]^{\frac{1 }{p }}[\Gamma(y)]^{\frac{1 }{q }}\,\,\,(\text{Holder's inequality})\\
         &\Rightarrow  \text{$ \ln\Gamma $ is convex on  $ (0,\infty) $.}
    \end{align*}
  \end{proof}
  \begin{theorem}
    If  $ f:(0,\infty)\rightarrow(0,\infty) $ satisfies:
    \begin{enumerate}[(a)]
        \item  $ f(x+1)=xf(x),\,\forall x\in (0,\infty) $ 
        \item f(1)=1
        \item  $ \ln{f} $ is convex on  $ (0,\infty) $  
    \end{enumerate} 
    then  $ f(x)=\Gamma(x) $ 
  \end{theorem}
  \begin{proof}
     $ \Gamma  $ satisfies (a), (b) and (c). So it is enough to prove that  $ f(x) $ is unique determined by (a), (b) and (c) for  $ \forall x>0 $.\\
     Actually, it's enough to prove this for  $ \forall x\in(0,1) $ as we use (a) and (b).
     Let  $ \varphi(x)=\ln{f(x)},\,x>0 $
     \begin{align*}
        \varphi \text{ convex }&\Rightarrow \ln{n }=\frac{\varphi(n+1)-\varphi(n)}{(n+1)-n} \leqslant 
        \frac{\varphi(n+1+x)-\varphi(n+1)}{(n+1+x)-(n+1)} \leqslant \frac{\varphi(n+2)-\varphi(n+1)}{(n+2)-(n+1)}=\ln{(n+1)}\\
        &\Rightarrow \varphi(x)=\lim\limits_{n\to\infty } \ln[\frac{n!n^x }{x(x+1)\cdots(x+n)}]\\
        &\Rightarrow\varphi(x) \text{ is unique determined on  $ (0,1) $ } 
     \end{align*}   
  \end{proof}
  \begin{corollary}
     $ \Gamma(x)=\lim\limits_{n\to\infty} \ln[\frac{n!n^x }{x(x+1)\cdots(x+n)}],\,x>0 $ 
  \end{corollary}