\section{Positive linear functionals and Radon measures}
\setcounter{subsection}{3}
\subsection{Regularity and Lusin's theorem}
\setcounter{theorem}{28}
\begin{corollary}
    \label{equivalence of measure sets in Radon measure}
    Let  $ E\subset X $ such that  $ E $ is contained in an open set with finite measure. Then the following are equivalent:
    \begin{enumerate}
        \item  $ E\in \mathfrak{M}  $ 
        \item For every $ \epsilon>0  $ there exists a compact set  $ K\subset X  $ and an open set  $ U\subset X  $ such that  $ K\subset E\subset U $ and  $ \mu(U\backslash K)<\epsilon $ 
        \item[2'] There exist a  $ \sigma $-compact set  $ A\subset X $ and a  $ G_\delta $  set  $ BA\subset X $ such that  $ A\subset E\subset B $ and $ \mu(B \backslash A)<\epsilon $ 
    \end{enumerate}
\end{corollary}
\subsection{Regularity beyond finite measures}
\setcounter{theorem}{36}
\begin{theorem}
     $ \mu  $ is  $ \sigma $-finite. Then  $ \forall E\subset X  $,  $ E\in \mathfrak{M } $  $ \Leftrightarrow  $  $ \forall   \epsilon,   \exists U \supset E  $ open in  $ X  $,  $ F\subset E  $ closed, such that  $ \mu(U \backslash F)<\epsilon $ 
\end{theorem}
\begin{theorem}
    Let  $ X  $ be second countable LCH.  $ \mu:\mathfrak{B}_X \rightarrow [0,+\infty]$ s.t. $ \mu(K)<+\infty,\forall K  $ compact. Then  $ \mu  $ is a Radon measure. 
\end{theorem}
\begin{proof}
    We need to prove a lemma:
    \begin{lemma}
    $ \mu  $ is inner regular on open sets.
    \end{lemma}
    The first proof is given by RM theorem.\\
    The second proof use the Cor \ref{equivalence of measure sets in Radon measure}\\
    The third proof shows that the regular sets are regular.
\end{proof}
\begin{theorem}
    If  $ f\in C([a,b],\mathbb{R }) $. Then  $ f  $ is Stieltjes integrable.  $ I_p:C([a,b],\mathbb{R })\rightarrow \mathbb{R },I_p(f)=f(a)\rho (a)+\int_{a }^{b }f \mathrm{d}\rho $  
\end{theorem}
\begin{lemma}
    Let  $ \rho:[a,b]\rightarrow \mathbb{R }_{ \geq 0 } $ be increasing. Assume that  $ a \leq c<d \leq b  $. Let  $ f\in C([a,b],[0,1]) $ s.t.  $ f|_{[a,c]}=1, f|_{[d,b]}=0  $. Then  $ \rho(c) \leq I_p(f) \leq\rho(d) $ 
\end{lemma}
\begin{theorem}[Riesz Representation Theorem]\label{Riesz Representation Theorem}
    \index{Riesz Representation Theorem}
    We have a bijection  $ \rho \mapsto I_\rho $ between increasing right continuous  $ \rho:[a,b]\rightarrow \mathbb{R }_{ \geq 0 } $ and positive linear functionals  $ \Lambda:C([a,b],\mathbb{R})\rightarrow \mathbb{R } $  
\end{theorem}
