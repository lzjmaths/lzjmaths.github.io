\documentclass{article}
\usepackage{amsthm}
\usepackage{amssymb}
\usepackage{enumerate}
\usepackage{amsmath}
\usepackage{extarrows}
\usepackage{mathrsfs}
\usepackage[UseMSWordMultipleLineSpacing,MSWordLineSpacingMultiple=1.4]{zhlineskip}
\usepackage{lipsum}
\usepackage{imakeidx}
\usepackage[linktoc=page]{hyperref}
\makeindex[columns=2,title=Index, options=-s example_style.ist]

\hypersetup{
    colorlinks=true, % 设置链接颜色
    linkcolor=blue, % 设置普通链接颜色
    citecolor=green, % 设置引用链接颜色
    urlcolor=red % 设置URL颜色
}
\title{Analysis2 Note}
\author{lin150117 }
\date{} 
\theoremstyle{definition} 
\newtheorem{definition}{Definition}[subsection]
\newtheorem{theorem}{Theorem}[section]
\newtheorem{example}[theorem]{Example}
\newtheorem{remark}[theorem]{Remark}
\newtheorem{corollary}[theorem]{Corollary}
\newtheorem{proposition}[theorem]{Proposition}
\newtheorem{lemma}[theorem]{Lemma}
\newtheorem{fact}[theorem]{Fact}
\DeclareMathOperator{\sign}{sign}
\DeclareMathOperator{\dom}{dom}
\DeclareMathOperator{\ran}{ran}
\DeclareMathOperator{\ord}{ord}
\DeclareMathOperator{\rank}{rank}
\DeclareMathOperator{\Span}{span}
\DeclareMathOperator{\img}{Im}
\DeclareMathOperator{\dd}{d\!}
\DeclareMathOperator{\Jac}{Jac}
\newcommand{\card}{\texttt{\#}}
\newcommand{\ie}{\emph{i.e. }}
\newcommand{\st}{\emph{s.t. }}

\usepackage{geometry}
\geometry{left=3.18cm, right=3.18cm, top=2.54cm, bottom=2.54cm}
\begin{document}
\maketitle
\tableofcontents
\setcounter{section}{24}
\section{Positive linear functionals and Radon measures}
\setcounter{subsection}{3}
\subsection{Regularity and Lusin's theorem}
\setcounter{theorem}{28}
\begin{corollary}
    \label{equivalence of measure sets in Radon measure}
    Let  $ E\subset X $ such that  $ E $ is contained in an open set with finite measure. Then the following are equivalent:
    \begin{enumerate}
        \item  $ E\in \mathfrak{M}  $ 
        \item For every $ \epsilon>0  $ there exists a compact set  $ K\subset X  $ and an open set  $ U\subset X  $ such that  $ K\subset E\subset U $ and  $ \mu(U\backslash K)<\epsilon $ 
        \item[2'] There exist a  $ \sigma $-compact set  $ A\subset X $ and a  $ G_\delta $  set  $ BA\subset X $ such that  $ A\subset E\subset B $ and $ \mu(B \backslash A)<\epsilon $ 
    \end{enumerate}
\end{corollary}
\subsection{Regularity beyond finite measures}
\setcounter{theorem}{36}
\begin{theorem}
     $ \mu  $ is  $ \sigma $-finite. Then  $ \forall E\subset X  $,  $ E\in \mathfrak{M } $  $ \Leftrightarrow  $  $ \forall   \epsilon,   \exists U \supset E  $ open in  $ X  $,  $ F\subset E  $ closed, such that  $ \mu(U \backslash F)<\epsilon $ 
\end{theorem}
\begin{theorem}
    Let  $ X  $ be second countable LCH.  $ \mu:\mathfrak{B}_X \rightarrow [0,+\infty]$ s.t. $ \mu(K)<+\infty,\forall K  $ compact. Then  $ \mu  $ is a Radon measure. 
\end{theorem}
\begin{proof}
    We need to prove a lemma:
    \begin{lemma}
    $ \mu  $ is inner regular on open sets.
    \end{lemma}
    The first proof is given by RM theorem.\\
    The second proof use the Cor \ref{equivalence of measure sets in Radon measure}\\
    The third proof shows that the regular sets are regular.
\end{proof}
\begin{theorem}
    If  $ f\in C([a,b],\mathbb{R }) $. Then  $ f  $ is Stieltjes integrable.  $ I_p:C([a,b],\mathbb{R })\rightarrow \mathbb{R },I_p(f)=f(a)\rho (a)+\int_{a }^{b }f \mathrm{d}\rho $  
\end{theorem}
\begin{lemma}
    Let  $ \rho:[a,b]\rightarrow \mathbb{R }_{ \geq 0 } $ be increasing. Assume that  $ a \leq c<d \leq b  $. Let  $ f\in C([a,b],[0,1]) $ s.t.  $ f|_{[a,c]}=1, f|_{[d,b]}=0  $. Then  $ \rho(c) \leq I_p(f) \leq\rho(d) $ 
\end{lemma}
\begin{theorem}[Riesz Representation Theorem]\label{Riesz Representation Theorem}
    \index{Riesz Representation Theorem}
    We have a bijection  $ \rho \mapsto I_\rho $ between increasing right continuous  $ \rho:[a,b]\rightarrow \mathbb{R }_{ \geq 0 } $ and positive linear functionals  $ \Lambda:C([a,b],\mathbb{R})\rightarrow \mathbb{R } $  
\end{theorem}

\section{Theorems of Fubini and Tonelli for Radon measures}
\subsection{Products of Radon measure}
\begin{lemma}
     $ X  $ be LCH,  $ \Lambda:C_c(X)\rightarrow\mathbb{C} $ positive linear functional. For each precompact open  $ U\subset X  $,  $ \Lambda|_{C_c(U)}:C_c(U)\rightarrow \mathbb{C} $ is bounded.  
\end{lemma}
\begin{theorem}
     $ X_1,\cdots,X_N  $ with positive linear function  $ \Lambda_i:C_c(X_i)\rightarrow \mathbb{C} $. Then there exists a unique positive linear function  $ \Lambda:C_c(X_1\times\cdots\times X_N)\rightarrow \mathbb{C} $ such that   $ \forall f_i\in C_c(X_i) $, $ \Lambda(f_1\cdots f_N)=\Lambda_1(f_1)\cdots\Lambda_N(f_N)  $ \\where  $ f_1\cdots f_N:X_1\times\cdots\times X_N\rightarrow \mathbb{C}, (x_1,\cdots x_N)\mapsto f_1(x_1)\cdots f_N(x_N) $ 
\end{theorem}
\begin{definition}
    The completion of the Radon measure associated to  $ \Lambda_1\otimes\cdots\otimes\Lambda_N  $ is denoted by  $ \mu_1\times\cdots\times\mu_N $($ \mu_i $ is the completion of the Radon measure for  $ \Lambda_n $)  called \textbf{Radon Product}\index{Randon Product}
\end{definition}
\subsection{Theorems of Fubini and Tonelli}
\begin{theorem}[Tonelli's theorem]\index{Tonelli's theorem}
     $ f\in LSC_+(X\times Y) $. Then  $ \int_Y f\mathrm{d}v:X\rightarrow[0,+\infty] $  
\end{theorem}
\begin{theorem}[Tonelli's theorem]\index{Tonelli's theorem}
     Assume  $ \mu,\nu  $ are  $ \sigma $-finite. Let  $ f\in \mathcal{L}(X\times Y)  $ i.e.  $ f  $ is  $ (\mu\times \nu) $-measurable
     \begin{enumerate}[$ (a) $]
         \item  $ f(x,\cdot):Y\rightarrow [0,+\infty] $ is  $ \nu-measurable $ for  $ x\in X $ a.e.
         \item  $ x\mapsto \int_{Y}f(x,\cdot)\mathrm{d}v  $ is measurable.
         \item  $ \int_{X\times Y} f\mathrm{d}(\mu\times \nu)=\int_X\int_Yf\mathrm{d}\nu\mathrm{d}\mu$ 
     \end{enumerate} 
\end{theorem}
\begin{proposition}\label{condition of A times B measurable}
    Assume  $ \mu,\nu  $ are  $ \sigma  $-finte. Then for measurable  $ A\subset X, B\subset Y $,  $ A\times B      $  is  $ (\mu\times\nu) $-measurable, and  $ (\mu\times v)(A\times B)=\mu(A)\nu(B) $.  
\end{proposition}
\begin{example}
     $ \mu  $ is completion of Radon on  $ X  $,  $ (Y,2^Y,\nu ) $ counting measure.\\
     Then if  $ E\subset X\times Y  $ is open  $ \Rightarrow  $  $ \chi_E $ is lower semicontinuous. \\
      Tonelli's theorem  $ \Rightarrow (\mu\times\nu)(E)=\sum\limits_{y\in y}\nu(E_y) $\\
      Assume  $ \mu  $ is   $ \sigma  $-finite,  $ f:X\times Y\rightarrow [0,+\infty ] $ is Borel. Apply Prop \ref{condition of A times B measurable}, we have 
      \[\sum\limits_{y\in Y}\int_X f_y\mathrm{d}\mu=\int_X\sum\limits_{y\in Y}f_y\mathrm{d}\mu\tag{a}\]
      (a) is true if  $ f  $ is LCH or if  $ Y  $ is countable.\\
      Let  $ I=fin(2^Y) $,  $ \forall \alpha\in I $,  $ g_\alpha=\sum\limits_{y\in \alpha}f_\alpha $. We have
      \[\lim\limits_{\alpha}\int_{X}g_\alpha\mathrm{d}\mu =\int_X\lim\limits_\alpha g_\alpha \mathrm{d}\mu\]  
      if  $ f  $ is LSC or  $ Y =\mathbb{Z} $. 
\end{example}
\section{The marriage of Hilbert spaces and integral theory}
\subsection{The definition of  $ L^p $  spaces}
\begin{definition}
     $ f\in \mathcal{L} (X,\mathbb{C}),||f||_{L^p}=||f||_p=(\int_X|f|^p\mathrm{d}\mu)^{\frac{1}{p}} $ \index{ $\vert\vert f\vert\vert_{L^p},\vert\vert f\vert\vert_p$}
\end{definition}
\begin{theorem}
    Assume  $ f,g\in\mathcal{L}(X,\mathbb{C}) $ or  $ f,g\in \mathcal{L}_+(X) $. We have \textbf{Minkowski's inequality}   \[ ||f+g||_p \leq||f||_p+||g||_p \]\index{Minkowski's inequality}  
    And if $ \frac{1 }{p }+\frac{1}{q} $,  $ 1<p,q<+\infty $. We have \textbf{H{\"o}lder's inequality}
    \[|\int_X fg\mathrm{d}\mu| \leq||f||_p||g||_q\]
    \[||f+g||_p \leq||\,|f|+|g|\,||_p\]\index{H{\"o}lder's inequality}
\end{theorem}
\begin{definition}
    Let  $ \mathcal{L}^p(X,\mu)=\{f\in \mathcal{L}(X,\mathbb{C}):||f||_p<+\infty\} $. Then  $ ||\cdot||_p  $ is a semi-norm on  $ \mathcal{L}^p(X,\mu) $  \index{$ \mathcal{L}^p(X,\mu) $}\\
     $ L^p(X,\mu)=\mathcal{L}^p(X,\mu)/_{\{f\in \mathcal{L}^p(X,\mu):||f||_p=0\}} $ \index{$ L^p(X,\mu) $} is a NVS with norm  $ ||\cdot||_p $.
     \[||f||_p=0\Leftrightarrow \int |f|^p=0\Leftrightarrow f=0 \text{ a.e.}\] 
      $ L^p(X,\mu ) $ is the space of all  $ f\in \mathcal{L}(X,\mathbb{C}) $ satisfying  $ ||f||_p<+\infty $, but  $ f,g  $ are the same iff  $ f=g  $ a.e.  
\end{definition}
\begin{definition}
     $ f\in \mathcal{L}(X,\mathbb{C}) $. Define\index{$ \vert\vert f\vert\vert_{L^\infty},\vert\vert f\vert\vert_\infty $ in  $ \mathcal{L}(X,\mathbb{C}) $}
     \[||f||_{L^{\infty}}=||f||_\infty=\inf \{a\in \overline{\mathbb{R}}_{ \geq 0}:\mu\{|f|>a\}=0\}\]
     where  $ \{|f|>a\} =\{x\in X:||f(x)||>a\}$  
\end{definition}
\begin{proposition}
    Let  $ f,g\in \mathcal{L}(X,\mathbb{C}) $
    \begin{enumerate}[(a)]
        \item If  $ f=g  $ a.e., then  $ ||f||_\infty=||g||_\infty $ 
        \item  $ f=0  $ a.e. iff  $ ||f||_{L^\infty}=0 $ 
    \end{enumerate} 
\end{proposition}
\begin{proposition}
    Let  $ f\in \mathcal{L}(X,\mathbb{C}),\lambda=||f||_{L^\infty} $. Then 
    \[ \{a\in \overline{\mathbb{R} }\geq 0:\mu\{|f|>a\}=0\}=[\lambda,+\infty]\]
    In particular,  $ \lambda  $ is the smallest number s.t.  $ \{|f|>\lambda\} $ is null. 
\end{proposition}
\begin{corollary}
    Let  $ A=\{|f| \leq \lambda\} $. Then  $ X\backslash A $ is null, and  $ ||f\chi_A||_{l^\infty}=||f||_{L^\infty} $  
\end{corollary}
\begin{proposition}
    Let  $ (f_n) $ be in  $ \mathcal{L}(X,\mathbb{C}) $. TFAE
    \begin{enumerate}[(1)]
        \item  $ \lim\limits_{n \to \infty}||f_n||_{L^\infty}=0 $
        \item  $ \exists  $ measurable  $ A  $ with  $ \mu(A^c)=0 $ s.t.  $ f_n|_A $ uniformly converge to 0. 
    \end{enumerate} 
\end{proposition}
\begin{proposition}
    \[||f+g||_{L^\infty} \leq||f||_{L^\infty}+||g||_{L^\infty}\]
    \[||af||_{L^\infty}=|a|\cdot||f||_{L^\infty}\]
\end{proposition}
\begin{definition}
    \[\mathcal{L}^\infty(X)=\{f\in \mathcal{L}(X,\mathbb{C}):||f||_{l^\infty}<+\infty\}\]\index{$\mathcal{L}^\infty(X) $}
    And   
    \begin{align}
        L^\infty(X,\mu)&=\mathcal{L}^\infty(X)/_{\{f\in \mathcal{L}^\infty(X):||f||_{L^\infty}=0\}}\notag\\
        &=\mathcal{L}^\infty(X)/_{\{f\in \mathcal{L}^\infty(X):f=0\text{ a.e.}\}}\notag
    \end{align}\index{$ L^\infty(X) $}
\end{definition}
\subsection{Approximation in  $ L^p $ spaces}
\begin{theorem}
    Let  $ X  $ be LCH. Let  $ \mu  $ be (the completion of) a Radon measure on   $ X  $   $ 1 \leq p  < +\infty $ Then  $ C_c(X)  $ is dense in  $ L^p(X,\mu) $  
\end{theorem}
\begin{proof}[Hint]
    First we prove  $ ||f||_{l^\infty}<+\infty $, second we approximate  $ f  $ by  $ f\chi_{E_n} $  
\end{proof}
\begin{corollary}
    Let  $ e_n: x\mapsto e^{inx} $ in  $ C(S^1) $. If  $ 1 \leq p<+\infty $, then  $ e_n  $ spans a dense subspace of  $ L^p(S^1,\frac{m}{2\pi}) $ where  $ m  $ is the Lebesgue measure.  
\end{corollary}
\begin{theorem}
    Let  $ X  $ be second countable LCH.  $ \mu  $ is (the completion of ) a Radon measure on  $ X  $. Let  $ 1 \leq p<+\infty $. Then  $ L^p(X,\mu ) $ is seperable. 
\end{theorem}
\begin{proof}[Hint]
    First for  $ X  $ compact. We have proved that  $ C(X)  $ is  $ l^\infty  $-separable. Easy to check that it is true for  $ L^\infty $.\\
    For arbitary  $ X  $. Let  $ K_1\subset K_2\subset\cdots K_n\subset\cdots\subset X $,  $ \bigcup\limits_n K_n=X $. And  $ \lim\limits_n f\chi_{K_n}=f $. It suffices to prove that  $ \mu|_{K_n} $ is Radno measure.
\end{proof}
\begin{theorem}
    Let  $ (X,\mu ) $ be measurable,  $ 1 \leq p \leq+\infty $. Then  $ L^p(X,\mu)\cap S(X,\mathbb{C}) $ is dense in  $ L^p(X\mu) $.   
\end{theorem}
\begin{proof}[Note]
    Elements in  $ L^p\cap S $ are exactly: 
    \begin{equation*}
        \left\{
            \begin{aligned}
                \,{}&S(X,\mathbb{C}), {}&p=+\infty\\
                \,{}&\sum a_n\chi_{E_n},a_n\in \mathbb{C},\mu(E_n)<\infty,{}&p<+\infty
            \end{aligned}
        \right.
    \end{equation*}
    And we only need to check that for  $ f \geq 0 $. 
\end{proof}
\begin{proposition}
     $ L^\infty(X,\mu) $ is complete. 
\end{proposition}
In inner product space, we prove a similar theorem for completeness
\begin{theorem}
    If  $ V  $ is NVS, then  $ V  $ is complete  $ \Leftrightarrow  $ if  $ (v_n) $ in  $ V  $ s.t.  $ \sum ||v_n||<+\infty $, then  $ \sum v_n  $ converges. 
\end{theorem}
\subsection{The Riesz-Fischer Theorem}
\begin{theorem}[Riesz-Fischer Theorem]\index{Riesz-Fischer Theorem}
    If  $ 1 \leq  p<+\infty $. Then  $ L^p(X,\mu) $ is complete(Banach) space.\\
    Moreover, if  $ (f_n)  $ in  $ L^p(X,\mu) $,  $ f\in L^p(X,\mu) $ and  $ \lim\limits_n ||f-f_n||_{L^p}=0 $, then  $ (f_n ) $ has a subsequence converging a.e. to f.   
\end{theorem}
\begin{corollary}[Riesz-Fischer]
    We have a unitary  $ L^2([-\pi,\pi],\frac{m}{2\pi})\rightarrow l^2(\mathbb{Z}), f\mapsto \hat{f}  $. 
\end{corollary}
\subsection{Introduction to dualities in  $ L^p $ spaces}
We now assume  $ \frac{1}{p}+\frac{1}{q}=1 $,  $ 1 \leq p,q \leq +\infty $.  $ (X,\mu) $ measurable space.
\begin{proposition}
    Assume  $ \mu  $ is  $ \sigma  $-finte if  $ p=+\infty,q=1 $. Then  $ \exists  $ linear isometry  $ \Psi:L^p(X,\mu)\rightarrow L^q(X,\mu)^* $ s.t.  $ \forall F\in L^p,g\in L^q $,  \[\left<\Psi(f),g\right>=\int_X fg \mathrm{d}\mu \]  
\end{proposition}  
\begin{proposition}
     $ \left|\left<\Psi(f),g\right>\right| \leq ||f||_p\cdot||g||_q $  
\end{proposition}
\begin{example}
     $ (X,\mu) $ measurable space,  $ f\in L^\infty(X,\mu) $. Define
     \[M_f:L^2(X,\mu)\rightarrow L^2(X,\mu),g\mapsto fg\]
     Called \textbf{multiplication operator}\index{multiplication operator}\\
      $ M_x  $ has no eigenvalue, when  $ X=[0,1] $ 
\end{example}
\begin{theorem}
    Let  $ T\in\mathfrak{L}(\mathcal{H})  $ be self-adjoint,  $ a \leq T \leq B $. There exists  $ (\mu_i)_I $ of Randon measures on  $ [a,b] $  and unitary $ U:\mathcal{H}\rightarrow\bigoplus\limits_{i\in I}L^2([a,b],\mu_i) $ s.t.  $  UTU^{-1}=\bigoplus \limits_{i\in I}M_x $.   
\end{theorem}
\begin{definition}
    A \textbf{unitary representation}\index{unitary representation} of  $ \mathcal{A} $ on  $ \mathcal{H} $ is a linear  $ \pi:\mathcal{A}\rightarrow\mathfrak{L}(\mathcal{H}) $ s.t.  $ \pi(ab)=\pi(a)\pi(b) $,  $ \pi(1)=1_\mathcal{H} $, $ \pi(a^*)=\pi(a)^* $.  $ pi  $ is called a \textbf{unital *-homomorphism}\index{unital *-homomorphism}.\\
    If  $ \Omega\in \mathcal{H} $ s.t.  $ \pi(\mathcal{A})\Omega= \left\{ \pi(a)\Omega:a\in\mathcal{A}\right\}  $ is dense, we say  $ \pi  $ is a \textbf{cyclic representation}\index{unitary representation!cyclic representation}.  $ \Omega  $ is called a \textbf{cyclic vector}\index{unitary representation!cyclic vector}.\\
    If  $ \mathcal{K} $ is a closed linear  subspace of  $ \mathcal{H} $. If  $ \mathcal{K} $ is  $ \mathcal{A} $-invariant, i.e.  $ \pi(a)\mathcal{K}\subset \mathcal{K} $ for all  $ a\in \mathcal{A} $, we call  $ \mathcal{K} $ is a \textbf{subrepresentation}\index{unitary representation!subrepresentation}.    
\end{definition}


\begin{fact}
    If  $ \mathcal{K} $ is subrepresentation, then  $ \mathcal{H}\cong \mathcal{K}\oplus \mathcal{K}^\bot  $.    
\end{fact}
\begin{proposition}
    Let  $ \pi:\mathcal{A}\rightarrow\mathfrak{L}(\mathcal{H}) $ be unitary representation. then  $ \exists \, (\mathcal{H}_i)_{i\in I}  $  of unitary subrepresentation of  $ (\pi,\mathcal{H}) $ s.t.
    \begin{enumerate}[(a)]
        \item Every  $ \mathcal{H}_i $ is a cyclic representation.
        \item  $ \mathcal{H}_i\perp \mathcal{H}_j $ if  $ i\not=j $.
        \item  $ \mathrm{Span}\{\mathcal{H}_i:i\in I\} $ is dense in  $ \mathcal{H} $.     
    \end{enumerate} 
\end{proposition}
\begin{theorem}
    Let  $ X  $ be compact Hausdorff.  $ \pi:C(X)\rightarrow\mathfrak{L}(\mathcal{H}) $ cyclic representation with cyclic vector  $ \Omega $. Then  $ \exists $ Randon measure  $ \mu  $ on  $ X  $ and an unitary equivalence  $ U:(\mathcal{H},\pi)\rightarrow (L^2(X,\mu),M) $,  $ U\Omega=1 $ 
\end{theorem}
For  $ T\in \mathfrak{L}(\mathcal{H}) $, define 
\begin{align*}
    \pi_T:\mathbb{C}[x]&\rightarrow \mathfrak{L}\mathcal{H}\\
    f&\mapsto f(T)
\end{align*}
\begin{theorem}
    Let  $ T\in \mathfrak{L}(\mathcal{H}) $ be self-adjoint.  $ a \leq T \leq b $. Then  $ \pi_T:\mathbb{C}[x]\rightarrow\mathfrak{L}(\mathcal{H}) $ has operator norm  $ ||\pi_T|| \leq 1 $ if  $ \mathbb{C}[x] $ is equipped with  $ l^\infty[a,b] $.     
\end{theorem}
\begin{proposition}
    Let  $ f\in \mathbb{C}[x] $,  $ T_\alpha  $ is a net in  $ \mathfrak{L}(\mathcal{H}) $. If  $ T_\alpha\rightarrow T  $,  $ \sup||T_\alpha||<+\infty $. Then  $ f(T_\alpha)\rightarrow f(T) $    
\end{proposition}
\begin{proposition}
    If  $ \sup||T_\alpha||<+\infty $,  $ T_\alpha\rightarrow T,\eta_\beta\rightarrow \eta $, then  $ \lim\limits_{\alpha,\beta}T_\alpha\eta_\beta=T\eta $   
\end{proposition}
\section{From the implicit function theorem to differential manifolds}
\subsection{The inverse function theorem}
\begin{definition}
    Let  $ U\subset \mathbb{R}^n $ open,  $ V\subset \mathbb{R}^m $ open.  $ f: U\rightarrow V  $ is called a  \textbf{$ C^r $-diffeomorphism} ($ 0 \leq r \leq \infty $), if  $ f  $ is bijective and  $ f,f^{-1}\in C^r $.\index{$ C^r $-diffeomorphism} \\
    In fact,  $ m=n $ since  $ \Jac(f)\cdot\Jac(f^{-1})=I $ 
\end{definition}
\begin{theorem}[Inverse Function Theorem]\index{Inverse Function Theorem}
    Let  $ \Omega\subset \mathbb{R}^n $ open. Let  $ \varphi:\Omega\rightarrow\mathbb{R}^n $ be  $ C^r $-map,  $ 1 \leq r \leq +\infty $.  Let  $ p\in \Omega $. Assume  $ \mathrm{d}\varphi|_p:\mathbb{R}^n\rightarrow \mathbb{R}^n $ is invertible. Then  $ \exists  $ a neighborhood  $ U\subset  \Omega $  of  $ p $, neighborhood  $ V\subset \mathbb{R}^n $ of  $ q=\varphi(p) $ s.t.  $ \varphi:U\rightarrow V $ is  $ C^r $-diffeomorphism.  
\end{theorem} 
\begin{lemma}
    First prove it is true for  $ \varphi  $ bijective.
\end{lemma}
\begin{corollary}
     $ r \geq 1  $,  $ \Omega\subset \mathbb{R}^n $ open.  $ \varphi:\Omega\rightarrow \mathbb{R}^n $ injective  $ C^r $-map s.t.  $ \Jac(\varphi) $  is invertible everywhere. Then  $ \varphi(\Omega) $ is open in  $ \mathbb{R}^n $,  $ \varphi:\Omega\rightarrow \varphi(\Omega) $ is a  $ C^r $-diffemmorphism.     
\end{corollary}
\subsection{The Implicit Theorem}
\begin{corollary}
     $ \Omega\subset \mathbb{R}^d\times\mathbb{R}^k $ open,  $ (x,y)=(x^1,\cdots,x^d,y^1,\cdots,y^k) $,  $ f=(f^1,\cdots,f^k):\Omega\rightarrow \mathbb{R}^k $ is  $ C^r $ function, $ r \geq 1 $. Assume  $ \Jac_y(f) $ is invertible at  $ p\in \Omega $. Then  $ \exists\, $ neighborhood  $ U\subset \Omega  $ of $ p $ and open  $ V\subset \mathbb{R}^d\times\mathbb{R}^k $ s.t. we have  $ C^r $-diffemmorphism 
     \[(x^1,\cdots,x^d,f^1,\cdots,f^k):U\rightarrow V\]    
\end{corollary}


\begin{corollary}
     $ f=(f^1,\cdots,f^k):\Omega\rightarrow\mathbb{R}^k $, $ \Omega  $ open in  $ \mathbb{R}^d \times \mathbb{R}^k$. If  $ \Jac_y(f) $ invertible at  $ p\in \Omega $. Then  $ \exists\, p\in U\subset \Omega$,  $ V\subset\mathbb{R}^d\times \mathbb{R}^k $ s.t.
     \[(x^1,\cdots,x^d,f^1,\cdots,f^k)=(x,f):U\cong V\]
     is a diffeomorphism.   
\end{corollary}
\begin{definition}\label{def1}
    Let  $ M\subset \mathbb{R} $,  $ r \geq 1 $. We say  $ M  $ is an \textbf{(embedding) $ C^r $-submanifold} of  $ \mathbb{R}^n $ \index{(embedding) $ C^r $-submanifold} if for every  $ p\in M $,  $ \exists\,U\in Nbh_{\mathbb{R}^n}(p) $,  $ 0 \leq d \leq n,k=n-d $,  $ \exists\,C^r  $-functions
    \[( \varphi^1,\cdots,\varphi^d,f^1,\cdots,f^k):U\rightarrow \mathbb{R}^n\]
    satisfying:
    \begin{enumerate}[(a)]
        \item  $ (\varphi,f):U\cong V $ is a  $ C^r $-diffeomorphism.
        \item    $ (\varphi,f)^{-1}((\mathbb{R}^d\times 0)\cap V)=M\cap U $,(equivalently, $ (\varphi,0) $ is bijective) 
    \end{enumerate}    
\end{definition}
\begin{proposition}
     $ \varphi(M\cap U) $ is an open subset of  $ \mathbb{R}^d $. Moreover,  $ \forall h\in C^r(U,\mathbb{R}) $,  $ \exists\,!\, g:\varphi(M\cap U)\rightarrow\mathbb{R} $ s.t. 
     \begin{equation}
        h|_{M\cap U}=g\circ \varphi|_{M\cap U}\tag{$ \triangle $}
     \end{equation} 
     Moreover, any  $ g $ satisfying  $ (\triangle) $  is  $ C^r $ 
\end{proposition}
\begin{lemma}
    Let  $ \Omega\subset \mathbb{R}^n $ open,  $ \psi=(\psi^1,\cdots,\psi^n):\Omega\rightarrow \mathbb{R}^d $ is a $ C^r $ function.  Assume  $ \Omega,\psi $ satisfying a similar property as  $ U,\varphi $ in  Def \ref{def1}. Then  $ (U,\varphi|_{M\cap U}) $ and  $ (\Omega,\psi|_{M\cap \Omega}) $ are  $ C^r $-compatible, i.e.
    \[\Psi|_{M\cap,U,\Omega}\circ (\varphi|_{M\cap U\cap \Omega})^{-1}:\varphi(M\cap U\cap \Omega)\rightarrow \Psi(M\cap U\cap \Omega)\]   
    is a  $ C^r $-diffeomoephism of open subsets of  $ \mathbb{R}^d $.  
\end{lemma}

\begin{definition}
    Let  $ M  $ be a nonempty Hausdorff. A set  $ \mathcal{U}=\{(U_\alpha,\varphi_\alpha)_{\alpha\in \mathcal{A}}\} $ is called a  \textbf{$ C^r $-atlas}\index{$ C^r $-atlas} of  $ M  $ if 
    \begin{itemize}
        \item  $ M=\bigcup\limits_\alpha U_\alpha $,  $ U_\alpha $ is open,
        \item  $ \varphi_\alpha:U_\alpha\xrightarrow{\cong}\varphi(U_\alpha) $ is  homeomorphism,  $ \varphi(U_\alpha) $ is open in  $ \mathbb{R}^{d_\alpha} $
        \item  $ \forall \alpha,\beta\in\mathcal{A} $,  $ \varphi_\alpha $ and  $ \varphi_\beta $ are  $ C^r $ compatible, i.e. $ \varphi_\beta\circ\varphi_\alpha^{-1}:\varphi_\alpha(U_\alpha\cap U_\beta)\xrightarrow{\cong}\varphi_\beta(U_\alpha\cap U_\beta) $ is a  $ C^r $-diffemmorphism.       
    \end{itemize} 
     $ (M,\mathcal{U}) $ is called a  \textbf{$ C^r $-manifold}\index{$ C^r $-manifold}.\\
     We assume  $ M  $ is second countable.\\
     We call  $ C^\infty $-manifold differential manifold or smooth manifold.  
\end{definition}
\begin{definition}
     $ U_\alpha $ is  $ d_\alpha $-dimensional. If  $ p\in U_\alpha $,  $ \dim_p M=d_\alpha $.\\
     If  $ \dim_p M=d $ independent of  $ p $, we say  $ M  $ is equidimensional.   
\end{definition}

\begin{proposition}
     $ \forall d\in \mathbb{N} $,  $ U_d=\{x\in M:\dim_x M=d\} $ is  open and closed in  $ M $.\\
     In particular, if  $ M  $ is connected, then  $ M  $ is equidimensional.  
\end{proposition}
\begin{definition}
    A \textbf{$ C^r $-chart }\index{ $ C^r $-chart} on  $ (M,\mathcal{U}) $ is  $ (V,\psi) $ s.t.
    \begin{itemize}
        \item  $ V  $ is open in  $ M  $,  $ \psi(V ) $ open in  $ \mathbb{R}^d $.
        \item  $ \psi:V\rightarrow \psi(V) $ homeomorphism.
        \item  $ (V,\psi) $ is  $ C^r $-compatible with any member of  $ \mathcal{U} $.   
    \end{itemize} 
     $ \overline{\mathcal{U}}=\{C^r-\text{chart of }(M,\mathcal{U})\} $ is the maximal  $ C^r  $ atlas contain  $ \mathcal{U} $.\\
     A maximal  $ C^r $-atlas on  $ M  $ is called a  $ C^r  $-structure.\\  
\end{definition}
\begin{definition}
     $ M,N  $ are  $ C^r  $-manifolds,  $ F:M\rightarrow N $ is a  $ C^r $-map if  $ F  $ is continuous and if for every chart  $ (U,\varphi) $ of  $ M  $ and  $ (V,\psi) $ of  $ N  $, we have 
     \[\psi\circ F\circ\varphi^{-1}:\varphi(U\cap F^{-1}(V))\rightarrow \psi(V)\] is  $ C^r  $ function.\\
     If  $ F  $ is bijective,  $ F $ and  $ F^{-1}  $ are  $ C^r $, we sat  $ F  $ is a  $ C^r $-diffeomorphism.
\end{definition}

\begin{proposition}
    Let  $ M  $ be a  $ C^r $-submanifold of  $ N  $, then the inclusion map $ \iota:M\rightarrow N $ is  $ C^r $.  
\end{proposition}
\begin{proposition}
    Let  $ X, N  $ be  $ C^r $-submanifolds.  $ M  $ is  $ C^r $-submanifold of  $ N  $. Let  $ \iota:M\rightarrow N $. Let  $ F:X\rightarrow M $. Then  $ F  $ is  $ C^r $ iff  $ \iota\circ F  $ is  $ C^r $.    
\end{proposition}
\begin{example}
     $ P\subset M,Q\subset N $ submanifold, then  $ P\times Q\subset M\times N  $ submanifold.
\end{example}
\begin{theorem}[Implict Function Theorem]\index{Implict Function Theorem}
    Let  $ (x,y)=(x^1,\cdots,x^d,y^1,\cdots,y^k) $ be standard coordinates on   $ \mathbb{R}^d \times \mathbb{R}^k$.  $ \Omega\subset \mathbb{R}^d \times \mathbb{R}^k $ open. Let  $ M\subset \mathbb{R}^d \times \mathbb{R}^k  $. Assume  $ \exists\, C^r $  $ f=(f^1,\cdots,f^k):\Omega\rightarrow \mathbb{R}^k $ \st 
    \begin{enumerate}
        \item  $ M\cap \Omega=Z(f) $.
        \item  $ \Jac_y f $ is invertible at  $ p\in M $ 
    \end{enumerate}   
    Then  $ \exists\,  $ Neighborhood  $ p\in U\subset \Omega $ \st \, $ M\cap U  $ is a  $ C^r  $-submanifold of  $ \mathbb{R}^d \times \mathbb{R}^k $ and  $ (M\cap U,x|_{M\cap U}) $ is a chart on  $ M\cap U $.  
\end{theorem}
\section{Differential calculus on manifold}
Recall that for  $ V  $ is  $ \mathbb{F} $-vector space, with $ e_i,1 \leq i  \leq n $ basis. There is dual basis  $ e^i $ in  $ V^* $ \st\,  for  $ \xi \in V $,  $ \xi=\sum\limits_{i=1}^n  \left< \xi,e^i \right>  e_i $.  
\subsection{Tangent Space and Cotangent Space}
\begin{definition}
    For  $ C^\infty $-map  $ \gamma:(a,b)\rightarrow M $.  $ p\in M $. Define 
    \[T_p M =\{\text{smooth }\gamma:(-\epsilon,\epsilon)\rightarrow M,\gamma(0)=p\}/_{\sim}=\{\gamma'(t_0):\gamma(t_0)=p,t_0\in\mathbb{R}\}\]   
    where  $ \gamma_1\sim \gamma_2 $ if and only if  $ \exists $ chart  $ (U,\varphi) $ \st \, $ \Jac(\varphi\circ \gamma_1)|_0= \Jac(\varphi\circ \gamma_2)|_0 $ \\
     $ \gamma'(t_0) $ is the equivalance class of  $ t\mapsto \gamma(t+t_0) $ in  $ T_{\gamma(t_0)}M $     
\end{definition}
\begin{theorem}
    Let  $ p\in M $, for each chart  $ (U,\varphi^1,\cdots,\varphi^n) $ containing  $ p  $,  $ \exists $ bijection  $ \mathrm{d}\varphi|_p $ defined by  $ \mathrm{d}\varphi|_p:T_pM\rightarrow \mathbb{R}^n $,  $ \mathrm{d}\varphi|_p\cdot \gamma'(0)=\Jac(\varphi\circ \gamma)|_0 $ if  $ \gamma $ is smooth path and  $ \gamma(0)=p $. 
\end{theorem}
\begin{remark}
    In this way, we can define a  $ \mathbb{R} $-vector space structure on  $ TM $. cf. Def \ref{123}  
\end{remark}
\begin{definition}\label{123}
    \textbf{Tangent bundle}\index{Tangent bundle}
    \[TM=\bigsqcup\limits_{p\in M}T_pM\]
     $ X:M\rightarrow TM  $ is called a vector field if  $ \forall p\in M, X|_p\in T_pM $ 
\end{definition}
\begin{definition}
    For  $ (U,\varphi) $ is chart on  $ M  $. Define
    \[\partial_{\varphi^i}=\frac{\partial}{\partial \varphi^i}:U\rightarrow TM\]
    \[\partial_{\varphi^i}|_p=(\mathrm{d}\varphi|_p)^{-1}e_i\]
\end{definition}∆
\begin{theorem}
    Let  $ F:M\rightarrow N  $ be  $ C^\infty $. Then  $ \forall p\in M  $, let  $ q=F(p) $. Then  $ \exists  $ unique linear map  $ \mathrm{d}F|_p:T_pM\rightarrow T_q N $,  $ \mathrm{d}F|_p\cdot \gamma'(0)=(F\circ  \gamma)'(p) $   for  $ \gamma:(-\epsilon,\epsilon)\rightarrow M $ smooth,  $ \gamma(0)=p $.\\
    Moreover, if  $ (U,\varphi^1,\cdots,\varphi^m) $ and  $ (V,\psi^1,\cdots,\psi^n) $ are charts of  $ M,N $ containing  $ p,q $, then 
    \[\mathrm{d}F|_p\cdot(\frac{\partial}{\partial\varphi^1},\cdots,\frac{\partial}{\partial\varphi^m})
    _p=(\frac{\partial}{\partial\psi^1},\cdots,\frac{\partial}{\partial\psi^1})_q\cdot \Jac(\psi\cdot\varphi^{-1})|_{\varphi(p)}\]      
\end{theorem}
\begin{remark}
    It's hard to prove the existence of  $ \mathrm{d}F|_p $ but the chain rule.
\end{remark}
\begin{proposition}[Chain rule]
    \[\mathrm{d}(G\circ F)_p=\mathrm{d}G|_{F(p)}\cdot \mathrm{d}F|_p\]
\end{proposition}
\begin{definition}
     $ X :M\rightarrow TM $ is called a \textbf{smooth vector field}\index{smooth vector field} if TFEC true:
     \begin{enumerate}
        \item  $ \forall  $ chart  $ (U,\varphi^1,\cdots,\varphi^n) $ if  $ X|_U=\sum\limits_{i=1}^nX^i\frac{\partial }{\partial \varphi^i} $,  $ X^i:U\rightarrow \mathbb{R} $ smooth.
        \item  $ \exists  $ atlas  $ \mathcal{U} $ \st  $ \forall (U,\varphi)\in \mathcal{U} $,   $ X^i  $ is smooth.  
     \end{enumerate}
\end{definition}
\begin{definition}
    The \textbf{cotangent space}\index{cotangent space} is 
    \[T^*M=\bigsqcup\limits_{p\in M}T_p^*M\]
    where  $ T_p^*M  $ is the (real) dual space of  $ T_pM $.\\
     $ \omega :M\rightarrow T^*M $ is called a \textbf{1-form}\index{1-form} if  $ \forall p\in M, \omega|_p\in T_p^*M $   
\end{definition}
\begin{definition}
    \textbf{1-form}\index{1-form} $ \omega :M\rightarrow T^*M $ is called smooth if  $ \forall  $ open  $ U\subset M $,  $ \forall  $ smooth  $ X:U\rightarrow TM $,  $  \left< \omega,X \right>:p\in U\mapsto  \left< \omega|_p,X|p \right>   $ is smooth. 
\end{definition}
\begin{definition}
     $ f\in C^\infty(M,\mathbb{R}) $,  $ \forall p\in M $,
     \[\mathrm{d}f|_p:T_pM\rightarrow T_{f(p)}\mathbb{R}\cong \mathbb{R},\mathrm{d}f|_p\in T_p^*M\]
      $ \mathrm{d}f :M\rightarrow T^*M $ is 1-form.   
\end{definition}
\begin{proposition}
    Let  $ f\in C^\infty(M,\mathbb{R}) $,  $ \mathrm{d}f $ is smooth 1-form. Moreover, if  $ (U,\varphi^1,\cdots,\varphi^n) $  is chart
    \[\frac{\partial}{\partial\varphi^j}f|_p=\partial_j(f\circ\varphi^{-1})|_{\varphi(p)},p\in M\]
    \[\partial_{\varphi^j}f=(\partial_j(f\circ\varphi^{-1}))\circ \varphi\]
\end{proposition}
\begin{corollary}
     $ f\in C^\infty(M,\mathbb{R}) $,  $ (U,\varphi^1,\cdots,\varphi^n) $ chart, then 
     \[\mathrm{d}f=\sum\limits_{i=1}^n\frac{\partial f}{\partial\varphi^j}\mathrm{d}\varphi^j\]  
\end{corollary}
\begin{corollary}
     $ f,g\in C^{\infty}(M,\mathbb{R}) $, then  $ \mathrm{d}(fg)=\mathrm{d}f\cdot g+\mathrm{d}g\cdot f $  
\end{corollary}
\begin{definition}
    If  $ F:M\rightarrow N  $ smooth,  $ F^*|_p:T^*_{F(p)}N\rightarrow T_p^*M $ is defined by transpose of  $ \mathrm{d}F|_p :T_pM\rightarrow T_{F(p)}N$.\\
     $ F^*  $ is called \textbf{cotangent map}\index{cotangent map}\\
     $ F^*\cdot\omega' $ called \textbf{pullback }of  $ \omega' $ by  $ F  $.\\
     If  $ \omega:N\rightarrow T^*N $ is a 1-form, its \textbf{pullback} is  $ F^*\omega :M\rightarrow T^*M,p\mapsto F^*(\omega|_{F(p)})\in T^*_pM $.  
\end{definition}
\begin{proposition}
    Let  $ \omega:N\rightarrow T^*N $ be smooth 1-form, then  $ F^*\omega $ is smooth. Moreover, if  $ f\in C^\infty(N,\mathbb{R}) $,  $ F^*\mathrm{d}f=\mathrm{d}(f\circ F) $    
\end{proposition}
\begin{definition}
     $ F:M\rightarrow N $ is called a (smooth) \textbf{embedding} if  $ F(M) $ is a  $ C^\infty $-submanifold of  $ N  $, and  $ F  $ restricts to a diffeomorphism on  $ M  $.
\end{definition}
\begin{proposition}
    Let  $ F:M\rightarrow N$  be smooth embedding, then   $ \forall p\in M $,  $ \mathrm{d}F|_p:T_pM\rightarrow T_{F(p)}N $  is injective.   
\end{proposition}
\begin{theorem}
    Let  $ F:M\rightarrow N $ be smooth, let  $ q\in N $.  Assume  $ \forall p\in F^{-1}(q) $,  $ \mathrm{d}F|_p:T_pM\rightarrow T_qN $ is surjective($ F  $ is a submersion at  $ q $) Then  $ F^{-1}(q)  $ is a smooth submanifold of  $ M  $.\\
    Moreover,  $ T_p(F^{-1}(q)=\ker(\mathrm{d}F|_p) $\\
    In particular,  $ \dim_p F^{-1}(q)=\dim_pM-\dim_qN $  
\end{theorem}
\begin{theorem}
    Let  $ F:M\rightarrow N  $ be smooth. Let  $ p\in M $,  $ q=F(p) $,  $ \mathrm{d}F|_p:T_pM\rightarrow T_qN $ linear isomorphism. Then  $ \exists U\in Nbh(p) $,  $ V\in Nbh(q) $ \st  $ F  $ restructs to a diffeomorphism  $ F:U\rightarrow V $   
\end{theorem}
\subsection{Tensor product}
\subsection{Introduction}
\begin{proposition}\,\label{exact_tensor}
    \begin{enumerate}[(1)]
         \item $ 0\rightarrow W'\rightarrow W\rightarrow W'' $ is exact if and only if \\
          $ \forall V $  $ R  $-module,  $ 0\rightarrow \Hom(V,W')\rightarrow \Hom(V,W)\rightarrow \Hom(V,W'') $ is exact.
          \item  $ V'\rightarrow V\rightarrow V''\rightarrow 0$ is exact if and only if \\
          $ \forall W $  $  R  $-module,  $ 0\rightarrow \Hom(V'',W)\rightarrow\Hom(V,W)\rightarrow\Hom(V',W) $ is exact   
    \end{enumerate}
\end{proposition}
\begin{proposition}\label{coker exact sequence}
     $ V\xrightarrow{f}W\rightarrow U\rightarrow 0 $ is exact means  $ U=\Coker f:=W/\img f $  
\end{proposition}
\subsection{Tensor Product}
\begin{definition}
     $ \forall U,V  $  $ R  $-module,  $ \exists  $  $ R $-module  $ X  $ and  $ U\times V\xrightarrow{F}X $  $ R  $-bilinear map \st  $ \forall  $  $ U\times V\xrightarrow{G}W  $  $ R  $-bilinear map, there exists a unique  $ R  $-linear map  $ g:X\rightarrow W $ with  $ g\circ F=G $.\\
     Moreover,  pair $ (X,F) $  is unique up to unique isomorphism.\\
     Write  $ (U\otimes_R V,U\times V\rightarrow U\otimes V ) $ 
\end{definition}
\begin{remark}
    The existence of  $ X  $ is proved through quotient space  $ R^{\oplus U\times V}/V $ for some equivalent submodule  $ V $.  
\end{remark}
\begin{proposition}[Proper A]
    \,
    \begin{enumerate}[(1)]
        \item  $ R\otimes_R V\cong V $
        \item  $ V\otimes W\cong W\otimes V $
        \item  $ (V\oplus U)\otimes W\cong (U\otimes W)\oplus (V\otimes W) $
        \item  $ (U\otimes V)\otimes W=U\otimes (V\otimes W) $    
    \end{enumerate}
\end{proposition}
\begin{example}
     $ R^m\otimes R^n=R^{mn} $,  $ R^m\otimes V\cong V^m $  
\end{example}
\begin{proposition}[Proper B]
    \[\Hom(U\otimes V,W)\cong \Hom(U,\Hom(V,W))\]
\end{proposition}
\begin{remark}
    It is up to the commutative diagram in the definition.
\end{remark}
\begin{proposition}[Proper C]
    If  $ W'\rightarrow W\rightarrow W''\rightarrow 0 $ is exact, then  $ V\otimes W'\rightarrow V\otimes W\rightarrow V\otimes W''\rightarrow 0 $ is exact for all $ V  $  $ R  $-module.
\end{proposition}
\begin{remark}
    It is directly from Proper B and Prop \ref{exact_tensor}
\end{remark}
\begin{example}
     $ \mathbb{Z}\rightarrow \mathbb{Z}\rightarrow \mathbb{Z}/5\rightarrow 0 $ is exact\\
     So  $ \mathbb{Z}/4\otimes \mathbb{Z}\rightarrow \mathbb{Z}/4\otimes \mathbb{Z}\rightarrow \mathbb{Z}/4\otimes \mathbb{Z}/5\rightarrow 0 $ is exact.\\
     By Prop \ref{coker exact sequence},  $ \mathbb{Z}/4\otimes \mathbb{Z}/5=\mathbb{Z}/\mathbb{Z}=0 $ \\
     Similarly,  $ \mathbb{Z}/4\otimes \mathbb{Z}/6=\mathbb{Z}/2 $ 
\end{example}
There is a proposition to describe it.
\begin{proposition}
    \begin{align*}
        R/I\otimes V&\cong V/_{IV}\\
        R/I\otimes R/J&\cong R/_{(I+J)}
    \end{align*}
\end{proposition}
\begin{proposition}
    \begin{align*}
        R/I\otimes_R R'&\cong R'/_{IR'}\\
        R[x]\otimes_R  R'\cong R'[x]
    \end{align*}
\end{proposition}
Therefore,  $ \mathbb{C}\otimes_\mathbb{R}\mathbb{C}\cong \mathbb{R}[x]/(x^2+1)\otimes \mathbb{C}=\mathbb{C}[x]/(x^2+1)\cong \mathbb{C}\times \mathbb{C} $  
\printindex
\end{document}