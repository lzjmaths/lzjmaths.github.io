\begin{theorem}
    If  $ X  $ is a metric space. Then  $  X  $ is compact
     $ \Leftrightarrow $  $ X  $ is sequentially compact.
\end{theorem}
\begin{definition}
     $ X $ is a topo. space.  $ X  $ is called \textbf{net-compact} if every net in  $ X  $ has a convergent subnet.\\
      $ X  $ is called \textbf{countably compact} if every countable open cover of  $ X  $ has a finite subcover. 
\end{definition}
Easy to see that:\\
compact $ \Rightarrow  $ countably compact.\\
\begin{proposition}
    For topology space, net-compact  $ \Leftrightarrow $ compact.\\
    For metric space, Four compactness are equivalent. 
\end{proposition}
\begin{example}[Extreme Value Theorem]
    If  $ X  $ is compact,  $ f:X\rightarrow \mathbb{R } $ continuous, then f attains it max(and min).
\end{example}
It suffices to prove  $ f(X ) $ is bounded.

\underline{steps of proof}:\\
Step1: Prove finiteness locally;\\
Step2: Use compactness to go from local to global.
\begin{proposition}
     $ X  $ is a topo. space. TFAE:
     \begin{enumerate}
        \item[$ (1) $]  $ X  $ is compact.
        \item[$ (2) $](Increasing chain property) If  $ (U_\mu)_{\mu\in I} $ is an increasing net of open subsets of  $ X  $, s.t.  $ \cup_{\mu\in I}U_\mu=X $, then  $ \exists \mu\in I  $ s.t.  $ U_\mu=X $.
        \item[$ (3) $](Decreasing chain property) If  $ (U_\mu)_{\mu\in I} $ is an dncreasing net of closd subsets of  $ X  $, s.t.  $ \forall \mu\in I,E_\mu\not=\emptyset  $, then  $ \cap_{\mu\in I }E_\mu\not=\emptyset $    
     \end{enumerate}
\end{proposition}