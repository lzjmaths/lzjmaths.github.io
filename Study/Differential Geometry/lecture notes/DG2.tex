% !TEX root = lecture/Differential_Geometry.tex

\subsection{Lie Groups and Homogeneous Spaces}

\begin{definition}
    We say  $ G  $ is a \name{Lie group} if it is a topological group with a smooth structure such that the multiplication map $ \cdot : G \times G \to G $ and the inverse map  $ G\rightsquigarrow G  $  is smooth. 
\end{definition}
\begin{example}
     $ GL(n,\mathbb{R})=\{n\times n \text{ matrices with non-zero determinant}\}\subset \mathbb{R}^{n\times n } $\\
      $ O(n)=\{A\in GL(n,\mathbb{R})|AA^T=I\} $\\
       $ SO(n)=\{A\in O(n)|\det A=1\} $\\
        $ U(n)=\{A\in GL(n,\mathbb{C})|A\overline{A}^T=I\} $\\
         $ SU(n)=\{A\in U(n)|\det A=1\} $     
\end{example}
\index{$GL(n),O(n),SO(n),U(n),SU(n)$}
\begin{exercise}
    \begin{align}
        O(1)&\cong S^2 &SO(1)&\cong *\\
            SO(2)&\cong S^1 &SO(3)&\cong \mathbb{R}\mathbb{P}^3 \\
            SU(2)&\cong S^3 &U(n)&\cong S^1\times SU(n)
    \end{align}
    The last one is a diffeomorphism but do not preserve the multiplicatioin, \ie not an isomorphism of Lie group.
\end{exercise}
\begin{theorem}[Carton]
    Let  $ H  $ be a closed subgroup of Lie group  $ G  $. Then  $ H  $ is a Lie group. More precisely,  $ H  $ is topological manifold, carries a canonical smooth structure that makes  the multiplication and inverse smooth. Also,  $ G/H  $ is a smooth manifold
\end{theorem}
\begin{definition}
    Let  $ M  $ be a smooth manifold. We say   $ M  $ is a \name{homogeneous space} if  $ \exists  $ a Lie group 
     $ G  $ with a smooth transitive action $ \rho: G\times M \rightarrow M  $.
\end{definition}
\begin{definition}
    For  $ M  $ be a homogeneous space.
    The \name{isotropy} group of  $ x\in M  $ is defined as 
    \[Iso(x)=\{g\in G|gx=x\}\]
    closed subgroup of  $ G $\\
    Given any  $ x,x'\in M  $,  $ Iso(x)\cong Iso(x') $ because the group  action is transitive. \\
    Hence, we have a well-defined map 
    \begin{align}
        p:G/_{Iso(x)}&\rightarrow M \\
        g\mapsto gx
    \end{align}
\end{definition}
\begin{theorem}
     $ p  $ is always a diffeomorphism.
\end{theorem}
Therefore, we have this proposition
\begin{proposition}
    $ M  $ is a homogeneous space  $ \Leftrightarrow  $  $ M=G/H  $ for some closed subgroup  $ H  $.
\end{proposition}
\begin{example}
    If  $ M=S^n  $, let  $ G=SO(n+1)  $.\\
    Then  $ Iso(1,0,\cdots,0)\cong SO(n) $.\\
    So  $ S^n\cong SO(n+1)/(SO(n)) $. \\
    Similarly, we can prove  $ \mathbb{RP}^n\cong SO(n+1)/(O(n)) $,  $ \mathbb{CP}^n\cong SO(n+1)/(U(n))$\\
    The isotropy k dimensional linear subspaces of  $ \mathbb{R}^n  $ can be  $ O(k)\times O(n-k) $ if  $ G=O
    (n) $   
\end{example}
A connected closed surface is a homogeneous space if and only if it is diffeomorphic to  $ \mathbb{RP}^2,S^2,T^2 $ and Klein bottle. 
\begin{theorem}[Whithead]
    Any smooth manifold has a triangulation.
\end{theorem}
\begin{theorem}[Poincare-Hopf]
     $ G  $ is compact Lie group  $ \Rightarrow  $  $ \chi(G)=0 $. 
\end{theorem}
\begin{theorem}[Mostow2005]
     $ M  $ is a compact homogeneous space  $ \Rightarrow  $  $ \chi(M) \geq 0 $. 
\end{theorem}
\subsection{Bump Function and Partition of Unity}
\begin{theorem}[Urysohn smooth version]
    Given  $ M  $, closed disjoint  $ A,B  $,  $ \exists  $ smooth  $ f:M\rightarrow[0,1] $ \st  $ f|_A=0,f|_B=1 $.  
\end{theorem}
\begin{theorem}[Tietze]
    Given  $ M  $, closed  $ A  $, smooth  $ f:A\rightarrow \mathbb{R}^n  $, there exists smooth  $ \hat{f }:M\rightarrow \mathbb{R}^n $ \st  $ \hat{f}|_A=f $  
\end{theorem}
To prove these and much more result we need partition of unity theorem.

First we define bump function.
\begin{lemma}
    Let  $ U  $ be a neighbourhood of  $ p\in M $. Then  $ \exists   $ smooth  $ \sigma:M\rightarrow [0,1]  $ \st 
    \begin{enumerate}
        \item  $ \sigma \equiv 1  $ near  $ p $
        \item Supp $ \sigma \subset U  $  
    \end{enumerate}
    Such  $ \sigma  $ is called a \name{bump function } at  $ p  $, supported in  $ U $. 
\end{lemma}