% !TEX root = lecture/Differential_Geometry.tex

% \tikzset{external/export=false}
\begin{definition}
    A  $ k $-form  $ \omega $ is \textbf{closed}\index{closed form}  if  $ \omega\in\ker\left(\Omega^k(M)\xrightarrow{d}\Omega^{k+1}(M)\right) $.

    A  $ k $-form  $ \omega $ is \textbf{exact}\index{exact form}  if there exists a  $ (k-1) $-form  $ \eta $ such that  $ d\eta=\omega $, or equivalently,  $ \omega\in \img\left(\Omega^{k-1}(M)\xrightarrow{d}\Omega^k(M)\right) $.

    By Proposition \ref{proposition of pull back} (1), exact  $ k $-form are all closed.

So we may define the  $ k $-th  \name{de Rham cohomology} of  $ M $
\begin{equation}
    H^k_{\mathrm{DR}}(M):=\dps\frac{\dps\ker\left(d:\Omega^k(M)\xrightarrow{d} \Omega^{k+1}(M)\right)}{\dps\img\left(\Omega^{k-1}(M)\xrightarrow{d}\Omega^k(M)\right)}
\end{equation}
\end{definition}

By Proposition \ref{proposition of pull back} (2), we have  $ \forall F\in C^\infty(M,N) $,
$ \omega\in \Omega^k(N) $.

Then  $ \omega $ closed  $ \Rightarrow  $  $ F^*\omega $ is closed.  $ \omega $ exact $ \Rightarrow $  $ F^*\omega $ exact.

So  $ F $ induces a linear map
\begin{equation*}
    \begin{aligned}
        F^*:H^k_{\mathrm{DR}}(N)&\rightarrow H^k_{\mathrm{DR}}(M)\\
        [\omega]&\mapsto [F^*\omega]
    \end{aligned}
\end{equation*}
\begin{proposition}[Key properties of  $ H^k_{\mathrm{DR}}(M) $]\label{key properties of de Rham cohomology}
    \,
    \begin{enumerate}[label=(\arabic*)]
        \item  $ (F\circ G)^*=G^* \circ F^* $
        \item  $ (\id)^*=\id $.
        \item  $ F,G\in C^\infty(M,N) $,  $ F $ homotopic to  $ G $ $ \Rightarrow  $ $ F^*=G^* $
        \item If  $ F $ is a homotopy equivalence $ \Rightarrow $  $ F^*:H^k_{\mathrm{DR}}(M)\rightarrow H^k_{\mathrm{DR}}(N) $  is an isomorphism.
    \end{enumerate}
\end{proposition}
\begin{remark}
    Properties (3),(4) are nontrivial, which is the essential part of the theory of de Rham cohomology
\end{remark}
\begin{proposition}\label{Computation of cohomology at 0}
     $ H^0_{\mathrm{DR}}(M)\cong \mathbb{R} \left<\pi_0(M)\right>$, where  $ \pi_0(M)=\{\text{path component of  $ M $}\} $.
\end{proposition}
It suffices to prove the lemma that
\begin{lemma}
     $ \alpha\in \Omega^0(M)=C^\infty(M,\mathbb{R}^n) $. Then $ \alpha $ is closed iff  $ \alpha $  is constant on each component of  $ M $.
\end{lemma}
\begin{proof}
    The inverse part is trivial.

    Assume  $ \alpha $ is closed. Pick  $ p,q\in M $ in some path component.  $ \exists $ smooth path  $ \gamma:\mathbb{R}\rightarrow M $,  $ \gamma(0)=p,\gamma(1)=q $.

    $ \dd \alpha=0 $ $ \Rightarrow  $  $ \mathrm{d}(\gamma^*\alpha)=0 $ $ \Rightarrow  $  $ \dd(\alpha\circ \gamma)=0 $  $ \Rightarrow $ $ \dps\frac{\dd (\alpha\circ \gamma)}{\dd t}=0 $ $ \Rightarrow  $ $ \alpha\circ \gamma(1)=\alpha\circ\gamma(0) $.

    So  $ \alpha(p)=\alpha(q) $
\end{proof}
We have  $ H^k_{\mathrm{DR}}(M)\cong \mathrm{Ab}(\pi_1(M))\otimes_\mathbb{Z} \mathbb{R} $.  $  \mathrm{Ab}(\pi_1(M)) $ is the Abelian group of  $ \pi_1(M) $. In particular,  $ H^1_{\mathrm{DR}}(\mathbb R^2)=0 $,  $ H^1_{\mathrm{DR}}(\mathbb R^2\backslash \{0\})\neq 0 $.

Let us stop the discussion of de Rham cohomology for a moment, and move on to the next topic.
\section{Orientation and Integration of  Differential Form}
\subsection{Orientation on Manifold}
An orientation on a finite dimensional vector space  $ V $ is an equivalent class of ordered basis
\[\alpha=(\alpha_1,\cdots,\alpha_n)^T\sim\beta=(\beta_1,\cdots,\beta_n)^T\Leftrightarrow \det(\alpha\beta^T)>0\]
Each vector space has exactly two orientations. And we actually have the 1-1 correspondence

\[\{\text{orientation on  $ V $}\}\leftrightarrow(\Alt^n(V)\backslash\{0\})/_{\mathbb{R}^+}\]
\[[(e_1,\cdots,e_n)]\leftrightarrow [e_1^*\wedge\cdots\wedge e_n^*]\]
An \name{orientation form} on  $ M $ of dimension  $ n $  is a nowhere vanishing  $ \omega\in \Omega^n(M) $  \ie an orientation form is a nowhere vanishing section of  $ \Alt^n(TM) $.

Two orientation forms  $ \omega_1,\omega_2 $ are equivalent if  $ \exists f\in C^\infty(M,\mathbb{R}^+) $ \st  $ \omega_1=f\omega_2 $. An \name{orientation} on  $ M $ is an equivalent class of  orientation form.

An \name{orientation manifold} is a manifold that has an orientation.

An \name{oriented manifold} is a manifold equipped with an orientation.
\begin{example}
     $ |\pi_0(M)|=k $ $ \Rightarrow $  $ M $ has  $ 2^k $ orientations or no orientations.
\end{example}

\begin{example}
    \,
    \begin{enumerate}
        \item $ U\subset \mathbb{R}^n $ open.  $ U $ has a standard orientation, represented by the form  $ \dd x^1\wedge\cdots\wedge \dd x^n $. Denote this standard orientation as \name{$ \mathcal{O}_{\mathrm{std}} $}
        \item  $ (M,\mathcal{O}_M),(N,\mathcal{O}_N) $ oriented manifolds $ \Rightarrow $  $ (M\times N,\mathcal{O}_M\times \mathcal{O}_N) $. If  $ \mathcal{O}_M=[\omega_M],\mathcal{O}_N=[\omega_N] $, then  $ \mathcal{O}_M\times \mathcal{O}_N $ is defined by $ [\pi_M^*(\omega_M)\wedge\pi_N^*(\omega_N)] $. ($ \pi_M,\pi_N $ is the pullback of the projection map)
        \item  $ T^n ,S^n$ are orientable.
        \item  $ \mathbb{RP}^n $ orientable iff  $ n $ is odd.
    \end{enumerate}
\end{example}
\begin{proposition}
    Let  $ \mathcal{U}=\{U_\alpha\} $ be an open cover of  $ M $. Suppose we have an orientation  $ \mathcal{O}_\alpha $ on each  $ U_\alpha $ \st  $ \mathcal{O}_\alpha|_{U_\alpha\cap U_\beta}=\mathcal{O}_\beta|_{U_\alpha\cap U_\beta} $, $ \forall \alpha,\beta $. Then  $ \exists  $ unique orientation  $ \mathcal{O}_M $ on  $ M $ \st  $ \mathcal{O}_M|_{U_\alpha}=\mathcal{O}_\alpha $.
\end{proposition}
\begin{proof}
    For each  $ \alpha $, we have  $ \omega_\alpha\in \Omega^n(U_\alpha) $ nowhere-vanishing. And
    \begin{equation}
        \omega_\alpha|_{U_\alpha\cap U_\beta}=f_{\alpha\beta}\cdot\omega_\beta|_{U_\alpha\cap U_\beta},\,f_{\alpha\beta}:U_\alpha\cap U_\beta\rightarrow \mathbb{R}^+\label{eq:14}
    \end{equation}
    Take partition of unity subordinate to  $ \mathcal{U} $,  $ \{\varphi_\alpha\} $.

    Set  $ \omega=\dps\sum_\alpha\varphi_\alpha\cdot\omega_\alpha $. Then  $ \omega $ is nowhere-vanishing by \eqref{eq:14}.

    The uniqueness follows from the fact that  $ n $-form is equivalent if and only if it is equivalent on each chart.
\end{proof}
\begin{definition}
    Given  $ (M,\mathcal{O}_M),(N,\mathcal{O}_N) $.  $ f\in \Diff(M,N) $. Say  $ f $ is \name{orientation preserving} if  $ f^*(\mathcal{O}_N)=\mathcal{O}_M $.  $ f $ is \name{orientation reversing} if  $ f^*(\mathcal{O}_N)=-\mathcal{O}_M $.
\end{definition}