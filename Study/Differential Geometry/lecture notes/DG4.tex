\section{Tangent space and tangent vectors}
\subsection{Tangent Space}
Given  $ p\in M  $, consider the set  $ C_p^\infty(M)=\{\text{smooth function } V\rightarrow \mathbb{R}\}/_\sim $ where  $ f_1\sim f_2 $ if and only if  $ \exists $ neighbourhood  $ U $ of  $ p $,  $ f_1|_U=f_2|_U $.

 $ C_p^\infty(M) $ is the space of \name{genus of smooth function} near  $ p $. 

 
A \name{partial-derivative} of  $ p $ is a  $ \mathbb{R} $-linear map  $ D:C_p^\infty(M)\rightarrow \mathbb{R} $ that satisfies the Leibniz rule:
\[D(fg)=D(f)g(p)+f(p)D(g)\] 
\begin{definition}
    A \name{tangent vector} of  $ M  $ at  $ p  $ is a partial-derivative at  $ p $.
    
    Define the \name{tangent space}  $ T_pM=\{\text{all partial-derivative at  $ p $ }\} $, which is a  $ \mathbb{R} $-vector space.   
\end{definition}
\begin{proposition}
    For  $ M=U\subset \mathbb{R}^n $ open. We have  $ \{\dfrac{\partial }{\partial x_i}\} $ is a basis for  $ T_pU $.   
\end{proposition}
\begin{proposition}
    \[\frac{\partial }{\partial x^i}|_p=\sum\limits_{1 \leq i \leq n}\frac{\partial y^i}{\partial x^i}\cdot \frac{\partial }{\partial y^i}|_p\]
\end{proposition}
