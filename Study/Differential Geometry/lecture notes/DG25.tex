% !TEX root = lecture/Differential_Geometry.tex

\section{Classical Differential Geometry}
\subsection{The geometry of curves and surfaces}
A \name{parametrized curve} is a smooth map  $ \alpha:I=(a,b)\rightarrow \Rbb^3 $.  $ t(s)=\alpha'(s) $ is called the \subname{tangent vector}{parametrized curve}  at  $ \alpha(s) $.

A reparametrization of  $ \alpha $ means  $ (a',b')\xrightarrow{\cong}(a,b)\xrightarrow{\alpha}\Rbb^3 $.

We say  $ \alpha  $ is a \subname{regular}{parametrized curve} curve if  $ t(s)\neq 0 $,  $ \forall s\in I $.

We say  $ \alpha  $ is \subname{parametrized by its arc length}{parametrized curve} if  $ |t(s)|=1 $, $ \forall s\in I $. (It can be obtained by reparametrization)

Always assume  $ \alpha  $ is parametrized by its arc length.

Define  \name{$ K(s) $}$= |\alpha''(s)|\in \Rbb_{ \geq 0} $, the \name{curvature} of  $ \alpha $ at  $ \alpha(s) $.    

If  $ K(s)\neq 0 $, we may define the \name{normal vector}  $ n(s)=\dps\frac{\alpha''(s)}{K(s)} $. Then  $ 0=\dps\frac{\dd}{\dd s}\<\alpha'(s),\alpha'(s)\>=2\<\alpha''(s),\alpha'(s)\> $. $ \Rightarrow $  $ n(s)\perp t(s) $.   The \name{osculating plane} of  $ \alpha  $ at  $ \alpha(s) $ is  $ \Span\<t(s),n(s)\> $.   And \name{binormal vector}  $ b(s)=t(s)\times n(s) $. Then  $ t(s),n(s),b(s) $ is an orthogonormal basis  of  $ T_{\alpha(s)}\Rbb^3 $.  $ b'(s)=t'(s)\times n(s)+t(s)\times n'(s)\perp t(s),b(s) $. So  $ b'(s)=\tau(s)\cdot n(s) $ for some  $ \tau:I\rightarrow \Rbb $.  $ \tau(s) $ is called the \name{torsion} of  $ \alpha  $ at  $ \alpha(s) $.

\begin{proposition}[Frenet formula]\label{Frenet formula}
     $ \begin{cases}
        t'=Kn\\
        n'=-Kt-\tau b\\
        b'=\tau n
     \end{cases} $.
\end{proposition}
\begin{proof}
     $ n'(s)=(b(s)\times t(s))'=\tau(s) n(s)\times t(s)+b(s)\times(K(s)\cdot n(s))=-\tau(s)b(s)-K(s)t(s) $. The other two are straight forward. 
\end{proof}
\begin{theorem}
    Given any smooth function  $ K:I\rightarrow \Rbb_{>0} $,  $ \tau:I\rightarrow \Rbb $, there exists a curve  $ \alpha  $ parametrized by its arc length  $ s  $ \st  $ K(s) $ is its curvature and  $ \tau(s) $ is its torsion. Any two such curves differ by a rotation and a translation in  $ \Rbb^3 $. \ie  $ \forall  \alpha_1,\alpha_2$,  $ \exists \rho\in \mathrm{SO}(3) $,  $ c\in \Rbb^3 $ \st  $ \alpha_2=\rho\circ\alpha_1+C $.        
\end{theorem}
\begin{proof}
    Uniqueness: After rotation and translation, one can assume  $ \alpha_1(0)=\alpha_2(0) $,  $ t_1(0)=t_2(0) $,  $ n_1(0)=n_2(0) $,  $ b_1(0)=b_2(0) $. By  the uniqueness of ODE,  $ t_1(s)=t_2(s) $, $ \forall s\in I $. Hence  $ \alpha_1=\alpha_2 $.
    
    Existence: Solve the Frenet formula \ref{Frenet formula} with any initial  $ (t(0),n(0),b(0)) $. We get a local solution  $ t(s),n(s),b(s):I'\rightarrow \Rbb^3 $ with maximal domain. Want to show  $ I'=I $. Just need to show  $ |t(s)| $,  $ |n(s)| $,  $ |b(s)| $ are bounded when  $ s\in K\subset I $ in a compact set  $ K $.
    
    Then  $ \exists A\in \Rbb_{>0} $ \st  $ |K(s)|,|\tau(s)| \leq A  $, $ \forall s\in K $.
    \[\frac{\dd }{\dd s}(|t(s)|+|n(s)|+|b(s)|) \leq |t'(s)|+|n'(s)|+|b'(s)| \leq 2A(|t(s)+|n(s)|+|b(s)|)\]
    \[\Rightarrow |t(s)|+|n(s)|+|b(s)| \leq e^{2As}\cdot(|t(0)|+|n(0)|+|b(0)|)\]
    So  $ I'=I $ by continuity.
    
    Let  $ \alpha(s)=\alpha'(0)+\int_0^st(s)\dd x $. Then  $ \alpha  $ has the given curvature and torsion.  
\end{proof}

\subsection{Theory of surfaces in  $ \Rbb^3 $}
A smooth embedded surface  $ S\xhookrightarrow{\iota} \Rbb^3 $ is called a \name{regular surface}.

We identify  $ S  $ with  $ \iota(s) $. Then  $ T_pS\subset T_p\Rbb^3=\Rbb^3 $, $ \forall p\in S $.

 $ \forall p\in S $, we pick local chart  $ X:V\rightarrow U \opensub S\hookrightarrow \Rbb^3,(u,v)^T\mapsto (x(u,v),y(u,v),z(u,v))^T $, called a \subname{local parametrization}{regular surface} of  $ S $.   

 $ \subname{$ X_u $}{regular surface!local parametrization}:=X_*(\dps\frac{\partial}{\partial u})=(\dps\frac{\partial x}{\partial u},\dps\frac{\partial y}{\partial u},\dps\frac{\partial z}{\partial u})^T   $.

 $ \subname{$ X_v $}{regular surface!local parametrization}:=X_*(\dps\frac{\partial}{\partial v})=(\dps\frac{\partial x}{\partial v},\dps\frac{\partial y}{\partial v},\dps\frac{\partial z}{\partial v})^T   $.

The standard restrction on  $ \Rbb^3 $ restricts to a Riemann matrix on  $ S $ \ie  $ \forall p\in S $, we have a symmetric, positive definite bilinear form  $ \<-,-\>_p:T_pS\otimes T_pS\rightarrow\Rbb,\<\omega_1,\omega_2\>_p=\<\omega_1,\omega_2\>_{\Rbb^3} $.

$ \<-,-\> $ is determined by the quadratic form  $ \Romannumer1_p:T_pS\rightarrow \Rbb, \vec{v}\mapsto |\vec{v}|^2 $ called the \subname{1st fundamental form}{regular surface}

\begin{example}
     $ S=\mathrm{graph}(f)=\{(u,v,f(u,v))\} $. Then 
     \[X_u=\begin{pmatrix}
        1\\0\\\dps\frac{\partial f}{\partial u}
     \end{pmatrix},\,X_v=\begin{pmatrix}
        0\\1\\\dps\frac{\partial f}{\partial v}
     \end{pmatrix}\] 
     So  $ \dps \Romannumer1_p(aX_u+b X_v)=a^2+b^2+(a\frac{\partial f}{\partial u}+b\frac{\partial f}{\partial v})^2 $ 
\end{example}
Recall that the induced volume form on   $ \Rbb^2 $ is   $ \dd\Vol=\sqrt{\det(g)}\dd u\dd v=|X_u\times X_v|\dd u\dd v  $. 

\subsection{Gauss map}
From now on, assume  $ S  $ is oriented, pick oriented local parametrization  $ X $. Then  $ \forall p\in S $,  $ \exists N(p)\in \Rbb^3 $ \st  $ |N(p)|=1 $,  $ N(p)\perp T_pS $,  $ (X_u,X_v,N(p)) $ oriented basis of  $ \Rbb^3 $.  $ N(p)  $ is called the \subname{normal vector}{regular surface} of  $ S  $ at  $ p $. 
\name{Gauss map } is $ N:S\rightarrow S^2,p\mapsto N(p) $.

Take the differential at  $ p $,  $ \dd N_p=N_{*,p}:T_pS\rightarrow T_{N(p)}S^2=N(p)^{\perp}=T_pS $.

\begin{proposition}
     $ \dd N_p:T_pS\rightarrow T_pS $ is symmetric or equivalently self-adjoint. \ie 
     \[\<\dd N_p(\omega_1),\omega_2\>_p=\<\omega_1,\dd N_p(\omega_2)\>_p,\,\forall \omega_2 \in T_pS\]
\end{proposition}
\begin{proof}
    Take local parametrization $ X :V\rightarrow S\subset \Rbb^3 $,  $ (u,v)\mapsto p $.
    By linearity, it suffices to check  $ \omega_1=X_u $ and  $ \omega_2=X_v $.
    
     $ \<N(u,v),X_u\>=0 $  $ \Rightarrow\<\dps\frac{\partial N }{\partial v},X_{uv}\>+\<N(u,v),X_{uv}\>=0 $. So  $ \<\dd N_p(X_v),X_u\>=-\<N(u,v),X_{uv}\>=\<X_u,\dd N_p(X_u)\> $ 
\end{proof}
\begin{remark}\label{remark1}
    Here we actually obtain 
    \[-\<N_v,X_u\>=\<N,X_{u,v}\>\]
    Similarly, one can prove
    \[-\<N_u,X_v\>=\<N,X_{v,u}\>\]
    \[-\<N_u,X_u\>=\<N,X_{u,u}\>\]
    \[-\<N_v,X_v\>=\<N,X_{v,v}\>\]
\end{remark}
Define the quadratic form  $ \Romannumer2_p:T_pS\rightarrow \Rbb $,  $ \omega\mapsto -\<\dd N_p(\omega),\omega\> $. The induced form is  called the \subname{2nd fundamental form}{regular surface}  

\begin{definition}
    Given  $ p\in S $, a curve  $ \alpha:I\rightarrow S\subset \Rbb^3, 0\mapsto p $. Let  $ n_p  $ be the normal vector of  $ \alpha  $ at  $ p  $. Let  $ \theta $ be the angle between  $ n_p $ and  $ N(p) $.  $ K  $ is the curvature of  $ \alpha  $ at  $ p $.  $ K_n:=K\cdot\cos\theta $ is called the \subname{normal curvature}{regular surface}. Then if  $ \alpha $ is parametrized by arc length,  $ K_n=\<N(p),\alpha''(s)\> $. Since $ \<N(\alpha(s)),\alpha'(s)\>=0 $ $ \Rightarrow $ 
    \[\<\dd N_p(\alpha'(0)),\alpha'(0)\>+\<N(\alpha(0)),\alpha''(0)\>=0\Rightarrow K_n=-\<\dd N_p(\alpha'(0)),\alpha'(0)\>=\Romannumer2_p(\alpha'(0))\]

    
\end{definition}
\begin{theorem}[Meusnier]\label{Meusnier theorem}
    All curves in  $ S $ through  $ p\in S $, with same tangent vector  $ v\in T_pS $ at  $ p $,  $ |v|=1 $, have the same normal curvature  $ K_n=\Romannumer2_p(v) $. In particular,  $ L_v=\Span\<N_p,v\> $,  $ \alpha_v=L\cap S $. Then for  $ \alpha_v $,  $ K_n=\pm K\text{ at }p=\Romannumer2_p(v) $. 
    
    So we call  $ \Romannumer2_p(v) $ the normal curvature of  $ S $ in the direction of  $ v\in T_pS $.  
\end{theorem}
% \tikzset{external/export=true}