\subsubsection{Divergence of vector fields}
\begin{definition}
    For Riemannian metric space  $ (M,g) $,  $ X\in \Gamma(TM) $. Define  $ \div X\in C^\infty(M)$ by 
    \begin{equation}
        \dd(\iota_X \Vol)=\div(X)\Vol\label{eq:def of divergence}
    \end{equation}
    Then by Cartan's formula, 
    \begin{equation}\label{eq:Lie derivative of volume form}
        \mathcal{L}_X\Vol=\dd(\iota_X\Vol)=\div(X)\Vol
    \end{equation}
\end{definition}
Let  $ \varphi_t:U\rightarrow M $ be the flow for  $ X $.
\begin{equation}
    \frac{\dd }{\dd t}\varphi_t^*(\Vol)|_{t=t_0}=\varphi_{t_0}^*(\mathcal{L}_X\Vol)=\varphi_{t_0}^*(\div X\cdot \Vol)
\end{equation}  

If  $ D  $ is an integration domain of  $ M $,  a small ball for example. Then define 
\begin{equation}
    \name{$ \Vol(D) $}=\int_D\Vol
\end{equation}

Now  $ \Vol(\varphi_t(D))=\int_{\varphi_t(D)}\Vol=\int_D\varphi_t^*(\Vol) $.

\begin{equation}
    \begin{aligned}
        \frac{\dd }{\dd t}\Vol(\varphi_t(D))&=\int_D\varphi_t^*(\div X\Vol)\\
        &=\int_{\varphi_t(D)}\div X\Vol
    \end{aligned}
\end{equation}

If  $ \div X|_p>0 $ for all  $ p $, then  $ \varphi_t $ is volume increasing. If  $ \div X|_p=0 $ for all  $ p $, then  $ \varphi_t $ is volume preserving. If  $ \div X|_p<0 $ for all  $ p  $, then  $ \varphi_t  $ is volume decreasing. 

\subsubsection{Hamiltonian vector fields on symplectic manifolds}
A \name{symplectic structure} on  $ M  $  is a  $ 2 $-form  $ \omega\in \Omega^2(M) $, called \name{symplectic form} \st (1)  $ \dd\omega=0 $ (2) $ \omega  $ is non-degenerate everywhere. \ie For  $ \forall v\in T_pM\neq 0 $, there exists  $ w\in T_pM $ such that  $ \omega(v,w)=0 $,  $ \omega^{\frac{\dim M}{2}} $ nowhere vanishing equivalently.   

\begin{example}
    For  $ M=T^*N $, we have a canonical form  $ \alpha\in \Omega^1(M) $ defined as follows:

    $ \forall p\in T_q^*N\subset M $,  $ v\in T_pM $, it induces a canonical map  $ \pi:M\rightarrow N, p\mapsto q $. Then its pullback  $ \pi_*:T_pM\rightarrow T_qN,v\mapsto \pi_*v $ induces  $ \alpha_p(v)=p(\pi_*(v))\in\Rbb $.
    
    Then  $ \omega=\dd\alpha  $ is a canonical symplectic structure on  $ T^*N $. 
\end{example}
\begin{proof}
    Take local chart $ (U,x^1,\cdots,x^n) $ of  $ N $, 

    \begin{center}
        \begin{tikzcd}
            U\arrow[r,"x^i"]&\tilde{U}\opensub \Rbb^n\\
            M\supset T^*U\arrow[r,"\cong"]&T^*\tilde{U}\cong\tilde{U}\times \Rbb^n\opensub \Rbb^{2n}\\
            \dps\sum_{i=1}^np_i\dd x^i\arrow[r,mapsto]&(x^1(q),\cdots,x^n(q),p_1,\cdots,p_n)
        \end{tikzcd} 
    \end{center}
    
    Then  $ \alpha=\dps\sum_{i=1}^np_i\dd x^i $,  $ \omega=\dps\sum_{i=1}^n\dd p_i\wedge\dd x^i $.
    
    \[\omega^n=n!\dd p_1\wedge\dd x_1\wedge\dd p_2\wedge\dd x_2\wedge\cdots\wedge\dd p_n\wedge\dd x_n\]
    So  $ \omega  $ is non-degenerate everywhere. 
\end{proof}

\begin{definition}
    For  $ (M,\omega) $ symplectic,  $ f\in C^\infty(M) $,  $ \dd f\in \Omega^1(M) $.  $ \omega  $ is non-degenerate everywhere. Define  
    \begin{equation}
        \iota_{-}\omega:\Gamma(TM)\rightarrow \Gamma(T^*M), \, X\mapsto \mathcal{L}_X\omega
    \end{equation}
    
    Define the \name{Hamiltonian vector field}  $ X_f $ by $ \iota_{X_f}\omega=\dd f $. The flow generated by  $ X_f  $ is called the \name{Hamiltonian flow}  $ \phi:U\rightarrow M $. 
\end{definition}

By Cartan's formula 
\begin{equation}
    \mathcal{L}_{X_f}\omega=\iota_{X_f}\dd \omega+\dd (\iota_{X_f}\omega)=\dd (\dd f)=0
\end{equation}

So the symplectic form is preserved under the Hamiltonian flow.

The motivation to study Hamiltonian flow is that in classical mechanics,  $ N=\Rbb^n $,  $ M=T^*N=\Rbb^n\times \Rbb^n $ configuration space. The movement of a system of  $ m  $ partial is given by  $ \gamma:\Rbb\rightarrow M $,  $ \gamma  $ satisfies the Hamiltonian equation 
\[\gamma'(t)=X_H|_{\gamma(t)}\] 
where  $ H:M\rightarrow \Rbb $ is the Hamiltonian. \ie  $ \gamma  $ is an integral curve of the Hamiltonian flow.  

\subsection{Frobenius Theorem}
The motivation is to solve the PDE equation for  $ f:U\rightarrow \Rbb $ such that
\begin{equation}
    \begin{cases}
        \dps\frac{\partial f}{\partial x}=\alpha(x,y,f(x,y))\\
        \dps\frac{\partial f}{\partial y}=\beta(x,y,f(x,y))\\
    \end{cases}
\end{equation}
where initial value is  $ f(x_0,y_0)=z_0 $.

Consider vector fields on  $ \Rbb^3 $,  $ X_1=\dps\frac{\partial }{\partial x}+\alpha(x,y,z)\frac{\partial }{\partial z} ,X_2=\dps\frac{\partial  }{\partial y}+\beta(x,y,z)\frac{\partial  }{\partial z}$. Then  $ f  $ is a solution iff  $ N:\mathrm{graph}(f)=\{(x,y,f(x,y))\}\hookrightarrow\Rbb^3 $ satisfies  $ (x_0,y_0,z_0)\in N $,  $ T_pN=\Span(X_{1,p},X_{2,p}) $.

\begin{definition}
     $ M^n  $ is a smooth manifold. A  $ k $-dimensional \name{tangent distribution}  $ D  $ is a  $ k $-dimensional linear subspace  $ D_p\subset T_pM $ at each  $ p\in M $ \st  $ D=\dps\bigsqcup_{p\in M}D_p $ is a smooth subbundle of  $ TM $.      

     Equivalently, this means for any  $ p\in M  $, there exists a neighborhood $ U  $  of  $ p $, and smooth vector fields  $ Y_1,\cdots,Y_k $ on  $ U  $ \st  $ D_q=\Span\<Y_{1,q},\cdots,Y_{k,q}\> $,  $ \forall q\in U $.

     An immersed submanifold  $ \varphi:N\looparrowright M  $ is called an \name{integral manifold} for   $ D  $ if  $ T_qN=D_q $,  $ \forall q\in N $.  \ie  $ \varphi_*(T_qN)=D{\varphi(q)} $. Identify  $ N $ with  $ \varphi(N) $ to simplify notation.
    
\end{definition}

\begin{example}
    Every nowhere vanishing vector field is a  $ 1 $-dimensional distribution,  and the integral curve of it is the integral manifold.
\end{example}

\begin{example}
    For  $ \dps M=\mathbb{T}^2=\Rbb\times\Rbb/_{\Zbb\times \Zbb} $,  $ D=\dps\Span(\frac{\partial}{\partial x}) $. Then the integral submanifold is  $ S^1\times\{y\} $. For  $ D=\dps\Span(\frac{\partial }{\partial x}+\sqrt{2}+\frac{\partial }{\partial y}) $, integral submanifold  $ \{[y,x]|y-y_0=\sqrt{2}(x-x_0)\} $ is an immersed but not embedding submanifold dense in  $ \mathbb T^2 $. 
\end{example}
\begin{example}
     $ M=\Rbb^n  $, $ D=\Span\left\{\dps\frac{\partial }{\partial x^1},\cdots,\frac{\partial }{\partial x^k}\right\} $ has integral manifold  $ N=\Rbb^k\times\{\boldsymbol{c}\} $,  $ \boldsymbol{c}\in \Rbb^{n-k} $.  
\end{example}

\begin{example}
     $ M=\Rbb^n\backslash\{0\} $,  $ D_p=p^\perp $  is an  $ n-1 $  dimensional distribution with integral submanifold  $ N=S_r^{n-1}=\{\boldsymbol{x}:\|\boldsymbol{x}\|=r\} $.
\end{example}

\begin{example}
     $ M=\Rbb^3 $,  $ D=\Span\left\{\dps\frac{\partial}{\partial x}+y\frac{\partial }{\partial z},\frac{\partial }{\partial y}\right\} $. We claim that  $ D  $ has no integral submanifold.  
    
\end{example}
\begin{proof}[Proof of the Claim]
    Let  $ N  $ be an integral submanifold,  $ (0,0,0)\in N $.  $ \gamma:
    (-\epsilon,\epsilon)\rightarrow \Rbb^n,t\mapsto (t,0,0) $ is an  integral curve for  $ X_1=\frac{\partial}{\partial x}$. So  $ (x,0,0)\in N $, for  $ x\in (-\epsilon,\epsilon) $.
    
    Let  $ \eta:(-\epsilon',\epsilon')\rightarrow \Rbb^3,t\mapsto (x,t,0) $ be an integral curve for  $ X_2=\frac{\partial}{\partial y} $ starting at  $ (x,0,0) $. So for  $ (x,y,0) \in N$ for  $ x\in(-\epsilon,\epsilon),y\in(-\epsilon',\epsilon') $.
    
    That implies that  $ N  $ contains  $ X-Y $ plane near  $ (0,0,0) $ but now  $ T_pN=\Span\dps\<\frac{\partial }{\partial x},\frac{\partial}{\partial y}\>\neq D_p $.   
\end{proof}
\begin{remark}
    This example shows that a distribution doesn't have integral manifold always. This property that have integral manifold implies some "linear properties", or equivalently, it should be closed under the action by local flows in that case.
\end{remark}
\begin{definition}
    Let  $ D  $ be a  $ k $-dimensional distribution on  $ M $.
    
    \begin{enumerate}[label=(\arabic*)]
        \item We say  $ D $ is \name{involutive} if for any open  $ U\subset M $, any smooth sections  $ X,Y\in \Gamma(D|_U) $, we have  $ [X,Y]\in\Gamma(D|_U) $.
        \item We say  $ D $ is \name{integrable} for any  $ p\in M $, there exists an integral submanifold  $ N $ for  $ D $      \st  $ p\in N $
        \item We say  $ D $ is \name{completely integrable} if  $ \forall p\in M $, there exists local chart  $ (U,x^1,\cdots,x^n) $ around  $ p $ \st  $ D|_U=\dps\Span\<\frac{\partial}{\partial x^1},\cdots,\frac{\partial}{\partial x^k}\> $      
    \end{enumerate}
\end{definition}
\begin{theorem}[Local Frobenius]\label{thm:local-frobenius}
    Those definitions above is equivalent.
\end{theorem}