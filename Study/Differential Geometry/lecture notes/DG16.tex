% !TEX root = lecture/Differential_Geometry.tex

\begin{lemma}\label{sec6.1:orientation preserving is equivalent to detereminant positive at any point}
    $ U_1,U_2\subset \mathbb R^n $ open. Then  $ f:(U_1,\mathcal{O}_{\mathrm{std}})\rightarrow(U_2,\mathcal{O}_{\mathrm{std}}) $ is orientation preserving iff 
   \begin{equation*}
       \forall p\in  U_1,\det(\mathrm{D}f|_p)>0\quad \mathrm{D}f=\left(\dps\frac{\partial f^i}{\partial x^j}\right)_{1 \leq i,j \leq n}
   \end{equation*} 
\end{lemma}
\begin{proof}
   For  $ \stdO=\left[\dd x^1\wedge\cdots\wedge\dd x^n\right] $,
   \begin{equation*}
       \begin{aligned}
           f^*(\stdwedge{x})&=\stdwedge{f},\quad \dd f^i=\sum_{i=1}^n\frac{\partial f^i}{\partial x^i}\dd x^i\\
           &=\det(\mathrm{D}f)\stdwedge{x}
       \end{aligned}
   \end{equation*} 
   Then 
   \begin{equation*}
       \begin{aligned}
           \det(\mathrm{D}f)>0&\Leftrightarrow f^*(\stdwedge{x})\sim\stdwedge{x}\\
           &\Leftrightarrow f^*\stdO=\stdO\\
           &\Leftrightarrow\text{ $ f $ is orientation preserving }
       \end{aligned}
   \end{equation*}
\end{proof}
Given  $ (M,\mc O) $,  $ p\in M $, a basis  $ e_1,\cdots,e_n $ of  $ T_pM $ is called \textbf{oriented} \index{oriented basis}if  $ \mathcal{O}_p=[(e_1,\cdots,e_n)] $.

A chart  $ U\xrightarrow{\varphi}V\overset{\text{open}}{\subset}\mathbb{R}^n $ is \textbf{oriented}\index{oriented chart} if  $ \varphi^*(\stdO)=\mathcal{O}|_U $.

A smooth atlas  $ \left\{U_\alpha\xrightarrow{\varphi_\alpha}V_\alpha\right\}_{\alpha\in\mathcal{A}} $ is \textbf{oriented}\index{oriented atlas} if each chart  $ U_\alpha\xrightarrow{V_\alpha} $ is oriented.

A smooth atlas  $ \left\{U_\alpha\xrightarrow{\varphi_\alpha}V_\alpha\right\}_{\alpha\in\mathcal{A}}  $ is called \textbf{positive}\index{positive atlas} if  $ \forall \alpha,\beta\in\mathcal{A} $,
\[\varphi_{\alpha\beta}=\varphi_\beta\circ\varphi_\alpha':\varphi_\alpha(U_\alpha\cap U_\beta)\rightarrow\varphi_\beta(U_\alpha\cap U_\beta)\text{ is orientation preserving}\] 
By Lemma \ref{sec6.1:orientation preserving is equivalent to detereminant positive at any point}, this is equivalent to  $ \det(\mathrm{D}\varphi_{\alpha\beta}|_p)>0 $ for any  $ p\in \varphi_\alpha(U_\alpha\cap U_\beta) $.

\subsection{Integration on Oriented Manifold}
\paragraph{Goal:} Given  $ M,\mathcal{O} $,  $ \omega\in  \name{$ \Omega_c^n(M) $} =\{\text{compactly supported  $ n $-form on  $ M $}\} $. Then  $ \Supp(\omega)=\overline{\{p\in M|\omega_p\neq 0\in \Alt^n(T_pM)\}} $ is compact.

We hope to define  $ \int_M\omega\in\Rbb $.

For  $ M\opensub \Rbb^n $,  $ \mathcal{O}=\stdO $. Then  $ \forall \omega\in \Omega_c^n(M) $,
\[\omega=f\stdwedge{x},\,f\in C_c^\infty(M)\]
Define  $ \int_M\omega=\int_M f\dd \mu $ where  $ \mu $ is the standard Lebesgue measure on  $ \Rbb^n $.

\begin{lemma}\label{Lemma:invariant integration under orientation preserving map}
    $ U,V\opensub\Rbb^n $, $ \varphi:U\xrightarrow{\cong}V $ is orientation preserving. Then  $ \forall \omega\in \Omega_c^n(V) $, we have  $ \int_U\varphi^*(\omega)=\int_V\omega $.    
\end{lemma}
\begin{proof}
   If $ \omega=f\stdwedge{x} $, then 
   \begin{equation}
       \begin{aligned}
           \varphi^*(\omega)&=\varphi^*(f)\wedge\stdwedge{\varphi}\\
           &=(f\circ\varphi)\det\left(\frac{\partial \varphi^i}{\partial x^j}\right)_{1 \leq i,j \leq n}\stdwedge{x^1}
       \end{aligned}
   \end{equation}
   So  $ \int_U\varphi^*(\omega)=\dps\int_U (f\circ\varphi)\det\left(\frac{\partial \varphi^i}{\partial x^j}\right)_{1 \leq i,j \leq n}\dd \mu=\int_Vf\dd \mu=\int_V\omega$ 
     
\end{proof}
So we can define the integral over special  $ \omega $ and  general  $ M $.
\begin{definition}
   If  $ \omega\in\name{$ \Omega_s^n(M) $}=\{\text{$ n $-forms with "small" support}\}\\= \{\omega\in\Omega_c^n(M)|\exists\text{oriented chart  $ \varphi:U\xrightarrow{ \cong} $ \st }\Supp(\omega)\subset U\} $.
   
   We define  $ \int_M\omega:=\int_V\varphi^{-1,*}(\omega) $ 
\end{definition} 
\begin{claim}
   If  $ \Supp(\omega)\subset U_\alpha\cap U_\beta $, then  $ \int_{V_\alpha}\varphi_\alpha^{-1,*}(\omega)=\int_{V_\beta}\varphi_\beta^{-1,*}(\omega) $  
\end{claim}
\begin{proof}
   \begin{align*}
       \varphi_{\alpha\beta}:\varphi_\alpha(U_\alpha\cap U_\beta)&\xrightarrow{\cong}\varphi_\beta(U_\alpha\cap U_\beta)\\
       \varphi_\alpha^{-1,*}(\omega)&\mapsto \varphi_\beta^*(\omega)
   \end{align*}
   By Lemma \ref{Lemma:invariant integration under orientation preserving map}, we have  $ \int_{V_\alpha}\varphi_\alpha^{-1,*}(\omega)=\int_{V_\beta}\varphi_\beta^{-1,*}(\omega) $ 
\end{proof}
\begin{theorem}
   For any oriented  $ (M,\mathcal{O}) $,  $ \exists $ unique linear map  $ \int_M:\Omega_c^n(M)\rightarrow\Rbb $ that extends  $ \int_M:\Omega_s^n(M)\rightarrow \Rbb $.   
\end{theorem}
\begin{proof}
   \,

   
   {\noindent\textbf{Step1:} There exists an oriented atlas  $ \mathcal{U}=\{\varphi_\alpha:U_\alpha\rightarrow V_\alpha\opensub\Rbb^n\}_{\alpha\in \mathcal{A}} $.
   }
   Indeed, pick any smooth atlas  $ \mathcal{U}=\{U_\alpha\xrightarrow{\varphi_\alpha} V_\alpha\}_{\alpha\in\mathcal{A}} $. By replacing  $ \varphi_\alpha $ with  $ r\circ \varphi_\alpha $ where  $ r(x_1,\cdots,x_n)=(-x_1,\cdots,x_n) $. We can get the oriented atlas  $ \mathcal{U}' $.
   
   {\noindent\textbf{Step2.} Pick a partition of unity subordinate to  $ \mathcal{U} $, $ \{\varphi_\alpha:M\rightarrow [0,1]\} $ }

   Now we begin the main proof:

   Let  $ \omega_\alpha=\rho_\alpha\cdot \omega $.  $ \Supp(\omega_\alpha)\subset \Supp(\rho_\alpha)\cap \Supp(\omega)\subset U_\alpha $. And  $ \omega_\alpha\in\Omega_s^n(M) $ 
   \begin{claim}
        $ \omega_\alpha\neq=0 $ for only finite many  $ \alpha\in\mathcal{A} $ 
   \end{claim}  
   \begin{proof}
        $ \forall p\in \Supp(\omega) $,  $ \exists $ neighbourhood $ W_p $ only intersects  $ \Supp(\rho_\alpha) $ for finitely many  $ \alpha $.
       
        Since  $ \{W_p\}_{p\in\Supp(\omega)} $ is an open cover of  $ \Supp(\omega) $, by compactness,  $ \Supp(\omega) $ only intersects  $ \Supp(\rho_\alpha) $ for finitely many  $ \alpha $.
        
       Therefore,  $ \omega_\alpha\neq0 $ for only finitely many  $ \alpha $  
   \end{proof}
   By this claim, since  $ \dps\sum_{\alpha\in\mathcal{A}}\rho_\alpha=1 $ $ \Rightarrow $  $ \omega=\omega_{\alpha_1}+\cdots+\omega_{\alpha_n} $ for some  $ \alpha_1,\cdots,\alpha_k\in\mathcal{A} $.
   We may define 
   \begin{equation}
       \int_M\omega=\dps\sum_{\alpha\in\mathcal{A}}\int_M\omega_\alpha=\int_M\omega_{\alpha_1}+\cdots+\int_M\omega_{\alpha_k}\in \Rbb
   \end{equation}
   This proves existence  $ \int_{(M,\mathcal{O},\mathcal{U},\{\rho_\alpha\})}:\Omega_c^n(M)\rightarrow\Rbb $.
   
   \textbf{Uniqueness:} $ \forall \omega\in \Omega_c^n(M) $,  $ \exists \omega_1,\cdots,\omega_k\in\Omega_s^n(M) $,  $ \omega=\omega_1+\cdots+\omega_n $ as the claim proved.
   
   So  $ \dps\int_M\omega=\sum_{i=1}^n\int_M\omega_i $ is uniquely defined. 
\end{proof}
\begin{remark}\label{Remark:Every form with compact support can be expressed as the sum of form with small support}
   We actually obtain that each  $ \omega\in \Omega_c^n(M) $ can be expressed as  $ \dps\sum_{k=1}^n\omega_k $ where  $ \omega_k\in\Omega_s^n(M) $.   
\end{remark}
\begin{proposition}
   \,
   \begin{enumerate}
       \item  $ f:(M,\mathcal{O}_M)\xrightarrow{\cong}(N,\mathcal{O}_N) $. If  $ f $ is orientation preserving, then  $ \int_Mf^*(\omega)=\int_N\omega $. If  $ f $ is orientation reversing, then  $ \int_Mf^*(M)=-\int_N\omega $. 
       \item  $ \int_{(M,\mathcal{I})}\omega=-\int_{(M,\overline{\mathcal{O}})}\omega $.
       \item If  $ \Supp(\omega)\subset U\opensub M $, then  $ \int_M\omega=\int_U\omega $.       
   \end{enumerate}
\end{proposition}
\begin{proof}
   Leave as exercise.
\end{proof}

\subsection{Smooth Manifold with Boundary}
Now let  $ M $ be the smooth manifold with boundary,  $ \partial M=N $. \ie  $ M $ has a smooth atlas  $ \{\varphi_\alpha:U_\alpha\xrightarrow{\cong}V_\alpha\opensub\Rbb_{\leq 0}\times\Rbb^{n-1}\} $.

We can define  $ T_pM $,  $ TM $, $ \Alt^k(M) $, $ \Omega^k(M) $,  $ \Omega_c^k(M) $,  $ \Omega_s^k(M) $ and orientation similar as before.

For  $ p\in\partial M $,  $ X\in T_pM $ is called \name{outward}  if  $ \exists $ local chart  $ \varphi:U\xrightarrow{\cong}V $ around  $ p $ \st 
\[\varphi_{p,*}(X)=a_1\partial x^1+\cdots+a_n\partial x^n\text{ with }a_1>0\]   

Recall that if  $ M $ is  $ n $ dimensional manifold with boundary, then  $ N=\partial M $ is a  $ n-1 $ dimensional manifold without boundary.  

\begin{proposition}
   For any orientation  $ \mathcal{O}_M $ on  $ M $,  $ \exists $  a unique induced orientation  $ \mathcal{O}_N $ on  $ N=\partial M $  \st  $ \forall p\in N $,  $ X=T_pM $ outward,  $ e_2,\cdots,e_n\in T_pN $  is an oriented basis. Moreover,  $ (X,e_2,\cdots,e_n) $ is oriented basis of  $ T_pM $.       
\end{proposition}
\begin{proof}
   Take oriented atlas  $ \mathcal{U}=\{\varphi_\alpha:U_\alpha\xrightarrow{\cong}V_\alpha\}_{\alpha\in \mathcal{A}} $.
   
   We have  $ \varphi_\alpha(U_\alpha\cap U_\beta)\subset \{0\}\times \Rbb^{n-1} $.
   
   Define  $ \psi_\alpha:N\cap U_\alpha\xrightarrow{\cong}(\{0\}\times \Rbb^{n-1})\cap U_\alpha $. Then  $ \mathcal{U}'\{\psi_\alpha\}  $  is a smooth atlas for  $ N $.
   
    $ \mathcal{U} $ is oriented implies  $ \mathcal{U} $ is positive, so is  $ \mathcal{U} $. So there exists the unique  $ \mathcal{O}_N $ \st  $ \mathcal{U}' $ is  oriented.
\end{proof}
\begin{theorem}[Stokes' Theorem]\label{Stokes Theorem}
    $ M $ is  $ n $ dimensional manifold with boundary, oriented by  $ \mathcal{O}_M $.
     $ N=\partial M $, with induced orientation  $ \mathcal{O}_N $.  $ \iota:N\hookrightarrow M $ is the inclusion map. Then  $ \forall \omega\in \Omega_c^{n-1}(M) $, we have 
   \begin{equation}
       \int_M\dd \omega=\int_N \iota^*(\omega)
   \end{equation}       
\end{theorem}
\begin{proof}
    By Remark \ref{Remark:Every form with compact support can be expressed as the sum of form with small support}$ \forall \omega\in \Omega_c^{n-1}(M) $,  $ \omega=\omega_1+\cdots+\omega_k $,  $ \omega_j\in\Omega_s^n(M) $.
    
    By linearity, we may assume  $ \omega\in \Omega_s^n(M) $,  $ \Supp(\omega)\subset U_\alpha $ for chart  $ \varphi_\alpha:U_\alpha\xrightarrow{\cong}V_\alpha $.
    
     $ \varphi_\alpha^{-1,*}(\omega)\in\Omega_c^n(V_\alpha) $ induces  $ \omega'\in \Omega_c^n(\Rbb_{\leq 0}\times \Rbb^{n-1}) $ if we extend it by 0.
     
     By considering  $ \omega' $ instead of  $ \omega $, we may assume  $ M=\Rbb_{\leq 0}\times\Rbb^{n-1} $ and we just need to prove 
     \[\int_{\Rbb^{n-1}}\iota^*(\omega')=\int_{\Rbb_{\leq 0}\times \Rbb^{n-1}}\dd \omega'\]
     Let  $ \omega'=\sum_{k=1}^nf^k\dd x^1\wedge\cdots\wedge \hat{\dd x^k}\wedge\cdots\wedge\dd x^n $.
     
     By linearity, we may assume  $ \omega'=f^k\dd x^1\wedge\cdots\wedge\hat{\dd x^k}\wedge\cdots\wedge\dd x^n $.
     
    \begin{equation}
        \begin{aligned}
            \dd \omega'&=\left(\dps\sum_{i=1}^n\frac{\partial f^k}{\partial x^i}\dd x^i\right)\\
            &=\frac{\partial f^k}{\partial x^k}(-1)^{k-1}\dd x^1\wedge\cdots\wedge\dd x^n
        \end{aligned}
    \end{equation}
    
    For  $ k=1 $,  
    
    \begin{equation*}
        \begin{aligned}
            \int_{\Rbb_{ \leq 0}\times \Rbb^{n-1}}\dd \omega'&=\int_{\Rbb^{n-1}}\left(\int_{\Rbb_{\leq 0}}\frac{\partial f^1}{\partial x^1}\dd x^1\right)\dd x^2\wedge\cdots\wedge x^n\\
            &=\int_{\Rbb^{n-1}}\left(f^1(0,x_2,\cdots,x_n)-\dps\lim_{x_1\to\infty}f^1(x_1,x_2,\cdots,x_n)\right)\dd x^2\cdots\wedge \dd x^n\\
            &=\int_{\Rbb^{n-1}}f^1(0,x^2,\cdots,x^n)\dd x^2\wedge\dd x^3\wedge\cdots\wedge\dd x^n\\
            &=\int_{\Rbb^{n-1}}\iota^*(\omega')
        \end{aligned}
    \end{equation*}

    For  $ k\neq 1 $, 
    
    \begin{equation*}
        \begin{aligned}
            \int_{\Rbb_{\leq 0}\times \Rbb^{n-1}}\dd \omega'&=\int_{\Rbb_{\leq 0}\times \Rbb^{n-1}}\left(\int_{\Rbb_{\leq 0}}\frac{\partial f^k}{\partial x^k}\dd x^k\right)\\
            &=\int_{\Rbb^{n-1}}\left(\dps\lim_{x_k\to+\infty}f^k(x_1,x_2,\cdots,0,\cdots,x_n)-\dps\lim_{x_k\to-\infty}f^k(x_1,x_2,\cdots,x_n)\right)\\
            &\quad\dd x^1\wedge\dd x^2\cdots\wedge\hat{\dd x_k}\wedge\cdots\wedge \dd x^n\\
            &=0\\
            &=\int_{\partial M}\iota^*(\omega')
        \end{aligned}
    \end{equation*}
\end{proof}
