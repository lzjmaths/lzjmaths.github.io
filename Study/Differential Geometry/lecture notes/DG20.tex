%! TEX root = lecture/Differential_Geometry.tex

\begin{proof}
    Completely integrable $ \Rightarrow  $ Integrable: In  $ U  $, all submanifolds of the form  $ \Rbb^k\times\{ \mathbf{c}\} $ are integral submanifold for  $ D $.
    
    Integrable  $ \Rightarrow  $ Involutive:  For  $ X,Y\in \Gamma(D) $,   $ p\in M $. Let  $ \iota:N\looparrowright M $ be integral submanifold  $ \st p\in N $.
    
    Then  $ [X,Y]|_p=[X|_N,Y|_N]_p\in T_pN=D_p $, since  $ X ,Y  $ is  $ \iota $-related to  $ X|_N,Y|_N  $ resepectively.  So  $ [X,Y]\in \Gamma(D) $.
    
    Involutive  $ \Rightarrow  $ Completely integrable.  $ \forall p\in M  $, take local chart  $ (V,y^1,\cdots,y^n ) $. WLOG,  $ y(p)=0 $. Then  $ \pi=(y^1,\cdots,y^k):V\rightarrow \Rbb^k,p\mapsto \mathbf{0} $. It induces $ \pi_{q,*}:T_qV\rightarrow T_{\pi(q)}\Rbb^k=\Rbb^k $. 
    
    Since $ D_p$ has dimension   $ k $,
    by shrinking  $ V  $, and reordering the coordinate such that  $ D_q $ is  disjoint with  $\dps \frac{\partial}{\partial x^{k+1}},\cdots \frac{\partial}{\partial x^n}$,   we may assume  $ \pi_{q,*}|_{D_q}:D_q\xrightarrow{\cong}\Rbb^k,\,\forall q\in V $.
    
    Consider  $ \partial y^i\in \Gamma(T\Rbb^k),\,1 \leq i \leq k $. Let  $ X_i  $ be the unique section of  $ D|_V $ \st   $ \pi_*(X_i)=\partial y^i $.  $ X_i\in \Gamma(D|_V)\subset \Gamma(TM) $. Then  $ \partial y^i $ is  $ \pi  $-related to  $ X_i $.
    
    Then  $ 0=[\partial y^i,\partial y^j] $ is  $ \pi $-related to  $ [X_i,X_j] $. So  $ \pi_*[X_i,X_j]_q=0 ,\,\forall q\in V$ $ \Rightarrow  $  $ [X_i,X_j]_q=0,\,\forall q\in V $. 
    
    Thus, we obtain linear independent vector fields with  $ [X_i,X_j]=0,i\neq j $.
    
    By canonical form of commuting vector fields \ref{Canonical Form of Commuting Vector Field}, there exists local charts  $ (U,x^1,\cdots,x^n) $  \st  $ X_i=\partial x^i $,  $ 1 \leq i \leq k $. So  $ D|_U=\Span\{\partial x^1,\cdots,\partial x^k\} $   
\end{proof}
\begin{remark}
    Indeed, completely integrable means more. We can find an embedding integrable manifold locally for completely integrable.
\end{remark}
\begin{lemma}
    Let  $ D  $ be a  $ k $-dimensional distribution. Then  $ D $ is involutive iff there exists an open cover  $ \mathcal{U} $,   $ \forall U\in\mathcal{U} $,  $ \exists X_1,\cdots,X_k\in \Gamma(TU) $ \st  $ D|_U=\Span\{X_1,\cdots,X_k\} $ and  $ [X_i,X_j]\in \Gamma(D|_U) $    
\end{lemma}
\begin{proof}
    "$ \Rightarrow $" is followed by the definition.
    
    "$ \Leftarrow $" Just need to show  $ D|_U $ is involutive for each  $ U $.
    
    $ \forall X,Y\in \Gamma(D|_U) $,  $ X=\dps\sum_{i=1}^nf^iX_i,Y=\sum_{i=1}^n g^iX_i $,  $ f^i,g^i:U\rightarrow \Rbb $.
    Then 
    \begin{equation}
        \begin{aligned}
            [X,Y]&=\sum_{i,j}f^ig^j[X_i,X_j]+\sum_{i,j}f^iX_i(g^j)X_j-\sum_{i,j}g^jX_j(f)X_i\in \Gamma(D)
        \end{aligned}
    \end{equation}
\end{proof}
\begin{corollary}
    $ \dps\begin{cases}
        \dps\frac{\partial f}{\partial x}=\alpha(x,y,f(x,y))\\
        \dps\frac{\partial f}{\partial y}=\beta(x,t,f(x,y))
    \end{cases}, f(x_0,y_0)=z_0 $ has local solution for  $ \forall (x_0,y_0,z_0) $ iff  \[ \dps\frac{\partial \alpha}{\partial y}+\beta\frac{\partial \alpha}{\partial z}=\frac{\partial \beta}{\partial x}+\alpha\frac{\partial \beta}{\partial z} \] 
\end{corollary}
\begin{proof}
    $ \Rightarrow $ is because of  $ \dps\frac{\partial^2 f}{\partial x\partial y}=\frac{\partial^2 f}{\partial y\partial x} $.
    
    "$ \Leftarrow $" Consider  $ D=\Span\{\frac{\partial }{\partial x}+\alpha\frac{\partial }{\partial z},\frac{\partial}{\partial y}+\beta\frac{\partial }{\partial z}\} $   $ 2 $ dimensional distribution on  $ \Rbb^3 $.
    \[[\frac{\partial }{\partial x}+\alpha\frac{\partial}{\partial z},\frac{\partial }{\partial y}+\beta\frac{\partial }{\partial z}]=\left(\frac{\partial \beta}{\partial x}+\alpha\frac{\partial \beta}{\partial z}-\frac{\partial \alpha}{\partial y}-\beta\frac{\partial\alpha}{\partial z}\right)\dps\frac{\partial}{\partial z}=0\]  
    So  $ D  $ is involutive  $ \Rightarrow  $  $ \forall (x_0,y_0,z_0) $,  $ \exists N\hookrightarrow \Rbb^3 $ \st  $ (x_0,y_0,z_0) \in N$,  $ N  $ is an integrable manifold of  $ D $.
    
    $ N\hookrightarrow \Rbb^3\xrightarrow{(x,y,z)\mapsto (x,y)}\Rbb^2 $ is a submersion. So  $ N  $ can be locally written as  $ (x,y,f(x,y)) $ for  $ f:U\rightarrow \Rbb $   
\end{proof}

A  $ k $-dimensional \name{foliation} is a decomposition  $ M=\dps\bigcup_s N_s $ \st (1) Each  $ N_s  $ is an injective immersed  $ k $-dimensional submanifold. (2)  $ \forall p\in M $,  $ \exists  $ local chart  $ (U,x^1,\cdots,x^n) $  \st  $ \forall s\in S $,  $ N_s\cap U=\Rbb^k\times A_s $.  $ A_s $ is a countable subset of  $ \Rbb^{n-k} $,  $ A_s $ is a countable subset of  $ \Rbb^{n-k} $        
\begin{example}
     $ \mathbb{T}^2=\dps\bigcup_s N_s $,  $ N_s=\Rbb\hookrightarrow \mathbb{T}^2, \,t\mapsto (x_0+t,\sqrt{2}t) $  
\end{example}
\begin{theorem}[Global Frobenius]
     $ D  $ is an involutive  $ k $-dimensional distribution  $ \Rightarrow $  $ D  $ induces a  $ k $-dimensional foliation  $ M=\dps\bigcup_s N_s $ such that  each  $ N_s  $ is a maximal integral submanifold of  $ D $. 
\end{theorem}

\section{de-Rham Cohomology}
\subsection{Basic Definition}
A \name{cochain complex} over  $ \Zbb $ is a graded abelian group  $ C=\dps\bigotimes_{b\in \Zbb}C^n $ with a deg-1 map  $ \dd=\dps\bigotimes_n\left(\dd^n:C^n\rightarrow C^{n+1}\right) $ \st  $ \dd^2=0 $.

$ \dd $ is  called \name{differential} or \name{boundary map}

The  $ k $-th \name{homology}
\begin{equation}
    H^k(C,d):=\frac{\ker(\dd^k:C^k\rightarrow C^{k+1})}{\Imag(\dd^{k-1}:C^{k-1}\rightarrow C^{k})}
\end{equation} 
The denominator is the set of  \name{$ k $-boundary}. The numerator is the set of  \name{$ k $-cycles} 
\begin{example}
    $ C_{DR}^*(M)=\dps\bigotimes_{n\in \Zbb}\Omega^n(M) $ is homology equipped with the exterior differential $ \dd $. And 
    \begin{equation}
        H^k(C_{DR}^*(M),\dd)=H_{DR}^*\cong H^k(M;\Rbb)
    \end{equation}
    where  $ H^k(M;\Rbb) $ is the singular cohomology.
\end{example}
Given cochain complexed  $ (A,\dd_A),(B,\dd_B) $, a \name{chain map}  $ f:(A,\dd_A)\rightarrow (B,\dd_B) $ is a degree 0(\ie  $ f(A^n)\subset B^n $) group homomorphism \st  $ \dd_B\circ f=f\circ \dd_A $.

If $ f  $ is a chain map, then  $ f $ maps  $ k $-cycles to  $ k $-cycles and  $ k $-boundariise to  $ k $-boundaries. So  $ f $ induces  $ f^*:H^k(A,\dd_A)\rightarrow H^k(B,\dd_B) $.
\begin{example}
    $ f:\in C^\infty(M,N) $ $ \Rightarrow  $  $ f^*:\Omega^k(N)\rightarrow \Omega^k(M) $  induces a chain map  $ f^*:C_{DR}^*(N)\rightarrow C_{DR}^*(M) $ $ \Rightarrow  $ $ f^*:H^*_{DR}(N)\rightarrow H_{DR}^*(M) $ satisfies  $ (f\circ g)^*=g^*\circ f^*,(\id)^*=\id $    
\end{example}  

Given chain maps  $ f,g:(A,\dd_A)\rightarrow (B,\dd_B) $, a \name{chain homotopy} from  $ f  $ to  $ g  $ is a deg-(-1) map  $s: A\rightarrow B $  \st  $ \dd_Bs+s\dd_A=f-g $. In this case we write  $ f\overset{s }{\simeq} g $.

\begin{center}
    \begin{tikzcd}
        A^{k-1}\arrow[r]\arrow[d,"f"]\ar[d,swap,"g"]&A^k\arrow[r]\arrow[d,"f"]\ar[d,swap,"g"]\arrow[ld,swap,"s"]&A^{k+1}\arrow[d,"f"]\ar[d,swap,"g"]\arrow[swap,ld,"s"]\\
        B^{k-1}\arrow[r]&B^k\arrow[r]&B^{k+1}
    \end{tikzcd}
\end{center}
