% Sample tex file for usage of iidef.sty
% Homework template for Inference and Information
% UPDATE: October , 2017 by Xiangxiang
% UPDATE: 2203/2018 by zhaofeng-shu33
\documentclass[a4paper]{article}
\usepackage[T1]{fontenc}
\usepackage{amsmath, amssymb, amsthm}
% amsmath: equation*, amssymb: mathbb, amsthm: proof
\usepackage{moreenum}
\usepackage{mathtools}
\usepackage{url}
\usepackage{graphicx}
\usepackage{subcaption}
\usepackage{booktabs} % toprule
\usepackage[mathcal]{eucal}
\usepackage{dsfont}
\usepackage{ctex}
\usepackage{setspace}  
\setstretch{1.6}








\theoremstyle{definition}
\newtheorem{definition}{Definition}[section]
\newtheorem{example}[definition]{Example}
\newtheorem{exercise}[definition]{Exercise}
\newtheorem{remark}[definition]{Remark}
\newtheorem{observation}[definition]{Observation}
\newtheorem{assumption}[definition]{Assumption}
\newtheorem{convention}[definition]{Convention}
\newtheorem{priniple}[definition]{Principle}
\newtheorem{notation}[definition]{Notation}
\newtheorem*{axiom}{Axiom}
\newtheorem{coa}[definition]{Theorem}
\newtheorem{srem}[definition]{$\star$ Remark}
\newtheorem{seg}[definition]{$\star$ Example}
\newtheorem{sexe}[definition]{$\star$ Exercise}
\newtheorem{sdf}[definition]{$\star$ Definition}
\newtheorem{question}{Question}




\newtheorem{problem}{Problem}
%\renewcommand*{\theprob}{{\color{red}\arabic{section}.\arabic{prob}}}
\newtheorem{rprob}[problem]{\color{red} Problem}
%\renewcommand*{\thesprob}{{\color{red}\arabic{section}.\arabic{sprob}}}
% \newtheorem{ssprob}[prob]{$\star\star$ Problem}



\theoremstyle{plain}
\newtheorem{theorem}[definition]{Theorem}
\newtheorem{Conclusion}[definition]{Conclusion}
\newtheorem{thd}[definition]{Theorem-Definition}
\newtheorem{proposition}[definition]{Proposition}
\newtheorem{corollary}[definition]{Corollary}
\newtheorem{lemma}[definition]{Lemma}
\newtheorem{sthm}[definition]{$\star$ Theorem}
\newtheorem{slm}[definition]{$\star$ Lemma}
\newtheorem{claim}[definition]{Claim}
\newtheorem{spp}[definition]{$\star$ Proposition}
\newtheorem{scorollary}[definition]{$\star$ Corollary}


\newtheorem{condition}{Condition}
\newtheorem{Mthm}{Main Theorem}
\renewcommand{\thecondition}{\Alph{condition}} % "letter-numbered" theorems
\renewcommand{\theMthm}{\Alph{Mthm}} % "letter-numbered" theorems


%\substack   multiple lines under sum
%\underset{b}{a}   b is under a


% Remind: \overline{L_0}



\usepackage{calligra}
\DeclareMathOperator{\shom}{\mathscr{H}\text{\kern -3pt {\calligra\large om}}\,}
\DeclareMathOperator{\sext}{\mathscr{E}\text{\kern -3pt {\calligra\large xt}}\,}
\DeclareMathOperator{\Rel}{\mathscr{R}\text{\kern -3pt {\calligra\large el}~}\,}
\DeclareMathOperator{\sann}{\mathscr{A}\text{\kern -3pt {\calligra\large nn}}\,}
\DeclareMathOperator{\send}{\mathscr{E}\text{\kern -3pt {\calligra\large nd}}\,}
\DeclareMathOperator{\stor}{\mathscr{T}\text{\kern -3pt {\calligra\large or}}\,}
%write mathscr Hom (and so on) 

\usepackage{aurical}
\DeclareMathOperator{\VVir}{\text{\Fontlukas V}\text{\kern -0pt {\Fontlukas\large ir}}\,}

\newcommand{\vol}{\text{\Fontlukas V}}
\newcommand{\dvol}{d~\text{\Fontlukas V}}
% perfect Vol symbol

\usepackage{aurical}








\newcommand{\fk}{\mathfrak}
\newcommand{\mc}{\mathcal}
\newcommand{\wtd}{\widetilde}
\newcommand{\wht}{\widehat}
\newcommand{\wch}{\widecheck}
\newcommand{\ovl}{\overline}
\newcommand{\udl}{\underline}
\newcommand{\tr}{\mathrm{t}} %transpose
\newcommand{\Tr}{\mathrm{Tr}}
\newcommand{\End}{\mathrm{End}} %endomorphism
\newcommand{\idt}{\mathbf{1}}
\newcommand{\id}{\mathrm{id}}
\newcommand{\Hom}{\mathrm{Hom}}
\newcommand{\cond}[1]{\mathrm{cond}_{#1}}
\newcommand{\Conf}{\mathrm{Conf}}
\newcommand{\Res}{\mathrm{Res}}
\newcommand{\res}{\mathrm{res}}
\newcommand{\KZ}{\mathrm{KZ}}
\newcommand{\ev}{\mathrm{ev}}
\newcommand{\coev}{\mathrm{coev}}
\newcommand{\opp}{\mathrm{opp}}
\newcommand{\Rep}{\mathrm{Rep}}
\newcommand{\Dom}{\mathrm{Dom}}
\newcommand{\loc}{\mathrm{loc}}
\newcommand{\con}{\mathrm{c}}
\newcommand{\uni}{\mathrm{u}}
\newcommand{\ssp}{\mathrm{ss}}
\newcommand{\di}{\slashed d}
\newcommand{\Diffp}{\mathrm{Diff}^+}
\newcommand{\Diff}{\mathrm{Diff}}
\newcommand{\PSU}{\mathrm{PSU}(1,1)}
\newcommand{\Vir}{\mathrm{Vir}}
\newcommand{\Witt}{\mathscr W}
\newcommand{\Span}{\mathrm{Span}}
\newcommand{\pri}{\mathrm{p}}
\newcommand{\ER}{E^1(V)_{\mathbb R}}
\newcommand{\prth}[1]{( {#1})}
\newcommand{\bk}[1]{\langle {#1}\rangle}
\newcommand{\bigbk}[1]{\big\langle {#1}\big\rangle}
\newcommand{\Bigbk}[1]{\Big\langle {#1}\Big\rangle}
\newcommand{\biggbk}[1]{\bigg\langle {#1}\bigg\rangle}
\newcommand{\Biggbk}[1]{\Bigg\langle {#1}\Bigg\rangle}
\newcommand{\GA}{\mathscr G_{\mathcal A}}
\newcommand{\vs}{\varsigma}
\newcommand{\Vect}{\mathrm{Vec}}
\newcommand{\Vectc}{\mathrm{Vec}^{\mathbb C}}
\newcommand{\scr}{\mathscr}
\newcommand{\sjs}{\subset\joinrel\subset}
\newcommand{\Jtd}{\widetilde{\mathcal J}}
\newcommand{\gk}{\mathfrak g}
\newcommand{\hk}{\mathfrak h}
\newcommand{\xk}{\mathfrak x}
\newcommand{\yk}{\mathfrak y}
\newcommand{\zk}{\mathfrak z}
\newcommand{\pk}{\mathfrak p}
\newcommand{\hr}{\mathfrak h_{\mathbb R}}
\newcommand{\Ad}{\mathrm{Ad}}
\newcommand{\DHR}{\mathrm{DHR}_{I_0}}
\newcommand{\Repi}{\mathrm{Rep}_{\wtd I_0}}
\newcommand{\im}{\mathbf{i}}
\newcommand{\Co}{\complement}
%\newcommand{\Cu}{\mathcal C^{\mathrm u}}
\newcommand{\RepV}{\mathrm{Rep}^\uni(V)}
\newcommand{\RepA}{\mathrm{Rep}(\mathcal A)}
\newcommand{\RepN}{\mathrm{Rep}(\mathcal N)}
\newcommand{\RepfA}{\mathrm{Rep}^{\mathrm f}(\mathcal A)}
\newcommand{\RepAU}{\mathrm{Rep}^\uni(A_U)}
\newcommand{\RepU}{\mathrm{Rep}^\uni(U)}
\newcommand{\RepL}{\mathrm{Rep}^{\mathrm{L}}}
\newcommand{\HomL}{\mathrm{Hom}^{\mathrm{L}}}
\newcommand{\EndL}{\mathrm{End}^{\mathrm{L}}}
\newcommand{\Bim}{\mathrm{Bim}}
\newcommand{\BimA}{\mathrm{Bim}^\uni(A)}
%\newcommand{\shom}{\scr Hom}
\newcommand{\divi}{\mathrm{div}}
\newcommand{\sgm}{\varsigma}
\newcommand{\SX}{{S_{\fk X}}}
\newcommand{\DX}{D_{\fk X}}
\newcommand{\mbb}{\mathbb}
\newcommand{\mbf}{\mathbf}
\newcommand{\bsb}{\boldsymbol}
\newcommand{\blt}{\bullet}
\newcommand{\Vbb}{\mathbb V}
\newcommand{\Ubb}{\mathbb U}
\newcommand{\Xbb}{\mathbb X}
\newcommand{\Kbb}{\mathbb K}
\newcommand{\Abb}{\mathbb A}
\newcommand{\Wbb}{\mathbb W}
\newcommand{\Mbb}{\mathbb M}
\newcommand{\Gbb}{\mathbb G}
\newcommand{\Cbb}{\mathbb C}
\newcommand{\Nbb}{\mathbb N}
\newcommand{\Zbb}{\mathbb Z}
\newcommand{\Qbb}{\mathbb Q}
\newcommand{\Pbb}{\mathbb P}
\newcommand{\Rbb}{\mathbb R}
\newcommand{\Ebb}{\mathbb E}
\newcommand{\Dbb}{\mathbb D}
\newcommand{\Hbb}{\mathbb H}
\newcommand{\cbf}{\mathbf c}
\newcommand{\Rbf}{\mathbf R}
\newcommand{\wt}{\mathrm{wt}}
\newcommand{\Lie}{\mathrm{Lie}}
\newcommand{\btl}{\blacktriangleleft}
\newcommand{\btr}{\blacktriangleright}
\newcommand{\svir}{\mathcal V\!\mathit{ir}}
\newcommand{\Ker}{\mathrm{Ker}}
\newcommand{\Cok}{\mathrm{Coker}}
\newcommand{\Sbf}{\mathbf{S}}
\newcommand{\low}{\mathrm{low}}
\newcommand{\Sp}{\mathrm{Sp}}
\newcommand{\Rng}{\mathrm{Rng}}
\newcommand{\vN}{\mathrm{vN}}
\newcommand{\Ebf}{\mathbf E}
\newcommand{\Nbf}{\mathbf N}
\newcommand{\Stb}{\mathrm {Stb}}
\newcommand{\SXb}{{S_{\fk X_b}}}
\newcommand{\pr}{\mathrm {pr}}
\newcommand{\SXtd}{S_{\wtd{\fk X}}}
\newcommand{\univ}{\mathrm {univ}}
\newcommand{\vbf}{\mathbf v}
\newcommand{\ubf}{\mathbf u}
\newcommand{\wbf}{\mathbf w}
\newcommand{\CB}{\mathrm{CB}}
\newcommand{\Perm}{\mathrm{Perm}}
\newcommand{\Orb}{\mathrm{Orb}}
\newcommand{\Lss}{{L_{0,\mathrm{s}}}}
\newcommand{\Lni}{{L_{0,\mathrm{n}}}}
\newcommand{\UPSU}{\widetilde{\mathrm{PSU}}(1,1)}
\newcommand{\Sbb}{{\mathbb S}}
\newcommand{\Gc}{\mathscr G_c}
\newcommand{\Obj}{\mathrm{Obj}}
\newcommand{\bpr}{{}^\backprime}
\newcommand{\fin}{\mathrm{fin}}
\newcommand{\Ann}{\mathrm{Ann}}
\newcommand{\Real}{\mathrm{Re}}
\newcommand{\Imag}{\mathrm{Im}}
%\newcommand{\cl}{\mathrm{cl}}
\newcommand{\Ind}{\mathrm{Ind}}
\newcommand{\Supp}{\mathrm{Supp}}
\newcommand{\Specan}{\mathrm{Specan}}
\newcommand{\red}{\mathrm{red}}
\newcommand{\uph}{\upharpoonright}
\newcommand{\Mor}{\mathrm{Mor}}
\newcommand{\pre}{\mathrm{pre}}
\newcommand{\rank}{\mathrm{rank}}
\newcommand{\Jac}{\mathrm{Jac}}
\newcommand{\emb}{\mathrm{emb}}
\newcommand{\Sg}{\mathrm{Sg}}
\newcommand{\Nzd}{\mathrm{Nzd}}
\newcommand{\Owht}{\widehat{\scr O}}
\newcommand{\Ext}{\mathrm{Ext}}
\newcommand{\Tor}{\mathrm{Tor}}
\newcommand{\Com}{\mathrm{Com}}
\newcommand{\Mod}{\mathrm{Mod}}
\newcommand{\nk}{\mathfrak n}
\newcommand{\mk}{\mathfrak m}
\newcommand{\Ass}{\mathrm{Ass}}
\newcommand{\depth}{\mathrm{depth}}
\newcommand{\Coh}{\mathrm{Coh}}
\newcommand{\Gode}{\mathrm{Gode}}
\newcommand{\Fbb}{\mathbb F}
\newcommand{\sgn}{\mathrm{sgn}}
\newcommand{\Aut}{\mathrm{Aut}}
\newcommand{\Modf}{\mathrm{Mod}^{\mathrm f}}
\newcommand{\codim}{\mathrm{codim}}
\newcommand{\card}{\mathrm{card}}
\newcommand{\dps}{\displaystyle}
\newcommand{\Int}{\mathrm{Int}}
\newcommand{\Nbh}{\mathrm{Nbh}}
\newcommand{\Pnbh}{\mathrm{PNbh}}
\newcommand{\Cl}{\mathrm{Cl}}
\newcommand{\diam}{\mathrm{diam}}
\newcommand{\eps}{\varepsilon}
\newcommand{\Vol}{\mathrm{Vol}}
\newcommand{\LSC}{\mathrm{LSC}}
\newcommand{\USC}{\mathrm{USC}}
\newcommand{\Ess}{\mathrm{Rng}^{\mathrm{ess}}}
\newcommand{\Jbf}{\mathbf{J}}
\newcommand{\SL}{\mathrm{SL}}
\newcommand{\GL}{\mathrm{GL}}
\newcommand{\Lin}{\mathrm{Lin}}
\newcommand{\ALin}{\mathrm{ALin}}
\newcommand{\bwn}{\bigwedge\nolimits}
\newcommand{\nbf}{\mathbf n}
\newcommand{\dive}{\mathrm{div}}












\numberwithin{equation}{problem}
% count the eqation by section countation


\DeclareMathOperator{\sign}{sign}
\DeclareMathOperator{\dom}{dom}
\DeclareMathOperator{\ran}{ran}
\DeclareMathOperator{\ord}{ord}
\DeclareMathOperator{\img}{Im}
\DeclareMathOperator{\dd}{d\!}
\newcommand{\ie}{ \textit{ i.e. } }
\newcommand{\st}{ \textit{ s.t. }}



%---------------优雅的插入MATLAB代码---------%
\usepackage{lipsum,zhlipsum} %生成一些测试文本
\usepackage{listings,matlab-prettifier} % MATLAB 美化包
\lstset{
        style=Matlab-editor,
        numbers      = left,
        frame        = single,
}

\usepackage[thehwcnt = 6]{iidef}
\thecourseinstitute{Tsinghua University}
\thecoursename{Numberical Analysis}
\theterm{Fall 2024}
\hwname{Homework}
\begin{document}
\courseheader
\name{Lin Zejin}
\rule{\textwidth}{1pt}
\begin{itemize}
\item {\bf Collaborators: \/}
  I finish this homework by myself. 
%   If you finish your homework all by yourself, make a similar statement. If you get help from others in finishing your homework, state like this:
%   \begin{itemize}
%   \item 1.2 (b) was solved with the help from \underline{\hspace{3em}}.
%   \item Discussion with \underline{\hspace{3em}} helped me finishing 1.3.
%   \end{itemize}
\end{itemize}
\rule{\textwidth}{1pt}

\vspace{2em}
 
\sloppy
\pagenumbering{arabic}

% put your code here
\begin{problem}
    \begin{align*}
        \hat{r}-r&=Ax-A\hat{x}\\
        &=A(A^TA)^{-1}A^Tb-A(A^TA+F)^{-1}A^Tb\\
        &=A(A^TA+F)^{-1}\left((A^TA+F)-(A^TA)\right)(A^TA)^{-1}A^Tb\\
        &=A(A^TA+F)^{-1}Fx
    \end{align*}
    We know from the lecture that
    \[\mathcal{K}_2(A)^2=\cond{2}(A^TA)=\dps\frac{|\rho(A^TA)|}{|\sigma_n(A)|^2}=\frac{||A||_2^2}{|\sigma_n(A)|^2}\]
    Noticed that  $ \dps||(A^TA)^{-1}||_2=||A^\dagger||_2^2=\frac{1}{|\sigma_n(A)|^2} \leq \frac{1}{2||F||_2} $, hence 
    \[||(A^TA+F)^{-1}||_2  \leq \frac{||(A^TA)^{-1}||_2}{1-||F||_2\cdot||(A^TA)^{-1}||_2} \leq \frac{||A^\dagger||_2^2}{\frac{1}{2}}\]
    Therefore
    \begin{align*}
        ||\hat{r}-\hat{r}||_2& \leq ||A||_2\cdot||(A^TA+F)^{-1}||_2\cdot||F||_2\cdot||x||_2\\
        & \leq 2||A||_2\cdot||A^\dagger||_2^2\cdot||F||_2\cdot||x||_2\\
        &=2\mathcal{K}_2(A)^2\frac{||F||_2}{||A||_2}||x||_2
    \end{align*}
\end{problem}

\begin{problem}
    \begin{align*}
        ||x-\hat{x}||_2&=||(A^TA)^{-1}f||_2\\
        & \leq ||(A^TA)^{-1}||\cdot||f||_2\\
        & \leq ||A^\dagger||_2^2\cdot\epsilon ||A||_2||b||_2\\
        &=\epsilon\mathcal{K}_2(A)^2\dps\frac{||A||_2||b||_2}{||A||_2^2}\\
        &=\epsilon\mathcal{K}_2(A)^2 \frac{||A||_2||b||_2}{||A^TA||_2}
    \end{align*}
    Then\[\dps\frac{||x-\hat{x}||_2}{||x||_2} \leq\epsilon\mathcal{K}_2(A)^2 \frac{||A||_2||b||_2}{||A^TA||_2||x||_2} \leq \epsilon\mathcal{K}_2(A)^2 \frac{||A||_2||b||_2}{||A^TAx||_2}=\epsilon\mathcal{K}_2(A)^2 \frac{||A||_2||b||_2}{||A^Tb||_2}\]
\end{problem}

\begin{problem}
    We already have method tp compute  $ A^TA,A^Tb $, and we do Gauss elimination to it to get an equation:
    \[U'x=b'\]
    where  $ U $ is upper triangular matrix, with 
    \[
        U'=\begin{bmatrix}
            U & B\\
            0 & 0
        \end{bmatrix},
        b'=\begin{bmatrix}
            \hat{b'}\\
            0
        \end{bmatrix}
    \] 
     $ U $ is the upper triangular matrix with all diagonal elements $ 1 $.
     
     Then solutions of  $ A^TAx=A^Tb $ are
     \[x=\begin{bmatrix}
        U^{-1}(\hat{b'}-Bx')\\
        x'
     \end{bmatrix}=\begin{bmatrix}
        U^{-1}\hat{b'}\\
        0
     \end{bmatrix}-\begin{bmatrix}
        U^{-1}B\\
        -I
     \end{bmatrix}x'
     \]
     
     Therefore the solution of least norm is exactly the least square solution of 
     \[
        \begin{bmatrix}
            U^{-1}B\\
            -I
         \end{bmatrix}x'=\begin{bmatrix}
            U^{-1}\hat{b'}\\
            0
         \end{bmatrix}
     \] 
     which can be solved by the least square method.

     Then we can find  $ x $ such that the 2-norm is least among all least square solution of  $ Ax=b $. 
\end{problem}

\begin{problem}
    For each step of the Givens transformation, we make one element in one row be $ 0 $.
    
    The step between  $ x,y $  should compute 
    \[t=\frac{y}{x},\,s=\sgn(x)(1+t^2)^{\frac{1}{2}},c=st\]
    So there is about  $ 3 $ times of computation.

    And in total, if we want to get an upper triangular matrix, it should be  $ 3(n-1) $.
    
    So the Complexity is  $ o(n) $. 
\end{problem}

\begin{problem}
    \begin{enumerate}
        \item[(a)] \begin{equation}
            A=\begin{bmatrix}
                \dps\frac{\sqrt{2}}{2}\alpha+\frac{1}{2}&-\dps\frac{\sqrt{2}}{2}\alpha&\dps\frac{\sqrt{2}}{2}\alpha-\dps\frac{1}{2}\\
                \\
                -\dps\frac{\sqrt{6}}{6}\alpha+\dps\frac{\sqrt{3}}{2}&\dps\frac{\sqrt{6}}{6}\alpha&\dps-\frac{-\sqrt{6}}{6}\alpha-\frac{\sqrt{3}}{2}
            \end{bmatrix}
        \end{equation}
        \begin{equation}
            A^\dagger=\begin{bmatrix}
                 \dps\frac{\alpha + \sqrt{2}}{4\alpha}& \dps-\frac{\sqrt{2}}{4\alpha}&    \dps       -\frac{\alpha-\sqrt{2}}{4\alpha} \\
                 \dps\frac{3\sqrt{3}\alpha - \sqrt{6}}{12\alpha}&\dps \frac{6}{12\alpha}& \dps-\frac{3\sqrt{3}\alpha + \sqrt{6}}{12\alpha}        
            \end{bmatrix}
        \end{equation}
        Then the solution is 
        \[x=A^\dagger b=\begin{bmatrix}
            
         1\\
         \sqrt{3}        
        \end{bmatrix}\]
        \item[(b)]
         \[\mathcal{K}(A)^2=\cond{2}(A^TA)=\dps\frac{|\lambda_1|}{|\lambda_n|}=\dps\frac{\max(2,2x^2)}{\min(2,2x^2)}\]
         since the eigen value of  $ A^TA $ is  $ 2x^2 $ and  $ 2 $ 
         \item[(3)]
         The obtained data is in the appendix \textbf{A} and main code is in the appendix \textbf{B}.

         So easy to see that The Cholesky method is more efficient but have no accuracy. And also it causes wrong message when  $ x=10^9 $.
         
         
         However, the QR method using Givens transformation is more accurate, actually, it is very accurate even if  $ x=10^9 $. And it also costs more time. 
         \newpage
         \appendix
         \section{Obtained Data}
         \label{sec:data}
         \begin{enumerate}
            \item  $ x=10^5 $  
        \\
        Cholesky method
        \\
        ans =
        
           1.4142e+05
        \\
        
        d1 =
        
           1.0e+05 *
        
            2.0000    0.0000
        \\
        cost 0.004769s
        \\Gauss method
        \\
        ans =
        
           1.4142e+05
        \\
        
        d1 =
        
           1.0e+05 *
        
            2.0000    0.0000
        \\
        cost 0.001538s
        \\QR Givens method
        \\
        ans =
        
            0.3660
        \\
        
        d2 =
        
            1.0000    1.7321
\\
            cost 0.002946 s\\
            \item  $ x=10^7 $\\
        
Cholesky method
\\
ans =

   1.4149e+07


d1 =

   1.0e+07 *

    2.0010    0.0000

cost 0.002820s

Gauss method

ans =

   1.4149e+07


d1 =

   1.0e+07 *

    2.0010    0.0000

cost 0.000503 s

QR Givens method

ans =

    0.3660


d2 =

    1.0000    1.7321

cost 0.001068 s
            \item  $ x=10^9 $\\


Cholesky method\\
There has to be some problem
\\ans =

   2.3094e+09
\\

d1 =

   1.0e+09 *

    3.2660    0.0000
\\
cost 0.003693 s\\
Gauss method\\
There has to be some problem\\
ans =

   2.3094e+09
\\

d1 =

   1.0e+09 *

    3.2660    0.0000
\\
cost 0.000513 s\\
QR givens method\\

ans =

    0.3660
\\

d2 =

    1.0000    1.7321
\\
cost 0.001796 s  
         \end{enumerate}
        \section{Source Code}
        \label{sec:code}
         \lstinputlisting[caption=QR\underline{ }Givens.m]{code/QR_Givens.m}
         \lstinputlisting[caption=Cholesky.m]{code/Cholesky.m}
         \lstinputlisting[caption=Gauss.m]{code/Gauss.m}
         \lstinputlisting[caption=solution.m]{code/solution.m}
         The code to compute the least square solution.
         \lstinputlisting[caption=least\underline{ }square]{code/sys2.m}
         The method to compute  $ (a),(b) $ 
         \lstinputlisting[caption=compute.m]{code/compute.m}
    \end{enumerate}
    
\end{problem}
\end{document}