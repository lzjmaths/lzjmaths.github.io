\subsection{Composite field}
\begin{definition}
    A \textbf{composite field}\index{composite field} of  $ K_1  $ and  $ K_2  $ (over  $ F  $) is a field  $ L/F  $ together with  $ F $-homomorphism  $ u_1:K_1\hookrightarrow L,u_2:K_2\hookrightarrow L$ s.t.  $ L  $ is generated by  $ u_1(K_1),u_2(K_2) $.\\
    Often write  $ L=K_1K_2 $.   
\end{definition}
\begin{theorem}
    A composition field exists and is unique up to  $ F  $-isomorphism.
\end{theorem}
\begin{proof}
    Let  $ R:=K_1\otimes_F K_2  $.  $ M  $ be its maximal ideal. Then $ L:=R/M  $ is a field.
\end{proof}
\begin{theorem}[Galois theory for composite fields]
    \,\begin{enumerate}[(1)]
        \item  $ K/F  $ finite Galois and  $ F'/F  $ any field extension. Set  $ K'=KF'  $. Then  $ K'/F'  $ is finite Galois and 
        \[\Gal(K'/F')\xrightarrow{\cong}\Gal(K/_{K\cap F'}),\sigma\mapsto \sigma|_K\]
        is an isomorphism. In particular,  $ [K':F']=[K:K\cap F'] $
        \item  $ K_1,K_2/F  $ is finite Galois. Then  $ K_1K_2,K_1\cap K_2 $ are Galois over F and 
         \[\Gal(K_1K_2/_{K_1\cap K_2})\rightarrow \Gal(K_1/_{K_1\cap K_2})\times \Gal(K_2/_{K_1\cap K_2}),\sigma\mapsto (\sigma|_{K_1},\sigma|_{K_2})\] 
         is an isomorphism. 
    \end{enumerate}
\end{theorem}
\begin{proof}
    (1)If  $ f(x)\in F[x] $ separable\st  $ K  $ is the splitting field of  $ F $. Then  $ K'=KF'  $ is the splitting field of  $ f(x)\in F'[x] $. i.e.  $ K'/F'  $ is finite Galois.\\
    Now take  $ \sigma\in \Gal(K'/F') $.  $ K/F  $ is normal  $ \Rightarrow  $  $ \sigma|_K:K\rightarrow K'  $ has image in  $ K  $.\\
    So  $ \Gal(K'/F')\rightarrow \Gal(K/_{K\cap F }) $ is well-defined.\\
    If  $ \sigma|_K  $ is identity, then  $ \sigma=id  $ on  $ K'  $  $ \Rightarrow  $ the map is injective,\\
    Now since  $ K^{\Gal(K'/F')}=K\cap F'  $(if not, then it is a simple extension and there is a contradiction from extension lemma)  $ \Rightarrow  $  $ \Gal(K'/F')=\Gal(K/_{K\cap F'}) $.  \\
    (2) Take  $ f_i\in F[x]  $ separable \st  $ K_i  $ is the splitting field.\\
    Let  $ f_i=g_ih $,  $ (g_1,g_2)=1 $.\\
    Then  $ K_1K_2  $ is the splitting field of separable polynomial  $ g_1g_2h\in F[x] $. So  $ K_1K_2  $ is finite Galois extension.\\
    The map is easy to check that it is well-defined and injective.\\
     $ \left|\Gal(K_1K_2/_{K_1\cap K_2})\right|=[K_1K_2:K_1\cap K_2]=[K_1K_2:K_2][K_2:K_1\cap K_2]=[K_1:K_1\cap K_2][K_2:K_1\cap K_2] $ by (1). So it is isomorphism. 
\end{proof}
\subsection{traces and norms}
\begin{definition}
    Let  $ K/F  $ be a finite extension. For  $ \alpha\in K  $, write  $ m_\alpha:K\rightarrow K,x\mapsto \alpha x $\\
    \[\Tr_{K/F}(\alpha):=\Tr(m_\alpha)\in F  \]
    \[\N_{K/F}(\alpha):=\det(m_\alpha)\in F\]
    Call them trace and norm of  $ \alpha  $ respectively.
\end{definition}
\begin{proposition}
    \,
    \begin{enumerate}[(1)]
        \item   $ \Tr  $ is  $ F  $-linear,  $ N(\alpha\beta)=N(\alpha)N(\beta),N(a\alpha)=a^{[K:F]}N(\alpha) $  
        \item (transitivity) For  $ L/K/F  $,  $ \Tr_{L/F}=\Tr_{K/F} \circ\Tr_{L/K}$,  $ N_{L/F}=N_{K/F}\circ N_{L/K} $ 
        \item Assume  $ K/F  $ is separable and write  $ \Hom_F(K,\overline{F})=\{\sigma_1,\cdots,\sigma_n\} $,  $ n=[K:F] $.\\
        Then  $ \Tr_{K/F}(\alpha)=\sum \sigma_i(\alpha) $,  $ N_{K/F}(\alpha)=\sigma_1(\alpha)\cdots\sigma_n(\alpha) $(Note that if  $ K/F  $ is Galois,  $ \Gal(K/F)=\{\sigma_1,\cdots,\sigma_n\} $)
        \item(non-degeneracy of trace pairing) Assume  $ K/F  $ separable. The  $ F $-bilinear map  $ K\times K\rightarrow F,(\alpha,\beta)\mapsto \Tr_{K/F}(\alpha\beta)  $ is non-degenerate, \ie  $ \forall \alpha\not=0\in K  $,  $ \exists \beta\in K $ \st  $ \Tr(\alpha\beta)\not=0 $        
    \end{enumerate}
\end{proposition}
\begin{proof}
    (1)(2) is easy to check\\
    (3) Write  $ K=F(\beta) $,  $ f(x)\in F[x] $ minimal polynomial of  $ \beta $.\\
    Write  $ f(x)=(x-\beta_1)\cdots(x-\beta_n)  $ in  $ \overline{F}[x] $.\\
    By extension lemma,  $ \Hom_F(K,\overline{F})=\{\sigma_1,\cdots,\sigma_n\} $ \st  $ \sigma_i(\beta)=\beta_i $.\\
    We know that  $ \overline{F}\otimes_F K\cong \overline{F}\otimes_F\frac{F[x]}{(f(x))}\cong \frac{\overline{F}[x]}{(f(x))} $.\\
    And we can conclude that  $ \gamma\otimes \delta\mapsto (\gamma\sigma_1(\delta),\cdots,\gamma\sigma_n(\delta)) $.\\
    Now  $ id\otimes m_\alpha:\overline{F}\otimes_F K\rightarrow \overline{F}\otimes_F K, \gamma\otimes \delta\mapsto \gamma\otimes \alpha\delta $.\\
     $ \Tr(m_\alpha)=\Tr(id\otimes m_\alpha),\det(m_\alpha)=\det(id\otimes m_\alpha) $.\\
     So in  $ \overline{F}\otimes K\cong \overline{F}^n $ we can easily find the answer.\\
     (4)In char 0 case, take  $ \beta=\frac{1 }{\alpha} $. In            
\end{proof}
\subsection{Advanced theorem in Galois theory}
\begin{theorem}[linear independence of  $ F  $-homomorphism]\index{linear independence of  $ F $-homomorphism}
     $ K,L  $  $ F  $ field extension. Let  $ \sigma_1,\cdots,\sigma_n:K\rightarrow L $ be distinct  $ F  $-homomorphism.  $ a_1,\cdots,a_n\in L $. \\
     If  $ \forall x\in K  $,  $ a_1\sigma_1(x)+\cdots+a_n\sigma_n(x)=0 $, then  $ a_1=\cdots=a_n=0 $    
\end{theorem}
\begin{theorem}[Hilbert 90]
    Let  $ K/F  $ be finite Galois extension with  $ G=\Gal(K/F) $.\\
    Let  $ A:G\rightarrow K^\times  $ be a set map \st  $ \forall \sigma\tau\in G $,  $ A(\sigma\tau)=A(\sigma)\sigma(A(\tau)) $.\\
    Then  $ \exists \alpha\in K^\times  $ \st  $ A(\sigma)=\frac{\sigma(\alpha )}{\alpha},\forall \sigma\in G $.    
\end{theorem}
\begin{proof}
    Apply linear independence of  $ F  $-homomorphism,  $ \exists \beta\in K^\times $ \st  $ \sum_{\tau\in G} A(\tau)\tau(\beta)\not=0 $ 
\end{proof}
\begin{remark}
    If we consider group cohomology  $ H^0(G,M)=M^G $, $ H^1(G,M)=\{\text{1-cocycle } A:G\rightarrow M\} $. Then Hilbert 90 tells us if  $ K/F  $ finite Galois,  $ H^1(\Gal(K/F),K^\times)=\{1\} $ 
\end{remark}
\begin{corollary}
    Assume  $ K/F  $ is finite cyclic Galois \ie  $ G=\Gal(K/F)=  \left< \sigma \right>  $ cyclic.\\
    If  $ \alpha\in K^\times $ satisfies  $ N_{K/F}(\alpha)=1 $, then  $ \exists \beta\in K^\times  $ \st  $ \alpha=\frac{\sigma(\beta )}{\beta} $.  
\end{corollary}
\begin{proof}
    Set  $ A(\sigma^m)=\alpha\sigma(\alpha)\cdots\sigma^{m-1}(\alpha)\in K^\times $.\\
    Then  $ A  $ defines a 1-cocycle  $ G= \left< \sigma  \right>\rightarrow K^\times $.(By using  $ N(\alpha)=1 $)\\
    By Hilbert 90,  $ \exists \beta\in K^\times  $ \st  $ A(\sigma)=\frac{\sigma(\beta )}{\beta} $.  
\end{proof}
\subsection{Kummer theory}
\begin{theorem}[Kummer's theorem]\index{Kummer's theorem}
    Assume  $ n\in F^\times  $ and  $ \psi_n\in F $.
    \begin{enumerate}[(1)]
        \item For  $ a\in F^\times $, write  $ \sqrt[n]{a}\in \overline{F} $ for a root of  $ x^n-a $.\\
         Then  $ F(\sqrt[n ]{a })/F $ is finite cyclic Galois.\\
         Moreover,  $ d:=[F(\sqrt[n ]{a }):F] $.\\
         Then  $ d|n   $,  $ (\sqrt[n ]{a})^i\not\in F $ for  $ i<d  $ and  $ (\sqrt[n ]{a})^d\in F $. 
         \item Let  $ K/F  $ be a finite cyclic Galois extension of degree of  $ d|n $. Then  $ \exists a\in F^\times  $ \st   order of  $ \overline{a}\in F^\times/_{F^{\times n}} $ is  $ d  $ and  $ K=F(\sqrt[n ]{a }) $ \\
         or    $ \exists b\in F^\times  $ \st   order of  $ \overline{b}\in F^\times/_{(F^{\times})^d} $ is  $ d  $ and  $ K=F(\sqrt[n ]{b })  $ 
    \end{enumerate}
\end{theorem}
\begin{proof}
    (1)From  $ \psi_n\in F  $ we know that  $ K/F  $ is the splitting field of separable polynomial $ x^n-a $.\\
    So  $ K=F(\sqrt[n ]{a }) $ is finite Galois.\\
    Consider  $ \iota:G\rightarrow \mathbb{Z}/n, \sigma\mapsto m\st \sigma(\sqrt[n ]{a })=\sqrt[n ]{a }\psi_n^m $.\\
    Then  $ \iota $ is a group isomorphism.\\
    We can check that  $ \sigma((\sqrt[n]{a})^l)=(\sqrt[n ]{a })^l $ is and only if  $ d|l $. So we end the proof of (1) \\
    (2)Let  $ \psi_d=\psi_n^{n/d} \in F$, so WLOG we can assume  $ d=n $.\\
    Then  $ K/F  $ is finite cyclic of degree  $ n  $.\\
     $ N_{K/F}(\psi_n)=\prod\limits_\tau\in G \tau(\psi_n)=\psi^{[K:F]}=1 $.\\
     By the corollary to Hilbert 90,  $ \exists \alpha\in K^\times  $ \st  $ \psi_n=\frac{\sigma(\alpha)}{\alpha} $,  $ \sigma\in G  $ generator.\\
     So  $ \sigma(\alpha)=\alpha\cdot \psi_n $ so  $ \sigma(\alpha^n)=\alpha^n $ \ie  $ \alpha^n\in F^\times $.\\
     Set  $ a=\alpha^n \in F^\times$ and we get the answer.     
\end{proof}
\subsection{Solvability by radicals}
Assume  $ \Char(F)=0 $.  $ M  $ is the splitting field of a polynomial  $ f(x)\in F[x] $  \\
\begin{definition}
    Say  $ f  $ is \textbf{solvable by  radicals }\index{solvable by radicals} if  $ \exists F=K_0\subset K_1\subset\cdots\subset K_m=K $ \st  $ K_{i+1}=K_i(\sqrt[n_i]{\alpha_i}) $,  $ \alpha_i\in K_i^\times $,  $ M\subset K $    
\end{definition}
Recall that  $ G  $ is solvable if  $ G\triangleright D(G)\triangleright \cdots \triangleright D^m(G)=\{1\} $, where  $ D(G)=[G,G]=<ghg^{-1}h^{-1}> $ 
\begin{proposition}
    \,
    \begin{enumerate}[(1)]
        \item  $ G  $ is solvable if and only if  $ \exists G\triangleright G_1\triangleright\cdots\triangleright G_m=1 $ \st  $ G_i/G_{i+1}  $ is abelian.  $ \Leftrightarrow  $  $ \exists G\triangleright G_1\triangleright\cdots\triangleright G_m=1 $ \st  $ G_i/G_{i+1}\cong\mathbb{Z}/p_i $ cyclic group of prime order.
        \item Every subgroup or quotient group of solvable group is solvable.
        \item  $ N\triangleleft G $ if  $ N,G/N  $ is solvable, then  $ G  $ is solvable. 
    \end{enumerate}
\end{proposition}
\begin{theorem}
     $ f  $ is solvable by radicals if and only if  the Galois group is a solvable group 
\end{theorem}
\begin{example}
     $ f(x)=(x-\alpha_1)(x-\alpha_2)(x-\alpha_3)(x-\alpha_4)(x-\alpha_5) $ for  $ \alpha_1,\alpha_2,\alpha_3\in \mathbb{R} $,  $ \alpha_4,\alpha_5\not\in \mathbb{R} $.\\
     Then  Galois group is  $ f  $ is  $ S_5  $ which is non-solvable  $ \Rightarrow  $  $ f  $ is not solvable by radicals.     
\end{example}
\subsection{Algebraic closure}
\begin{theorem}[Steinitz]
    An algebraic closure exists and is unique up to isomorphism.
\end{theorem}