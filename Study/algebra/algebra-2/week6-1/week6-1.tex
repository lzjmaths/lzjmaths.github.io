In this subsection we will discuss  $ R  $ integral domain.
\begin{definition}
    Let  $ f\in R  $ be nonzero and nonunit.\\ Say  $ f  $ is \textbf{irreducible}\index{irreducible element} if  $ f=gh \Rightarrow  $ either $ g $  or  $ h  $ is a unit.\\Say  $ f  $ is a \textbf{prime }\text{prime element} if  $ f|gh \Rightarrow f|g  $ or  $ f|h  $\\
    Here,  $ a|b  $ means $ \exists c\in R  $ such that  $ ac=b  $  
\end{definition}
\begin{proposition}
     $ f\in R  $ 
     \begin{enumerate}
        \item  $ f  $ is irreducible $ \Leftrightarrow (f)  $ is maximal among proper principal ideals.
        \item  $ f  $ is prime if and only if  $ (f)  $ is a prime ideal.
     \end{enumerate}
\end{proposition}
\begin{proposition}\,
    \begin{enumerate}
        \item A prime element is irreducible.
        \item If  $ R  $ is a PID(i.e. every ideal is principal)\index{principal ideal domain}, then an irreducible element is prime.
    \end{enumerate}
\end{proposition}
\begin{proposition}
    For  $ \varphi:R\rightarrow R'  $ ring homomorphism.
    \begin{enumerate}
        \item  $ J\subset R'  $ ideal  $ \rightarrow \varphi^{-1}J\subset R $ ideal.
        \item  $ J\subset R'  $ prime ideal  $ \rightarrow \varphi^{-1}J\subset R $ prime ideal.
        \item Maximal ideal is not preserved between homomorphism.
    \end{enumerate}
\end{proposition}
\begin{example}
     $ \mathbb{Z }[i]/(3)\backsimeq \mathbb{Z}[x]/(x^2+1,3)\backsimeq \mathbb{F}_3[x]/(x^2+1) $.\\
     Since  $ x^2+1 $ have no root in  $ \mathbb{F}_3  $,   $ x^2+1  $ is irreducible, hence prime in  $ \mathbb{f}_3[x] $. Therefore  $ \mathbb{Z}[i]/(3)\backsimeq \mathbb{F}_3[x]/(x^2+1) $ is an integral domain. Then  $ 3  $ is prime in  $ \mathbb{Z}[i] $.  
\end{example}
\begin{example}
    $ \mathbb{Z}\sqrt{-5}/(2)\backsimeq \mathbb{Z}[x]/(x^2+5,2)\backsimeq \mathbb{F}_2[x]/(x^2+5) $.\\
    Since  $ x^2+5 $ is not prime. Therefore   $ 2  $ is not prime in  $ \mathbb{Z}\sqrt{-5}/(2) $.
\end{example}
\begin{example}
    However  $ 2  $ is irreducible.\\
    Set  $ P=(2,1+\sqrt{-5}) $.
    \begin{claim}\,
        \begin{enumerate}
            \item  $ P  $ is a prime.
            \item  $ P^2=(2) $ 
        \end{enumerate}
    \end{claim} 
    Let  $ Q=(3,1+\sqrt{-5}) $. Then  $ Q,\overline{Q}  $ are maximal.\\  $ Q\overline{Q}=(3),PQ=(1+\sqrt{-5}),P\overline{Q}=(1-\sqrt{-5})$\\
     $ \Rightarrow (6)=P^2Q\overline{Q} $ called \textbf{prime ideal factorization}\index{prime ideal factorization}(i.e.  $ \mathbb{Z}[\sqrt{-5}] $ is a Dedekind domain)
\end{example}
\begin{example}
     $ R=\mathbb{C}[x,y] $.\\
    \begin{fact}
         $ Spec R:=\{0\}\cup \{(f(x,y)):f\in R\,irreducible\}\cup\{(x-a,y-b),a,b\in \mathbb{C} \} $ 
    \end{fact}
\end{example}
\begin{theorem}[Hilbert Nullstellensate]\index{Hilbert Nullstellensate}
    Maximal ideals of  $ \mathbb{C}[x,y] $  are  $ (x-a,y-b) $. 
\end{theorem}
\subsection{Zorn's lemma}
\begin{theorem}
    If  $ I\subset R  $ is a proper ideal,  $ \exists   $ maximal ideal  $ \mathfrak{M} $ s.t.  $ \mathfrak{M}\supset I $.\\
    In particular, if  $ R\not=0  $, then  $ Spec R\not=0 $  
\end{theorem}
\begin{definition}
    Say a partially ordered set  $ S  $ is inductive, if every totally ordered subset  $ s'\subset S  $ has an upper bound.
\end{definition}
\begin{theorem}[Zorn's lemma]\index{Zorn's lemma}
    Every nonempty inductive partially ordered set has a maximal element.
\end{theorem}