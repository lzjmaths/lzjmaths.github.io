\documentclass{article}
\usepackage{amsthm}
\usepackage{amssymb}
\usepackage{enumerate}
\usepackage{amsmath}
\usepackage{extarrows}
\usepackage{mathrsfs}
\usepackage[UseMSWordMultipleLineSpacing,MSWordLineSpacingMultiple=1.4]{zhlineskip}
\usepackage{lipsum}
\title{Algebra-1 Note}
\author{lin150117 }
\date{}
\newtheorem{definition}{Definition}[subsection]
\newtheorem{theorem}{Theorem}[subsection]
\newtheorem{example}{Example}
\newtheorem{remark}{Remark}
\newtheorem{corollary}{Corollary}
\newtheorem{proposition}{Proposition}

\begin{document}
\setlength{\parindent}{0pt}
\paragraph{Exercise 3.4}


$ \varphi(y+1-(x-1)^3)=0 $. Let  $ l:=y+1-(x-1)^3 $ \\
Then for  $ f\in \mathbb{C}[x,y] $, let  $ f=ml+n $, where  $ m\in \mathbb{C}[x,y],n\in\mathbb{C}[x] $. $ (\ast ) $ \\
Then  $ \varphi(f)=\varphi(n) $. $ \Rightarrow $ 
\[\varphi(f)=0\Leftrightarrow \varphi(n)=0\Leftrightarrow n(t+1)=0\Leftrightarrow n=0\]    
Hence  $ K=l\mathbb{C}[x,y]=(l) $.\\
By Correspondence theorem, the ideal  $ I  $ of  $ \mathbb{C}[x,y] $ that contains  $ K  $  gets an ideal  $ I/K $ of  $ \mathbb{C}[x,y]/K $. And by  $ (\ast)  $,  $ \mathbb{C}[x,y]/K=\mathbb{C}[x] $.\\
By Prop 11.3.22,  $ I/K=(\overline{x}) $ for some  $ \overline{x}\in  \mathbb{C}[x,y]/K  $. Then  $ I=(x,l) $ since  $ \forall i\in I,\overline{i}=\overline{x}\cdot \overline{f},\overline{f}\in \mathbb{C }[x,y]/K $ $ \Rightarrow  $  $ i=(x+k_1)(f+k_2)+k_3\in (x,l) $ for some $ k_1,k_2,k_3\in K=(l) $.  

\paragraph{Exercise3.9}

If  $ x  $ is nilpotent, assume   $ x^n=0 $. Then  $ (1-(-x))(1+(-x)+(-x)^2+\cdots+(-x)^{n-1})=1-(-x)^n=1 $  $ \Rightarrow $  $ (1+x)^{-1}=(1-(-x))^{-1}=(1+(-x)+(-x)^2+\cdots+(-x)^{n-1}) $ is a unit.

(b)Since  $ a  $ is nilpotent, then  $ a^n=0  $ for sufficient large  $ n  $. In particular,  $ \exists m\in \mathbb{N} $ s.t.  $ a^{p^m }=0 $. Noticed that  $ p|\binom{p^m}{k}  $ for all  $ 1 \leq k \leq p^m-1 $. Then
\[(1+a)^{p^m}=\sum_{k=0}^{p^m}\binom{p^m}{k}a^k=1+a^{p^m}=1\] 



