\documentclass{article}
\usepackage{amsthm}
\usepackage{amssymb}
\usepackage{enumerate}
\usepackage{amsmath}
\usepackage{extarrows}
\usepackage{mathrsfs}
\usepackage[UseMSWordMultipleLineSpacing,MSWordLineSpacingMultiple=1.4]{zhlineskip}
\usepackage{lipsum}
\title{Algebra-1 Note}
\author{lin150117 }
\date{}
\newtheorem{definition}{Definition}[subsection]
\newtheorem{theorem}{Theorem}[subsection]
\newtheorem{example}{Example}
\newtheorem{remark}{Remark}
\newtheorem{corollary}{Corollary}
\newtheorem{proposition}{Proposition}

\begin{document}
\setlength{\parindent}{0pt}
\paragraph{Exercise 3.4}


$ \varphi(y+1-(x-1)^3)=0 $. Let  $ l:=y+1-(x-1)^3 $ \\
Then for  $ f\in \mathbb{C}[x,y] $, let  $ f=ml+n $, where  $ m\in \mathbb{C}[x,y],n\in\mathbb{C}[x] $. $ (\ast ) $ \\
Then  $ \varphi(f)=\varphi(n) $. $ \Rightarrow $ 
\[\varphi(f)=0\Leftrightarrow \varphi(n)=0\Leftrightarrow n(t+1)=0\Leftrightarrow n=0\]    
Hence  $ K=l\mathbb{C}[x,y]=(l) $.\\
By Correspondence theorem, the ideal  $ I  $ of  $ \mathbb{C}[x,y] $ that contains  $ K  $  gets an ideal  $ I/K $ of  $ \mathbb{C}[x,y]/K $. And by  $ (\ast)  $,  $ \mathbb{C}[x,y]/K=\mathbb{C}[x] $.\\
By Prop 11.3.22,  $ I/K=(\overline{x}) $ for some  $ \overline{x}\in  \mathbb{C}[x,y]/K  $. Then  $ I=(x,l) $ since  $ \forall i\in I,\overline{i}=\overline{x}\cdot \overline{f},\overline{f}\in \mathbb{C }[x,y]/K $ $ \Rightarrow  $  $ i=(x+k_1)(f+k_2)+k_3\in (x,l) $ for some $ k_1,k_2,k_3\in K=(l) $.  

\paragraph{Exercise3.9}

If  $ x  $ is nilpotent, assume   $ x^n=0 $. Then  $ (1-(-x))(1+(-x)+(-x)^2+\cdots+(-x)^{n-1})=1-(-x)^n=1 $  $ \Rightarrow $  $ (1+x)^{-1}=(1-(-x))^{-1}=(1+(-x)+(-x)^2+\cdots+(-x)^{n-1}) $ is a unit.

(b)Since  $ a  $ is nilpotent, then  $ a^n=0  $ for sufficient large  $ n  $. In particular,  $ \exists m\in \mathbb{N} $ s.t.  $ a^{p^m }=0 $. Noticed that  $ p|\binom{p^m}{k}  $ for all  $ 1 \leq k \leq p^m-1 $. Then
\[(1+a)^{p^m}=\sum_{k=0}^{p^m}\binom{p^m}{k}a^k=1+a^{p^m}=1\] 


\paragraph{Exercise4.4}
If there is an isomorphism  $ \varphi:\mathbb{Z}[x]/(2x^2+7) \rightarrow \mathbb{Z}[x]/(x^2+7)$, then  $ \varphi(\overline{1})=\overline{1}\Rightarrow -\overline{7}=-7\varphi(\overline{1})=\varphi(-\overline{7}) $($ -7 $ is the number -7). \\
 $ \Rightarrow $  $ -\overline{7}=\varphi(-\overline{7})=\varphi(\overline{2x^2})=2\varphi(\overline{x^2}) $ \\
  $ \Rightarrow  $  $ -7=2t+m(x^2+7) $ where  $ \varphi(\overline{x^2})=\overline{t},m\in \mathbb{Z}[x] $. But the sum of coefficients of RHS is even but LHS is odd. That's contradiction.\\
  So there is no isomorphism.

\paragraph{Exercise5.6}
(a) For every  $ \beta=\sum_{k=0}^n a_k\alpha^k $, $ \beta=(a\alpha)^n \sum_{k=0}^n a_k\alpha^k=\alpha^n\cdot \sum_{k=0 }^{n }a_k(a\alpha)^k\cdot a^{n-k}=\alpha^n\sum_{k=0 }^{n }a_k\cdot a^{n-k}=\alpha^nb$ for some  $ b\in R  $.

(b) For  $ b\in R  $,  $ b=(\alpha a)^n b= \alpha^n a^n b $. Then if  $ a^nb=0  $,  $ b   $ equals 0 in  $ R' $. If  $ b=0  $ in  $ R' $, i.e.  $ b=l(a\alpha-1) $ for some  $ l\in R'  $.  $ (a)\Rightarrow l=\alpha^k t  $ where  $ t\in R  $. Then  $ b=\alpha^kt(a\alpha -1) \Rightarrow a^{k+1}b=t(a-a)=0$.

(c)If  $ a  $ is a nilpotent,  $ (b)  $  $ \Rightarrow  $  $ b=0  $ in  $ R'  $ for  $ b\in R  $ . Thus  $ R'=R[x]/(ax-1)=\{0\} $.

If  $ R'=R[x]/(ax-1)=\{0\} $, then  $ 1=0  $ in  $ R'  $ $ \Rightarrow $   $ \exists n, a^n\cdot 1=0 $ by  $ (b)  $ $ \Rightarrow   $  $ a  $ is a nilpotent.

\paragraph{Exercise8.3}
 $ \mathbb{F}_2[x]/(x^3+x+1)=\{\overline{x^2},\overline{x},\overline{1},\overline{x^2+x},\overline{x+1},\overline{x^2+1},\overline{x^2+x+1},0\} $ and we have
 \[\overline{x^2}\cdot\overline{x^2+x+1}=\overline{x^2+x}+\overline{x+1}+\overline{x^2}=1,\overline{x}\cdot\overline{x^2+1}=-1=1\]
 \[\overline{x+1}\cdot\overline{x^2+x}=\overline{x+1}+\overline{x^2}+\overline{x^2}+\overline{x}=1\]
 so every nonzero element in  $ \mathbb{F}_2[x]/(x^3+x+1) $ is a unit. Therefore  $ \mathbb{F}_2[x]/(x^3+x+1) $ is a field.

Noticed that  $ \overline{x^2+x+1}\cdot \overline{x^2+x-1}=\overline{x^2+x}^2-1=\overline{x^4}+2\overline{x^3}+\overline{x^2}-1=\overline{-x^2-x}+2\overline{-x-1}+\overline{x^2}-1=0$. Then  $ \overline{x^2+x+1} $ has no inverse $ \Rightarrow \mathbb{F}_3[x]/(x^3+x+1) $ is not a field.

6.For  $ p=\sum_{n=0}^{\infty}a_nt^n,q=\sum_{n=0}^{b_n}t^n $, 
\[p\cdot q=\sum_{n=0}^{\infty}(\sum_{i=0}^n a_i\cdot b_{n-i})t^n\] 
Then  $ p  $ is a unit impiles that  $ a_0b_0=1 $  $ \Rightarrow  $  $ a_0  $ is a unit in  $ R $.

Conversely, if  $ a_0  $ is a unit in  $ R  $, define a sequence $ \{b_n\} $ s.t.
\[b_0=a_0^{-1},b_n=a_0^{-1}\cdot (-\sum_{i=1}^n a_ib_{n-i})\]
Then  $ \forall n \geq 1, \sum_{i=0}^n a_i\cdot b_{n-i}=0  $.
Therefore  $ q=\sum_{n=0}^\infty b_nt^n=p^{-1}  $  $ \Rightarrow   $  $ p  $ is a unit.

Conclusion:  $ p $ is a unit if and only if  $ p=\sum_{n=0}^\infty a_nt^n  $ where  $ a_0  $ is a unit in  $ R  $.  
\end{document}