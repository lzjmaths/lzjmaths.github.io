\subsection{presentation}
Given  $ f:R^n\rightarrow R^m  $  $ R $-linear map.\\
 $ M:=R^m/\img f $ is  $ R-module $.\\
 In this case,  $ f  $ is called a \textbf{presentaion} of  $ M $
 and if  $ M  $ admits such a presentation, we say  $ M  $ is of finite presentation.\\
 Usually, by change of basis, we get better presentation(for  $ R  $ Euclidean domain) through elementary row/column operation.
 \paragraph{Note.}Now, we may assume  $ R  $ is a PID.    
 \begin{theorem}
    If  $ V  $ is a free  $ R $-module of rank  $ n  $, and  $ W\subset V  $ is a submodule, then  $ \exists m \leq n $,  $ \exists v_1,\cdots,v_n $  $ R $-basis of  $ V  $,  $ \exists b_1|b_2\cdots|b_m\in R \,\not=0$ s.t.  $ W  $ is free of rank  $ m  $ with basis  $ b_1v_1,\cdots,b_mv_m $.\\
    Moreover,  $ \{(b_1)\supset\cdots\supset(b_m )\} $ is unique.        
 \end{theorem}
 \begin{theorem}
    If  $ V  $ is finite generated over a  PID  $ R  $, then  $ V\cong R/(b_1)\oplus\cdots\oplus R/(b_l)\oplus R^k  $ for  $ b_1|\cdots|b_l $. 
 \end{theorem}
 In particular, for  $ R=\mathbb{Z} $,  $ R $-module is exactly abelian group. Every finitely generated abelian group is isom to  $ \oplus\mathbb{Z}_{n_i}\oplus \mathbb{Z}^k $   
 \subsection{Rational canonical form and Jordan canonical form}
 \begin{theorem}
    For  $ T=A\in GL_n(K) $,  $ \exists P\in GL_n(K) $ s.t.
    \[P^{-1}AP=\begin{pmatrix}
        C_1 & 0 & \cdots & 0\\
        0 & C_2 & \cdots & 0\\
        \vdots & \vdots & \ddots & \vdots\\
        0 & 0 & \cdots & C_n
    \end{pmatrix}\] 
    where  $ C_i $ is comparion matrix of  $ f_i(x) $ 
 \end{theorem}