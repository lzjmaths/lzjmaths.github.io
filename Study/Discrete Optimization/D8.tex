%!TEX root = /lecture/Discrete_Optimistic.tex

\begin{lemma}
    Every time inner WHILE terminates, max-flow value is less than  $ \mathrm{val}(f)+m\Delta $. 
\end{lemma}

\begin{corollary}
    Each inner WHILE iterates  $  \leq 2m $. The times complexity is  $ O(m^2\log C) $  
\end{corollary}
\begin{proof}[Proof of Lemma]
    We let  $ A $ be all nodes reachable from  $ s $ and  $ B=S\setminus A $.  
    \begin{align*}
        \val(f)&=\sum_{e\in E\text{ from $ A $ to  $ B $ }}f(e)-\sum_{e\in E\text{ from  $ B $ to  $ A $ }}f(e)\\
        &>\sum_{e\in E\text{ from $ A $ to  $ B $ }}(c(e)-\Delta)-\sum_{e\in E\text{ from  $ B $ to  $ A $ }}(\Delta)\\
        &=c(A,B)-\sum_{e\in E\text{ between  $ A,B $ }}\Delta\\
        & \geq c(A,B)-m\Delta\\
        & \geq \mathrm{MaxFlow}-m\Delta
    \end{align*}
\end{proof}


\begin{algorithm}
    \caption{Shortest Augment Path}
    \begin{algorithmic}
        \STATE Initiate $ f\leftarrow 0 $.
        \WHILE{ $ \exists s\rightarrow t $ path in  $ G_f $}
            \STATE Find  $ P:s\rightarrow t $ in  $ G_f $ using least number of edges.
            \STATE Augment  $ (f,P) $.   
        \ENDWHILE 
    \end{algorithmic}
\end{algorithm}
\begin{lemma}
    Length of the shortest augmenting path never decreases.
\end{lemma}
\begin{lemma}\label{shortest augment path lemma -strongest version}
    After  $   \leq  m$ iterations, length of the shortest augmenting path strictly increases. Time complexity is  $ O(nm^2) $ 
\end{lemma}

\begin{proof}
    Assume $ f\xrightarrow{\mathrm{augment(f,P)}}f' $ 
    Denote  $ l(u), l'(u) $ as the length of the shortest  $ s\to u $ path in  $ G_f,G_{f'} $ respectively.
    
    Our goal is to  prove  $ l(u) \leq l'(u) $.

    $ l(u) $ determines "distance" to  $ s $. 
    
    Define the level graph as the set of  all  $ (u,v)\in E(G_f) $ such that  $ l(u)+1=l(v) $. 
    
    Call edges not belong to level graph as \textit{back edge}.

    \paragraph{Observation} Consider any  $ e\in E(G_{f'})\setminus E(G_f) $,  $ e $ must be a back edge in  $ G_f $.   

    Choose   $ u $ such that  $ l'(u)<l(u) $ and  $ l'(u) $ minimized.
    
    If  $ (v,u) $ is the edge in the shortest path of  $ G_{f'} $
    \[l(v) \leq l'(v)=l'(u)-1<l(u)-1 \leq l(u)-2\]
    so  $ (u,v) $ is not a back edge in  $ G_f $, hence  $ (v,u)\not\in E(G_{f'}) $, which causes contradiction.
\end{proof}

\begin{lemma}
    After  $  \leq m $ augmentation,  $ \exists u $,  $ l(u) $ strictly increases. It goes on no more $    n^2$     times, so the time complexity is  $ O(n^2m^2) $. 
\end{lemma}
\begin{proof}
    This lemma is much easier than the previous lemma. 

    Noticed that each augmentation adds back edges and removes at least one edges in level graph.
\end{proof}
\begin{proof}[Proof of Lemma \ref{shortest augment path lemma -strongest version}]
    \begin{claim}
        During the period when  $ l(t)  $ doesn't increase, the added edges in residual graph does not appear in shortest augmenting path.
    \end{claim}
    
    Suppose for contradiction:  $ \exists j<i $,  $ l_j(t)=l_i(t) $,  $ \exists  $  $ (v,u)  $ appears in the shortest augmenting path  $ P $ in  $ G_{f_i} $ and  $ l_j(v)=l_j(u)+1 $.
    
    Choose the edge  $ (v,u)  $ with smallest  $ i $ and then with largest  $ l_i(u) $.
    
    Then  $ l_i(u) \geq l_j(u)+2 $.
    So 

    \begin{align*}
        l_j(t)& \leq l_j(u)+|P[u\rightarrow t]|\\
        & \leq l_i(u)+|P[u\to t]|-2\\
        &\leq l_i(t)-2
    \end{align*}
    
\end{proof}

\paragraph{Recent work} [Chen et at al. '2022]
we can do in  $ O(m^{1+o(1)}) $.

\subsection{Appllication}

\subsubsection{Bipartite Matching}
\begin{example}[Bipartite Matching]
    For Bipartite graph $ G=(U,V,E) $, a matching  $ M\subset E $, we want to find  $ M $ to maximize  $ |M| $.    
\end{example}

We can construct two virtual nodes $ s,t  $ such that  $ s\to  $ all nodes in  $ U $ and  $ t\to  $ all nodes in  $ V $, with capacity  $ 1 $.

Then the maximum capacity of flow in the augmented graph is what we need.

\textbf{So the meaning of capacity can be generalized as the number of one node who can accommonded}

Now consider so-called "perfect matching", \ie  $ |M|=|U|=|V| $.

Note that  $ \exists $  perfect matching \st  $ \forall S\subset U $,  $ |\Gamma(S)| \geq |S| $.  
