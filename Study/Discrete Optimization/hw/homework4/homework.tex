% Sample tex file for usage of iidef.sty
% Homework template for Inference and Information
% UPDATE: October 12, 2017 by Xiangxiang
% UPDATE: 22/03/2018 by zhaofeng-shu33
\documentclass[a4paper]{article}
\usepackage[T1]{fontenc}
\usepackage{amsmath, amssymb, amsthm}
% amsmath: equation*, amssymb: mathbb, amsthm: proof
\usepackage{moreenum}
\usepackage{mathtools}
\usepackage{url}
\usepackage{graphicx}
\usepackage{subcaption}
\usepackage{booktabs} % toprule
\usepackage[mathcal]{eucal}
\usepackage{dsfont}

\usepackage{setspace}  
\setstretch{1.6}








\theoremstyle{definition}
\newtheorem{definition}{Definition}[section]
\newtheorem{example}[definition]{Example}
\newtheorem{exercise}[definition]{Exercise}
\newtheorem{remark}[definition]{Remark}
\newtheorem{observation}[definition]{Observation}
\newtheorem{assumption}[definition]{Assumption}
\newtheorem{convention}[definition]{Convention}
\newtheorem{priniple}[definition]{Principle}
\newtheorem{notation}[definition]{Notation}
\newtheorem*{axiom}{Axiom}
\newtheorem{coa}[definition]{Theorem}
\newtheorem{srem}[definition]{$\star$ Remark}
\newtheorem{seg}[definition]{$\star$ Example}
\newtheorem{sexe}[definition]{$\star$ Exercise}
\newtheorem{sdf}[definition]{$\star$ Definition}
\newtheorem{question}{Question}




\newtheorem{problem}{Problem}
%\renewcommand*{\theprob}{{\color{red}\arabic{section}.\arabic{prob}}}
\newtheorem{rprob}[problem]{\color{red} Problem}
%\renewcommand*{\thesprob}{{\color{red}\arabic{section}.\arabic{sprob}}}
% \newtheorem{ssprob}[prob]{$\star\star$ Problem}



\theoremstyle{plain}
\newtheorem{theorem}[definition]{Theorem}
\newtheorem{Conclusion}[definition]{Conclusion}
\newtheorem{thd}[definition]{Theorem-Definition}
\newtheorem{proposition}[definition]{Proposition}
\newtheorem{corollary}[definition]{Corollary}
\newtheorem{lemma}[definition]{Lemma}
\newtheorem{sthm}[definition]{$\star$ Theorem}
\newtheorem{slm}[definition]{$\star$ Lemma}
\newtheorem{claim}[definition]{Claim}
\newtheorem{spp}[definition]{$\star$ Proposition}
\newtheorem{scorollary}[definition]{$\star$ Corollary}


\newtheorem{condition}{Condition}
\newtheorem{Mthm}{Main Theorem}
\renewcommand{\thecondition}{\Alph{condition}} % "letter-numbered" theorems
\renewcommand{\theMthm}{\Alph{Mthm}} % "letter-numbered" theorems


%\substack   multiple lines under sum
%\underset{b}{a}   b is under a


% Remind: \overline{L_0}



\usepackage{calligra}
\DeclareMathOperator{\shom}{\mathscr{H}\text{\kern -3pt {\calligra\large om}}\,}
\DeclareMathOperator{\sext}{\mathscr{E}\text{\kern -3pt {\calligra\large xt}}\,}
\DeclareMathOperator{\Rel}{\mathscr{R}\text{\kern -3pt {\calligra\large el}~}\,}
\DeclareMathOperator{\sann}{\mathscr{A}\text{\kern -3pt {\calligra\large nn}}\,}
\DeclareMathOperator{\send}{\mathscr{E}\text{\kern -3pt {\calligra\large nd}}\,}
\DeclareMathOperator{\stor}{\mathscr{T}\text{\kern -3pt {\calligra\large or}}\,}
%write mathscr Hom (and so on) 

\usepackage{aurical}
\DeclareMathOperator{\VVir}{\text{\Fontlukas V}\text{\kern -0pt {\Fontlukas\large ir}}\,}

\newcommand{\vol}{\text{\Fontlukas V}}
\newcommand{\dvol}{d~\text{\Fontlukas V}}
% perfect Vol symbol

\usepackage{aurical}








\newcommand{\fk}{\mathfrak}
\newcommand{\mc}{\mathcal}
\newcommand{\wtd}{\widetilde}
\newcommand{\wht}{\widehat}
\newcommand{\wch}{\widecheck}
\newcommand{\ovl}{\overline}
\newcommand{\udl}{\underline}
\newcommand{\tr}{\mathrm{t}} %transpose
\newcommand{\Tr}{\mathrm{Tr}}
\newcommand{\End}{\mathrm{End}} %endomorphism
\newcommand{\idt}{\mathbf{1}}
\newcommand{\id}{\mathrm{id}}
\newcommand{\Hom}{\mathrm{Hom}}
\newcommand{\cond}[1]{\mathrm{cond}_{#1}}
\newcommand{\Conf}{\mathrm{Conf}}
\newcommand{\Res}{\mathrm{Res}}
\newcommand{\res}{\mathrm{res}}
\newcommand{\KZ}{\mathrm{KZ}}
\newcommand{\ev}{\mathrm{ev}}
\newcommand{\coev}{\mathrm{coev}}
\newcommand{\opp}{\mathrm{opp}}
\newcommand{\Rep}{\mathrm{Rep}}
\newcommand{\Dom}{\mathrm{Dom}}
\newcommand{\loc}{\mathrm{loc}}
\newcommand{\con}{\mathrm{c}}
\newcommand{\uni}{\mathrm{u}}
\newcommand{\ssp}{\mathrm{ss}}
\newcommand{\di}{\slashed d}
\newcommand{\Diffp}{\mathrm{Diff}^+}
\newcommand{\Diff}{\mathrm{Diff}}
\newcommand{\PSU}{\mathrm{PSU}(1,1)}
\newcommand{\Vir}{\mathrm{Vir}}
\newcommand{\Witt}{\mathscr W}
\newcommand{\Span}{\mathrm{Span}}
\newcommand{\pri}{\mathrm{p}}
\newcommand{\ER}{E^1(V)_{\mathbb R}}
\newcommand{\prth}[1]{( {#1})}
\newcommand{\bk}[1]{\langle {#1}\rangle}
\newcommand{\bigbk}[1]{\big\langle {#1}\big\rangle}
\newcommand{\Bigbk}[1]{\Big\langle {#1}\Big\rangle}
\newcommand{\biggbk}[1]{\bigg\langle {#1}\bigg\rangle}
\newcommand{\Biggbk}[1]{\Bigg\langle {#1}\Bigg\rangle}
\newcommand{\GA}{\mathscr G_{\mathcal A}}
\newcommand{\vs}{\varsigma}
\newcommand{\Vect}{\mathrm{Vec}}
\newcommand{\Vectc}{\mathrm{Vec}^{\mathbb C}}
\newcommand{\scr}{\mathscr}
\newcommand{\sjs}{\subset\joinrel\subset}
\newcommand{\Jtd}{\widetilde{\mathcal J}}
\newcommand{\gk}{\mathfrak g}
\newcommand{\hk}{\mathfrak h}
\newcommand{\xk}{\mathfrak x}
\newcommand{\yk}{\mathfrak y}
\newcommand{\zk}{\mathfrak z}
\newcommand{\pk}{\mathfrak p}
\newcommand{\hr}{\mathfrak h_{\mathbb R}}
\newcommand{\Ad}{\mathrm{Ad}}
\newcommand{\DHR}{\mathrm{DHR}_{I_0}}
\newcommand{\Repi}{\mathrm{Rep}_{\wtd I_0}}
\newcommand{\im}{\mathbf{i}}
\newcommand{\Co}{\complement}
%\newcommand{\Cu}{\mathcal C^{\mathrm u}}
\newcommand{\RepV}{\mathrm{Rep}^\uni(V)}
\newcommand{\RepA}{\mathrm{Rep}(\mathcal A)}
\newcommand{\RepN}{\mathrm{Rep}(\mathcal N)}
\newcommand{\RepfA}{\mathrm{Rep}^{\mathrm f}(\mathcal A)}
\newcommand{\RepAU}{\mathrm{Rep}^\uni(A_U)}
\newcommand{\RepU}{\mathrm{Rep}^\uni(U)}
\newcommand{\RepL}{\mathrm{Rep}^{\mathrm{L}}}
\newcommand{\HomL}{\mathrm{Hom}^{\mathrm{L}}}
\newcommand{\EndL}{\mathrm{End}^{\mathrm{L}}}
\newcommand{\Bim}{\mathrm{Bim}}
\newcommand{\BimA}{\mathrm{Bim}^\uni(A)}
%\newcommand{\shom}{\scr Hom}
\newcommand{\divi}{\mathrm{div}}
\newcommand{\sgm}{\varsigma}
\newcommand{\SX}{{S_{\fk X}}}
\newcommand{\DX}{D_{\fk X}}
\newcommand{\mbb}{\mathbb}
\newcommand{\mbf}{\mathbf}
\newcommand{\bsb}{\boldsymbol}
\newcommand{\blt}{\bullet}
\newcommand{\Vbb}{\mathbb V}
\newcommand{\Ubb}{\mathbb U}
\newcommand{\Xbb}{\mathbb X}
\newcommand{\Kbb}{\mathbb K}
\newcommand{\Abb}{\mathbb A}
\newcommand{\Wbb}{\mathbb W}
\newcommand{\Mbb}{\mathbb M}
\newcommand{\Gbb}{\mathbb G}
\newcommand{\Cbb}{\mathbb C}
\newcommand{\Nbb}{\mathbb N}
\newcommand{\Zbb}{\mathbb Z}
\newcommand{\Qbb}{\mathbb Q}
\newcommand{\Pbb}{\mathbb P}
\newcommand{\Rbb}{\mathbb R}
\newcommand{\Ebb}{\mathbb E}
\newcommand{\Dbb}{\mathbb D}
\newcommand{\Hbb}{\mathbb H}
\newcommand{\cbf}{\mathbf c}
\newcommand{\Rbf}{\mathbf R}
\newcommand{\wt}{\mathrm{wt}}
\newcommand{\Lie}{\mathrm{Lie}}
\newcommand{\btl}{\blacktriangleleft}
\newcommand{\btr}{\blacktriangleright}
\newcommand{\svir}{\mathcal V\!\mathit{ir}}
\newcommand{\Ker}{\mathrm{Ker}}
\newcommand{\Cok}{\mathrm{Coker}}
\newcommand{\Sbf}{\mathbf{S}}
\newcommand{\low}{\mathrm{low}}
\newcommand{\Sp}{\mathrm{Sp}}
\newcommand{\Rng}{\mathrm{Rng}}
\newcommand{\vN}{\mathrm{vN}}
\newcommand{\Ebf}{\mathbf E}
\newcommand{\Nbf}{\mathbf N}
\newcommand{\Stb}{\mathrm {Stb}}
\newcommand{\SXb}{{S_{\fk X_b}}}
\newcommand{\pr}{\mathrm {pr}}
\newcommand{\SXtd}{S_{\wtd{\fk X}}}
\newcommand{\univ}{\mathrm {univ}}
\newcommand{\vbf}{\mathbf v}
\newcommand{\ubf}{\mathbf u}
\newcommand{\wbf}{\mathbf w}
\newcommand{\CB}{\mathrm{CB}}
\newcommand{\Perm}{\mathrm{Perm}}
\newcommand{\Orb}{\mathrm{Orb}}
\newcommand{\Lss}{{L_{0,\mathrm{s}}}}
\newcommand{\Lni}{{L_{0,\mathrm{n}}}}
\newcommand{\UPSU}{\widetilde{\mathrm{PSU}}(1,1)}
\newcommand{\Sbb}{{\mathbb S}}
\newcommand{\Gc}{\mathscr G_c}
\newcommand{\Obj}{\mathrm{Obj}}
\newcommand{\bpr}{{}^\backprime}
\newcommand{\fin}{\mathrm{fin}}
\newcommand{\Ann}{\mathrm{Ann}}
\newcommand{\Real}{\mathrm{Re}}
\newcommand{\Imag}{\mathrm{Im}}
%\newcommand{\cl}{\mathrm{cl}}
\newcommand{\Ind}{\mathrm{Ind}}
\newcommand{\Supp}{\mathrm{Supp}}
\newcommand{\Specan}{\mathrm{Specan}}
\newcommand{\red}{\mathrm{red}}
\newcommand{\uph}{\upharpoonright}
\newcommand{\Mor}{\mathrm{Mor}}
\newcommand{\pre}{\mathrm{pre}}
\newcommand{\rank}{\mathrm{rank}}
\newcommand{\Jac}{\mathrm{Jac}}
\newcommand{\emb}{\mathrm{emb}}
\newcommand{\Sg}{\mathrm{Sg}}
\newcommand{\Nzd}{\mathrm{Nzd}}
\newcommand{\Owht}{\widehat{\scr O}}
\newcommand{\Ext}{\mathrm{Ext}}
\newcommand{\Tor}{\mathrm{Tor}}
\newcommand{\Com}{\mathrm{Com}}
\newcommand{\Mod}{\mathrm{Mod}}
\newcommand{\nk}{\mathfrak n}
\newcommand{\mk}{\mathfrak m}
\newcommand{\Ass}{\mathrm{Ass}}
\newcommand{\depth}{\mathrm{depth}}
\newcommand{\Coh}{\mathrm{Coh}}
\newcommand{\Gode}{\mathrm{Gode}}
\newcommand{\Fbb}{\mathbb F}
\newcommand{\sgn}{\mathrm{sgn}}
\newcommand{\Aut}{\mathrm{Aut}}
\newcommand{\Modf}{\mathrm{Mod}^{\mathrm f}}
\newcommand{\codim}{\mathrm{codim}}
\newcommand{\card}{\mathrm{card}}
\newcommand{\dps}{\displaystyle}
\newcommand{\Int}{\mathrm{Int}}
\newcommand{\Nbh}{\mathrm{Nbh}}
\newcommand{\Pnbh}{\mathrm{PNbh}}
\newcommand{\Cl}{\mathrm{Cl}}
\newcommand{\diam}{\mathrm{diam}}
\newcommand{\eps}{\varepsilon}
\newcommand{\Vol}{\mathrm{Vol}}
\newcommand{\LSC}{\mathrm{LSC}}
\newcommand{\USC}{\mathrm{USC}}
\newcommand{\Ess}{\mathrm{Rng}^{\mathrm{ess}}}
\newcommand{\Jbf}{\mathbf{J}}
\newcommand{\SL}{\mathrm{SL}}
\newcommand{\GL}{\mathrm{GL}}
\newcommand{\Lin}{\mathrm{Lin}}
\newcommand{\ALin}{\mathrm{ALin}}
\newcommand{\bwn}{\bigwedge\nolimits}
\newcommand{\nbf}{\mathbf n}
\newcommand{\dive}{\mathrm{div}}




\usepackage{algorithm}
\usepackage{algorithmic}


\newcommand{\OPT}{\mathrm{OPT}}



\numberwithin{equation}{problem}
% count the eqation by section countation


\DeclareMathOperator{\sign}{sign}
\DeclareMathOperator{\dom}{dom}
\DeclareMathOperator{\ran}{ran}
\DeclareMathOperator{\ord}{ord}
\DeclareMathOperator{\img}{Im}
\DeclareMathOperator{\dd}{d\!}
\newcommand{\ie}{ \textit{ i.e. } }
\newcommand{\st}{ \textit{ s.t. }}


\usepackage[numbered,framed]{matlab-prettifier}
\lstset{
  style              = Matlab-editor,
  captionpos         =b,
  basicstyle         = \mlttfamily,
  escapechar         = ",
  mlshowsectionrules = true,
}

\usepackage[thehwcnt = 1]{iidef}
\usepackage{tikz} % Required for tikzpicture environment
\thecourseinstitute{Tsinghua University}
\thecoursename{Discrete Optimistic}
\theterm{Fall 2024}
\hwname{Homework}
\usepackage{geometry}
\geometry{left=1.5cm,right=1.5cm,top=2.5cm,bottom=2.5cm}
\begin{document}

\courseheader
\name{Lin Zejin}
\rule{\textwidth}{1pt}
\begin{itemize}
\item {\bf Collaborators: \/}
  I finish this homework by myself. 
%   If you finish your homework all by yourself, make a similar statement. If you get help from others in finishing your homework, state like this:
%   \begin{itemize}
%   \item 1.2 (b) was solved with the help from \underline{\hspace{3em}}.
%   \item Discussion with \underline{\hspace{3em}} helped me finishing 1.3.
%   \end{itemize}
\end{itemize}
\rule{\textwidth}{1pt}

\vspace{2em}


\sloppy
\pagenumbering{arabic}

\begin{problem}
    (a) When  $ \OPT \geq c $, assume with  $ \frac{1}{T} $ algorithm  $ A $  outputs a solution of value at least  $ s $.  $ T\in O(poly(n)) $  Run algorithm  $ A $ for  $ T\cdot n $ iterations. Then with  $ (1-\frac{1}{T})^{Tn}<e^{-n} $ probability, the algorithm  $ A $  outputs a solution of value less than  $ s $.

    So with at least  $ 1-e^{-n} $ probability, the algorithm  $ A $  outputs a solution of value at least  $ s $.

    (b)
    \[\begin{aligned}
        s=\Ebb[outputs]& \leq \mathrm{Pr}[outputs \geq s-\frac{1}{n^a}]\cdot poly(n)+(1-\mathrm{Pr}[outputs \geq s-\frac{1}{n^a}])\cdot (s-\frac{1}{n^a})
    \end{aligned}\]
    Then
    \[\mathrm{Pr}[outputs \geq s-\frac{1}{n^a}]  \geq \frac{\frac{1}{n^a}}{poly(n)-s+\frac{1}{n^a}}=\frac{1}{n^a (poly(n)-s)+1}\]
    Here we end the proof.
\end{problem}
\begin{problem}
  (a) Use the original greedy algorithm  $ \lceil \ln(n/\OPT)\rceil \cdot \OPT $ times, there are at most 
  \[(1-1/\OPT)^{\lceil \ln(n/\OPT)\rceil \cdot \OPT}\cdot n \leq e^{-\ln(n/\OPT)}\cdot n=\OPT\]
  elements that do not cover.

  So suffices to find at most $ \OPT $  sets to cover those elements in polynomial time.

  Here we obtain a  $ \lceil \ln(n/\OPT)\rceil+1 $-factor approximation algorithm.
  
  (b) If  the optimum solution covers  $ c\times 100\% $ elements, denote  $ S $ as the set of elements that optimum solution covers.
  
  For  $ t $-step, there exists a set in optimum solution that covers  $ (1-\frac{1}{k})\cdot \text{ elements in  $ S $ uncovered} $.
  
  By induction, we can prove greedy algorithm covers  $(1-(1- \frac{1}{k})^t)\cdot c $ elements in  $ t $-step.
  
  Therefore, greedy algorithm returns a value larger than 
  \[(1-(1-\frac{1}{k})^k)c  \geq (1-\frac{1}{e})c\]

  So greedy algorithm is a  $ (1-\frac{1}{e}) $-factor approximation algorithm.
  
  (c) If  $ e_i $ is covered by  $ S_i $ in the solution, denote 
  \[p(e_i)=\frac{\omega(S_i)}{\text{number of uncovered elements that  $ S_i $ would cover}}\]
  Then 
  \[\sum \omega(S_i)=\sum_{i=1}^n p(e_i)\]

  We prove that if  $ e_i $ is covered in  $ k $-step, then  $ p(e_i) \leq \frac{\OPT}{n-k+1} $.
  
  If the optimal sets are  $ O_1,O_2,\cdots,O_{p} $,  then  $ \OPT=\dps\sum_{i=1}^p\omega( O_i) $.
  
  For any  $ O_i=\{x_k,x_{k-1},\cdots, x_1\} $, wlog, we assume the algorithm covers  $ x_k,x_{k-1},\cdots,x_1 $ in order.
  
  Then  at the start of the iteration in which the algorithm covers element  $ x_j $  of  $ O_i $, at least  $ i $ elements that do not covered in  $ O_i $. Since  $ p(x_i) $ takes  minimum in the equation, 
  \[p(x_i) \leq \frac{\omega(O_i)}{i}\]
  
  Therefore,  $ \dps\sum_{i=1}^k p(x_i) \leq H_D\omega(O_i) $ 
  % Now, assume the greedy algorithm has covered elements in  $ C $ so far. Then we know the uncovered elements  $ U\setminus C $,
  % \[|U\setminus C| \leq \sum_{i=1}^p |O_i\cap (U\setminus C)\]  

  % Since  $ p(e_i) $ takes minimum in its definition, \ie 
  % \[p(e_i) \leq \frac{\omega(O_i)}{|O_i\cap (U\setminus C)|}\]
  % Therefore,
  % \[p(e_i)\cdot|O_i\cap (U\setminus C)| \leq \omega(O_i)\]
  % So
  % \[p(e_i)|U\setminus C| \leq p(e_i)\cdot\sum_{i=1}^p |O_i\cap (U\setminus C)| \leq \sum_{i=1}^p\omega(O_i)\]
  % Hence, 
  % \[p(e_i) \leq \frac{\OPT}{|U\setminus C|} \leq \frac{\OPT}{n-k+1}\]
  
  Then 
  \[\mathrm{Val}=\sum_{i=1}^D p(e_i) \leq (1+\frac{1}{2}+\cdots+\frac{1}{D})\OPT\]

  (d) Consider the set  $ [D] $ and  $ S_i=\{i\},S_{D+1}=[D] $. Equipped with 
  \[\omega(S_i)=\frac{1}{i},i=1,2,\cdots, D,\omega(S_{D+1})=1+\epsilon\]

  Then in  $ t $-step,  $ \dps\frac{1}{D-t+1}<\frac{1+\epsilon}{D-t+1} $, so algorithm chooses  $ S_{D-t+1} $. Therefore, the value of algorithm is  $ H_D $ but  $ \OPT=1+\epsilon $.     
\end{problem}

\begin{problem}
  (a) Let  $ k=c $,  $ U=\{1,2,\cdots,c\}^q $ where  $ q $ large enough. Introduce 
  \[S_{i,b}=\{e\in U:e_i=b\}, \,i\in [q],b\in [c]\]
  Choose  $ S_{i,t},1 \leq t \leq c $ and the coverage is  $ 1 $.
  
   $ x_{i,b}^*=\frac{1}{q} $ will also achieves coverage  $ 1 $.  

  Then we cover each  $ j\in U $    with probability 
  \[1-(1-\frac{1}{c})^c\]
  So the expected coverage of rounding is 
  \[1-(1-\frac{1}{c})^c\]
  AS  $ c $ large enough, the expected  coverage of rounding is  $ 1-\frac{1}{e} $.
  
  
  (b) With instance  $ k=c $,  $ U=\{0,1\}^q $,  $ n=2^q $ and 
  \[S_{i,b}=\{e\in U:e_i=b\}, \,i\in [q],b=0,1\]
  
  The LP solution  $ x_{i,b}^*=\frac{1}{q} $. 

   $ \alpha x_{i,b}^*=\frac{(1-\epsilon)\ln n}{q}=(1-\epsilon)\ln 2<\ln 2 $.
   
  Then 
  \[\mathrm{Pr}[j \text{ is covered}]=1-(1-\alpha x_{i,b}^*)^q<1-(1-\ln 2)^q<1-(2^{-1.5})^{\log_2 n}=1-n^{-3/2}\]

  So 
  \[\mathrm{Pr}[U\text{ is all covered}]<(1-n^{-3/2})^n<1-n^{-\frac{1}{2}+\epsilon}\]
  as  $ n $ large enough. So the randomized rounding algorithm may not be able to find a feasible solutionwith probability at least $ n^{-\frac{1}{2}+\epsilon} $. 

\end{problem}

\begin{problem}
  (a)
  

  (b) No, since the rounding algorithm gives a solution with expected value large than  $ (1-\frac{1}{e})\mathrm{LP} $. So 
  \[\OPT \geq (1-\frac{1}{e})\mathrm{LP}\]
  always holds. 
\end{problem}

\begin{problem}
  (a)  Suffices to prove the decision problem that if there exists a clique of size  $ k $ is whether or not NP-complete.

  For an 3-SAT instance  with clauses  $ c_1,\cdots,c_m $ and literals $ x_1,\cdots,x_n $, we can construct a graph  $ G $ with vertices  $ c_1,\cdots,c_m,x_1,\bar{x}_1,x_2,\bar{x}_2,\cdots,x_n,\bar{x}_n $ and additional clauses  $ x_{j_1}\wedge x_{j_2}\wedge x_{j_3} $ if  there is some  $ c_j $ is composed by $ x_{j_1},x_{j_2},x_{j_3} $ or  some of its nagation.   Let  $ k=2m+n $  
  
  First construct a complete graph. 
  
  Remove the edges between  $ x_i $ and  $ \bar{x}_i $  

  Remove those edges that connnects  $ c_j=a\vee  b\vee c  $ and  $ \bar{a}\wedge \bar{b}\wedge \bar{c} $.

  For a clause  $ a\wedge b\wedge c $, which means  $ a,b,c $ holds in the same time, we remove the edges between  $ a\wedge b\wedge c $ and those additional clauses that contains one of  $ \bar{a},\bar{b},\bar{c}$     

  Remove the edges between  $ a\wedge b\wedge c $ and  $ \bar{a},\bar{b},\bar{c} $  
  
  Then, that a graph is clique is equivalent to this conditions:

  (1) If  $ a\wedge b\wedge c $ belongs to it, then additional clauses that contains  $ \bar{a},\bar{b},\bar{c} $ cannot belong to the same clique, and  $ c_j=\bar{a}\vee \bar{b}\vee\bar{c} $ cannot belong if it exists.
  
  (2) If  $ x_i $ belongs to it, then  $ \bar{x}_i $ cannot belong to the same clique and those additional clauses contains  $ \bar{x}_i $   cannot belong to the same clique.


  Certainly, the size of the clique in this graph cannot be larger than  $ 2m+n $ since each clauses  $ c_j $  corresponds to  $ 8 $ additional clauses but at most one of them is contained. 
   
  Therefore, if  $ 3-SAT $ is satifiable, then we  choose all  true literals and   all clauses that is true in the solution. Then  $ c_j $ will be chosen all and exactly one of eight addtional clauses that corresponds to  $ c_j $ will be chosen, so  $ k=2m+n $ is reachable. 
  
  If  there exists a clique of size  $ k=2m+n $, which means choosing  $ n $ of  independent literals,  $ c_j $ and exactly one of eight additional clauses that corresponds to  $ c_j $. Those clauses will be TRUE when the literals that is chosen is assuemd TRUE.
  
  So it is a feasible instance for max-clique problem.

  Therefore, It is NP-Hard.

  (b) 
  
  If there is a  $ 1-\epsilon $-approximation polynomial algorithm for graph in (a) and returns a solution. 

  First, we prove that we can find a solution in polynomial time that contains $ n $ independent literals. That's because, each additional clause  $ a\wedge b\wedge c $  implies that  $ a,b,c  $ is TRUE, which will not cause a contradiction by the clique assumption. So we actually can choose those TRUE literals and other random literals that is not mentioned. Since we want to return a max-clique, we actually can return a solution that contains  $ n  $ independent literals. 
  
  Second, we prove we can find a solution in polynomial time that contains  $ m  $ additional clauses, \ie exactly one of eight additional clauses that corresponds to  $ c_j $ is contained.

  Otherwise, if all of eight additional clauses that contains  $ a,b,c $ or their negation  are not contained. Then since we find a solution that contains  $ n $ independent literals, we can choose an additional clause  $ a'\wedge b'\wedge c' $ such that  $ a'\in\{a,\bar{a}\},b'\in\{b,\bar{b}\},c'\in \{c,\bar{c}\} $ and  $ a',b',c' $  are chosen. And we can remove vertices  $ \bar{a'}\vee \bar{b'}\vee \bar{c'} $  if exists. Then it remains a clique of the same size. After  $ O(m) $, we actually obtain a solution that contains  $ n $ independent literals and  $ m $ additional clauses.  
  
  The graph we obtain actually finds a solution in  $ 3-SAT $, with value 
  \[\frac{(1-\epsilon)(n+2m)-n-m}{m}=\frac{(1-2\epsilon)m-\epsilon n}{m}>1-3\epsilon\]
  if  $ n<m $.
  
  3-SAT for $ n \geq m $ is P-solved. So we find a  $ (1-3\epsilon) $-approximation polynomial algorithm for  $ 3-SAT $.
  
  PCP theorem implies  $ 3\epsilon>\epsilon_0 $. So  $ \exists \epsilon_1>0 $  such that  $ (1-\epsilon_1) $-approximation polynomial algorithm for  max-clique is NP-hard.
  
  For a graph  $ G=(V,E) $, define 
  \[G^{\otimes t}=(V^{\otimes t},E^{\otimes t})\]
  where 
  \[V^{\times t}=\{(v_1,v_2,\cdots,v_t):v_i\in V\}\]
  $ (v_1,\cdots,v_t),(u_1,\cdots,u_t) $ are connected iff  $ (v_i,u_i)\in E $, $ \forall i=1,2,\cdots,t $.  
  
  Then  $ S\subset G^{\times t} $ is a clique iff 
  \[S^i=\{v_i:(v_1,\cdots,v_i,\cdots,v_t)\in S\}\]
  are all  clique.
  
  Since 
  \[|S| \leq \prod_{i=1}^t|S_i|\]
  $ \Rightarrow $  $ \exists $  $ |S_i| \geq |S|^{1/t} $.
  
  Therefore, if we find a solution with value  $ (1-\epsilon)^t $ in  $ G^{\otimes t} $, then we find a solution with value   $ 1-\epsilon $.
  
  $ (1-\epsilon) $-approximation is NP-Hard  $ \Rightarrow  $  $ (1-\epsilon)^t$-approximation is NP-Hard.
  
  So  $ \forall \delta>0 $,  $ \delta $-approximation is NP-Hard.  

  

\end{problem}

\begin{problem}
  Consider the verifier reads one of  $ 3-CNF $ uniformly and returns the result of this 3-CNF under the value given by prover. 

  If there exists a GAP-3SAT  solution with value  $ 1 $, then  verifier always accepts.

  If there is  a GAP-3SAT solution with value less than  $ s $, then  verifier accepts with probability less than  $ s $.
  
  So  $ \mathrm{GAP-3SAT}_{1,s} $  $ \in \mathrm{PCP}_{1,s}[O(\log n),3] $.
  
  Therefore, for any  $ NP $ problem  $ \mathcal{L} $,  $ \mathrm{GAP-3SAT}_{1,s} $ NP-Hard  $ \Rightarrow  $  $ \mathcal{L} \leq _p \mathrm{GAP-3SAT}_{1,s} $.
  
  So 
  \[\mathcal{L}\in \mathrm{PCP}_{1,s}[O(\log n),3] \leq _p\mathrm{PCP}_{1,\frac{1}{2}}[O(\log n),O(1)]\]

  In the lecture we prove that  $ NP \geq _p\mathrm{PCP}_{1,\frac{1}{2}}[O(\log n),O(1)]  $.

  So  $ NP=\mathrm{PCP}_{1,\frac{1}{2}}[O(\log n),O(1)] $. \ie PCP theorem holds. 
\end{problem} 

\begin{problem}
  There is a counter-example for  $ U=\{u_1,u_2\},V=\{v_1,v_2\},K=L=4 $ and  $ G $ is fully connected. 
$ [K],[L]\leftrightarrow \{u,v\}\times\{1,2\} $ 

$ \pi_{(u_i,v_j)}=\{((u,i),(u,i)),((u,i'),(u,i)),((v,j),(v,j)),((v,j'),(v,j))\} $ where  $ \{i,i'\}=\{j,j'\}=\{1,2\} $  

Clearly,  $ \OPT(H)=\frac{1}{2} $ since two edges from vertices in  $ V $ cannot be satisfied both.

However,  $ \OPT(H^{\otimes 2})=\frac{1}{2} $.

Let  $ \sigma((u_{i_1},u_{i_2}))=(((u,i_1),(v,i_1))) $,  $ \sigma((v_{j_2},v_{j_2}))=((u,j_1),(v,j_2)) $.  

So verifier accepts if  $ i_1=j_2 $.
\end{problem}
\end{document}