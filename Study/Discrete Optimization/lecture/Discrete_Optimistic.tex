% !TeX spellcheck = en_US
% !TEX program = pdflatex
\documentclass[12pt,b5paper,notitlepage]{article}
\usepackage[b5paper, margin={0.5in,0.65in}]{geometry}
%\usepackage{fullpage}
\usepackage{mathtools} % perfect math mode, but seems not well for underbrace. It is recommended to use \begingroup in the file.
\let\underbrace\overbrace\relax
\usepackage{amsmath,amscd,amssymb,amsthm,mathrsfs,amsfonts,layout,indentfirst,graphicx,caption,mathabx, stmaryrd,appendix,calc,imakeidx,upgreek,amsbsy,thmtools} % mathabx for \wtidecheck
%\usepackage{ulem} %wave underline
\usepackage[dvipsnames]{xcolor}
\usepackage{palatino}  %template

\usepackage{stmaryrd}


\usepackage{slashed} % Dirac operator
\usepackage{mathrsfs} % Enable using \mathscr
%\usepackage{eufrak}  another template/font
\usepackage{extarrows} % long equal sign, \xlongequal{blablabla}
% \usepackage{pdfMsym}
\usepackage{enumitem} % enumerate label change e.g. [label=(\alph*)]  shows (a) (b) 

\usepackage{comment} % hide something


% \usepackage{titlesec}
% \usepackage{etoolbox}
% \pretocmd{\section}{\refstepcounter{section}\label{sec:\thesection}}{}{}
% \AtBeginEnvironment{theorem}{%
% 	\refstepcounter{theorem}%
% 	\edef\theoremname{\theoremname}% 获取定理的名字(如果有的话)
% 	\@ifundefined{theoremname}
%     {% 如果没有名字,只使用编号
%      \label{thm:\thesection.\thetheorem}%
%     }
%     {% 如果有名字,标签包含编号和名字
%      \label{thm:\thesection.\thetheorem.\theoremname}%
%     }
% }

%%%%%%%%%%%%%%%%%%%%%%%%%%%%%%

%\usepackage{fontspec}
%\setmainfont{Palatino Linotype}
%\usepackage{emoji}


% emoji, use lualatex  remove \usepackage{palatino}

%%%%%%%%%%%%%


\usepackage{CJK}   % Chinese package
\usepackage{pifont}




\usepackage{csquotes} % \begin{displayquote}   \begin{displaycquote}  for quotation
\usepackage{epigraph}   %\epigraph{}{}  for quotation
%\pmb  mandatory math bold 

\usepackage{fancyhdr} % date in footer



% 设置章节标题格式
% \titleformat{\section}[hang]{\normalfont\huge\bfseries}{\thesection}{2pc}{}

% \pagestyle{fancy}
% \fancyhead[L]{\rightmark}  % 左侧显示章节名
% \fancyhead[C]{}           % 中间清空
% \fancyhead[R]{}           % 右侧清空
\pagestyle{plain}

%\usepackage{soul}  %\ul underline break line automatically

\usepackage{ulem}  % \uline  underline break line   also    \uwave

\usepackage{relsize} % use \mathlarger \larger \text{\larger[2]$...$} to enlarge the size of math symbols

\usepackage{verbatim}  % comment environment


\usepackage{halloweenmath} % Interesting halloween math symbols

%%%%%%%%%%%%%%%%%%%%%%%%%%%%%%
\usepackage{tcolorbox}
\tcbuselibrary{theorems}
% box around equations   \tcboxmath
%%%%%%%%%%%%%%%%%%%%%%%%%%%%%%%%%%




%%%%%%%%%%%%%%%%%%%%%%%%%%%%%
% circled colon and thick colon \hcolondel and \colondel

\usepackage{pdfrender}

\newcommand*{\hollowcolon}{%
	\textpdfrender{
		TextRenderingMode=Stroke,
		LineWidth=.1bp,
	}{:}%
}

\newcommand{\hcolondel}[1]{%
	\mathopen{\hollowcolon}#1\mathclose{\hollowcolon}%
}
\newcommand{\colondel}[1]{%
	\mathopen{:}#1\mathclose{:}%
}

%%%%%%%%%%%%%%%%%%%%%%%%%%%%%%%%


\usepackage{setspace}  
\setstretch{1.6}



\usepackage{tikz}
\usetikzlibrary{fadings}
\usetikzlibrary{patterns}
\usetikzlibrary{shadows.blur}
\usetikzlibrary{shapes}
% \usetikzlibrary{graphs,graphdrawing,circular}

% \usetikzlibrary{graphs, graphdrawing, shapes.geometric, arrows}

\usepackage{tikz-cd}
\usepackage[nottoc]{tocbibind}   % Add  reference to ToC


\makeindex


% The following set up the line spaces between items in \thebibliography
\usepackage{lipsum}  
\let\OLDthebibliography\thebibliography
\renewcommand\thebibliography[1]{
	\OLDthebibliography{#1}
	\setlength{\parskip}{0pt}
	\setlength{\itemsep}{2pt} 
}


%\hyperref{page.10}{...}

\allowdisplaybreaks  %allow aligns to break between pages
\usepackage{latexsym}
\usepackage{chngcntr}
\usepackage[colorlinks,linkcolor=blue,anchorcolor=blue, linktocpage,
%pagebackref
]{hyperref}
\hypersetup{ urlcolor=cyan,
	citecolor=[rgb]{0,0.5,0}}


\setcounter{tocdepth}{2}	 %hide subsections in the content


\counterwithin{figure}{section}

\counterwithin*{footnote}{section}   % Footnote numbering is recounted from the beginning of each subsection




% \pagestyle{plain}

% \captionsetup[figure]
% {
% 	labelsep=none	
% }
% 控制图表结构











\theoremstyle{definition}
\newtheorem{definition}{Definition}[section]
\newtheorem{example}[definition]{Example}
\newtheorem{exercise}[definition]{Exercise}
\newtheorem{remark}[definition]{Remark}
\newtheorem{observation}[definition]{Observation}
\newtheorem{assumption}[definition]{Assumption}
\newtheorem{convention}[definition]{Convention}
\newtheorem{priniple}[definition]{Principle}
\newtheorem{notation}[definition]{Notation}
\newtheorem*{axiom}{Axiom}
\newtheorem{coa}[definition]{Theorem}
\newtheorem{srem}[definition]{$\star$ Remark}
\newtheorem{seg}[definition]{$\star$ Example}
\newtheorem{sexe}[definition]{$\star$ Exercise}
\newtheorem{sdf}[definition]{$\star$ Definition}
\newtheorem{question}{Question}
\theoremstyle{remark}
\newtheorem{note}{Note}
\newtheorem*{claim}{Claim}


\newtheorem{problem}{\color{red}Problem}[section]
%\renewcommand*{\theprob}{{\color{red}\arabic{section}.\arabic{prob}}}
\newtheorem{sprob}[problem]{\color{red}$\star$ Problem}
%\renewcommand*{\thesprob}{{\color{red}\arabic{section}.\arabic{sprob}}}
% \newtheorem{ssprob}[prob]{$\star\star$ Problem}



\theoremstyle{plain}
\newtheorem{theorem}[definition]{Theorem}
\newtheorem{Conclusion}[definition]{Conclusion}
\newtheorem{thd}[definition]{Theorem-Definition}
\newtheorem{proposition}[definition]{Proposition}
\newtheorem{corollary}[definition]{Corollary}
\newtheorem{lemma}[definition]{Lemma}
\newtheorem{sthm}[definition]{$\star$ Theorem}
\newtheorem{slm}[definition]{$\star$ Lemma}

\newtheorem{spp}[definition]{$\star$ Proposition}
\newtheorem{scorollary}[definition]{$\star$ Corollary}
\newtheorem{fact}[definition]{Fact}

\newtheorem{cond}{Condition}
\newtheorem{Mthm}{Main Theorem}
\renewcommand{\thecond}{\Alph{cond}} % "letter-numbered" theorems
\renewcommand{\theMthm}{\Alph{Mthm}} % "letter-numbered" theorems


%\substack   multiple lines under sum
%\underset{b}{a}   b is under a


% Remind: \overline{L_0}



\usepackage{calligra}
\DeclareMathOperator{\shom}{\mathscr{H}\text{\kern -3pt {\calligra\large om}}\,}
\DeclareMathOperator{\sext}{\mathscr{E}\text{\kern -3pt {\calligra\large xt}}\,}
\DeclareMathOperator{\Rel}{\mathscr{R}\text{\kern -3pt {\calligra\large el}~}\,}
\DeclareMathOperator{\sann}{\mathscr{A}\text{\kern -3pt {\calligra\large nn}}\,}
\DeclareMathOperator{\send}{\mathscr{E}\text{\kern -3pt {\calligra\large nd}}\,}
\DeclareMathOperator{\stor}{\mathscr{T}\text{\kern -3pt {\calligra\large or}}\,}
%write mathscr Hom (and so on) 

\usepackage{aurical}
\DeclareMathOperator{\VVir}{\text{\Fontlukas V}\text{\kern -0pt {\Fontlukas\large ir}}\,}

\newcommand{\vol}{\text{\Fontlukas V}}
\newcommand{\dvol}{d~\text{\Fontlukas V}}
% perfect Vol symbol

\usepackage{aurical}
\usepackage[T1]{fontenc}








\newcommand{\fk}{\mathfrak}
\newcommand{\mc}{\mathcal}
\newcommand{\wtd}{\widetilde}
\newcommand{\wht}{\widehat}
\newcommand{\wch}{\widecheck}
\newcommand{\ovl}{\overline}
\newcommand{\udl}{\underline}
\newcommand{\tr}{\mathrm{tr}} %transpose
\newcommand{\Tr}{\mathrm{T}}
\newcommand{\End}{\mathrm{End}} %endomorphism
\newcommand{\idt}{\mathbf{1}}
\newcommand{\id}{\mathrm{id}}
\newcommand{\Hom}{\mathrm{Hom}}
\newcommand{\Conf}{\mathrm{Conf}}
\newcommand{\Res}{\mathrm{Res}}
\newcommand{\res}{\mathrm{res}}
\newcommand{\KZ}{\mathrm{KZ}}
\newcommand{\ev}{\mathrm{ev}}
\newcommand{\coev}{\mathrm{coev}}
\newcommand{\opp}{\mathrm{opp}}
\newcommand{\Rep}{\mathrm{Rep}}
\newcommand{\diag}{\mathrm{diag}}
\newcommand{\Dom}{\mathrm{Dom}}
\newcommand{\loc}{\mathrm{loc}}
\newcommand{\con}{\mathrm{c}}
\newcommand{\uni}{\mathrm{u}}
\newcommand{\ssp}{\mathrm{ss}}
\newcommand{\di}{\slashed d}
\newcommand{\Diffp}{\mathrm{Diff}^+}
\newcommand{\Diff}{\mathrm{Diff}}
\newcommand{\PSU}{\mathrm{PSU}(1,1)}
\newcommand{\Vir}{\mathrm{Vir}}
\newcommand{\Witt}{\mathscr W}
\newcommand{\Span}{\mathrm{Span}}
\newcommand{\pri}{\mathrm{p}}
\newcommand{\ER}{E^1(V)_{\mathbb R}}
\newcommand{\prth}[1]{( {#1})}
\newcommand{\bk}[1]{\langle {#1}\rangle}
\newcommand{\bigbk}[1]{\big\langle {#1}\big\rangle}
\newcommand{\Bigbk}[1]{\Big\langle {#1}\Big\rangle}
\newcommand{\biggbk}[1]{\bigg\langle {#1}\bigg\rangle}
\newcommand{\Biggbk}[1]{\Bigg\langle {#1}\Bigg\rangle}
\newcommand{\GA}{\mathscr G_{\mathcal A}}
\newcommand{\vs}{\varsigma}
\newcommand{\Vect}{\mathrm{Vec}}
\newcommand{\Vectc}{\mathrm{Vec}^{\mathbb C}}
\newcommand{\scr}{\mathscr}
\newcommand{\sjs}{\subset\joinrel\subset}
\newcommand{\Jtd}{\widetilde{\mathcal J}}
\newcommand{\gk}{\mathfrak g}
\newcommand{\hk}{\mathfrak h}
\newcommand{\xk}{\mathfrak x}
\newcommand{\yk}{\mathfrak y}
\newcommand{\zk}{\mathfrak z}
\newcommand{\pk}{\mathfrak p}
\newcommand{\hr}{\mathfrak h_{\mathbb R}}
\newcommand{\Ad}{\mathrm{Ad}}
\newcommand{\DHR}{\mathrm{DHR}_{I_0}}
\newcommand{\Repi}{\mathrm{Rep}_{\wtd I_0}}
\newcommand{\im}{\mathbf{i}}
\newcommand{\Co}{\complement}
%\newcommand{\Cu}{\mathcal C^{\mathrm u}}
\newcommand{\RepV}{\mathrm{Rep}^\uni(V)}
\newcommand{\RepA}{\mathrm{Rep}(\mathcal A)}
\newcommand{\RepN}{\mathrm{Rep}(\mathcal N)}
\newcommand{\RepfA}{\mathrm{Rep}^{\mathrm f}(\mathcal A)}
\newcommand{\RepAU}{\mathrm{Rep}^\uni(A_U)}
\newcommand{\RepU}{\mathrm{Rep}^\uni(U)}
\newcommand{\RepL}{\mathrm{Rep}^{\mathrm{L}}}
\newcommand{\HomL}{\mathrm{Hom}^{\mathrm{L}}}
\newcommand{\EndL}{\mathrm{End}^{\mathrm{L}}}
\newcommand{\Bim}{\mathrm{Bim}}
\newcommand{\BimA}{\mathrm{Bim}^\uni(A)}
%\newcommand{\shom}{\scr Hom}
\newcommand{\divi}{\mathrm{div}}
\newcommand{\sgm}{\varsigma}
\newcommand{\SX}{{S_{\fk X}}}
\newcommand{\DX}{D_{\fk X}}
\newcommand{\mbb}{\mathbb}
\newcommand{\mbf}{\mathbf}
\newcommand{\bsb}{\boldsymbol}
\newcommand{\blt}{\bullet}
\newcommand{\Vbb}{\mathbb V}
\newcommand{\Ubb}{\mathbb U}
\newcommand{\Xbb}{\mathbb X}
\newcommand{\Kbb}{\mathbb K}
\newcommand{\Abb}{\mathbb A}
\newcommand{\Wbb}{\mathbb W}
\newcommand{\Mbb}{\mathbb M}
\newcommand{\Gbb}{\mathbb G}
\newcommand{\Cbb}{\mathbb C}
\newcommand{\Nbb}{\mathbb N}
\newcommand{\Zbb}{\mathbb Z}
\newcommand{\Qbb}{\mathbb Q}
\newcommand{\Pbb}{\mathbb P}
\newcommand{\Rbb}{\mathbb R}
\newcommand{\Ebb}{\mathbb E}
\newcommand{\Dbb}{\mathbb D}
\newcommand{\Hbb}{\mathbb H}
\newcommand{\cbf}{\mathbf c}
\newcommand{\Rbf}{\mathbf R}
\newcommand{\wt}{\mathrm{wt}}
\newcommand{\Lie}{\mathrm{Lie}}
\newcommand{\btl}{\blacktriangleleft}
\newcommand{\btr}{\blacktriangleright}
\newcommand{\svir}{\mathcal V\!\mathit{ir}}
\newcommand{\Ker}{\mathrm{Ker}}
\newcommand{\Cok}{\mathrm{Coker}}
\newcommand{\Sbf}{\mathbf{S}}
\newcommand{\low}{\mathrm{low}}
\newcommand{\Sp}{\mathrm{Sp}}
\newcommand{\Rng}{\mathrm{Rng}}
\newcommand{\vN}{\mathrm{vN}}
\newcommand{\Ebf}{\mathbf E}
\newcommand{\Nbf}{\mathbf N}
\newcommand{\Stb}{\mathrm {Stb}}
\newcommand{\SXb}{{S_{\fk X_b}}}
\newcommand{\pr}{\mathrm {pr}}
\newcommand{\SXtd}{S_{\wtd{\fk X}}}
\newcommand{\univ}{\mathrm {univ}}
\newcommand{\vbf}{\mathbf v}
\newcommand{\ubf}{\mathbf u}
\newcommand{\wbf}{\mathbf w}
\newcommand{\CB}{\mathrm{CB}}
\newcommand{\Perm}{\mathrm{Perm}}
\newcommand{\Orb}{\mathrm{Orb}}
\newcommand{\Lss}{{L_{0,\mathrm{s}}}}
\newcommand{\Lni}{{L_{0,\mathrm{n}}}}
\newcommand{\UPSU}{\widetilde{\mathrm{PSU}}(1,1)}
\newcommand{\Sbb}{{\mathbb S}}
\newcommand{\Gc}{\mathscr G_c}
\newcommand{\Obj}{\mathrm{Obj}}
\newcommand{\bpr}{{}^\backprime}
\newcommand{\fin}{\mathrm{fin}}
\newcommand{\Ann}{\mathrm{Ann}}
\newcommand{\Real}{\mathrm{Re}}
\newcommand{\Imag}{\mathrm{Im}}
%\newcommand{\cl}{\mathrm{cl}}
\newcommand{\Ind}{\mathrm{Ind}}
\newcommand{\Supp}{\mathrm{Supp}}
\newcommand{\Specan}{\mathrm{Specan}}
\newcommand{\red}{\mathrm{red}}
\newcommand{\uph}{\upharpoonright}
\newcommand{\Mor}{\mathrm{Mor}}
\newcommand{\pre}{\mathrm{pre}}
\newcommand{\rank}{\mathrm{rank}}
\newcommand{\Jac}{\mathrm{Jac}}
\newcommand{\emb}{\mathrm{emb}}
\newcommand{\Sg}{\mathrm{Sg}}
\newcommand{\Nzd}{\mathrm{Nzd}}
\newcommand{\Owht}{\widehat{\scr O}}
\newcommand{\Ext}{\mathrm{Ext}}
\newcommand{\Tor}{\mathrm{Tor}}
\newcommand{\Com}{\mathrm{Com}}
\newcommand{\Mod}{\mathrm{Mod}}
\newcommand{\nk}{\mathfrak n}
\newcommand{\mk}{\mathfrak m}
\newcommand{\Ass}{\mathrm{Ass}}
\newcommand{\depth}{\mathrm{depth}}
\newcommand{\Coh}{\mathrm{Coh}}
\newcommand{\Gode}{\mathrm{Gode}}
\newcommand{\Fbb}{\mathbb F}
\newcommand{\sgn}{\mathrm{sgn}}
\newcommand{\Aut}{\mathrm{Aut}}
\newcommand{\Modf}{\mathrm{Mod}^{\mathrm f}}
\newcommand{\codim}{\mathrm{codim}}
\newcommand{\card}{\mathrm{card}}
\newcommand{\dps}{\displaystyle}
\newcommand{\Int}{\mathrm{Int}}
\newcommand{\Nbh}{\mathrm{Nbh}}
\newcommand{\Pnbh}{\mathrm{PNbh}}
\newcommand{\Cl}{\mathrm{Cl}}
\newcommand{\diam}{\mathrm{diam}}
\newcommand{\eps}{\varepsilon}
\newcommand{\Vol}{\mathrm{Vol}}
\newcommand{\LSC}{\mathrm{LSC}}
\newcommand{\USC}{\mathrm{USC}}
\newcommand{\Ess}{\mathrm{Rng}^{\mathrm{ess}}}
\newcommand{\Jbf}{\mathbf{J}}
\newcommand{\SL}{\mathrm{SL}}
\newcommand{\GL}{\mathrm{GL}}
\newcommand{\Lin}{\mathrm{Lin}}
\newcommand{\ALin}{\mathrm{ALin}}
\newcommand{\bwn}{\bigwedge\nolimits}
\newcommand{\nbf}{\mathbf n}
\newcommand{\dive}{\mathrm{div}}
\newcommand{\Alt}{\mathrm{Alt}}
\newcommand{\Sym}{\mathrm{Sym}}



\renewcommand{\epsilon}{\varepsilon}
% \renewcommand{\phi}{\varphi}





\usepackage{tipa} % wierd symboles e.g. \textturnh
\newcommand{\tipar}{\text{\textrtailr}}
\newcommand{\tipaz}{\text{\textctyogh}}
\newcommand{\tipaomega}{\text{\textcloseomega}}
\newcommand{\tipae}{\text{\textrhookschwa}}
\newcommand{\tipaee}{\text{\textreve}}
\newcommand{\tipak}{\text{\texthtk}}
\newcommand{\mol}{\upmu}
\newcommand{\dmol}{d\upmu}




\usepackage{tipx}
\newcommand{\tipxgamma}{\text{\textfrtailgamma}}
\newcommand{\tipxcc}{\text{\textctstretchc}}
\newcommand{\tipxphi}{\text{\textqplig}}















\numberwithin{equation}{section}
% count the eqation by section countation



\title{Discrete Optimization}
\author{
	{{\bf Instructor:} Yuan Zhou}\\
	{{\bf Notes Taker:} Zejin Lin}\\
	{\small \sc Tsinghua University.}\\
	{\small linzj23@mails.tsinghua.edu.cn}\\
	{\small \href{lzjmaths.github.io}{lzjmaths.github.io}}
}

\DeclareMathOperator{\sign}{sign}
\DeclareMathOperator{\dom}{dom}
\DeclareMathOperator{\ran}{ran}
\DeclareMathOperator{\ord}{ord}
\DeclareMathOperator{\img}{Im}
\DeclareMathOperator{\argmin}{argmin}
\DeclareMathOperator{\argmax}{argmax}
\newcommand{\OPT}{\mathrm{OPT}}
\newcommand{\dd}{\mathrm{d}}
\newcommand{\ie}{ \textit{ i.e. } }
\newcommand{\st}{ \textit{ s.t. }}
\newcommand{\name}[1]{\textbf{#1}\index{#1}}
\newcommand{\subname}[2]{\textbf{#1}\index{#2!#1}}
\newcommand{\stdO}{\mathcal{O}_{\mathrm{std}}}
\newcommand{\stdwedge}[1]{\dd {#1}^1\wedge\cdots\wedge\dd {#1}^n}
\newcommand{\stddownwedge}[1]{\dd {#1}_1\wedge\cdots\wedge\dd {#1}_n}

\newcommand{\opensub}{\overset{\text{open}}{\subset}}
\newcommand{\chart}[2]{({#1}_\alpha,{#2}^1,\cdots,{#2}^n)}
\newcommand{\<}{\left<}
\renewcommand{\>}{\right>}
\renewcommand{\|}{\Vert}
\renewcommand{\hat}[1]{\widehat{#1}}
\renewcommand{\tilde}[1]{\widetilde{#1}}
\newcommand{\Romannumer}[1]{\mathrm{\uppercase\expandafter{\romannumeral#1}}}
\newcommand{\rot}{\mathrm{rot}}

\renewcommand{\div}{\mathrm{div}}
%\makeindex[columns=2,title=Index, options=-s example_style.ist]
\usepackage{pgfplots}
\pgfplotsset{width=7cm,compat=1.18}
\usepackage{algorithm}
\usepackage{algorithmic}

\usepackage{chemarrow}

% \usepackage{tikz-3dplot} 
% \renewcommand{\xrightarrow}[2]{\autorightarrow{#1}{#2}}

% \excludecomment{proof}
% \excludecomment{remark}
% \excludecomment{definition}
% \excludecomment{example}
% \excludecomment{exercise}

\tikzset{
  curarrow/.style={
  rounded corners=8pt,
  execute at begin to={every node/.style={fill=red}},
    to path={-- ([xshift=-50pt]\tikztostart.center)
    |- (#1) node[fill=white] {$\scriptstyle d$}
    -| ([xshift=50pt]\tikztotarget.center)
    -- (\tikztotarget)}
    }
}

\begin{document}
\sloppy
\pagenumbering{arabic}
\maketitle
\tableofcontents
\newpage

\section{Regression}
\begin{equation}\label{Lasso}
    \min_{\omega\in \Rbb^m}\frac{1}{2N}\|\Phi\omega-y\|^2+\lambda C(\omega)
\end{equation}

\name{Lasso}:  $ C=\|\omega\|_1 $. \name{Ridge regression}: $ C=\|\omega\|_2 $.

\name{subgradient} of  $ f $:
\[\partial f(x_0)=\{g|f(x) \geq f(x_0)+g^T(x-x_0)\}\] 
In particular, 
\[\partial |x|=\begin{cases}
    1, & x>0\\
    -1, & x<0\\
    [-1,1], & x=0
\end{cases}\]

\subsection{Binary classification problem}
\name{one-hot encoding} for the output  $ \{\binom{1}{0},\binom{0}{1} \} $. It can be understood as the probability for each class  and can take continuous values.

A \name{linear hypothesis space} is  $ \{u(x):u=\omega^Tx,x\in \Rbb^n,\omega\in \Rbb^n\} $.

\name{Softmax}:Map the extracted feature  $ u $ to the space of one-hot codes 
\[\mu=\frac{1}{1+e^{-u}},\quad 1-\mu=\frac{e^{-u}}{1+e^{-u}}=\frac{1}{1+e^{u}}\] 

\begin{equation}\label{KL distance}
    KL(p,q)=\int p(\log p - \log q)
\end{equation}
For  $ p $ real probability, to minimize \eqref{KL distance}, suffices to minimize 
\[-\int p\log q_\theta\dd x=-\sum_{x_i}\log q_\theta(x_i)\] 
which is called \name{Maximum likelihood} (\name{cross entropy})
\[-\sum \log p(y_i|x_i,\omega)=\sum -y_i\log \mu_i-(1-y_i)\log (1-\mu_i)\]
We reduce to minimize the thing above.

\subsection{Gradient Descent}
\[J(\theta)=\sum_{i=1}^N L(f_\theta(x_i),y_i),\quad \theta^{t+1}=\theta^t-\eta_t\frac{\partial J(\theta)}{\partial \theta}|_{\theta=\theta^t}\]
%!TEX root = /lecture/Discrete_Optimistic.tex

\subsection{Single-Source Shortest Path}
\begin{example}[Single-Source Shortest Path(SSSP)]
    Input: Graph $ G=(V,E,w) $,  $ V $ is the set of point and  $ E $ is the set of edge with direction  and  $ \omega:E\rightarrow \Rbb_{ \geq 0} $.
    
    We want to find a path from  $ s $ to  $ t $ with minimum total cost.
\end{example}
\paragraph{Dijkstra's Algorithm}
Choose  $ s  $ as a source.  $ d[s]=0,d[u]=\begin{cases}
    \omega(s,u)&\text{if  $ (s,u)\in E $}\\
    +\infty &\text{otherwise}
\end{cases} $, $ S=\{s\} $ first. To record the path, we can use  $ \mathrm{Pred}[u]\leftarrow s $.  

\begin{algorithm}
    \caption{Dijkstra's Algorithm}
    \label{alg:dijkstra}
    \begin{algorithmic}[1]
    \WHILE{$ S\neq V $}
        \STATE Choose  $ u\in \dps\arg\min_{\not\in S}\{d[x]\} $.\label{Choose u min}
        \STATE Update  $ S\leftarrow S\cup\{u\} $.
        \FOR{each $ x\in V-S $,$ (u,x)\in E $}
            \STATE$ d[x]\leftarrow\min\{d[x],d[u]+\omega(u,x)\} $.
            \IF{ $ d[u]+\omega(u,x)<d[x] $ }
                \STATE  $ d[x]\leftarrow d[u]+\omega(u,x) $  
                \STATE  $ \mathrm{Pred}[x]\leftarrow u $  
            \ENDIF
        \ENDFOR 
    \ENDWHILE
    \end{algorithmic}
\end{algorithm}

\begin{theorem}[Invariant]
    $ \forall u\in S $, $ d[u] $ is the shortest path distance  $ s\leadsto u $   
\end{theorem}
\begin{proof}
    Induction on  $ |S| $.
    
    For  $ |S|=1 $ true.

    \textbf{Induction Step: } Every time executing \ref{Choose u min} in Algorithm \ref{alg:dijkstra}, we need to prove  $ d[u] $ is the shortest distance  $ s\leadsto u $.

    If  $ v=\mathrm{Pred}[u]\in S $, then  $ d[u]=d[v]+\omega(v,u) $.
    
    For any path from  $ s $  to  $ u $, there exists  $ (\alpha,\beta)\in E $ such that  $ \alpha\in S,\beta\not\in S $. Then 
    \begin{align*}
        \mathrm{length}(P)& \geq \mathrm{length}(P[s\rightarrow\beta])\\
        & =\mathrm{length}(P[s\rightarrow\alpha])+\omega(\alpha,\beta)\\
        & \geq d[\alpha]+\omega(\alpha,\beta)\\
        & \geq d[\beta] \geq d[u]
    \end{align*}   
\end{proof}

\begin{remark}
    The straightforward implementation of Dijkstra's Algorithm is of  $ O(|v|^2) $.
    
    If we use priority queue: $ Q $ with priority  $ Q.\pi() $. It has some methods:
    \begin{itemize}
        \item ExtractMin: Return  $ \dps\arg\min_{x\in Q}\{Q.\pi(x)\} $ and remove  $ x  $ from  $ Q $.
        \item DecreaseKey: Update  $ Q.\pi(v) $ with newkey. 
    \end{itemize}  
    The time complexity is  $ |V|\times $ ExtractMin  $ + $ $ |E|\times  $  DecreaseKey    
    \begin{center}
        \begin{tabular}{|c|c|c|c|}
            \hline
            Runtime & ExtractMin & DecreaseKey & Dijkstra\\ \hline
            Simple Array   &  $ O(|V|) $   &  $ O(1) $ & $ O(|v|^2) $    \\ \hline
            Binary Heap   &  $ O(\log|V|) $    &  $ O(\log |V|) $ & $ O(|E|\cdot \log|V|) $   \\ \hline
            Fibonacci Heap & $ O(\log|V|) $ &  $ O(1) $ (amorized)&   $ O(|E|+|V|\log|V|) $ \\ \hline
        \end{tabular}
    \end{center}
\end{remark}
\subsection{Minimum Spanning Tree}
\begin{example}[Minimum Spanning Tree (MST)]
    Input: Connected, undirected graph $ G=(V,E,\omega) $.
    
    \begin{definition}[Spanning Tree]
        $ T\subset E $ is a \name{spanning tree}  if  $ |T|=|V|-1 $,  $ G'=(V,T) $ is connected.
        
        \textbf{Goal of MST } Find spanning tree  $ T $ so that  $ \dps\omega(T)=\sum_{e\in T}\omega(e) $ minimized.  
    \end{definition}
    \begin{theorem}[Cayley Theorem]
        The number of spanning trees of  $ n $-vertex complete graph is  $ n^{n-2} $  
    \end{theorem}
    A \name{cut}  $ (S,V-S) $ has a \name{cutset} of  $ S $ $ =\{e=(u,v):u\in S,v\not\in S\} $.  
\end{example}

For \textbf{empirical loss} 
\[J(\theta)=\frac{1}{N}\sum_{i=1}^N J_i(\theta),J_i(\theta)L(f_\theta(y_i),x_i)\]
\name{Stochastic Gradient Descent}:only compute gradients over a mini batch for each epoch
\[\theta_{k+1}=\theta_k-\eta\frac{1}{B}\sum_{i\in I_B} \nabla  J_i(\theta_k)\]
$ I_B $, called \name{mini batch} are randomly sampled from the training data indexes.

The motivation to sample is that for distribution  $ x $,  $ \dps\frac{x_1+\cdots+x_N}{N} $ has the same mean  $ \mu  $ but less variance  $ \frac{1}{N}\sigma^2 $, so in order to have more randomness and greatly reduce the computational cost, we choose less ammount of data.

Randomness can help avoid getting stuck in  local minimum.

SGD: $ x_{t+1}=x_t-\alpha \nabla f(x_t) $.

SGD+\name{momentum}:
\[v_{t+1}=\rho v_t+\nabla f(X_t)\]
\[x_{t+1}=x_t-\alpha v_{t+1}\]
$ v_t $ is the momentum which helps accelerate convergence by accumulating the gradients of past steps and smoothing out the oscillations, where  $ \rho=0.9  $ or  $ 0.99 $.  

\subsection{Adaptive learning rate}

\[r_t=r_{t-1}+\nabla f(x_t)\bigodot  \nabla f(x_t)\]
\[x_{t+1}=x_t-\frac{\alpha}{\sqrt{r_t+\epsilon}}\bigodot  \nabla f(x_t)\]
For frequent features, the  updates will be smaller, and for rare features, the updates will be larger.

\paragraph{Notation}  $ \odot $ is  the multiplication for each component, which means:


\[\begin{pmatrix}
    a_1\\
    \vdots\\
    a_n
\end{pmatrix}\odot\begin{pmatrix}
    b_1\\
    \vdots\\
    b_n
\end{pmatrix}=\begin{pmatrix}
    a_1b_1\\
    \vdots\\
    a_nb_n
\end{pmatrix}\]

\section{MLP}

\name{Exponential Moving Averaging(EMA)}
\[r_t=\beta r_{t-1}+(1-\beta)\nabla f(x_t)\odot \nabla f(x_t)\]
\[x_{t+1}=x_t-\frac{\alpha}{\sqrt{r_t+\epsilon}}\bigodot  \nabla f(x_t)\]

RMSProp uses a moving average, avoiding overly aggresive decay complared with AdaGrad.

\name{Adaptive Moment Estimation(Adam)}:RMAProp+Momentum 
\[g_t=\nabla f(x_t)\]
\[v_t=\beta_1^t v_{t-1}+(1-\beta_1^t)g_t, r_t=\beta_2^t r_{t-1}+(1-\beta_2^t)g_t\odot g_t\]
\[v_t=\frac{v_t}{1-\beta_1^t},E_t=\frac{E_t}{1-\beta_2^t}\]
\[x_{t+1}=x_t-\frac{\alpha}{\sqrt{r_t+\epsilon}}\odot v_t\]

\section{Vanishing or Exploding Gradient}
The gradient of the Sigmoid function is very small most of the time, leading to vanishing gradients

We want to avoid exploding or vanishment in gradient.

\name{Sigmoid}: $ \dps\sigma(z)=\frac{1}{1+e^{-z}} $.

\name{Tanh}: $ \dps\sigma(z)=\frac{e^{z}-e^{-z}}{e^z+e^{-z}} $.

\name{ReLU}:  $ \dps\mathrm{ReLU}(z)=\begin{cases}
    z,&z>0\\
    0,&\text{otherwise}
\end{cases} $.

\name{LeakyReLU}: $ \dps \mathrm{LeakyReLU}(z)=\begin{cases}
    z,&z>0\\
    az,&\text{otherwise}
\end{cases} $.
The gradient neither vanishes nor explodes; it is computationally fast, but some neurons may not be activated.

LeakyReLU solves the issue with ReLU and is the most commonly used.

Consider the back propagation  $ \dps\frac{\partial J}{\partial x}=\frac{\partial J}{\partial y}\frac{\partial y}{\partial x}=\frac{\partial J}{\partial x}W$. Cumulate after multi-layers
\[\mathrm{Var}(\frac{\partial J}{\partial x})=\prod_i n_l \mathrm{Var}(W_l)\mathrm{Var}(\frac{\partial J}{\partial x_l})\]
We want   $ n_l\mathrm{Val}(W_l)\sim 1 $. 

After normalizing their variances, the updating becomes more steady and efficient.

To avoid variance becoming too small or too large in deep layers, normalize features in the network 
\[\hat{x}_i=\frac{x_i-\Ebb x_i}{\sqrt{\mathrm{var}(x_i)}}\]
\name{Batch Normalization}: normalize features across samples within each batch.





\begin{definition}
    Given directed  $ G=(V,E)  $ and  $ r\in V  $,  $ n=|V|,m=|E| $,  $ F\subset E $ is an \name{arborescence} if 
    \begin{itemize}
        \item  $ F $ is a spanning tree if ignoring directions
        \item   $ \forall v\in V $,  $ \exists $ unique  path  $ r\rightarrow v $ in  $ F $.    
    \end{itemize}  
    Or equivalently,  $ F $ has no directed cycles and every node  $ v\neg r $ has a unique incoming edge.  
\end{definition}

For this problem, WLOG we can assuem that the root  $ r $  has no in-degree and assume  $ \omega \geq 0 $.  

\newcommand{\cheap}{\mathrm{cheap}}
For each  $ n\neq r $, let 
\[\mathrm{cheap}(v)=\argmin_{e=(u,v)\in E}\{\omega(e)\}\] 
\begin{claim}
    Let  $ F=\{\cheap(v)|v\neq r\} $.  $ F $ is arborescense $ \Rightarrow  $  $ F $ is min-cost.   
\end{claim}

Define  $ \omega_r(u,v)=\omega(u,v)-\omega(\cheap(v)) $. Suffices to find the min-cost arborescence under  $ \omega_r $. 

If  $ F $ is not an arborescense, then  $ \exists $ a directed cycle  $ C $ with all edges of weight  $ 0 $.    


Using the contraction view, if we contract "$ 0 $-cycle" and keep this process recursively. By taking degrees carefully we can easily confirm the legallity of the contraction view. Then suffices to prove it is indeed the min-cost arborescence when we expand after.

\begin{theorem}
    The min-cost arborescence  $ \tilde{F} $ when we apply contraction to  $ 0- $cycle is exactly the min-cost arborescence in the original graph after expanding.  
\end{theorem}
\begin{lemma}
    $ \exists $ min-cost  $ F^* $ \st only 1 edge in  $ F^* $ entering  $ C $.   
\end{lemma}
\begin{proof}
    Our goal is to prove  $ \omega_r(F) \leq \omega_r(F^*) $. 

    Let  $ F^*_C=F^*\cap (C\times C) $. Then  $ |F^*_C|=|C|-1 $.
    
    Apply  $ C $-contraction to  $ F^*\setminus F^*_C $ we obtain an arborescence of  $ \tilde{G} $. (Easy to check) So 
    \[\sum_{e\in F^*\setminus F^*_C}\omega_r(e) \geq \sum_{e\in \tilde{F}}\omega_r(e)\]
    So  $ \omega_r(F^*) \geq \omega_r(F) $ 
\end{proof}
\begin{proof}[Proof of Lemma]
    Choose any  $ v\in C $.
    
    Let  $ (x,y)\in r\rightarrow v $ be the first edge entering  $ C $.
    
    Delete the edge entering  $ C\setminus\{y\} $ and add the edge of circle  except the edge entering  $ y $.
    
    Then it is  an arborescence of less cost.
\end{proof}
\section{Dynamic Programming}
\subsection{Weighted Interval Scheduling}
\begin{example}[Weighted Interval Scheduling]
    Input: $ n $ jobs,  $ \{(s_i,f_i),\omega_i\}_{i=1}^n $. Want to find  $ \sum\omega_{i_k} $ maximum.  
\end{example}

To make the structure simpler, we WLOG assume  $ s_1 \leq s_2 \leq \cdots \leq s_n $. 
\begin{algorithm}
    \caption{ $ \mathrm{Search}(i) $ }
    \begin{algorithmic}[1]
        \STATE  $ j\leftarrow \min \id >i,s_j \geq f_i $. 
        \STATE Return  $ \max\{\mathrm{Search}(j)+\omega_i,\mathrm{Search}(i+1)\} $.
    \end{algorithmic}
\end{algorithm} 
We may find that there is a lot of repetitive computation. We can record each  $ \mathrm{Search}(i) $


\begin{algorithm}
    \caption{ $ \mathrm{Search-Memorization}(i) $}
    \begin{algorithmic}[1]
        \STATE If  $ i>n $, RETURN 0
        \STATE If  $ i\neq $  bottom, RETURN  $ F[i] $.
        \STATE  $ j(i)\leftarrow \min\{j|s_j \geq f_i\} $.
        \STATE  $ F[i]\leftarrow \max\{\mathrm{Search-M}(j(i))+\omega_i,\mathrm{Search-M}(i+1)\}  $ 
        \STATE RETURN  $ F[i] $
    \end{algorithmic}
\end{algorithm}

It can be written as 
\[\begin{cases}
    F[i]=\max\{F[j(i)]+\omega_i,F[i+1]\}\\
    F[n+1]=0
\end{cases}\]

Such an equation is called \name{Bellman Equation}. So Dynamic Programming is a method to solve the problem by finding the optimal solution of each subproblem. We sometimes need to record the optimal solution of each subproblem to avoid repetition.

\subsection{Segmented Least Square}

\begin{example}[Least Square]
    We have  $ n  $ points  $ \{(x_i,y_i)\}_{i=1}^n $. We want to find a line  $ y=ax+b $ to minimize 
    \begin{equation}
        \mathrm{SSE}=\sum_{i=1}^n[y_i-(ax_i+b)]^2
    \end{equation}  
    Actually, \[ \begin{cases}
        a=\dps\frac{n\sum_ix_iy_i-\left(\sum_ix_i\right)\left(\sum_iy_i\right)}{n\sum_ix_i^2-\left(\sum_ix_i\right)^2}\\
        \\
        b=\dps\frac{\sum_iy_i-a \sum_ix_i}{n}
    \end{cases} \]
\end{example}

\begin{example}[Segmented Least Square]
    Input:  $ \{(x_i,y_i)\}_{i=1}^n $,  $ c>0 $.
    
    Goal: Minimize  $ l=E+cL $ for piecewise line, where  $ c $ is the \name{hyperparameter},  $ L  $ is the number of the segments.
\end{example}

WLOG, assume  $ x_1<x_2<\cdots<x_n $. 

We can define its subproblem as 
\[\mathrm{OPT}[i]:\text{min loss}\]
when in put is  $ (x_1,y_1),\cdots,(x_i,y_i) $.

Find solution  $ \mathrm{OPT}[n] $. The boundary condition is  $ \mathrm{OPT}[1]=\mathrm{OPT}[2]=c $ and the \textbf{Bellman Equation} is 
\[\OPT[i]=\min_{1 \leq j \geq i}\{\OPT[j-1]+l_{ji}+c\}\] 

\subsection{Knapsack Problem}
\begin{example}[Knapsack Problem]
    Input:  $ n $ items,  $ w_i,v_i $ for its weight and value. The capacity of knapsack is  $ w $. 

    If assume integral weight, then denote  $ \OPT[i,w] $ as the optimal total value when in put is first knapsack capacity is  $ w $. 

    The \textbf{Bellman Equation} is 
    \[\OPT[i,w]=\begin{cases}
        \OPT[i-1,w]&w<w_i\\
        \max\{\OPT[i-1,w],v_i+\OPT[i-1,w-w_i]\},w \geq w_i
    \end{cases}\]

    It has time complexity  $ O(nw) $, which is not a polynomial algorithm.

    We can find another Value-Based DP: (Also assume integral values)
    \begin{center}
        $ \OPT[i,v] $: choose min weight items. 
    \end{center}
    from item  $ 1,2,\cdots,i $ so that total value  $  \geq v $.
    
    The final solution for maxmial $ v $ \st  $ \OPT[n,v] \leq w $.  

    \[OPT[i,v]=\min\begin{cases}
        \OPT[i-1,v]\\
        w_i+\OPT[i-1,(v-v_i)^+]
    \end{cases}\]
     $ \OPT[0,v]=\begin{cases}
        0&v=0\\
        +\infty&v>0
    \end{cases} $ 
    The time complexity is  $ O(n^2v) $.
    
    Now we consider a \textbf{ $ \alpha $-approximation algorithm} that   $ \mathrm{ALG} \geq \alpha\cdot \OPT $ for  $ \alpha\in (0,1] $.

    Let  $ \epsilon=1-\alpha $. 

    \begin{algorithm}
        \caption{Knapsack Problem}
        \begin{algorithmic}[1]
            \STATE Assume WLOG  $ w_i \leq W $ so that  $ V \geq \OPT $.
            \STATE Set  $ K=\dps\frac{\epsilon V}{n} $. Let  $ v_i'=\left[\dps\frac{v_i}{K}\right] $
            \STATE Run value-based DP to find optimal solution  $ T $ for  $ I' $ 
            \STATE Return  $ T $ as a solution to  $ I $.       
        \end{algorithmic}
    \end{algorithm}
    It is a feasible solution and 
    \begin{align*}
        \sum_{i\in T}v_i'&=\OPT(I')\\
        & \geq v(S;I'),\,\forall \text{feasible} S\\
        & \geq v(T^*,I')\\
        &=\sum_{i\in T^*}v_i'\\
        &=\sum_{i\in T^*}\left[\frac{v_i}{K}\right]\\
        & \geq \sum_{i\in T^*}\left(\frac{v_i}{K}-1\right)\\
        & \geq \frac{1}{k}\sum_{i\in T^*}\sum_{i\in T^*}v_i-n\\
        &=\frac{1}{K}\OPT(I)-n
    \end{align*}
    So  $ \mathrm{ALG} \geq \dps\sum_{i\in T}K\cdot v_i' \geq \OPT(I)-nK \geq (1-\epsilon)\OPT(I) $. 

    The time complextity is  $ O(n^2V')=O(n^2\frac{V}{K})=O(n^3\epsilon^{-1}) $.
    
    \begin{remark}
        The time complextity depends on the accuracy  $ \epsilon $ instead of the maximum value  $ V $ since the accuracy is based on scale.
        
        In other words,  $ \epsilon^{-1} $ in time complexity represents not only accuracy but also the "size" of scale. 
    \end{remark}

    \paragraph{Fully Polynomial-Time Approximation Scheme(FPTAS)}
     $ \forall \epsilon $,  $ \exists  $ $ (1-\epsilon) $-approximation algorithm with time complexity  $ f(n,\epsilon)=\mathrm{poly}(n,\frac{1}{\epsilon}) $.
    
    \paragraph{PTAS}:  $ \forall \epsilon $,  $ \exists $   $ (1-\epsilon) $-approximation in time  $ f_\epsilon(n)=\mathrm{poly}(n) $. For this algorithm, it is   $ (n\cdot 2^{\frac{1}{\epsilon}},n^{\frac{1}{\epsilon}})$.

\end{example}




%!TEX root = /lecture/Discrete_Optimistic.tex

\subsection{RNA Secondary Structure}
\begin{example}[RNA Secondary Structure]
    RNA is a string  $ b_1b_2\cdots b_n $  where  $ b_i\in \{A,C,G,U\} $.
    
    The secondary structure is what fold to form "base pairs" including:
    \[U\cdots A,A\cdots U,C\cdots G,G\cdots C\]
    Mathematically, second structure represented by set of base pairs  $ S=\{(i,j)\} $, 
    \begin{enumerate}[label=*)]
        \item  $ \forall (i,x)\in S,(b_i,b_j)\in \{U\cdots A,A\cdots U,C\cdots G,G\cdots C\} $
        \item no sharp turns:  $ \forall (i,j)\in S $,  $ i<j-4 $,
        \item non-crossing:  $ \forall (i,j),(k,l)\in S $, cannot have  $ i<k<j<l $.     
    \end{enumerate} 
    Goal: Maximize  $ |S| $. 
\end{example}

A direct idea is to construct those subproblems:

\[\OPT[i,j]=\max_{i \leq k<j-4}\begin{cases}
    \OPT[i,j-1] &b_j \text{ not matched}\\
    1+\OPT[i,k-1]+\OPT[k+1,j-1] &b_j\text{ matched with }b_k
\end{cases}\]
\[\OPT[i,j]=0\text{ when }i \leq j<i+4\]


\subsection{Sequence Alignment(Edit Distance)}

\begin{example}
    For a wrong-spelled word, what cost do we need to make it right, using the gap and mismatch.

    Or what is its edit distance to the correct word.

    Mathematically, for string  $ (a_1\cdots a_n),(b_1\cdots b_m) $, a matching  $ M=\{(i,j)\}$ such that there is no  $ (i_1,j_1),(i_2,j_2)\in M $ \st  $ i_1<i_2 $ but  $ j_2<j_1 $.     Define its cost 
    \[\mathrm{cost}(M)=\sum_{(i,j)\in M}\alpha_{a_i b_j}+\sum_{i\in [n],i\text{ not in  $ M $}}+\sum_{\substack{
        j\in [m]\\
        j\text{ not in  $ M $}}}\delta\] 

    $ \dps \sum_{(i,j)\in M}\alpha_{a_i b_j}$ is the mismatch cost and  $ \sum_{i\in [n],i\text{ not in  $ M $}}+\sum_{j\in [m],j\text{ not in  $ M $}}\delta $ is the gap cost 
\end{example}

Define  $ \OPT[i,j] $ is the edit distance between  $ a_1a_2\cdots a_i $ and  $ b_1b_2\cdots b_j $.



\[\OPT[i,j]=\min_{1 \leq k \leq j}=\begin{cases}
    \delta+\OPT[i-1,j]&a_i\text{ not matched}\\
    \alpha_{a_ib_k}+\delta\cdot (j-k)+\OPT[i-1,k-1]&a_i\text{matched with  $ b_k $ }
\end{cases}\]

However, for each case it can be divided into three cases:
\[\OPT[i,j]=\min\begin{cases}
    \OPT[i-1,j-1]+\alpha_{a_ib_j}\\
    \OPT[i-1,j]+\delta\\
    \OPT[i,j-1]+\delta
\end{cases}\]

The question is, if we need to trace the matching process, the space complexity is  $ O(nm) $, too large.

Here we use binary search.
\begin{algorithm}
    \caption{Binary Search}
    \begin{algorithmic}[1]
        \STATE Compute  $ A[j]=d[(0,0)\rightarrow(\frac{n}{2},j)] $ and  $B[j]= d[(\frac{n}{2},j)\rightarrow(n,m)] $,
        \STATE find  $ j^*=\argmin_j A[j]+B[j] $.
        \STATE Run the sub-process  $ (0,0)\rightarrow(\frac{n}{2},j^*) $ and  $ (\frac{n}{2},j^*)\rightarrow(n,m) $ 
    \end{algorithmic}
\end{algorithm}

The complexity is still  $ O(nm)+\frac{1}{2}O(nm)+\cdots+\frac{1}{2^k}O(nm)=O(nm) $.

\subsection{Matrix Multiplication}

\begin{example}[Matrix Multiplication]
    Consider  $ M_1\cdot M_2\cdots M_k $ where  $ M_i $ is a  $ n_{i-1}\times n_i $ matrix. 

    We want to find the optimal multiplicative order such that the time cost is minimal.
\end{example}

Denote  $ \OPT[i,j] $ is the min from  $ M_i $ to  $ M_j $.

Using the binary tree, consider the last multiplication

\[\OPT[i,j]=\min_{i \leq l<j}\{\OPT[i,l]+\OPT[l+1,j]+n_{i-1}n_ln_j\}\]





\printindex
\newpage
\listoftheorems[ignoreall, show={theorem,proposition}]
\end{document}
