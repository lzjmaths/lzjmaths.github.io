\begin{corollary}
    $ \forall \epsilon>0 $,  $ \mathrm{Max3SAT}_{1-\epsilon,\frac{7}{8}+\epsilon} $ is NP-Hard.  
\end{corollary}
\begin{proof}
    Reduce the instance of  $ \mathrm{3LIN} $ to  $ \mathrm{3SAT} $ as follows:
    
    \begin{align*}
        \mathrm{3LIN}(I)&\rightarrow \mathrm{3SAT}(J)\\
        x_i+x_j+x_k\equiv0&\rightarrow \begin{cases}
            \bar{x}_i\vee\bar{x}_j\vee \bar{x}_k\\
            \bar{x}_i\vee x_j\vee x_k\\
            x_i\vee\bar{x}_j\vee x_k\\
            x_i\vee x_j\vee \bar{x}_k 
        \end{cases}\\
        x_i+x_j+x_k\equiv 1&\rightarrow \begin{cases}
            x_i\vee x_j\vee x_k\\
            x_i\vee \bar{x}_j\vee \bar{x}_k\\
            \bar{x}_i\vee x_j\vee \bar{x}_k\\
            \bar{x}_i\vee \bar{x}_j\vee x_k
        \end{cases}
    \end{align*}
    Then  $ \OPT(I) \geq 1-\epsilon $  $ \Rightarrow  $  $ \OPT(J) \geq 1-\epsilon $.
    
    $ \OPT(I)<1/2+\epsilon $  $ \Rightarrow  $  $ \OPT(J)<7/8+\epsilon $ 
\end{proof}

\subsection{Hardness of Max-Cut}
\begin{corollary}
    $ \forall \epsilon>0 $,  $ \dps\left(\frac{16}{17}+\epsilon\right) $-approximation Max-Cut is NP-Hard.
\end{corollary}
\begin{remark}\label{16/17 Hardness of Max-Cut}
    Assume \textbf{Unique Games Conjecture},  $ \alpha_{GW}+\epsilon $-approximation Max-Cut  $ \not\in \mathrm{P} $.  
\end{remark}

\begin{definition}
    \name{Unique Label Cover Game} is the label cover game $ (K,K) $  that  $ \pi_{uv} $ is a permutation over  $ [K] $.
\end{definition}
\begin{fact}
    $ \mathrm{ULC}\in \mathrm{P} $ 
\end{fact}
\begin{conjecture}
    $ \forall \delta>0 $,  $ \exists K>0 $ such that  $ \mathrm{ULC}_{1-\delta,\delta}\not\in P $   
\end{conjecture}

\begin{proof}[Proof of Remark \ref{16/17 Hardness of Max-Cut}]
    \,\\

    \textbf{Cut Dictatorship Test}:  $ f:\{\pm 1\}^n\rightarrow\{\pm 1\} $.
    
    Cut test:  $ f(\vec{x})\neq f(\vec{y}) $.
    
    Parameter: $ \rho $.
    \begin{enumerate}
        \item Uniformly sample  $ \vec{x}\sim \{\pm 1\}^n $
        \item Sample  $ \vec{\mu}\in \{\pm 1\}^n $
        \[\mu_i=\begin{cases}
            1 & \text{w.p. }\frac{1+\rho}{2}\\
            -1 & \text{w.p. } \frac{1-\rho}{2}
        \end{cases}\,(\Ebb\mu_i=\rho)\]
        Test  $ f(\vec{x})\neq f(\vec{x}\vec{\mu}) $ 
    \end{enumerate}  

    \begin{align*}
        \mathrm{Pr}[f\text{ pass}]&=\Ebb_{\vec{x},\vec{\mu}}\left[\frac{1-f(\vec{x})f(\vec{x}\vec{\mu})}{2}\right]\\
        &=\frac{1}{2}-\frac{1}{2}\Ebb_{\vec{x},\vec{\mu}}\left(\sum_S\hat{f}(S)\chi_S(\vec{x})\right)\left(\sum_T\hat{f}(T)\chi_T(\vec{x}\vec{\mu})\right)\\
        &=\frac{1}{2}-\frac{1}{2}\sum_{S,T}\hat{f}(S)\hat{f}(T)\left(\sum_{\vec{x}}\chi_S(\vec{x})\chi_T(\vec{x})\right)\left(\sum_{\vec{\mu}}\chi_T(\mu)\right)\\
        &=\frac{1}{2}-\frac{1}{2}\sum_S\hat{f}(S)^2\rho^{|S|}\\
        &=\frac{1}{2}-\frac{1}{2}\Sbb_\rho(f)
    \end{align*}
    
    We investigate some major functions:
    \begin{enumerate}
        \item Dictators: $ f(\vec{x})=x_i $
        \[\mathrm{Pr}[f\text{ pass}]=\frac{1}{2}-\frac{1}{2}\rho\]
        \item Constant functions:  $ f\equiv 0,f\equiv 1 $.
        \[ \mathrm{Pr}[f\text{ pass}]=0\]
        \item Parity functions:  $ f=\chi_S(\vec{x}) $.
        \[\mathrm{Pr}[f\text{ pass}]=\frac{1}{2}-\frac{1}{2}\rho^{|S|}\]
        \item Majority function:  $\dps f(\vec{x})=\sgn\left(\frac{x_1+\cdots +x_n}{\sqrt{n}}\right) $ 
        \[\begin{pmatrix}
            \frac{x_1+\cdots +x_n}{\sqrt{n}}\\
            \frac{\sum x_i\mu_i}{\sqrt{n}}
        \end{pmatrix}\rightarrow \mathcal{N}(0,\begin{bmatrix}
            1&\rho\\
            \rho&1
        \end{bmatrix})\]
        \[\mathrm{Pr}[f\text{ passes}]\approx\mathop{Pr}_{\begin{pmatrix}
        g_1\\
        g_2
    \end{pmatrix}\sim \mathcal{N}(0,\begin{bmatrix}
         1&\rho\\
        \rho&1
    \end{bmatrix})}[\sgn(g_1)\neq \sgn(g_2)]=\frac{\arccos\rho}{\pi}\]

    \item Linear Threshold function (LFT)
    \[f(\vec{x})=\sgn(\vec{a}^T\vec{x}),\|\vec{a}\|_2=1\]
    Then Majority function is a special case  $ \vec{a}=(1/\sqrt{n},\cdots,1/\sqrt{n}) $  of LFT.

    Dictator function is also a special case  $ \vec{a}=(0,0,\cdots,0,1,0\cdots,0) $  of LFT.
    \end{enumerate}
    
    \begin{theorem}[High-Dim Berry Esseen]
        Let  $ \vec{Y}_1,\cdots,\vec{Y}_n $ be independent random variables in  $ \Rbb^d $, where  $ \Ebb Y_i=\vec{0}, \dps\sum_{i=1}^n\mathrm{Val}[Y_i]=I_d $  Let  $ \vec{Y}=\vec{Y}_1+\cdots+\vec{Y}_n $. For all measurable convex sets  $ A\subset \Rbb^d $, we have 
        \[\left|\mathrm{Pr}[\vec{Y}\in A]-\mathop{Pr}_{\vec{g}\sim N(0,I_d)}[\vec{g}\in A]\right| \leq (42d^{1/4}+16)\sum_{i=1}^n\Ebb\|Y_i\|^3_2\]    
    \end{theorem}

    \begin{lemma}
        If  $ f=\sgn(\vec{a}^T\vec{x}) $,  $ \|a\|_2=1 $. Then 
        \[\mathrm{Pr}[f\text{ passes}]=\frac{\arccos\rho}{\pi}\pm \frac{O(1)}{\sqrt{(1-\rho^2)^3}}\max_i|a_i|\]  
    \end{lemma}

    Completeness: If  $ f $ is a dictator function, then  $ \mathrm{Pr}[\text{pass}]=\dps\frac{1-\rho}{2} $.
    
    Soundness:  $ f $ far from dictator, then $ \mathrm{Pr}[text{pass}]\sim \dps\frac{\arccos\rho}{\pi} $ $ \Leftrightarrow $ $ \mathrm{Pr}[\text{pass}]>\frac{\arccos\rho}{\pi}+\epsilon $, then  $ f $ has a few "influential dimensions" which can be decoded for  $ ULC $. 
    
    \begin{definition}[Influence]
        Given  $ f:\{\pm 1\}^n\rightarrow\{\pm 1\} $. For any  $ i\in [n] $, define 
        \[\begin{aligned}
            \mathrm{Inf}_i(f)&=\mathop{Pr}_{\vec{x}\sim \{\pm 1\}^n}\left[f(\vec{x})\neq f(\vec{x}^{\otimes i})\right]\\
            &=\Ebb_{\vec{x}\sim \{\pm 1\}^n}\left[\frac{1-f(\vec{x})f(\vec{x}^{\otimes i})}{2}\right]\\
            &=\frac{1}{2}-\frac{1}{2}\Ebb_{\vec{x}}\left(\sum_S\hat{f}(S)\chi_S(\vec{x})\right)\left(\sum_T\hat{f}(T)\chi_T(\vec{x}^{\otimes i})\right)\\
            &=\frac{1}{2}-\frac{1}{2}\sum_{S,T}\hat{f}(S)\hat{f}(T)\Ebb_{\vec{x}}\chi_S(\vec{x})\chi_T(\vec{x}^{\otimes i})\\
            &=\frac{1}{2}-\frac{1}{2}\sum_{i\in S}\hat{f}(S)-\frac{1}{2}\sum_{i\not\in S}\hat{f}(S)^2\\
            &=\sum_{i\in S}\hat{f}(S)^2
        \end{aligned}\]  
    \end{definition}
    \begin{example}
        Let  $ f $ be the majority function over  $ n=2k+1 $ variables. Then 
        \[\mathrm{Inf}_i(f)=\mathop{Pr}_{\vec{x}\sim \{\pm1\}^n}[\mathrm{maj}(\vec{x})\neq \mathrm{maj}(\vec{x}^{\otimes i})]=\frac{\binom{2k}{k}}{2^{2k}}=\frac{(2k)!}{2^{2k}(k!)^2}\approx\frac{\sqrt{2\pi (2k)}\cdot\left(\frac{2k}{e}\right)^{2k}}{2^{2k}(2\pi k)\left(\frac{k}{e}\right)^{2k}}=\frac{\sqrt{4\pi k}}{2\pi k}=\Theta(\frac{1}{\sqrt{n}})\]  
    \end{example}
    \begin{theorem}[Majoiry is Stablest]
        Let  $ \rho\in (-1,0) $,  $ \epsilon>0 $. Then  $ \exists \tau=\tau(\rho,\epsilon)>0 $ such that  $ \forall f:\{\pm 1\}\rightarrow\{\pm 1\} $, if  $ \mathrm{Inf}_i(f) \leq \tau $,  $ \forall i $, then 
        \[\Sbb_{\rho}(f)>1-\frac{2\arccos\rho}{\pi}-\epsilon\]
        Therefore,
        \[\mathrm{Pr}[f(\vec{x})\neq f(\vec{x}\vec{\mu})]=\frac{1}2{}-\frac12\Sbb_{\rho}(f)<\frac{\arccos\rho}{\pi}+\frac{\epsilon}{2}\]      
    \end{theorem}
\end{proof}


