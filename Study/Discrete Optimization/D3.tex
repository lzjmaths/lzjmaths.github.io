%!TEX root = /lecture/Discrete_Optimistic.tex

\begin{claim}
    Any cycle  $ C $ and cutset $ D $ has intersection  $ |C\cap D| $ even.   
\end{claim}

\textbf{Fundamental Cycle}: Given  $ G $ and spanning tree  $ T\subset E $, for each  $ e\in E\setminus T $, the unique cycle in  $ T\cup\{e\} $ is called \name{Fundamental cycle}.

\begin{claim}
    For a fundamental cycle  $ C $ related with  $ e $, $ \forall f\in C\cap T $,  $ (T\cup \{e\})\setminus\{f\} $ is  also a spanning tree.
\end{claim}
If  $ T  $ is  $ MST $, then  $ \omega(e) \geq \omega(f) $.  


\textbf{Fundamental Cut}: Spanning tree  $ T\subset E $. For each  $ f\in T $,  $ T\setminus\{f\} $   has two connected components, whose cutset is called \name{fundamental cut}.

\begin{claim}
    $ \forall e\in D\setminus T $,  $ (T\cup\{e\})\setminus\{f\} $ is  a spanning tree. 
\end{claim}
If  $ T  $ is MST, then  $ \omega(e) \geq \omega(f) $.

\paragraph{MST Algorithm}
There are some rules.
\textbf{Red rule}:Let  $ C  $ a cycle without red edges. Select an uncolored edge in  $ C   $ with max weight and color it red.

\textbf{Blue rule}: Let  $ D  $ be a cutset without blue edges. Select an un  colored edge in  $ D $ with min weight and color it blue.

\textbf{Greedy Algorithm}:Apply red or blue rules in any order iteratively until all edges colored.

\begin{theorem}
    Greedy algorithm terminates and  blue edges from MST.
\end{theorem}
\begin{proof}
    Observed that during the algorithm, blue edges always from a forest.
\end{proof}

\paragraph{Invariant}  $ \exists  $ MST  $ T^*  $ \st  $ T^* $ contains all  blue edges and no red edges.   
\begin{proof}
    Proof by induction. If there is a MST  $ T^* $ contains all blue edges no red edges now. If we apply blue rule, with cutset  $ D $ and  $ f\in D $ but  $ f\not\in T^* $, then for fundamental cycle  $ C $ of  $ f $,  $ \forall e\in C\cap T $,  $ \omega(e) \geq \omega(f) $. Since  $ C $ has even edges in the cutset by the claim,  $ \exists e\in C\cap T $ \st  $ e\in D $, which contradicts the fact that  $ f $ is the edge in cutset  $ D $        with min weight.

    The case that we  apply red rule is similar.
\end{proof}

\begin{algorithm}
    \caption{Prim's Algorithm}
    \label{alg:Prim}
    \begin{algorithmic}[1]
    \STATE Initialize  $ S\leftarrow \{s\} $.
    \WHILE{$ n-1 $ times}
        \STATE Choose   $ e  $ be the min weight edge in the cutset $ (S,V\setminus S) $  
        \STATE add  $ e $ to  $ T $, another endpoint of  $ e $ to  $ S $. 
    \ENDWHILE
    \end{algorithmic}
\end{algorithm}


\begin{remark}
    It is compatible with the simple idea: Each time chooses the min weight edge. However, it is more powerful since we only need to do this process in the cutset.

    It is similar to Dijkstra's Algorithm. So its time complexity is  $ O(|E|+|V|\log |V|) $ 
\end{remark}



\begin{algorithm}
    \caption{Kruskal's Algorithm}
    \label{alg:Kruskal}
    \begin{algorithmic}[1]
        \STATE Consider edges in weight increasing order.
        \STATE Add each edge to  $ T $ if not introducing a cycle. 
    \end{algorithmic}
\end{algorithm}
\begin{remark}
    The first step need time complexity  $ O(|E|\log |E|) $.

    The second step need time complexity  $ O(|E|\cdot \alpha(|V|)) $ using \textbf{Union-Find} data structure. 
\end{remark}

WLOG we can assume edge weights are distinct.
\begin{algorithm}
    \caption{Boruvka's Algorithm}
    \label{alg:Boruvka}
    \begin{algorithmic}[1]
        \WHILE{ $ <(n-1) $ blue edges}
            \STATE Simultaneously apply blue rule to each blue compunent.
        \ENDWHILE
    \end{algorithmic}
\end{algorithm}

\begin{claim}
    WHILE loop iterates  $  \leq O(\log|V|) $.
    
    So time complexity is  $ O(|E|\log|V|) $. 
\end{claim}

\begin{remark}
    There is a "contraction View". For each step, we can view each component as a single point with edges to other components.

    If the graph is \textbf{Planar Graph}, then  $ |E| \leq O(V) $. At the  $ i $-th WHILE iteration,   $ |V_i| \leq \dps\frac{|V|}{2^{i-1}} $,  $ |E_i| \leq O(|V_i|) $. 
    
    So the time complexity is  $ \dps\sum_{i}O(|E_i|) \leq \sum_i O\left(\dps\frac{|V|}{2^{i-1}}\right) \leq O(|V|) $ which is linear! 
\end{remark}
Using the contraction view, we can get another algorithm:

\paragraph{Prim+Boruvka} 
\begin{itemize}
    \item Run Boruvka for  $ k $ iterates.
    \item Run Prim on the contracted graph.
\end{itemize}

\begin{remark}
    \,

    For step 1, time  complexity is  $ k\cdot|E| $.

    For step 2, time complexity is   $ |E|+\dps\frac{|V|}{2^k}\cdot\log\frac{|V|}{2^k} $.

    So the total time complexity is  $ k|E|+\dps\frac{|V|}{2^k}\cdot\log\frac{|V|}{2^k} $.

    Choose  $ k=\log_2\log_2|V| $, it  comes to  $ (\log\log|V|)\cdot |E|+\dps\frac{|V|}{\log_2|V|}\cdot\log_2|V| \leq O(|E|\log\log|V|+|V|) $.
\end{remark}

\subsection{Minimum Arborescence}
\begin{example}[Minimum Arborescence]
    Input: Directed  $ G=(V,E) $, source  $ s\in V $ and weight  $ \omega:E\rightarrow\Rbb $. 
    
    We want to find an \textbf{arborescence} $ T=(V,E) $ with root  $ r $ of  minimum total weight.
\end{example}

