 % !TeX spellcheck = en_US
% !TEX program = pdflatex
\documentclass[12pt,b5paper,notitlepage]{article}
\usepackage[b5paper, margin={0.5in,0.65in}]{geometry}
%\usepackage{fullpage}
\usepackage{amsmath,amscd,amssymb,amsthm,mathrsfs,amsfonts,layout,indentfirst,graphicx,caption,mathabx, stmaryrd,appendix,calc,imakeidx,upgreek,amsbsy,thmtools} % mathabx for \wtidecheck
%\usepackage{ulem} %wave underline
\usepackage[dvipsnames]{xcolor}
\usepackage{palatino}  %template

\usepackage{slashed} % Dirac operator
\usepackage{mathrsfs} % Enable using \mathscr
%\usepackage{eufrak}  another template/font
\usepackage{extarrows} % long equal sign, \xlongequal{blablabla}
\usepackage{enumitem} % enumerate label change e.g. [label=(\alph*)]  shows (a) (b) 


%%%%%%%%%%%%%%%%%%%%%%%%%%%%%%

%\usepackage{fontspec}
%\setmainfont{Palatino Linotype}
%\usepackage{emoji}


% emoji, use lualatex  remove \usepackage{palatino}

%%%%%%%%%%%%%


\usepackage{CJK}   % Chinese package





\usepackage{csquotes} % \begin{displayquote}   \begin{displaycquote}  for quotation
\usepackage{epigraph}   %\epigraph{}{}  for quotation
%\pmb  mandatory math bold 

\usepackage{fancyhdr} % date in footer

%\usepackage{soul}  %\ul underline break line automatically

\usepackage{ulem}  % \uline  underline break line   also    \uwave

\usepackage{relsize} % use \mathlarger \larger \text{\larger[2]$...$} to enlarge the size of math symbols

\usepackage{verbatim}  % comment environment


\usepackage{halloweenmath} % Interesting halloween math symbols

%%%%%%%%%%%%%%%%%%%%%%%%%%%%%%
\usepackage{tcolorbox}
\tcbuselibrary{theorems}
% box around equations   \tcboxmath
%%%%%%%%%%%%%%%%%%%%%%%%%%%%%%%%%%





%%%%%%%%%%%%%%%%%%%%%%%%%%%%%
% circled colon and thick colon \hcolondel and \colondel

\usepackage{pdfrender}

\newcommand*{\hollowcolon}{%
	\textpdfrender{
		TextRenderingMode=Stroke,
		LineWidth=.1bp,
	}{:}%
}

\newcommand{\hcolondel}[1]{%
	\mathopen{\hollowcolon}#1\mathclose{\hollowcolon}%
}
\newcommand{\colondel}[1]{%
	\mathopen{:}#1\mathclose{:}%
}

%%%%%%%%%%%%%%%%%%%%%%%%%%%%%%%%


\usepackage{setspace}  
\setstretch{1.6}



\usepackage{tikz}
\usetikzlibrary{fadings}
\usetikzlibrary{patterns}
\usetikzlibrary{shadows.blur}
\usetikzlibrary{shapes}

\usepackage{tikz-cd}
\usepackage[nottoc]{tocbibind}   % Add  reference to ToC


\makeindex


% The following set up the line spaces between items in \thebibliography
\usepackage{lipsum}  
\let\OLDthebibliography\thebibliography
\renewcommand\thebibliography[1]{
	\OLDthebibliography{#1}
	\setlength{\parskip}{0pt}
	\setlength{\itemsep}{2pt} 
}


%\hyperref{page.10}{...}

\allowdisplaybreaks  %allow aligns to break between pages
\usepackage{latexsym}
\usepackage{chngcntr}
\usepackage[colorlinks,linkcolor=blue,anchorcolor=blue, linktocpage,
%pagebackref
]{hyperref}
\hypersetup{ urlcolor=cyan,
	citecolor=[rgb]{0,0.5,0}}


\setcounter{tocdepth}{2}	 %hide subsections in the content


\counterwithin{figure}{section}

\counterwithin*{footnote}{section}   % Footnote numbering is recounted from the beginning of each subsection



\pagestyle{plain}

\captionsetup[figure]
{
	labelsep=none	
}













\theoremstyle{definition}
\newtheorem{definition}{Definition}[section]
\newtheorem{example}[definition]{Example}
\newtheorem{exercise}[definition]{Exercise}
\newtheorem{remark}[definition]{Remark}
\newtheorem{observation}[definition]{Observation}
\newtheorem{assumption}[definition]{Assumption}
\newtheorem{convention}[definition]{Convention}
\newtheorem{priniple}[definition]{Principle}
\newtheorem{notation}[definition]{Notation}
\newtheorem*{axiom}{Axiom}
\newtheorem{coa}[definition]{Theorem}
\newtheorem{srem}[definition]{$\star$ Remark}
\newtheorem{seg}[definition]{$\star$ Example}
\newtheorem{sexe}[definition]{$\star$ Exercise}
\newtheorem{sdf}[definition]{$\star$ Definition}
\newtheorem{question}{Question}
\theoremstyle{remark}
\newtheorem*{note}{Note}
\newtheorem*{claim}{Claim}




\newtheorem{problem}{\color{red}Problem}[section]
%\renewcommand*{\theprob}{{\color{red}\arabic{section}.\arabic{prob}}}
\newtheorem{sprob}[problem]{\color{red}$\star$ Problem}
%\renewcommand*{\thesprob}{{\color{red}\arabic{section}.\arabic{sprob}}}
% \newtheorem{ssprob}[prob]{$\star\star$ Problem}



\theoremstyle{plain}
\newtheorem{theorem}[definition]{Theorem}
\newtheorem{Conclusion}[definition]{Conclusion}
\newtheorem{thd}[definition]{Theorem-Definition}
\newtheorem{proposition}[definition]{Proposition}
\newtheorem{corollary}[definition]{Corollary}
\newtheorem{lemma}[definition]{Lemma}
\newtheorem{sthm}[definition]{$\star$ Theorem}
\newtheorem{slm}[definition]{$\star$ Lemma}
\newtheorem{spp}[definition]{$\star$ Proposition}
\newtheorem{scorollary}[definition]{$\star$ Corollary}


\newtheorem{cond}{Condition}
\newtheorem{Mthm}{Main Theorem}
\renewcommand{\thecond}{\Alph{cond}} % "letter-numbered" theorems
\renewcommand{\theMthm}{\Alph{Mthm}} % "letter-numbered" theorems


%\substack   multiple lines under sum
%\underset{b}{a}   b is under a


% Remind: \overline{L_0}



\usepackage{calligra}
\DeclareMathOperator{\shom}{\mathscr{H}\text{\kern -3pt {\calligra\large om}}\,}
\DeclareMathOperator{\sext}{\mathscr{E}\text{\kern -3pt {\calligra\large xt}}\,}
\DeclareMathOperator{\Rel}{\mathscr{R}\text{\kern -3pt {\calligra\large el}~}\,}
\DeclareMathOperator{\sann}{\mathscr{A}\text{\kern -3pt {\calligra\large nn}}\,}
\DeclareMathOperator{\send}{\mathscr{E}\text{\kern -3pt {\calligra\large nd}}\,}
\DeclareMathOperator{\stor}{\mathscr{T}\text{\kern -3pt {\calligra\large or}}\,}
%write mathscr Hom (and so on) 

\usepackage{aurical}
\DeclareMathOperator{\VVir}{\text{\Fontlukas V}\text{\kern -0pt {\Fontlukas\large ir}}\,}

\newcommand{\vol}{\text{\Fontlukas V}}
\newcommand{\dvol}{d~\text{\Fontlukas V}}
% perfect Vol symbol

\usepackage{aurical}
\usepackage[T1]{fontenc}








\newcommand{\fk}{\mathfrak}
\newcommand{\mc}{\mathcal}
\newcommand{\wtd}{\widetilde}
\newcommand{\wht}{\widehat}
\newcommand{\wch}{\widecheck}
\newcommand{\ovl}{\overline}
\newcommand{\udl}{\underline}
\newcommand{\tr}{\mathrm{t}} %transpose
\newcommand{\Tr}{\mathrm{Tr}}
\newcommand{\End}{\mathrm{End}} %endomorphism
\newcommand{\idt}{\mathbf{1}}
\newcommand{\id}{\mathrm{id}}
\newcommand{\Hom}{\mathrm{Hom}}
\newcommand{\Conf}{\mathrm{Conf}}
\newcommand{\Res}{\mathrm{Res}}
\newcommand{\res}{\mathrm{res}}
\newcommand{\KZ}{\mathrm{KZ}}
\newcommand{\ev}{\mathrm{ev}}
\newcommand{\coev}{\mathrm{coev}}
\newcommand{\opp}{\mathrm{opp}}
\newcommand{\Rep}{\mathrm{Rep}}
\newcommand{\diag}{\mathrm{diag}}
\newcommand{\Dom}{\mathrm{Dom}}
\newcommand{\loc}{\mathrm{loc}}
\newcommand{\con}{\mathrm{c}}
\newcommand{\uni}{\mathrm{u}}
\newcommand{\ssp}{\mathrm{ss}}
\newcommand{\di}{\slashed d}
\newcommand{\Diffp}{\mathrm{Diff}^+}
\newcommand{\Diff}{\mathrm{Diff}}
\newcommand{\PSU}{\mathrm{PSU}(1,1)}
\newcommand{\Vir}{\mathrm{Vir}}
\newcommand{\Witt}{\mathscr W}
\newcommand{\Span}{\mathrm{Span}}
\newcommand{\pri}{\mathrm{p}}
\newcommand{\ER}{E^1(V)_{\mathbb R}}
\newcommand{\prth}[1]{( {#1})}
\newcommand{\bk}[1]{\langle {#1}\rangle}
\newcommand{\bigbk}[1]{\big\langle {#1}\big\rangle}
\newcommand{\Bigbk}[1]{\Big\langle {#1}\Big\rangle}
\newcommand{\biggbk}[1]{\bigg\langle {#1}\bigg\rangle}
\newcommand{\Biggbk}[1]{\Bigg\langle {#1}\Bigg\rangle}
\newcommand{\GA}{\mathscr G_{\mathcal A}}
\newcommand{\vs}{\varsigma}
\newcommand{\Vect}{\mathrm{Vec}}
\newcommand{\Vectc}{\mathrm{Vec}^{\mathbb C}}
\newcommand{\scr}{\mathscr}
\newcommand{\sjs}{\subset\joinrel\subset}
\newcommand{\Jtd}{\widetilde{\mathcal J}}
\newcommand{\gk}{\mathfrak g}
\newcommand{\hk}{\mathfrak h}
\newcommand{\xk}{\mathfrak x}
\newcommand{\yk}{\mathfrak y}
\newcommand{\zk}{\mathfrak z}
\newcommand{\pk}{\mathfrak p}
\newcommand{\hr}{\mathfrak h_{\mathbb R}}
\newcommand{\Ad}{\mathrm{Ad}}
\newcommand{\DHR}{\mathrm{DHR}_{I_0}}
\newcommand{\Repi}{\mathrm{Rep}_{\wtd I_0}}
\newcommand{\im}{\mathbf{i}}
\newcommand{\Co}{\complement}
%\newcommand{\Cu}{\mathcal C^{\mathrm u}}
\newcommand{\RepV}{\mathrm{Rep}^\uni(V)}
\newcommand{\RepA}{\mathrm{Rep}(\mathcal A)}
\newcommand{\RepN}{\mathrm{Rep}(\mathcal N)}
\newcommand{\RepfA}{\mathrm{Rep}^{\mathrm f}(\mathcal A)}
\newcommand{\RepAU}{\mathrm{Rep}^\uni(A_U)}
\newcommand{\RepU}{\mathrm{Rep}^\uni(U)}
\newcommand{\RepL}{\mathrm{Rep}^{\mathrm{L}}}
\newcommand{\HomL}{\mathrm{Hom}^{\mathrm{L}}}
\newcommand{\EndL}{\mathrm{End}^{\mathrm{L}}}
\newcommand{\Bim}{\mathrm{Bim}}
\newcommand{\BimA}{\mathrm{Bim}^\uni(A)}
%\newcommand{\shom}{\scr Hom}
\newcommand{\divi}{\mathrm{div}}
\newcommand{\sgm}{\varsigma}
\newcommand{\SX}{{S_{\fk X}}}
\newcommand{\DX}{D_{\fk X}}
\newcommand{\mbb}{\mathbb}
\newcommand{\mbf}{\mathbf}
\newcommand{\bsb}{\boldsymbol}
\newcommand{\blt}{\bullet}
\newcommand{\Vbb}{\mathbb V}
\newcommand{\Ubb}{\mathbb U}
\newcommand{\Xbb}{\mathbb X}
\newcommand{\Kbb}{\mathbb K}
\newcommand{\Abb}{\mathbb A}
\newcommand{\Wbb}{\mathbb W}
\newcommand{\Mbb}{\mathbb M}
\newcommand{\Gbb}{\mathbb G}
\newcommand{\Cbb}{\mathbb C}
\newcommand{\Nbb}{\mathbb N}
\newcommand{\Zbb}{\mathbb Z}
\newcommand{\Qbb}{\mathbb Q}
\newcommand{\Pbb}{\mathbb P}
\newcommand{\Rbb}{\mathbb R}
\newcommand{\Ebb}{\mathbb E}
\newcommand{\Dbb}{\mathbb D}
\newcommand{\Hbb}{\mathbb H}
\newcommand{\cbf}{\mathbf c}
\newcommand{\Rbf}{\mathbf R}
\newcommand{\wt}{\mathrm{wt}}
\newcommand{\Lie}{\mathrm{Lie}}
\newcommand{\btl}{\blacktriangleleft}
\newcommand{\btr}{\blacktriangleright}
\newcommand{\svir}{\mathcal V\!\mathit{ir}}
\newcommand{\Ker}{\mathrm{Ker}}
\newcommand{\Cok}{\mathrm{Coker}}
\newcommand{\Sbf}{\mathbf{S}}
\newcommand{\low}{\mathrm{low}}
\newcommand{\Sp}{\mathrm{Sp}}
\newcommand{\Rng}{\mathrm{Rng}}
\newcommand{\vN}{\mathrm{vN}}
\newcommand{\Ebf}{\mathbf E}
\newcommand{\Nbf}{\mathbf N}
\newcommand{\Stb}{\mathrm {Stb}}
\newcommand{\SXb}{{S_{\fk X_b}}}
\newcommand{\pr}{\mathrm {pr}}
\newcommand{\SXtd}{S_{\wtd{\fk X}}}
\newcommand{\univ}{\mathrm {univ}}
\newcommand{\vbf}{\mathbf v}
\newcommand{\ubf}{\mathbf u}
\newcommand{\wbf}{\mathbf w}
\newcommand{\CB}{\mathrm{CB}}
\newcommand{\Perm}{\mathrm{Perm}}
\newcommand{\Orb}{\mathrm{Orb}}
\newcommand{\Lss}{{L_{0,\mathrm{s}}}}
\newcommand{\Lni}{{L_{0,\mathrm{n}}}}
\newcommand{\UPSU}{\widetilde{\mathrm{PSU}}(1,1)}
\newcommand{\Sbb}{{\mathbb S}}
\newcommand{\Gc}{\mathscr G_c}
\newcommand{\Obj}{\mathrm{Obj}}
\newcommand{\bpr}{{}^\backprime}
\newcommand{\fin}{\mathrm{fin}}
\newcommand{\Ann}{\mathrm{Ann}}
\newcommand{\Real}{\mathrm{Re}}
\newcommand{\Imag}{\mathrm{Im}}
%\newcommand{\cl}{\mathrm{cl}}
\newcommand{\Ind}{\mathrm{Ind}}
\newcommand{\Supp}{\mathrm{Supp}}
\newcommand{\Specan}{\mathrm{Specan}}
\newcommand{\red}{\mathrm{red}}
\newcommand{\uph}{\upharpoonright}
\newcommand{\Mor}{\mathrm{Mor}}
\newcommand{\pre}{\mathrm{pre}}
\newcommand{\rank}{\mathrm{rank}}
\newcommand{\Jac}{\mathrm{Jac}}
\newcommand{\emb}{\mathrm{emb}}
\newcommand{\Sg}{\mathrm{Sg}}
\newcommand{\Nzd}{\mathrm{Nzd}}
\newcommand{\Owht}{\widehat{\scr O}}
\newcommand{\Ext}{\mathrm{Ext}}
\newcommand{\Tor}{\mathrm{Tor}}
\newcommand{\Com}{\mathrm{Com}}
\newcommand{\Mod}{\mathrm{Mod}}
\newcommand{\nk}{\mathfrak n}
\newcommand{\mk}{\mathfrak m}
\newcommand{\Ass}{\mathrm{Ass}}
\newcommand{\depth}{\mathrm{depth}}
\newcommand{\Coh}{\mathrm{Coh}}
\newcommand{\Gode}{\mathrm{Gode}}
\newcommand{\Fbb}{\mathbb F}
\newcommand{\sgn}{\mathrm{sgn}}
\newcommand{\Aut}{\mathrm{Aut}}
\newcommand{\Modf}{\mathrm{Mod}^{\mathrm f}}
\newcommand{\codim}{\mathrm{codim}}
\newcommand{\card}{\mathrm{card}}
\newcommand{\dps}{\displaystyle}
\newcommand{\Int}{\mathrm{Int}}
\newcommand{\Nbh}{\mathrm{Nbh}}
\newcommand{\Pnbh}{\mathrm{PNbh}}
\newcommand{\Cl}{\mathrm{Cl}}
\newcommand{\diam}{\mathrm{diam}}
\newcommand{\eps}{\varepsilon}
\newcommand{\Vol}{\mathrm{Vol}}
\newcommand{\LSC}{\mathrm{LSC}}
\newcommand{\USC}{\mathrm{USC}}
\newcommand{\Ess}{\mathrm{Rng}^{\mathrm{ess}}}
\newcommand{\Jbf}{\mathbf{J}}
\newcommand{\SL}{\mathrm{SL}}
\newcommand{\GL}{\mathrm{GL}}
\newcommand{\Lin}{\mathrm{Lin}}
\newcommand{\ALin}{\mathrm{ALin}}
\newcommand{\bwn}{\bigwedge\nolimits}
\newcommand{\nbf}{\mathbf n}
\newcommand{\dive}{\mathrm{div}}








\usepackage{tipa} % wierd symboles e.g. \textturnh
\newcommand{\tipar}{\text{\textrtailr}}
\newcommand{\tipaz}{\text{\textctyogh}}
\newcommand{\tipaomega}{\text{\textcloseomega}}
\newcommand{\tipae}{\text{\textrhookschwa}}
\newcommand{\tipaee}{\text{\textreve}}
\newcommand{\tipak}{\text{\texthtk}}
\newcommand{\mol}{\upmu}
\newcommand{\dmol}{d\upmu}




\usepackage{tipx}
\newcommand{\tipxgamma}{\text{\textfrtailgamma}}
\newcommand{\tipxcc}{\text{\textctstretchc}}
\newcommand{\tipxphi}{\text{\textqplig}}















\numberwithin{equation}{section}
% count the eqation by section countation



\title{Complex Analysis}
\author{{\sc Lin150117}
	\\
	{\small \sc Tsinghua University.}\\
	{\small linzj23@mails.tsinghua.edu.cn}
}

\DeclareMathOperator{\sign}{sign}
\DeclareMathOperator{\dom}{dom}
\DeclareMathOperator{\ran}{ran}
\DeclareMathOperator{\ord}{ord}
\DeclareMathOperator{\img}{Im}
\DeclareMathOperator{\dd}{d\!}
\newcommand{\ie}{ \textit{ i.e. } }
\newcommand{\st}{ \textit{ s.t. }}
\newcommand{\name}[1]{\textbf{#1}\index{#1}}
\newcommand{\subname}[2]{\textbf{#1}\index{#2!#1}}

\begin{document}
\sloppy
\pagenumbering{arabic}
\maketitle
\tableofcontents
\newpage

%put your code here
\section{Smooth Manifold}
\begin{definition}[Topological manifold]
    A space  $ M  $ is called a topological manifold if 
    \begin{enumerate}
        \item locally Euclidean
        \item Hausdorff
        \item second countable
    \end{enumerate}
\end{definition}
\begin{definition}[Smooth Manifold]
     A smooth structure is given by an equivalence class of smooth atlas  $ \{(U_\alpha,\varphi_\alpha)\} $ \st  $ \varphi_{\alpha\beta}:\varphi_\alpha(U_\alpha\cap U_\beta)\rightarrow \varphi_\beta(U_\alpha\cap U_\beta) $  is smooth  $ \forall \alpha,\beta $.  $ M=\cup U_\alpha $.\\
     A \name{smooth manifold} is a topological manifold with a smooth structure.\\
     Define when a continuous map  $ f:M_1\rightarrow M_2 $ is smooth if  $ \forall (U_1,\varphi_1)\in\mathcal{A}_1,(U_2,\varphi_2)\in\mathcal{A}_2 $, we have  $ \varphi_2\circ f\circ \varphi_1^{-1}:\varphi_1(U_1\cap U_2)\rightarrow \varphi_2(U_1\cap U_2) $ is smooth. 
\end{definition}
\begin{definition}
    Given  $ (M_1,\mathcal{A}_1),(M_2,\mathcal{A}_2) $. A homeomorphism  $ f:M_1\rightarrow M_2 $ is called a diffeomorphism if   $ f $, $ f^{-1} $  is smooth. \\
    In this case we say  $ (M_1,\mathcal{A}_1),(M_2,\mathcal{A}_2) $ are diffeomorphism. 
\end{definition}
\begin{theorem}[Kervaire]
     $ \exists  $ 1 10-dimensional topological manifold without smooth manifold.
\end{theorem}
\begin{theorem}[Milnor]
     $ \exists  $ a smooth manifold  $ M  $ \st  $ M\cong S^7 $ but not in diffeomorphism meaning.  
\end{theorem}
\begin{theorem}[Kervaire-Milnor]
     $ \exists  $ 28 smooth structures (up to orientation preserving diffeomorphism) on  $ S^7 $ 
\end{theorem}
\begin{theorem}[Morse-Birg]
    On  $ S^7  $. If  $ n \leq 3  $, then any  $ n  $-dimensional topological manifold  $ M  $ has a unique smooth structure up to diffeomorphism.
\end{theorem}
\begin{theorem}[Stallings]
    If  $ n\not=4  $, then  $ \exists  $ a unique smooth structure on  $ \mathbb{R}^n  $ up to diffeomorphism.
\end{theorem}
\begin{theorem}[Donaldson-Freedom-Gompf-Faubes]
     $ \exists  $ uncountable smooth structures on  $ \mathbb{R}^4 $ up to diffeomorphism. 
\end{theorem}
\begin{definition}[topological manifold with boundary]
    A space  $ M  $ is called a topological manifold with boundary if 
    \begin{enumerate}
        \item  $ M  $ is Hausdorff
        \item  $ M  $ is second countable 
        \item  $ \forall  p\in M $,  $ \exists   $ a neighbourhood  $ U  $ of  $ p  $ and a homeomorphism  $ \varphi:U\rightarrow V    $  where  $ V  $ is open in  $ \mathbb{H}^n $ 
    \end{enumerate}
    We say a manifold  $ M  $ is closed if  $ M  $ is compact and  $ \partial M  $ is empty.
\end{definition}
Our motivation for studying manifold is to study the space of solution for equations.
\begin{question}
    Given  $ f:\mathbb{R}^n\rightarrow \mathbb{R} $ smooth,  $ q\in \mathbb{R}^n $, when is  $ f^{-1}(q)  $ is a smooth manifold?
\end{question}

For  $ f:U\rightarrow \mathbb{R}^n $ smooth,  $ U  $ open in  $ \mathbb{R}^m $,  the differential of  $ f  $ at  $ p\in U  $ denoted as  $ \mathrm{d}f(p) $.  
\begin{definition}
    We say  $ p\in U  $ is a \textbf{regular point}\index{regular point} of  $ f  $ if  $ \mathrm{d}f(p)  $ is surjective. Otherwise we say  $ p\in U  $ is a \textbf{critical point}\index{critical point}.\\
    A point  $ q\in \mathbb{R}^n  $ is called a \textbf{regular value}\index{regular value} of  $ f  $ if  $ \forall  p\in f^{-1}(q)  $ ,  $ p  $ is a regular point of  $ f $.\\
    A point  $ q\in \mathbb{R}^n  $ is called a \textbf{critical value}\index{critical value} of  $ f  $ if  $ \forall  p\in f^{-1}(q)  $ ,  $ p  $ is a critical point of  $ f $.
\end{definition}
\begin{theorem}[Implicit function theorem]
    If  $ p\in U  $ is a regular point of  $ f:U\rightarrow \mathbb{R}^n  $. Then there exists 
    \begin{itemize}
        \item An open neighbourhood  $ V  $ of  $ p  $ in  $ U  $
        \item An open subset  $ V'  $ of  $ \mathbb{R}^m $
        \item  A diffeomorphism  $ \varphi:V\rightarrow V'  $ such that  $ P\circ \varphi=f $ where  $ P  $ is the projection from  $ \mathbb{R}^m $ to  $ \mathbb{R}^n $. 
    \end{itemize}
    In other words, near a regular point, we can do local coordinate change to turn  $ f  $ into the projection.
\end{theorem}
\begin{remark}
    In particular, we have a homeomorphism
    \[ f^{-1}(f(p))\cap V \xrightarrow[\text{restriction of  $ \varphi $ }]{\cong }\{(x_1,\dots,x_m)\in V'|(x_1,\cdots.x_n)=f(p)\}\]
    \ie if we set  $ M=f^{-1}(f(p)) $, then  $ (M\cap V,\varphi_p) $ is a chart that contains  $ p  $.  
\end{remark}
\begin{corollary}
    If  $ q  $ is a regular value of  $ f:U\rightarrow \mathbb{R}^n $ then  $ f^{-1}(q) $ is a smooth manifold.
\end{corollary}
\begin{remark}
    It suffices to show that the corresponding charts are compatible.
\end{remark}
\begin{theorem}[Sard]
    If  $ f:U\rightarrow \mathbb{R}^n $ is a smooth map, then the set of critical values of  $ f $ has measure $  0 $.
\end{theorem}
\begin{remark}
    For a "generic"  $ q  $,  $ f^{-1}(q)  $ is a manifold of dimension  $ m-n $. 
\end{remark}
\begin{corollary}
    If  $ f:U\rightarrow\mathbb{R}^n $ is smooth and  $ m<n  $ then  $ f(U ) $ has measure $  0  $. 
\end{corollary}

\subsection{Power Series}
\subsubsection{Power series}
\begin{theorem}[Abel's theorem]
If  $ \sum a_n  $ converges, then  $ f(z)=\sum a_nz^n\rightarrow f(1)  $ as  $ z\rightarrow 1  $ in such a way that  $ \frac{|1-z| }{1-|z| } $ remains bounded.
\end{theorem}
\subsection{Exponential, Trigonometric and logorithmic functions}
\subsubsection{Exponential and Trigonometric function}
The \name{exponential function } is defined as the solution if the differential equation
\[\left\{
    \begin{aligned}
        f'(z)=f(z)\\
        f(0)=1
    \end{aligned}
\right.\]
We denote  $ e^z=\exp z=\sum\limits_{n=0}^\infty \frac{z^n}{n!} $.

The \name{trigonometric function} are defined by 
\[\cos z=\frac{e^{iz}+e^{-iz}}{2}\qquad \sin z=\frac{e^{iz}-e^{-iz}}{2i}\]
\subsubsection{Logorithmic Functions}
The \name{logorithmic function}  $ \ln  $ is defined by  $ z=\ln w  $ is a root of the equation  $ e^z=w $.\\
For  $ w\not=0 $, we write  $ z=x+iy $, then 
\[e^{x+iy}=w\Leftrightarrow 
\left\{
    \begin{aligned}
        e^x=|w|\\
        e^{iy}=\frac{w }{|w|}
    \end{aligned}
\right.\]   
The first equation has a unique solution  $ x=\ln|w|  $.

The second equation  $ e^{iy}=\dfrac{w }{|w|}  $ has a unique solution  $ y_0\in [0,2\pi) $.

If we write  $ w=re^{i\theta} $, then  $ x=\ln w,y=\theta =\arg w $.

Thus, for  $ w\not=0  $, we have 
\[\ln w=\ln|w|+i\arg w\]
The function  $ \ln  $ is actually not single-valued. But we can define a single-valued function  $ Ln $ 

We define 
\[a^b=\exp(b\ln a)\]   
We will prove  $ Ln  $ is analytic in  $ \mathbb{C}-(-\infty,0] $ but not continuous in  $ (-\infty,0] $.

$ Ln  $ is the principal branch of the logithm.   
\section{Conformal Mappings}
\subsection{Basic topology}
\subsubsection{Connectedness}
\begin{theorem}
    A nonempty open set in  $ \mathbb{C} $ is connected iff any two of its points can be joined by a polygon  which lies in the set,\ie Connectedness is equivalent to Path Connectedness
\end{theorem}
An nonempty connected subset is called a \name{region}
\subsubsection{Compactness}
\begin{definition}
    A set  $ X  $ is \name{totally bounded } if  $ \forall \epsilon>0  $,  $ X  $ can be covered by finitely many balls of radius  $ \epsilon  $ 
    
\end{definition}
\begin{theorem}
    A set is compact iff it is complete and totally bounded.
\end{theorem}
\begin{theorem}
    A subset  $ X\subset $ is compact iff every infinite sequence of  $ X  $ has a limit point in  $ X  $.
\end{theorem}
\subsubsection{Continuous Functions}
\begin{theorem}
    Continous function maps connected space to connected space.
\end{theorem}
\begin{theorem}
    Continous function maps compact space to compact space.
\end{theorem}
\subsection{Conformality, geometric consequences of the existence of a derivative}
\subsubsection{Arcs and closed curves}
The equation of an \name{arc} r in  $ \mathbb{C} $ can  be represented by one of the terms
\begin{itemize}
    \item  $ x=x(t),y=y(t) $, $ \alpha \leq t  \leq \beta$,  $ x,y $ are continuous at  $ t $
    \item  $ z(t)=x(t)+iy(t) $, $ \alpha \leq t \leq \beta $.
    \item The  continuous mapping  $ \gamma:[\alpha,\beta]\rightarrow \mathbb{C} $. 
\end{itemize}
For a non-decreasing function  $ \varphi:[\alpha,\beta]\rightarrow[\alpha,\beta] $,  $ z=z(\varphi(t)),\alpha' \leq \tau \leq \beta' $ is \name{change of parameter}  of  $ z(t) $.

The change is \name{reversible } iff  $ \varphi  $ is strictly increasing. 

If  $ \gamma  $ is differentiable, then call  $ \gamma  $ a \name{curve}.

$ \gamma  $ is \subname{simple }{curve}, or a   \textbf{Jordan curve}\index{curve!Jordan curve}, if  $ \gamma $ is injective.

$ \gamma  $ is   \textbf{closed curve}\index{curve!closed curve} if  $ \gamma(0)=\gamma(1) $.  
\subsubsection{Analytic Functions in Regions}
A function  $ f  $ is analytic on an arbitrary set  $ A  $ if it is  the restriction to  $ A  $ of a function which is analytic in some open set containing  $ A  $. 
\begin{theorem}
    An analytic function in a region(\ie open and connected) $ \Omega $ whose derivative is 0 must reduce to a constant. The same hold if the real part, the imaginary part, the modulus, or the argument is constant.
\end{theorem}
\subsection{Conformal Mappings}
Suppose  $ f:\Omega\rightarrow \mathbb{C}$ is analytic in  $ \Omega $.  $ r_1=z_1(t),r_2=z_2(t) $, $ \alpha \leq t \leq \beta $.

 $ z_0=z_1(t_0)=z_2(t_0') ,z_1'(t_0)\not=0,z_2'(\hat{t_0})\not=0,\alpha<t_0,\hat{t_0}<\beta$.
 
  $ f'(z_0)\not=0, w_1(t)=f(z_1(t_0)),w_2=f(z_2(\hat(t_0))) $
  
   $ \Gamma_1=\{w_1(t)|\alpha \leq t \leq \beta\} $,  $ \Gamma_2=\{w_2(t)|\alpha \leq t \leq \beta\}$        
   
Then 
\begin{align*}
    w_1'(t)&=f'(z_1(t))z_1'(t)\\
    w_2'(t)&=f'(z_2(\hat{t}))z_2'(\hat{t})
\end{align*}
 $ \Rightarrow  $
 \begin{align*}
    w'_1(t_0)\not=0&,w_2'(t_0)\not=0\\
    \arg w_1'(t_0)&=\arg f'(z_1(t_0))z_1'(t_0)\\
    \arg w_2'(t_0)&=\arg f'(z_2(\hat{t_0}))z_2'(\hat{t_0})
 \end{align*} 
So the "angle"  $ \arg w_1'(t_0)-\arg w_2'(\hat{t_0}=\arg z_1(t_0)-\arg z_2(\hat{t_0}) $ remains the same. \\
Now we give the definition.
\begin{definition}
     $ w=f(z) $ is said to be \name{conformal} in  $ \Omega $ if  $ f  $ is analytic in  $ \Omega  $ and  $ f'(z)\not=0 $ for  $ \forall z\in \Omega$.    
\end{definition}
Easy to prove that linear change of scale at  $ z_0 $  is independent of the direction.\\
\ie  $ |f'(z_0)|=\lim\limits_{z\to z_0}\frac{\delta \sigma}{\delta s} $ 
\subsection{Length and Area}
The \name{length} of a differentiable arc  $ \gamma $ with the equation  $ z(t)=x(t)+iy(t) $, $ a \leq t \leq b $
\[L(\gamma)=\int_{a }^{b }\sqrt{(x'(t))^2+(y'(t))^2}\mathrm{d}t=\int_{a }^{b }|z'(t)|\mathrm{d}t\]   
For  $ \Gamma=f(\gamma) $ where  $ f  $ conformal mapping.\\
Then 
\[L(\Gamma)=\int_{a}^{b}|f'(z(t))|\cdot|z'(t)|\mathrm{d}t\]
The \name{area} of  $ E\subset \mathbb{R} $ is  $ A(E)=\int \int_{E}\mathrm{d}x\mathrm{d}y $\\
Then by the differentiable functional transformation, the area $ \hat{E}=f(E) $ is 
\[A(\hat{E})=\int \int_E|u_xv_y-u_yv_x|\mathrm{d}x\mathrm{d}y\]  
If  $ f  $ is the conformal mapping of an open set containing  $ E  $, then by Caucht-Riemann equation
\[A(\hat{E})=\int\int_E|f'(z)|^2\mathrm{d}x\mathrm{d}y\]
\section{M{\"o}bius Transformation}
Recall that a \name{M{\"o}bius transformation} is a function of the form
\[w=s(z)=\frac{az+b}{cz+d},\quad ad-bc\not=0\]
Then it has an inverse  $ z=S^{-1}(w)=\dfrac{dw-b}{-cw+a} $.\\
We may define  $ S(\infty)=\lim\limits_{z\to \infty}S(z)=\frac{a }{c} $, $ S(\frac{-d }{c})=\infty $ \\
With these definition,  $ S:\hat{\mathbb{C}}\rightarrow\hat{\mathbb{C}} $ is a topological mapping. Here one may use the chordal metric to define the topology.\\
\[S'(z)=\frac{ad-bc}{(cz+d)^2}\]
Then  $ S  $ is conformal in  $ \hat{\mathbb{C}}-\{-\frac{d }{c},\infty\} $.\\
$ w=z+\alpha  $ is called a \name{parallel translation}.

$ w=kz  $ with  $ |k|=1 $ is a \name{rotation}.

$ w=kz $ with  $ k>0  $ is a \name{homothetic transformation}.

$ x=\frac{1 }{z }  $ is called an \name{inversion}.
\begin{proposition}
    Every M{\"o}bius transformation is a composition of the above four operations.
\end{proposition}    
\subsection{Cross ratio}
For three distinct points  $ z_2,z_3,z_4\in\hat{\mathbb{C}} $,we can find a M{\"o}bius transformation  $ S $ such that  $ S(z_2)=0,S(z_3)=1,S(z_4)=\infty $.
\begin{lemma}
    The M{\"o}bius transformation satisfying the above conditions is unique.
\end{lemma}
The \name{cross ratio} $ (z_1,z_2,z_3,z_4) $ is the image  $ z_1 $ under the M{\"o}bius transformation which maps  $ z_2  $ to 1, $ z_3  $ to 0 and $ z_4  $ to  $ \infty $.
\begin{theorem}
    If  $ z_1,z_2,z_3,z_4\in \hat{\mathbb{C}} $  are distinct, and  $ T  $ is any M{\"o}bius transformation, then $ (Tz_1,Tz_2,Tz_3,Tz_4)=(z_1,z_2,z_3,z_4) $. 
\end{theorem} 
\begin{lemma}
    Let  $ T  $ be a M{\"o}bius transformation,  $ T(\mathbb{R}) $ is either a circle or a straight line.
\end{lemma}  
\begin{theorem}
    The cross ratio  $ (z_1,z_2,z_3,z_4) $ is real iff the four points lie on a circle or a straight line.
\end{theorem}
\begin{remark}
    One may prove the theorem by elementary geometry
\end{remark}

\begin{theorem}
    A M{\"o}bius transformation maps circles into circles.
\end{theorem}
\subsubsection{Symmetry}
Suppose  $ T  $ is a M{\"o}bius transformation which maps  $ \hat{\mathbb{R}}  $ onto a circle  $ C $. \\
We say that  $ w=Tz  $ and  $ w^*=T\bar{z} $ are \name{symmetric} w.r.t.  $ C $.
\begin{remark}
    This definition is independent of  $ T $. Suppose  $ S  $ is another M{\"o}bius transformation which maps  $ \hat{\mathbb{R}} $ onto  $ C $, then  $ S^{-1}T $ maps  $ \hat{\mathbb{R}} $ to  $ \hat{\mathbb{R}} $, and this  $ S^{-1}w=S^{-1}Tz $ and  $ S^{-1}w^*=S^{-1}T\bar{z} $ are conjugate.       
\end{remark}  
The points  $ z  $ and  $ z^* $ are \name{symmetric w.r.t C through  $ z_1,z_2,z_3 $} iff  $ (z^*,z_1,z_2,z_3)=\overline{(z,z_1,z_2,z_3)} $.

This can be another definition. 

Note that only the points on  $ C  $ are symmetric to themselves.

The mapping  $ z\mapsto z^* $ is 1-1 and is called \name{reflection} w.r.t. $ C $.

\paragraph{Geometric Meaning of Symmetry} 
\,

Case1:  $ C  $ is a straight line. We may assume  $ z_3=\infty $. 

$ z,z^* $ are symmetric w.r.t.  $ C $ if and only if 
\[\frac{z^*-z_2}{z_1-z_2}=\frac{\bar{z}-\bar{z_2}}{\bar{z_1}-\bar{z_2}}\]
Then 
\[|z^*-z_2|=|z-z_2|,\quad \forall z_2\in C\text{ and }z_2\neq\infty\]
\[\img\frac{z^*-z_2}{z_1-z_2}=\img\frac{\bar{z}-\bar{z_2}}{\bar{z_1}-\bar{z_2}}\]   
So  $ C  $ is the bisecting normal of the segment between  $ z  $ and  $ z^
 * $.\\
Case2: $ C  $ is the circle  $ |z-a|=R $.

Then for  $ \forall $ distinct  $ z_1,z_2,z_3\in\mathbb{C} $,  $ \overline{(z,z_1,z_2,z_3)}=\overline{(z-a,z_1-a,z_2-a,z_3-a)}\\=(\bar{z}-\bar{a},\bar{z_1}-\bar{a},\bar{z_2}-\bar{a},\bar{z_3}-\bar{a})=(\bar{z}-\bar{a},\dfrac{R^2}{z_1-a},\dfrac{R^2}{z_2-a},\dfrac{R^2}{z_3-a})=(\dfrac{R^2}{\bar{z}-\bar{a}},z_1-a,z_2-a,z_3-a)\\=(\dfrac{R^2 }{\bar{z}-\bar{a}},z_1,z_2,z_3) $.\\
Then the symmetric point of  $ z  $ w.r.t.  $ C  $ is 
\[z^*=\dfrac{R^2}{\bar{z}-\bar{a}}+a\]
or\[(z^*-a)(\bar{z}-\bar{a})=R^2\]
  $ \Rightarrow $ 
  \[\begin{cases}
    |z^*-a|\cdot|z-a|=R^2\\
    \dfrac{z^*-a}{z-a}=\dfrac{(z^*-a)(\bar{z}-\bar{a})}{|z-a|^2}>0
  \end{cases}\]    
  \begin{theorem}[The Symmetric principle]
    If a M{\"o}bius transformation maps a circle  $ C_1  $ onto a circle  $ C_2 $, then it transforms any pair of symmetric points w.r.t.  $ C_1 $ into a pair of symmetric points w.r.t.  $ C_2 $.   
  \end{theorem}
  \begin{proof}
    Case1:  $ C_1=\hat{\mathbb{R}} $. Let  $ T  $ be the M{\"o}bius transformation which maps  $ \hat{\mathbb{R}} $ onto   $ C_2 $.  $ \forall z\in \mathbb{C}$, by definition,  $ w=Tz  $ and  $ w^*=T\bar{z} $ are symmetric w.r.t.  $ C_2 $.\\
    Case2: $ C_1  $ is a general circle. Let  $ T:C_1\rightarrow C_2 $ and  $ S:\mathbb{R}\rightarrow C_2 $ be M{\"o}bius transformation.
    
    Suppose  $ w,w^* $ are symmetric w.r.t.  $ C_1 $. Then there exists  $ z $ \st  $ w=Sz,w^*=S\bar{z} $.\\
    Then we can find  $ Tw=TSz,Tw^*=TS\bar{z} $ are symmetric w.r.t.  $ C_2 $ since  $ TS:\hat{\mathbb{R}}\rightarrow C_2 $      
  \end{proof}
  \begin{remark}
    \begin{enumerate}
        \item The M{\"o}bius transformation from  $ C_1  $ to  $ C_2 $ satisfies  $ z_1\mapsto w,z_2\mapsto w_2,z_3\mapsto w_3 $  where  $ z_1,z_2,z_3\in C_1 $,  $ w_1,w_2,w_3\in C_2 $ is given by 
        \[(w,w_1,w_2,w_3)=(z,z_1,z_2,z_3,)\]  
        \item The M{\"o}bius transformation from  $ C_1 $ to  $ C_2 $ satisfies  $ z_1\mapsto w_1 $,  $ z_2\mapsto w_2 $ where  $ z_1\in C_1,z_2\not\in C_1 $,  $ w_1\in C_2,w_2\not\in C_2 $ is given by 
        \[(w,w_1,w_2,w_2^*)=(z,z_1,z_2,z_2^*)\]    
    \end{enumerate}
  \end{remark}
\subsubsection{Steiner Circles, circular net}
For  $ S(z)=\dps\frac{az+b}{cz+d} $,  $ S'(z)=\frac{ad-bc }{(cz+d)^2} $.

A point  $ z\not\in  $ a circle  $ C $ is said to on the \name{right}(\name{left}, \textbf{resp.}) of  $ C $ if  $ \img(z,z_1,z_2,z_3)>0 $($ \img(z,z_1,z_2,z_3)<0 $)

\begin{remark}
    \,
    \begin{enumerate}
        \item This agrees with everyday use since  $ (i,1,0,\infty)=i $ 
        \item This distinct between left and right is the same for all triples, while the meaning may be reversed.
        
        (If  $ C=\hat{\mathbb{R}} $, then  $ (z,z_1,z_2,z_3)=\dps\frac{az+b}{cz+d} $ with  $ a,b,c,d\in\mathbb{R} \Rightarrow \img(z,z_1,z_2,z_3)=\dps\frac{ad-bc}{|cz+d|^2}\img(z)$)
        \item We can define an absolute positive orientation of all finite circles by requiring that  $ \infty $ should be lie to the right of the oriented circles. 
    \end{enumerate}
    Consider a M{\"o}bius transformation of the form 
    \[w=k\cdot\frac{z-a}{z-b}\]
    Here,  $ z=a\mapsto w=0,z=b\mapsto w=\infty $.
    
    Then circles through  $ a,b $ maps to straight line through  $ 0,\infty $.
    
    The concentric circle about the origin, $ |w|=\rho $, correspond to circles with the equation 
    \[\dps\left|\frac{z-a}{z-b}\right|=\frac{\rho}{|k|}\]
    These are the circles of \name{Apollonius} with limit points  $ a $ and  $ b $. 
\end{remark}
Denote by  $ C_1 $ the circles through  $ a,b $ and  $ C_2 $ the circles of Apollonius with these limit points. The configuration formed by all the circles  $ C_1 $ and  $ C_2 $ is called the \name{Steiner circles}(or \name{circular net})
\begin{theorem}
    \,
    \begin{enumerate}
        \item[(a)] There is exactly one  $ C_1 $ and one  $ C_2 $ through each point in  $ \hat{\mathbb{C}}\backslash\{a,b\} $
        \item[(b)] Every  $ C_1 $  meets every  $ C_2 $ under right angle.
        \item[(c)] Reflection in a  $ C_1 $ transforms every  $ C_2 $ into itself and every  $ C_1 $ into another  $ C_1 $.
        \item[(d)] The limit points  $ a,b $ are symmetric w.r.t. each  $ C_2 $, but not w.r.t. other circles.     
    \end{enumerate}
\end{theorem}   
\begin{proof}
    If the limit points are  $ 0,\infty $, those properties are trivial in the  $ w $-plane. The general case follows since all properties are invariant under M{\"o}bius transformations.
\end{proof}
\section{Elementary Conformal mapping}
\begin{example}
    $ w=z^\alpha $ where  $ \alpha>0 $.
    
    Let  $ S(u_1,u_2) $ with  $ 0<\varphi_2-\varphi_1 \leq 2\pi $ be  $ \{z\in \mathbb{C}:z\not=0,\varphi_1<\arg(z)<\varphi_2\} $ where  $ \arg(z) $ can be chosen as any value of it.   

    Then  $ S(\varphi_1,\varphi_2) $ is a region.
    
    In this region, a unique value of  $ w=z^\alpha $ is defined by  $ \arg w=\alpha\arg z $.
    
    This function is analytic with  $ \dps\frac{\mathrm{d}w}{\mathrm{d}z}=\alpha\dps\frac{w}{z} $.
    
    This function is  $ 1-1 $ only if  $ \alpha(\varphi_2-\varphi_1) \leq 2\pi $.  
\end{example}
\begin{example}
    $ w=e^z  $  maps  $ \{z\in \mathbb{C}:\dps-\frac{\pi }{2}<\img(z)<\frac{\pi}{2}\} $ onto  $ \{w\in\mathbb{C}:\Real( w)>0\} $ 
\end{example}
\begin{example}
    $ w=\dps\frac{z-1}{z+1} $ maps  $ \{z\in \mathbb{C}:\Real(z)>0\} $ onto  $ \{ww\in \mathbb{C}:|w|<1\} $   
\end{example}
\begin{example}
    \begin{equation}
        \mathbb{C}\backslash[-1,1]\xrightarrow{z_1=\frac{z+1}{z-1}}\mathbb{C}\backslash(-\infty,0]\xrightarrow{z_2=\sqrt{z_1}}\{\Real(z_2)>0\}\xrightarrow{w=\frac{z_2-1}{z_2+1}}\{w\in\mathbb{C}:|w|<1\}
    \end{equation}
\end{example}
\subsection{Elementary Riemann surfaces}
\begin{example}
    $ w=z^n $,  $ n\in \mathbb{Z}_+ $ and  $ n>1 $.
    
    There is a 1-1 correspondence between each angle  $ \dps\frac{(k-1)2\pi}{n}<\arg z<\frac{k\cdot 2\pi}{n},k=1,2,\cdots,n $ and while  $ w $-plane except for the positive real axis.  
\end{example}
\begin{example}
    $ w=e^z $. This function maps each parallel strip $ (k-1)2\pi<\img z<k\cdot 2\pi, k\in \mathbb{Z} $ onto a sheet with a  cut along the positive axis.
\end{example}
\section{Complex Integration}
\subsection{Fundamental Theorems}
\subsubsection{Line integral and rectifiable arcs}
Let  $ f(t)=u(t)+iv(t) $ be a complex-valued defined on  $ t\in[a,b]\subset\mathbb{R} $ where $ u,v $ are real-valued functions.
If  $ f $ is continuous on  $ [a,b] $, we may define the \name{integral}
\[\int_a^bf(t)\mathrm{d}t:=\int_a^bu(t)\mathrm{d}t+i\int_a^bv(t)\mathrm{d}t\]   


Let  $ \gamma $ br a piecewise differential arc in  $ \mathbb{C} $ with the equation  $ z=z(t),a \leq t \leq b $. If  $ f  $ is continuous on  $ \gamma $, then  $ f(z(t)) $ is conitnuous on  $ [a,b] $, and we define
\begin{equation}
    \int_\gamma f(z)\mathrm{d}z=\int_a^bf(z(t))z'(t)\mathrm{d}t\label{integration of complex curve}
\end{equation}  
The integral defined in \ref{integration of complex curve} is independent of the parametrization of  $ \gamma $. Suppose that anther parrametrization of  $ \gamma $ is  $ \gamma:(\alpha,\beta)\rightarrow \mathbb{C} ,\tau\mapsto z(t(\tau))$, where  $ t:(\alpha,\beta)\rightarrow (a,b),\tau\mapsto t(\tau) $ is piecewise differentiable. Then we have
\begin{equation}
    \int_a^b f(z(t))z'(t)\mathrm{d}t=\int_\alpha^\beta f(z(t(\tau)))z'(t(\tau))t'(\tau)\mathrm{d}t=\int_\alpha^\beta f(z(t(\tau)))\frac{\mathrm{d}z(t(\tau))}{\mathrm{d}\tau}\mathrm{d}\tau
\end{equation}
\newline
For an arc  $ \gamma $ with equation  $ z=z(t),a \leq t \leq b $, we define  $ -\gamma $ by  $ z=z(-t),-b \leq t \leq a $.

Then we have 
\begin{align*}
    \int_{-\gamma}f(z)\mathrm{d}z&=\int_{-b}^{-a}f(z(-t))\dps\frac{\mathrm{d}z(-t)}{\mathrm{d}t}\mathrm{d}t\\
    &=-\int_{-a}^{-b}f(z(-t))z'(-t)\mathrm{d}t\\
    &=-\int_{a}^bf(z(\tau))z'(\tau)\mathrm{d}\tau\\
    &=-\int_\gamma f(z)\mathrm{d}z
\end{align*}
So we have those properties:
\begin{proposition}
    \,
    \begin{enumerate}
        \item[(a)]  $ \dps\int_{-\gamma}f(z)\mathrm{d}z=-\int_\gamma\mathrm{d}z $
        \item[(b)] Let  $ f $ and  $ g $ be two continuous functions on the piecewise differentiable arc  $ \gamma $, then 
        \[\int_\gamma(\lambda_1f+\lambda_2g)\mathrm{d}z=\lambda_1\int_\gamma f\mathrm{d}z+\lambda_2\int_\gamma g\mathrm{d}z,\forall \lambda_1,\lambda_2\in\mathbb{C}\]      
        \item[(c)] If  $ \gamma $ can be subdivided into two pieces differentiable arcs  $ \gamma_1 $ and  $ \gamma_2 $, and  $ f $ is continuous on  $ \gamma_1 $ , then
        \[\int_\gamma f\mathrm{d}z=\int_{\gamma_1} f\mathrm{d}z+\int_{\gamma_2} f\mathrm{d}z\]
        \item[(d)]  $ (c) $ implies that the integral of a closed curve doesn't depend on the starting point on the curve 
    \end{enumerate}
\end{proposition}
\begin{example}
    Evaluate  $ \dps\int_\gamma\frac{1}{z-a}\mathrm{d}z $ where  $ \gamma $ is the circle centered at  $ a\in\mathbb{C} $ with radius  $ R $. 
    
    Let  $ z=z(t)=a+Re^{it} $. Then the integral is  $ 2\pi i $  
\end{example}
\subsubsection{The fundamental theorem of Calculus for integrals in  $ \mathbb{C} $}
The line integral w.r.t.  \subname{$ \bar{z} $}{integral} is defined by 
\[\int_\gamma f(z)\overline{\mathrm{d}{z}}=\overline{\int_\gamma \overline{f(z)}\mathrm{d}z} \] 
With this notation, line integrals w.r.t.  $ x=\Real(z) $ and  $ y=\Imag(z) $ can be defined by 
\[\int_\gamma f(z)\mathrm{d}x=\dps\frac{1}{2}[\int_\gamma f(z)\mathrm{d}z+\int_\gamma f(z)\overline{\mathrm{d}z}]\]
\[\int_\gamma f(z)\mathrm{d}y=\dps\frac{1}{2 i}[\int_\gamma f(z)\mathrm{d}z-\int_\gamma f(z)\overline{\mathrm{d}z}]\]
if we write  $ f(z)=\mu+i\nu $, we have 
\[\int_\gamma f(z)\mathrm{d}z=\int_\gamma f(z)\mathrm{d}x+i\int_\gamma f(z)\mathrm{d}y=\int_\gamma(\mu\mathrm{d}x-\nu\mathrm{d}y)+i\int_\gamma(\nu\mathrm{d}x+\mu\mathrm{d}y)\]   
\begin{remark}
    It is followed by the intuition. We can view the integration as the multiplication between  $ f $ and  $ \mathrm{d}z $. 
\end{remark}

The integral w.r.t. \subname{arc length}{integral} is defined by 
\[\int_\gamma f(z)|\mathrm{d}z|=\int_a^bf(z(t))|z'(t)|\mathrm{d}t\]
This integral is again independent of the parametrization. It is easy to check 
\[\int_{-\gamma}f(z)|\mathrm{d}z|=\int_\gamma f(z)|\mathrm{d}z|\]
Now we define \name{length} of a curve  $ \gamma $: $ L(\gamma)=\int_\gamma |\mathrm{d}z| $ 

We have the inequality: 
\[\dps\left|\int_\gamma f\mathrm{d}z\right| \leq \int_\gamma |f| \cdot|\mathrm{d}z| \leq L(\gamma)\cdot \sup\limits_{z \leq \gamma}|f(z)|\]
The length of an arc  $ \gamma $ ($ z=z(t) $) can also be defined as the least upper bound of all sums 
\[\dps\sum_{i=1}^n |z(t_i)-z(t_{i-1})|\]
where  $ a=t_0<t_1<\cdots<t_n=b $ 
If this least upper bound is finite, we say that the arc is \name{rectifiable}

It is easy to show that piecewise differentiable arcs are rectifiable.

The integral of a continuous function  $ f $ on a rectifiable arc may be defined as 
\[\dps\int_\gamma f(z)\mathrm{d}z=\lim \sum_{k=1}^nf(z(\psi_k))[z(t_k)-z(t_{k-1})]\] 
\begin{theorem}
    Let  $ \gamma\subset \mathbb{C} $ be a region, and  $ P,Q $ two (possibly complex-valued) functions that are continuous on  $ \Omega $,  $ \gamma $ closed curve. The integral  $ \int_\gamma p(x,y)\mathrm{d}x+Q(x,y)\mathrm{d}y $ depends only on the end point of  $ \gamma $ iff there exists a function   $ U(x,y) $ on  $ \Omega $ with  $ \dps\frac{\partial U}{\partial x}=P,\frac{\partial U}{\partial y}=Q $.  
\end{theorem}
\begin{proof}
    "$ \Leftarrow $": If such a  $ U $ exists, then 
    \[\int_\gamma P\mathrm{d}x+Q\mathrm{d}y=\int_\gamma\frac{\partial U}{\partial x}\mathrm{d}x+\frac{\partial U}{\partial y}\mathrm{d}y=\int_\gamma\frac{\mathrm{d}U}{\mathrm{d}t}\mathrm{d}t=U(\gamma(b))-U(\gamma(a))\] 
    "$ \Rightarrow $": Fix a point  $ (x_0,y_0)\in \Omega $. We define  $ U(x,y)=\int_\gamma P\mathrm{d}x+Q\mathrm{d}y$ where  $ \gamma $ is any curve between  $ (x_0,y_0) $ and  $ (x,y) $. Easy to check that it is true.     
\end{proof}
\begin{theorem}[Fundamental theorem of Calculus for integrals on  $ \mathbb{C} $]
    Let  $ f $ be continuous on a region  $ \Omega $  containing  $ \gamma $.  $ \int_\gamma f\mathrm{d}z $ depends on the endpoints iff  $ f $ is the derivative of an analytic function  $ F $ in  $ \Omega $.     
\end{theorem}
\begin{remark}
    We will prove  $ \dps\int_\gamma f\mathrm{d}z=F(\omega_2)-F(\omega_1) $ where  $ \gamma $ begins at  $ \onemga_1 $ and ends at  $ \omega_2 $.    
\end{remark}
\begin{proof}
    Transform  the line integration into  the composition of two real integration.
\end{proof}
\begin{corollary}
    If  $ F $ is analytic on  $ \Omega $ with  $ F'=f $, and  $ \gamma $ is a closed curve in  $\Omega $, then  $ \int_\gamma f\mathrm{d}z=0 $. Conversely if  $ f $  is continuous on  $ \Omega $ and  $ \int_\gamma f\mathrm{d}z=0 $ for any closed curve in  $ \Omega $, then  $ f $ is the derivartive of an analytic function  $ F $ in  $\Omega $.          
\end{corollary}



\subsubsection{Cauchy's theorem for a rectangle}
There is some notes in this section:

 $ R $ is the rectangle in  $ \mathbb{C} $,  $ R=\{x+iy\in\mathbb{C}:a \leq x \leq b,c \leq y \leq d\} $. And  $ \partial R $ is boundary curve oriented in the counterclockwise direction.
 \begin{theorem}[Cauch's theorem for a rectangle]\label{Cauchy's theorem for a rectangle}
    If  $ f $ is analytic on an open set which contains  $ R $, then  $ \dps\int_{\partial R}f(z)\mathrm{d}z=0 $   
\end{theorem}
\begin{proof}
    For  $ \forall $ rectangle  $ \tilde{R}  $ inside  $ R $, we define  $ {Z}(\tilde{R})=\dps\int_{\partial \tilde{R}}f(z)\mathrm{d}z $. Then  $ Z(R)=Z(R_1)+Z(R_2) $ if  $ R $ is divided into  $ Z_1,Z_2 $.  
    
    Since we can divide  $ R $ into four equal rectangles, and find a rectangle with  $ |{Z}(R^{(1)})| \geq \frac{1}{4}|{Z}(R)| $.
    Then repeat the above steps and we obtain a sequence of nested rectangles
     $ R\supset R^{(1)}\supset\cdots $ with the property 
   \begin{equation}
        {Z}(R^{(n)}) \geq \frac{1}{4}|{Z}(R^{(n-1)})| \geq \cdots \geq \frac{1}{4^{n}}{Z}(R)\label{eq:2}
   \end{equation}
      $ \forall \delta>0 $,  $ \exists n\in \mathbb{N} $ \st  $ R^{(n)}\subset\{z\in\mathbb{C}:|z-z_0|<\delta\},\forall n \geq N $, where  $ z_0 $ is the limit of  $ R^{(n)} $ as  $ n\rightarrow \infty $.
      
       $ f $ is analytic  in  $ R $ $ \Rightarrow $  $ \forall \epsilon $,  $ \exists \delta>0$ \st 
      \begin{equation}
        \left|\dps\frac{f(z)-f(z_0)}{z-z_0}-f'(z_0)\right|<\epsilon,\forall z\text{ with }|z-z_0|<\delta\label{eq:1}
      \end{equation}
       We assume that  $ \delta $ satisfies both conditions. We have 
       \[{Z}(R^{(n)})=\int_{\partial R^{(n)}}f(z)\mathrm{d}z=\int_{\partial R^{(n)}}[f(z)-f(z_0)-(z-z_0)f'(z_0)]\mathrm{d}z\]
       \[\Rightarrow |{Z}(R^{(n)})| \leq \epsilon \int_{\partial R^{(n)}}|z-z_0|\mathrm{d}z \text{ by \ref{eq:1}} \]   
       Let  $ d_n $ be the length of diagonal of  $ R^{(n)} $,  $ L_n $ be the length of its perimeter. Then  $ |z-z_0| \leq d_n,\,\forall z\in \partial R^{(n)} $.  
       
       $ \Rightarrow |{Z}(R^{(n)})| \leq \epsilon d_nL_n=\epsilon \dps\frac{D}{2^n}\cdot\frac{L}{2^n} $ where  $ D $,  $ L $ are the diameter and perimeter of  $ R $.
       
        $ \Rightarrow  $  $ |{Z(R)}| \overset{\ref{eq:2}}{ \leq }4^n|{Z}(R^{(n)})| \leq \epsilon DL \Rightarrow Z(R)=0 $ since  $ \epsilon $ is arbitary.  
\end{proof}    
 We will next prove the following stronger theorem:
\begin{theorem}[stronger version of Cauchy's theorem for a rectangle]\label{stronger version of Cauchy's theorem for a rectangle}
    Let  $ f $ be analytic on  $ R'=R\backslash\{\psi_1,\cdots,\psi_m\}, m\in \mathbb{N} $. If  $ \dps\lim\limits_{z\to\psi_j}(z-\psi_j)f(z)=0,\forall 1 \leq j \leq m $, then  $ \dps\int_{\partial R}f(z)\mathrm{d}z=0 $.    
\end{theorem}
\begin{proof}
    WLOG, we may assume  $ f $ is not analytic at only one point  $ \psi\in R $. If we put  $ psi  $ into a small rectangle $ S_0 $, then the previous theorem tells us  $ \int_{
        \partial R
    }f(z)\mathrm{d}z=\int_{\partial S_0}f(z)\mathrm{d}z $.
    
     $ \forall \epsilon>0 $, we may choose  $ S_0 $ small enough such that  $ |f(z)| \leq \dps\frac{\epsilon}{|z-\epsilon},\,\forall z\in\partial S_0 $
     
      $\dps \Rightarrow|\int_{\partial R}f(z)\mathrm{d}z \leq \epsilon \int_{\partial S_0}\frac{|\mathrm{d}z|}{|z-\psi|} \leq \epsilon \frac{1}{\frac{l}{2}}\cdot 4l=8\epsilon $ 
      
       $ \Rightarrow $ $ \int_{\partial R}f(z)\mathrm{d}z=0 $ since  $ \epsilon $ is arbitrary.   
\end{proof}
\subsubsection{Cauchy's Theorem for a disk}
$ \Delta:=\{z\in \mathbb{C}:|z-z_0|<R\} $ where  $ R>0 $.
\begin{theorem}[Cauchy's Theorem for a disk]\label{Cauchy's theorem for a disk}
   If  $ f $ is analytic in an open disk  $ \Delta $, then  $ \int_\gamma f(z)\mathrm{d}z=0 $  for closed curve  $ \gamma $ in  $ \Delta $.  
\end{theorem}  
\begin{proof}
   Suppose the center of  $ \Delta $ is  $ z_0=x_0+iy_0 $,  $ z=x+iy $. We define 
   \[F(z)=\int_\gamma f(z)\mathrm{d}z\]
   where  $ \gamma $ is the horizontal line segment from  $ z_0 $ to  $ (x,y_0) $ added with vertical line segment from  $ (x,y_0) $ to  $ z $. We have
    
\begin{equation}
      \frac{\partial F}{\partial y}=\lim_{\delta y\to 0}\frac{F(x,y+\delta y)-F(x,y)}{\delta y}=\lim_{\delta y\to 0}\frac{1}{\delta y}\int_{\delta\gamma}f(z)\mathrm{d}z=if(z)
\end{equation}
   By Cauchy' theorem on rectangles, one has  $ F(z)=-\int_{\tilde{\gamma}}f(z)\mathrm{d}z $, where  $ \tilde{\gamma} $ is the vertical line segment from  $ z_0 $ to  $ (x_0,y) $ added with horizontal line segment from  $ (x_0,y) $ to  $ z $.
   
   Similarly,  $ \dps\frac{\partial F}{\partial x}=f(z) $.
   
   $ \Rightarrow \frac{\partial F}{\partial x}=-i\frac{\partial F}{\partial y} $ $ \Rightarrow  $ $ F  $   is analytic in  $ \Delta $  with derivative  $ f $.
   By Fundamental Theorem \ref{FTC for integrals} of Calulus $ \Rightarrow  $  $ \int_\gamma f(z)\mathrm{d}z=0 $ for  $ \forall  $ closed curve in  $ \Delta $.    
\end{proof}
Here is a stronger version.
\begin{theorem}[stronger version of Cauchy's Theorem for a disk]\label{stronger version of Cauchy's theorem for a disk}
    Let  $ f $ be analytic in a region  $ \Delta'=\Delta\backslash\{\psi_1,\cdots,\psi_m\} $ with  $ m\in \mathbb{N} $. If  $ f $ satisfies  $ \dps\lim\limits_{z\to \psi_j}(z-\psi_j)f(z)=0,\forall 1\leq j\leq m $, then  $ \dps\int_\gamma f(z)\mathrm{d}z=0,\forall \gamma $ closed in  $ \Delta'  $     
\end{theorem}
\begin{proof}
    It is similar to the above proof.
    
    For the case no  $ \psi_j $ lies on  $ x=x_0 $ and  $ y=y_0 $, we can find a similar curve $ \gamma $ with last segment is a vertical one. Let  $ F(z)=\int_\gamma f(z)\mathrm{d}z $. And continue the process of proof of the previous theorem.  
    
    For the case that  $ \exists $  $ \psi_j $ lies on the lines  $ x=x_0,y=y_0 $, we actually can move the center to another point \st no  $ \psi_j $ lies on the lines  $ x=x_0',y=y_0'$.   
\end{proof}
\subsection{Cauchy's integral formula}
\subsubsection{Index of a point with resect to a closed curve}
\begin{lemma}
    If the piecewise differentiable closed curve  $ \gamma $ does not pass through  $ z\in \mathbb{C} $, then the value of the integral  $ \int_\gamma \frac{\mathrm{d}\zeta}{\zeta -z}$  is a multiple of  $ 2\pi i $. 
\end{lemma}
\begin{proof}
     $ \gamma:\zeta=\zeta(t),\alpha \leq t \leq \beta $.  $ h(t)=\int_\alpha^t\frac{\zeta'(x)}{\zeta(s)-z}\mathrm{d}s $.
     
      $ z\in \gamma $  $ \Rightarrow  $  $ h $ is defined and continuous on  $ [\alpha,\beta] $. For all  $ t $ \st  $ \zeta'(t) $ is continuous, we have 
      \[h'(t)=\frac{\zeta'(t)}{\zeta(t)-z}\Rightarrow \frac{\mathrm{d}}{\mathrm{d}t}\left[e^{-h(t)}(\zeta(t)-z)\right]=0\]
      So  $ \dps e^{-h(t)}(\zeta(t)-z) $ is constant on  $ [\alpha,\beta] $.
      
      Then  $ e^{h(t)}=\dps\frac{\zeta(t)-z}{\zeta(\alpha)-z} $ $ \Rightarrow  $ $ e^{h(\beta)}=1 $ $ \Rightarrow  $ $ h(\beta)\in \{2k\pi i:k\in \mathbb{Z}\} $.
\end{proof}
 
 
  

\printindex
\listoftheorems[ignoreall, show={theorem,proposition}]
\end{document}