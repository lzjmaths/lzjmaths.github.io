%! TEX root = lecture/Complex_Analysis

\subsection{Power Series}
\subsubsection{Power series}
\begin{theorem}[Abel's theorem]
If  $ \sum a_n  $ converges, then  $ f(z)=\sum a_nz^n\rightarrow f(1)  $ as  $ z\rightarrow 1  $ in such a way that  $ \dps\frac{|1-z| }{1-|z| } $ remains bounded.
\end{theorem}
\subsection{Exponential, Trigonometric and Logorithmic Functions}
\subsubsection{Exponential and Trigonometric function}
The \name{exponential function } is defined as the solution if the differential equation
\[\left\{
    \begin{aligned}
        &f'(z)=f(z)\\
        &f(0)=1
    \end{aligned}
\right.\]
We denote  $ e^z=\exp z=\sum\limits_{n=0}^\infty \frac{z^n}{n!} $.

The \name{trigonometric function} are defined by 
\[\cos z=\frac{e^{iz}+e^{-iz}}{2}\qquad \sin z=\frac{e^{iz}-e^{-iz}}{2i}\]
\subsubsection{Logorithmic Functions}
The \name{logorithmic function}  $ \ln  $ is defined by  $ z=\ln w  $ is a root of the equation  $ e^z=w $.\\
For  $ w\neq0 $, we write  $ z=x+iy $, then 
\[e^{x+iy}=w\Leftrightarrow 
\left\{
    \begin{aligned}
        e^x=|w|\\
        e^{iy}=\frac{w }{|w|}
    \end{aligned}
\right.\]   
The first equation has a unique solution  $ x=\ln|w|  $.

The second equation  $ e^{iy}=\dfrac{w }{|w|}  $ has a unique solution  $ y_0\in [0,2\pi) $.

If we write  $ w=re^{i\theta} $, then  $ x=\ln w,y=\theta =\arg w $.

Thus, for  $ w\neq0  $, we have 
\[\ln w=\ln|w|+i\arg w\]
The function  $ \ln  $ is actually not single-valued. But we can define a single-valued function  $ Ln $ 

We define 
\[a^b=\exp(b\ln a)\]   
We will prove  $ Ln  $ is analytic in  $ \mathbb{C}-(-\infty,0] $ but not continuous in  $ (-\infty,0] $.

$ Ln  $ is the principal branch of the logithm.   