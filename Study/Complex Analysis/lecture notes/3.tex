\section{Conformal Mappings}
\subsection{Basic topology}
\subsubsection{Connectedness}
\begin{theorem}
    A nonempty open set in  $ \mathbb{C} $ is connected iff any two of its points can be joined by a polygon  which lies in the set,\ie Connectedness is equivalent to Path Connectedness
\end{theorem}
An nonempty connected subset is called a \name{region}
\subsubsection{Compactness}
\begin{definition}
    A set  $ X  $ is \name{totally bounded } if  $ \forall \epsilon>0  $,  $ X  $ can be covered by finitely many balls of radius  $ \epsilon  $ 
    
\end{definition}
\begin{theorem}
    A set is compact iff it is complete and totally bounded.
\end{theorem}
\begin{theorem}
    A subset  $ X\subset $ is compact iff every infinite sequence of  $ X  $ has a limit point in  $ X  $.
\end{theorem}
\subsubsection{Continuous Functions}
\begin{theorem}
    Continous function maps connected space to connected space.
\end{theorem}
\begin{theorem}
    Continous function maps compact space to compact space.
\end{theorem}
\subsection{Conformality, geometric consequences of the existence of a derivative}
\subsubsection{Arcs and closed curves}
The equation of an \name{arc} r in  $ \mathbb{C} $ can  be represented by one of the terms
\begin{itemize}
    \item  $ x=x(t),y=y(t) $, $ \alpha \leq t  \leq \beta$,  $ x,y $ are continuous at  $ t $
    \item  $ z(t)=x(t)+iy(t) $, $ \alpha \leq t \leq \beta $.
    \item The  continuous mapping  $ \gamma:[\alpha,\beta]\rightarrow \mathbb{C} $. 
\end{itemize}
For a non-decreasing function  $ \varphi:[\alpha,\beta]\rightarrow[\alpha,\beta] $,  $ z=z(\varphi(t)),\alpha' \leq \tau \leq \beta' $ is \name{change of parameter}  of  $ z(t) $.

The change is \name{reversible } iff  $ \varphi  $ is strictly increasing. 

If  $ \gamma  $ is differentiable, then call  $ \gamma  $ a \name{curve}.

$ \gamma  $ is \subname{simple }{curve}, or a   \textbf{Jordan curve}\index{curve!Jordan curve}, if  $ \gamma $ is injective.

$ \gamma  $ is   \textbf{closed curve}\index{curve!closed curve} if  $ \gamma(0)=\gamma(1) $.  
\subsubsection{Analytic Functions in Regions}
A function  $ f  $ is analytic on an arbitrary set  $ A  $ if it is  the restriction to  $ A  $ of a function which is analytic in some open set containing  $ A  $. 