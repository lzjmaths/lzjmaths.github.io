\documentclass[a4paper,10pt,twoside]{article}
\usepackage{geometry}
\geometry{a4paper,scale=0.85}
\usepackage{amsthm}
\usepackage{amssymb}
\usepackage{enumerate}
\usepackage{amsmath}
\usepackage{extarrows}
\usepackage{mathrsfs}
\usepackage[UseMSWordMultipleLineSpacing,MSWordLineSpacingMultiple=1.4]{zhlineskip}
\usepackage{lipsum}
\usepackage{ctex}
\title{解析几何}
\author{lin150117}
\date{}
\newtheorem{definition}{定义}[subsection]
\newtheorem{theorem}{定理}[subsection]
\newtheorem{example}{例子}[section]
\newtheorem{remark}{评论}
\newtheorem{corollary}{推论}
\newtheorem{proposition}{性质}
\begin{document}
\maketitle
\tableofcontents
\newpage
\setlength{\parindent}{0pt}
\section{椭圆的性质}
\subsection{基础性质}
\begin{proposition}
    已知:椭圆 $ \dfrac{x^2}{a^2}+\dfrac{y^2}{b^2}=1 $ 的左右焦点为 $ F_1,F_2 $, $ P(x_p,y_p) $ 为椭圆上一点, $ \angle F_1PF_2=\theta $ 
    \begin{enumerate}[ $ (1) $ ]
        \item  $ \angle F_1PF_2 $ 外角平分线的方程为\_
        \item 过 $ F_2 $ 作 $ \angle F_1PF_2 $ 外角平分线的垂线,其垂足的轨迹方程为\_
        \item 以焦半径 $ PF_2 $ 为直径的圆,与以椭圆长轴为直径的圆\_(位置关系)
    \end{enumerate} 
    设 $ \angle F_1PF_2 $ 的角平分线 $ PM $ 与 $ x  $ 轴交于点 $ M $, $ \angle F_1PF_2 $ 外角平分线 $ PN $ 与 $ x  $ 轴交于点 $ N $
    \begin{enumerate}
        \item[ $ (4) $ ]  $ x_M=\_,\,x_N=\_,|OM|\cdot|ON|=\_ $
        \item[ $ (5) $ ] 设 $ \triangle PF_1F_2 $ 的内心为 $ I  $, 则 $ \dfrac{MI}{IP}=\_ $\\点 $ I $ 的坐标为(用 $ x_P,y_P  $ 表示)\_\\点 $ I  $ 的轨迹方程为\_
        \item[ $ (6) $ ]  $ PF_2 $ 外旁心的轨迹方程为\_
        \item[ $ (7) $ ]  $ \triangle F_1PF_2 $ 的面积为\_(用 $ a,b,\theta $ 表示)
        \item[ $ (8) $ ]  $ |OP|^2+|PF_1|\cdot|PF_2|=\_ $    
    \end{enumerate}   
\end{proposition}
\begin{proposition}
     $ P(x_0,y_0) $ 为椭圆外一点,过 $ P  $ 作切线切于 $ AB $,则
     \begin{enumerate}[ $ (1) $ ]
        \item  $ AB  $ 的方程为\_
        \item 设 $ AB  $ 的中点为 $ M  $,则 $ OMP  $ 三点\_
        \item 若 $ OP  $ 交椭圆于 $ Q  $,则 $ \dfrac{x_Q^2}{x_Px_M}=\_ $
        \item 若 $ \angle APB =90^{\circ} $,则 $ P  $ 点的轨迹方程为\_
        \item 若 $ F_1,F_2  $ 为左右焦点,则 $ \angle AF_1P\_\_\angle BF_1P $   
     \end{enumerate} 
\end{proposition}
\begin{proposition}
     $ F  $ 为椭圆 $ \dfrac{x^2}{a^2}+\dfrac{y^2}{b^2}=1  $ 的右焦点, $ l  $ 为右准线, $ AB  $ 为过 $ F  $ 的弦, $ \angle AFx=\theta  $ 
     \begin{enumerate}[ $ (1) $ ]
        \item  $ |AF|=\_\_,\,|BF|=\_\_,\,|AB|=\_\_ $
        \item  $ \dfrac{1}{|AF|}+\dfrac{1}{|BF|}=\_ $
        \item 若 $ l  $ 上存在点 $ C  $,使得 $ \triangle ABC  $ 为正三角形,则离心率 $ e $ 的范围 \_
        \item 求 $ AB  $ 中点 $ M  $ 的轨迹方程\_
        \item 设 $ AA_1\perp l,BB_1\perp l  $,若 $ AB_1  $ 交 $ x  $ 轴于点 $ N  $,则 $ N  $ 的坐标为\_  
     \end{enumerate}
     设 $ A,B  $ 处椭圆的切线分别为 $ l_A,l_B  $
     \begin{enumerate}
        \item[ $ (6 ) $ ] 设 $ l_A,l_B  $ 交于点 $ D  $,则 $ D  $ 的轨迹方程为\_
        \item[ $ (7 ) $ ]  $ \angle AFD\_90^{\circ },\quad \angle ADB\_ 90^{} $  
     \end{enumerate}
\end{proposition}
\begin{proposition}
    设椭圆 $ \dfrac{x^2}{a^2}+\dfrac{y^2}{b^2}=1 $ 右焦点为 $ F  $,右准线为 $ l  $ 
    \begin{enumerate}[ $ (1) $ ]
        \item 弦 $ AB  $ 延长交 $ l  $ 于点 $ C  $,  $ \angle AFB $ 的外角平分线交 $ l  $ 于点 $ D  $,则 $ y_C\_y_D $
        \item 椭圆上点 $ E  $ 的切线与 $ l  $ 交于点 $ E  $,则 $ \angle EFE_1=\_  $
        \item  $ A  $ 为椭圆上一点,过 $ F  $ 的弦交椭圆于 $ MN $, $ AM $ 和 $ AN  $分别交右准线于 $ P,Q $, 则 $ \angle PFQ=\_ $
        \item 过 $ l  $ 上一点 $ P  $ 作椭圆的两条切线 $ PA,PB $,则直线AB经过定点\_
        \item 设 $ AB,CD  $ 为过焦点的两动弦, $ AC,BD  $ 交于点 $ M   $,则 $ M  $ 的轨迹方程为\_    
    \end{enumerate}
\end{proposition}
\begin{proposition}
     $ P(x_0,y_0) $ 为椭圆$ \dfrac{x^2}{a^2}+\dfrac{y^2}{b^2}=1  $ 内定点, $ x_0\in(0,a) $, $ Q $ 在椭圆上,  $ F  $ 为右焦点,则
     \begin{align*}
        (c|PQ|+a|QF|)_{min}&=\_\_\\
        |PQ|+|QF|&\in\_\_
     \end{align*} 
\end{proposition}
\begin{proposition}
    \begin{enumerate}[ $ (1) $ ]
        \item  $ M  $ 为 $ x=x_M  $ 上一动点,  $ A_1,A_2  $ 为左右顶点, $ A_1M,A_2M  $ 交椭圆于 $ S,T  $, $ ST $ 交 $ x  $ 轴于  $ N  $,则 $ x_M\cdot x_N=\_ $
        \item 过 $ P(x_P,0) $ 作直线交$ \dfrac{x^2}{a^2}+\dfrac{y^2}{b^2}=1 $ 于 $ AB  $,椭圆左右顶点为 $ M,N  $,  $ AM,BN  $ 交于点 $ R  $,则 $ x_P\cdot x_R=\_  $
        \item       过 $ P(x_P,0) $ 作直线交$ \dfrac{x^2}{a^2}+\dfrac{y^2}{b^2}=1 $ 于 $ AB  $, $ B  $ 关于 $ x  $ 轴的对称点为 $ B_1 $, $ AB_1  $ 交 $ x  $ 轴于 $ Q  $,则 $ x_P\cdot x_Q=\_  $
        \item 过 $ P(x_P,0) $ 作两条直线交$ \dfrac{x^2}{a^2}+\dfrac{y^2}{b^2}=1 $ 于 $ AB,CD  $,直线 $ AC,BD  $ 交于点 $ S  $,则 $ x_P\cdot x_S=\_ $  
    \end{enumerate}
\end{proposition}
\begin{remark}
    可以看见,这里的 $ Q,R,S $ 均在同一条直线 $ x=\dfrac{a^2}{x_P} $ 上。看看它的特殊情况是什么!
\end{remark}
\begin{proposition}
     $ P  $ 为椭圆$ \dfrac{x^2}{a^2}+\dfrac{y^2}{b^2}=1 $ 上一点, $ A,B  $ 为左右顶点, $ F_1,F_2  $ 为左右焦点, $ PF_1,PF_2  $ 交椭圆于 $ M,N  $,则
     \begin{enumerate}[ $ (1) $ ]
        \item  $ k_{AP }\cdot k_{PB}=\_ $ \quad (椭圆第三定义)
        \item  $ k_{AP}\cdot k_{AM }=\_ $
        \item 若 $ AM,BN  $ 交于 $ Q  $,则 $ Q  $ 点的方程为\_  
     \end{enumerate}
\end{proposition}
\begin{proposition}
    椭圆$ \dfrac{x^2}{a^2}+\dfrac{y^2}{b^2}=1 $ 上两点 $ A,B  $ 满足: $ OA\perp OB  $
    \begin{enumerate}
        \item[ $ (1) $ ]  $ \dfrac{1}{|OA|^2}+\dfrac{1}{|OB|^2}=\_ $
        \item[ $ (2) $ ] 设 $ OM\perp AB  $ 于点 $ M  $,则 $ M  $ 的轨迹方程为\_  
    \end{enumerate} 
\end{proposition}
\subsection{补充}
\begin{example}
    设椭圆$ \dfrac{x^2}{a^2}+\dfrac{y^2}{b^2}=1 $有两点距离为 $ m(\dfrac{2b^2}{a}<m<2a) $,试求他们的中点到 $ y  $ 轴的距离最大值 
\end{example}
\begin{example}
    椭圆上的点 $ A  $ 作两条倾斜角互补的直线与椭圆交于 $ BC  $,设过 $ A  $ 的切线为 $ l  $,设直线 $ l  $ 与直线 $ BC  $ 的斜率分别为 $ k_l  $ 和 $ k_{BC}  $,
    求证: $ k_l=-k_{BC } $ 
\end{example}
\begin{example}
    过椭圆内一点 $ A  $ 作两条倾斜角互补的直线,与椭圆交于 $ BC,DE  $,设直线 $ BD  $ 和直线 $ CE  $的斜率分别为 $ k_{BD} ,k_{CE}$,求证: $ k_{BD}+k_{CE}=0 $   
\end{example}
\begin{example}
    设 $ A,B  $ 为椭圆$ \dfrac{x^2}{a^2}+\dfrac{y^2}{b^2}=1 $的左右顶点。点 $ M (x_0,y_0 ) $ 为 $ x  $ 轴上方椭圆上一点,已知 $ \angle AMB =\gamma  $,求三角形 $ AMB  $ 的面积
\end{example}
\section{双曲线的性质}
\subsection{基础性质}
\begin{proposition}
    已知:双曲线 $ \dfrac{x^2}{a^2}-\dfrac{y^2}{b^2}=1 $ 的左右焦点为 $ F_1,F_2 $, $ P(x_0,y_0) $ 为椭圆上一点, $ \angle F_1PF_2=\theta $ 
    \begin{enumerate}[ $ (1) $ ]
        \item  $ |PF_1|=\_\_,\,|PF_2|=\_\_ $(用 $ x_0,y_0  $ 表示) 
        \item  $ \angle F_1PF_2 $ 角平分线的方程为\_
        \item 过 $ F_1 $ 作 $ \angle F_1PF_2 $ 角平分线的垂线,其垂足的轨迹方程为\_
        \item 以焦半径 $ PF_2 $ 为直径的圆,与以实轴为直径的圆\_(位置关系)
    \end{enumerate} 
    设 $ \angle F_1PF_2 $ 的角平分线 $ PM $ 与 $ x  $ 轴交于点 $ M $, $ \angle F_1PF_2 $ 外角平分线 $ PN $ 与 $ x  $ 轴交于点 $ N $
    \begin{enumerate}
        \item[ $ (4) $ ]  $ x_M=\_,\,x_N=\_,|OM|\cdot|ON|=\_ $
        \item[ $ (5) $ ] 设 $ \triangle PF_1F_2 $ 的内心为 $ I  $, 则点 $ I  $ 的轨迹方程为\_
        \item[ $ (6) $ ]  $ \triangle F_1PF_2 $ 中$ PF_2 $ 所对外旁心的轨迹方程为\_
        \item[ $ (7) $ ]  $ \triangle F_1PF_2 $ 的面积为\_(用 $ a,b,\theta $ 表示)
        \item[ $ (8) $ ]  $ |PF_1|\cdot|PF_2|-|OP|^2=\_ $    
    \end{enumerate}   
\end{proposition}
\begin{remark}
    双曲线和椭圆性质类似,对比一下即知。角平分线由于有到两边距离相同的性质以及内切圆的边长性质,和椭圆和双曲线的第一定义有一定联系,故有如上性质。
\end{remark}
\begin{proposition}
    $ P(x_0,y_0) $ 为双曲线外一点,过 $ P  $ 作切线切于 $ AB $,则
    \begin{enumerate}[ $ (1) $ ]
       \item  $ AB  $ 的方程为\_
       \item 设 $ AB  $ 的中点为 $ M  $,则 $ OMP  $ 三点\_
       \item 若 $ \angle APB =90^{\circ} $,则 $ P  $ 点的轨迹方程为\_
       \item 若 $ F_1,F_2  $ 为左右焦点,则 $ \angle AF_1P\_\_\angle BF_1P $   
       \item 设双曲线左右顶点为 $ M,N  $, $ AM,BN  $ 相交于点 $ Q  $,则 $ x_Q =\_ $ 
    \end{enumerate} 
\end{proposition}
\begin{proposition}
    $ F  $ 为双曲线 $ \dfrac{x^2}{a^2}-\dfrac{y^2}{b^2}=1  $ 的右焦点, $ l  $ 为右准线, $ AB  $ 为过 $ F  $ 的弦, 倾斜角为$ \theta  $ 
    \begin{enumerate}[ $ (1) $ ]
       \item  $ |AF|=\_\_,\,|BF|=\_\_,\,|AB|=\_\_ $
       \item   $ A,B  $ 在双曲线同一支上时, $ \theta  $ 的范围\_\\ $ A,B  $ 在双曲线不同支时, $ \theta  $ 的范围\_
       \item 求 $ AB  $ 中点 $ M  $ 的轨迹方程\_
       \item 设 $ AA_1\perp l,BB_1\perp l  $,若 $ AB_1  $ 交 $ x  $ 轴于点 $ N  $,则 $ N  $ 的坐标为\_  
    \end{enumerate}
    设 $ A,B  $ 处双曲线的切线分别为 $ l_A,l_B  $
    \begin{enumerate}
       \item[ $ (6 ) $ ] 设 $ l_A,l_B  $ 交于点 $ D  $,则 $ D  $ 的轨迹方程为\_, $ \angle AFD\_90^{\circ }$  
       \item[ $ (7 ) $ ] 设左右顶点分别为 $ C,D  $,则 $ k_{AC}\cdot k_{AD}=\_,\,k_{DA}\cdot k_{DB}= $ 
    \end{enumerate}
\end{proposition}
\begin{proposition}
    设双曲线 $ \dfrac{x^2}{a^2}-\dfrac{y^2}{b^2}=1  $ 的渐近线为 $ l_1,l_2 $,右焦点为 $ F  $ 
    \begin{enumerate}[ $ (1) $ ]
        \item 过 $ F  $ 作 $ FH\perp l_1  $,则 $ x_H =\_\_ \,y_H=\_\_ $\\若 $ FH  $ 与左右两支各有一个交点 $ B,A  $,则 $ e\in\_\_ $\\若 $ \vec{BH}=\lambda \vec{HA } $,则 $ e=\_\_ $ 
        \item 直线 $ l  $ 与双曲线, $ l_1,l_2 $ 的交点依次为 $ A,B,C,D  $,则 $ \dfrac{|AD|}{|BC|}=\_\_  $    
    \end{enumerate}
\end{proposition}
\begin{proposition}
    设双曲线 $ \dfrac{x^2}{a^2}-\dfrac{y^2}{b^2}=1  $ 的右焦点为 $ F  $,右准线为 $ l  $
    \begin{enumerate}[ $ (1) $ ]
        \item 直线 $ AB  $ 交双曲线的右支分别于 $ A,B  $,交 $ l  $ 于  $ M  $,  $ \angle BAF =\alpha ,\angle ABF=\beta  $,则 $ \angle AMF=\_ $
        \item 直线 $ AB  $ 交双曲线的左右两支分别于 $ A,B  $,交 $ l  $ 于 $ M  $, $ \angle BAF =\alpha ,\angle ABF=\beta$,则 $ \angle AMF=\_ $
        \item 以 $ F  $ 为圆心做半圆交左右两支分别于 $ A,B  $,设 $ C  $ 为双曲线左顶点,记 $ \angle AFC=\alpha,\angle BFC=\beta $,则 $ \cos \beta -\cos \alpha =\_\_ $  
    \end{enumerate} 
\end{proposition}
\section{抛物线的性质}
\begin{proposition}
    抛物线 $ y^2=2px(p>0) $ 的焦点为 $ F  $,准线为 $ l  $,  $ A(x_1,y_1),B(x_2,y_2) $ 是抛物线上的点 $ AA_1\perp l  $ 于 $ A_1 $, $ \angle A_1AF,\angle B_1BF $ 的平分线分别为 $ l_a,l_b $
    \begin{enumerate}[ $ (1) $ ]
        \item 求 $ l_a $ 的方程\_
        \item 设 $ l_a  $ 与 $ y  $ 轴交于点 $ M  $,则 $ y_M =\_\_  $, $ A_1,M,F  $ 三点\_\_,以焦半径为直径的圆与 $ y  $ 轴\_\_
        \item  $ AB  $ 的方程为 $ \_\_  $(用 $ y_1,y_2 $ 表示)它与 $ l_a  $ 方程有什么联系\_\_
        \item 设 $ l_a,l_b  $ 交于点 $ C  $,则 $ x_C=\_\_,\,y_C=\_\_  $\\设 $ C' $ 在抛物线上,过 $ C'  $ 的切线与 $ AB  $ 平行,则 $ y_{C'}=\_\_ $\\设 $ C''  $ 在抛物线上, $ y_A<y_C<y_B  $,当 $ C'' $ 到 $ AB  $ 距离最大时, $ y_{C''}=\_\_ $  
    \end{enumerate}     
\end{proposition}
\begin{proposition}
    抛物线 $ y^2=2px(p>0) $ 的焦点为 $ F  $,准线为 $ l  $,过 $ F  $ 的直线交抛物线于 $ A(x_1,y_1),B(x_2,y_2) $, $ AB $ 倾斜角为 $ \theta  $ 
    \begin{enumerate}[ $ (1) $ ]
        \item  $ x_1x_2=\_\_,\,y_1y_2=\_\_,\,|AB|=\_\_ $(用 $ \theta  $ 表示) $ | A_1OB $三点\_\_
        \item  $ \angle A_1FB_1=\_\_ $\\设 $ M_1  $ 为 $ A_1B_1 $中点,则 $ \angle AM_1B =\_\_$\\以 $ AB  $ 为直径的圆与 $ l  $ 相切
        \item  $ l  $ 与 $ x  $ 轴交于 $ F_1  $,则 $ \angle AF_1F\_\_\angle BF_1F $
        \item 若 $ S_{\triangle AA_1F}\cdot S_{\triangle BB_1F}=\lambda S_{\triangle AF_1B}^2 $,则 $ \lambda=\_\_ $
        \item  $ M_1F^2=k|AF|\cdot|BF| $, $ k=\_\_ $
        \item  $ AB  $ 中点 $ M  $ 的轨迹方程为\_\_          
    \end{enumerate} 
\end{proposition}
\begin{remark}
    抛物线在保有类似双曲线和椭圆的一些性质外还独有一些性质。这是因为抛物线计算上相对简单,因此许多量都可以直接表示,这里建议把直接可以表示的量尽量记住,抛物线的性质是最有可能用到的。
\end{remark}
\begin{proposition}
     $ P  $为抛物线 $ y^2=2px(p>0) $ 外一点,过点 $ P  $ 作切线切于 $ A(x_1,y_1),B(x_2,y_2) $ 
     \begin{enumerate}[ $ (1) $ ]
        \item  $ x_P=\_\_ ,\,y_P=\_\_ $(用 $ A,B  $ 坐标表示)
        \item 设 $ P  $ 坐标为 $ (x_0,y_0)  $,则 $ A,B  $ 方程为\_\_ \\当 $ P  $ 在准线上时, $ AB  $ 经过定点\_\_
        \item  $ |PF|^2\_\_|AF|\cdot|BF| $ 
        \item  $ \angle AFP\_\_\angle BFP $  
     \end{enumerate}
\end{proposition}
\begin{proposition}
    过抛物线 $ y^2=2px(p>0) $焦点 $ F  $ 作 $ l_1\perp l_2 $ 交抛物线于 $ A,B,C,D  $
    \begin{enumerate}
        \item 若 $ AB,CD  $的中点为 $ M,N  $,则 $ MN  $ 经过定点\_\_ 
        \item   若以 $ AB,CD $ 为直径的圆的公共线中点为 $ Q  $,则
         $ Q  $ 的轨迹方程为\_\_  
    \end{enumerate} 
\end{proposition}
\begin{proposition}
    设抛物线 $ y^2=2px(p>0) $ 顶点为 $ O  $,弦 $ AB  $ 与 $ x  $ 轴交于点 $ M $
    \begin{enumerate}
        \item 若 $ AO\perp BO  $, $ x_M=\_\_ $\\过 $ O  $作弦 $ AB   $ 的垂线,垂足为 $ H  $,则 $ H  $ 的轨迹方程为\_\_ 
        \item 若 $ M  $ 为定点 $ (m,0)  $, $ k_{OA}\cdot k_{OB}=\_\_  $ \\设 $ N(-m,0)  $ 是 $ M  $ 关于原点的对称点,则 $ \angle ANM\_\_\angle BNM $\\过 $ N  $ 的两条倾斜角互补的直线与抛物线分别交于 $ A_1,A_2,B_1,B_2 $,则 $ A_1B_2  $ 必经过定点\_\_ 
        \item 设 $ P(x_0,y_0 ) $ 为抛物线上一点,且  $ PA\perp PB  $,则 $ AB  $ 必经过定点 $ \_\_  $ 
        \item 抛物线上动点 $ Q  $满足: $ QA\perp QB,
        |QA|=|QB| $,则 $ S_{\triangle QBA } $ 的最小值为\_\_    
    \end{enumerate} 

\end{proposition}
\begin{proposition}
     $ \triangle ABC  $ 顶点在抛物线 $ y^2=2px(p>0) $上
     \begin{enumerate}
        \item 若 $ \triangle ABC  $ 为等边三角形,则重心的轨迹方程为\_\_
        \item 抛物线在这三点的切线围成的三角形的垂心的横坐标\_\_ 
     \end{enumerate}
\end{proposition}
\section{解析几何基本解法}
限于笔者水平,若有不当或情况未列出请指正!\\
解析几何关键要看思路的流畅性,即计算对象的不断变化,我们在以下题型中重点放在计算顺序上。通常的计算顺序都是两边凑,我们要时刻对“什么东西已经可以表示“保持警惕。建议把计算顺序在写完题后特别是不会的总结一下,非常推荐,这让你能理解更深一点解析几何的计算逻辑
\paragraph{定点定直线问题}总体而言,我们要去思考如何证明一条直线过顶点或一个点一定在一条直线上,以下是一些解决方式。笔者希望读者能从中理解“两边凑的思想”。

(一)直接猜:考虑一下 $ x  $ 轴和 $ y  $ 上的点,平行于 $  x,y $ 轴的直线是否符合条件
\begin{example}
    已知椭圆$ C:\dfrac{x^2}{4}+\dfrac{y^2}{2}=1 $,过点 $ P (0,1 ) $ 的动直线 $ l  $ 与椭圆交于 $ A,B  $两点,点 $ B  $ 关于 $ y  $ 轴的对称点为点 $ D  $ (不同于点 $ A $ ),证明:直线 $ AD  $ 恒过顶点,并求出该点坐标。 
\end{example}
\begin{remark}
    对称永远是数学的主题。一方面解析几何的一些对称性质大多还是需要一定经验的,但另一方面由 $ x  $ 轴开始,由 $ x/y  $ 轴结束是非常自然的,因为从坐标轴出发的往往计算上会简单一点,以及我们有很多表示法,过 $ x  $ 轴上一点的直线是表示上是很简单的。

    这里, $ P  $ 在 $ y  $ 轴上,根据对称性,我们让 $ l  $ 分别为两条对称的直线,那么定点只能是两个 $ AD  $ 相交而得,所以一定在 $ y  $ 轴上。
\end{remark}

(二)慎重考虑设出的对象,不一定是问题中第一条引出的直线,也可以是问题中问的直线
\begin{example}
    已知椭圆的标准方程为 $ \dfrac{x ^2 }{4}+\dfrac{3y^2}{4}=1 $。其左顶点为点 $ A  $,动点 $ B,C  $ 在椭圆上,且 $ AB\perp AC  $,求证:直线 $ BC  $ 恒过一定点。 
\end{example}
\begin{example}
    已知抛物线 $ x^2=4y $ 上一点 $ N (2,1) $。动点 $ A,B  $ 在椭圆上,且 $ NB\perp NA  $,求证:直线 $ AB  $ 恒过一定点。
\end{example}
\begin{example}
    已知椭圆的标准方程为 $ \dfrac{x ^2 }{4}+y^2=1 $。点 $ P(0,1) $,设直线 $ l  $ 不经过 $ P  $ 点不垂直于 $ x  $ 轴,且与 $ C  $ 相交于 $ A,B  $ 两点。若直线 $ PA,PB  $ 的斜率的和为 $ -1  $,求证:直线 $ l $ 过定点。
\end{example}
\begin{example}
    从双曲线 $ C:\dfrac{x^2}{4}-y^2=1 $ 上一点 $ A(4,\sqrt{3}) $ 引两条直线 $ AM,AN  $,分别交 $ C  $ 于 $ M,N  $,且 $ AM,AN  $ 的斜率为 $ k_1,k_2 $,满足 $ k_1\cdot k_2=1 $,判断直线 $ MN  $ 是否过定点。    
\end{example}
\begin{remark}
    这里就是解析几何思想的非常关键的东西:设出的对象是什么。我们肯定要确保未知字母的个数尽量不超过3,然而我们不能死脑筋,一定要以全题审视的视角看看把哪一个点,哪一条直线用未知数表示出来。当然,经验上谈一般就是设开头引出的直线或者最后问的直线。我们这道题就是设出最后的一条直线,这在此类问题中非常常见。其原因是:一条直线 $ y=mx+n  $ 过顶点就是在问 $ m,n  $ 的关系是什么,那么我们设出这条直线,去对照什么样的 $ m,n  $ 可以符合条件中做的一系列点。
\end{remark}
(三)发现特点,利用几何性质
\begin{example}
    设 $ F_1,F_2  $ 分别为椭圆 $ \dfrac{x^2}{a^2}+\dfrac{y^2 }{1-a^2 }=1 $ 的左右焦点,点 $ P  $ 为椭圆 $ E  $ 上的第一象限内的点,直线 $ F_2P  $ 交 $ y  $ 轴于点 $ Q  $,并且 $ PF_1\perp F_1Q  $。证明:当 $ a  $ 变化时,点 $ P  $ 在某定直线上
\end{example}
\begin{example}
    已知椭圆 $ C:\dfrac{x^2 }{18 }+\dfrac{y^2}{9}=1 $,过点 $ M(0,-1) $ 的动直线 $ l  $ 与椭圆C交于 $ A,B  $ 两点。试判断 $ AB  $ 直线的圆是否恒过定点,并说明理由。  
\end{example}
\begin{remark}
    思考从哪里入手,入手点和思路都是很重要的需要积累的经验。
\end{remark}
\paragraph{定值问题}
类似上述方法,定点问题和定值问题几乎是一样的,关键是对我们如何去设参数的考量。下面介绍一下点差法的运用。
\begin{example}
    已知椭圆$ C:9x^2+y^2=m^2(m>0) $,直线 $ l  $ 不过原点 $ O  $且不平行于坐标轴, $ l  $ 与 $ C  $ 有两个交点 $ A,B  $,线段 $ AB  $ 的中点为 $ M  $,证明:直线 $ OM  $的斜率与直线 $ l  $ 的斜率的乘积为定值。
\end{example}
\begin{example}[2020浙江21]
    已知椭圆 $ C_1:\dfrac{x^2}{2}+y^2=1 $,抛物线 $ C_2:y^2=2px(p>0) $,点 $ A  $是椭圆 $ C_1 $ 与抛物线 $ C_2  $ 的交点,过点 $ A  $ 的直线 $ l $ 交椭圆 $ C_1  $ 于点 $ B  $,交抛物线 $ C_2 $ 于点 $ M $( $ B,M  $ 不同于 $ A 1 $ )若存在 $ l  $ 使 $ M  $ 为线段 $ AB  $ 的中点,求 $ p  $ 的最大值。   
\end{example}
\begin{remark}
    这里有一种全新的思想:设点。我们可以称之为点参。事实上,回顾一下我们之前的思路,无非就是设出一条直线
     $ y=kx+b $ (但这条直线需要仔细琢磨)然后再设点 $ (x_1,y_1),(x_2,y_2) $ 用韦达定理建立起 $ x_1,x_2 $ 与 $ k,b  $ 的联系,之后再回答问题。这称之为线参。其实就是把参数的重点放在点上还是线上,即用点的坐标表示线还是用线的方程表示点(当然线的方程不能直接表示点,但不妨我们将韦达定理作为一种表示形式。不得不说,由于美妙的对称性,一般不太可能出现超出韦达定理的情况)
    \\点参和线参需要按照实际情况来使用,也需要结合一定的逻辑。例如本题中,很明显直线的作用仅仅只是把两点表示出来,而其实先点出两点再连这条线也是没有任何问题的,因此完全可以将线隐去,利用点满足的方程进行一定的操作。这同时也体现出一个思想:曲线上的点适合曲线的方程,且两者是等价关系。那么我们只要对两点的方程进行对照,消消乐即可(笔者还没有想出超出相减的做法)
    \\回到点差法的思想,基本上用于弦上的定比分点(多为中点)通过构建两条方程乘以系数作差以消去某些项。因此,点差法的关键是能不能消去一些没有用的,如常数项。
\end{remark}
\subsection{反思与总结}
\paragraph{注意事项}

一、一些特殊情况,如斜率不存在,仅有唯一交点等。

二、注意左支右支导致的多种情况,焦点位置(可能方程不是标准方程而在 $ y  $ 轴)引来结果多样

三、多注意对称性以及“地位”上的平等(例如一条直线交曲线两个点,那这两个点地位就是平等的)

\paragraph{基本手法}
点差法、平移坐标轴(对仅有斜率、边比的问题能简便运算,关键是选取原点使点利于计算)、齐次化(化简得到 $ f(x,y)=1 $ 后齐次化解出 $ \dfrac{y}{x } $,一般在平移后用)、参数方程(相当于三角换元,能将一些结构变得简单)

\paragraph{处理思想}

一、换角度看问题:如何用简洁好用的算式表达一个条件,这需要我们多元的看待一个对象。

二、合理安排计算顺序:从哪里计算很重要,是点参还是线参,用点表示线还是用线通过韦达定理表达交点。做题不能死板,既可以用已知量去表示一个量,也可以设出这个量列得有关方程。

三、倒推思想:先设最后得到的反推初始状态,可以列得方程。

四、方程思想:在原本方程式上进行代数变形,如点差法

\paragraph{小结}
解析几何是计算的艺术,第一点的误区是不能把解析几何看成非常多的技巧的组合,至少以上的内容涉及了几乎所有“在考场上想不到的人方法”,因此,熟练的计算代数式是要练就的功底。当然经验也很重要,你需要知道什么样的东西大概是能解出来的。如果做多了,很容易发现,有根号的东西往往不以加法形式存在,往往会保留到最后几步自然消去。三次及以上的方程几乎都能因式分解,一般的式子就是形如 $ x-y $ 的对称式或者一次式。

限于笔者水平,一些方法还没有面面俱到,但笔者认为数学思想比方法更为重要,关注重点,找到联系,从熟悉的地方出发逐个攻破就是很好的思想。正如我所说,想不到的方法我已经列在上面了,剩下的关键在能不能发现。一定的积累是非常必要的,笔者推荐有一本笔记本,专门记你觉得的好题,但关键不是记录题目,而是记录你的思考,读者可以在题下面写关于入手点,思想方法,思维逻辑,数学对象之间的联系,关注点,重点等等,或许笔者上面对一些题的评语会对你有帮助。笔者一直认为,每个人的思维不同,因此关注点不同会导致差距,这也就是不同人擅长不同东西的原因。但是我们要汲取别人以及答案的思维方式,重点要关注他们是怎么审视这道题的。具体的操作就有读者自行斟酌了。

任何东西都可以记,为的其实是让自己记住。人类忽视了很多很多的细节,我们要做的就是重新审视,从旧的中获得新的。这也是我们汲取经验的方式。
\end{document}