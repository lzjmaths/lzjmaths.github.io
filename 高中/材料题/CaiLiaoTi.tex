\documentclass[a4paper,10pt]{article}
\usepackage{amsthm}
\usepackage{amssymb}
\usepackage{enumerate}
\usepackage{amsmath}
\usepackage{extarrows}
\usepackage{mathrsfs}
\usepackage[UseMSWordMultipleLineSpacing,MSWordLineSpacingMultiple=1.4]{zhlineskip}
\usepackage{lipsum}
\usepackage{hyperref}
\usepackage{ctex}
\usepackage{geometry}
\geometry{top=2.54cm,bottom=2.54cm,left=3.18cm,right=3.18cm}
\hypersetup{
    colorlinks=true, % 设置链接颜色
    linkcolor=blue, % 设置普通链接颜色
    citecolor=green, % 设置引用链接颜色
    urlcolor=red % 设置URL颜色
}
\title{数学中的问题介绍}
\author{lin150117}
\date{}
\newtheorem{definition}{定义}[section]
\newtheorem{theorem}{定理}[section]
\newtheorem{example}{例子}
\newtheorem{remark}{评论}
\newtheorem{corollary}{推论}
\newtheorem{proposition}{性质}
\newtheorem{lemma}{引理}
\begin{document}
\setlength{\parindent}{0pt}
\maketitle
\tableofcontents
\newpage
\section{拓扑学导论}
\subsection{引言}
定义 $ D^2=\{(x,y)\in\mathbb{R}^2|x^2+y^2<1\} $ 为单位圆盘。 $ S^1=\{(x,y)\in\mathbb{R}^2|x^2+y^2=1\} $ 为单位圆盘的边界。 
\begin{definition}
    对 $ X,Y  $ 为 $ \mathbb{R}^2 $ 的子集。我们考虑映射$ f:X\rightarrow Y $,如果 $ f  $ 满足:\\ $ \forall \,\varepsilon>0,x\in 
    X,\exists\, \delta>0  $ 使得  $ \forall \,y\in Y,|x-y|<\delta  $ 我们有 $  |f(x)-f(y)|<\varepsilon $ 一定成立。\\我们称 $ f  $ 是\textbf{连续}的。
\end{definition}
\begin{remark}
    连续这里采用了 $ \varepsilon-\delta  $ 语言定义,读者可能理解上有一定困难,通俗地讲,当 $ y $ 无限接近于 $ x $ 时, $ f(y) $ 也会无限接近于 $ f(x) $。举个例子, $ f(x)=e^x  $ 就是连续的,如果你画图的话,这个现象就更直观了。 
\end{remark}
我们假定我们已经证明了定理:
\begin{theorem}[二维空间的零点存在定理]\label{定理1}
    对一个连续函数 $ f:D^2\rightarrow \mathbb{R}^2  $, 如果对所有 $ |x|=1 $, 存在 $ t>0,f(x)=tx $. 则存在 $ x\in D^2 $ 使得 $ f(x)=0 $. 
\end{theorem}


\begin{figure}[htb]
    \centering
    \includegraphics[width=0.4\textwidth]{拓扑学-课本.jpg}
    \caption{定理\ref{定理1}中的f}
    \label{定理中的f}
\end{figure}

这个定理其实很好理解,如果我们学过拓扑学的一些知识就会知道,若0不在 $ f  $ 的值域中,那么 $ 0 $ 周围会有一个“洞”,“洞内”的点均不在值域中。但是我们不可能将一个圆盘通过捏泥巴(连续的直观解释)的方式让它里面有洞,所以不可能零点不存在。如果我们相信这个定理,那么我们有如下推论:
\begin{theorem}[Brouwer fixed point theorem]
    任何连续函数 $ f:D^2\rightarrow D^2  $ 都存在不动点。也就是说, $ \exists x\in D^2 $ 使得 $ f(x)=0 $ 
\end{theorem}
\begin{remark}
    我们可以简单的理解这个定理,它其实的意思是:如果我们搅拌一个咖啡杯,我们假定咖啡杯表面的液体分子不会向下走。那么无论我们怎么搅拌,咖啡杯表面总是有一个点与最初的位置相同,这是因为我们搅拌的过程总是连续的。\\
    事实上,这个定理可以拓展到三维领域,因此,我们不需要做任何假设就可以得出一个数学上绝对正确的结论:如果我们在玩一个榨汁机,无论我们怎么在榨汁机里搅拌,榨汁机里总是有一个点与最初的位置相同。这就是拓扑学研究的事情。
\end{remark}
\begin{proof}
    假设 $ f  $ 没有不动点,那么对于任意的 $ x\in D^2 $,  $ f(x)  $ 与 $ x  $ 不是同一个点,因此我们可以画出从 $ f(x) $ 到 $ x  $ 的射线 $ l_x $。设 $ l_x  $ 与单位圆边界 $ S^1 $ 交于点 $ g(x) $。这样我们就构造出了一个函数 $ g:D^2\rightarrow D^2  $, $  x\mapsto g(x) $。\\
    容易验证这几点要求:
    \begin{enumerate}
        \item $ g(x) $ 是连续的。
        \item 对 $ x\in S^1 $, $ g(x) $ 满足 $ g(x)=x $.
        \item  $ g(x)\in S^1 $,故不存在 $ g(x)=0 $.  
    \end{enumerate}  
    但是由定理\ref{定理1}, $ g  $ 满足存在 $ x $,  $ g(x)=0 $。这与上面的第三点矛盾!\\
    这样我们就证明了问题。   
\end{proof}
\begin{figure}[htb]
    \centering
    \includegraphics[width=0.4\textwidth]{Brouwer.jpg}
    \caption{$ g(x) $的构造}
    \label{Brouwer定理}
\end{figure}
\begin{remark}
    这个证明非常的漂亮,特别是运用了反证法这一事情,居然有利用反证法证明存在性的方法!尽管这个定理的证明由Brouwer给出,但是他在后期并不承认这个证明。这是数学中非常著名的\textbf{直觉主义}学派,他们宣称,数学的对象只能通过直接证明而不能通过反证法。因此存在性证明必须是构造性的。这从这个定理中也可以理解:你证明了不存在不动点是有问题的,但却完全不能描述它的大概位置。那么凭什么认为这个无中生有的不动点存在呢。
\end{remark}

笔者介绍了一个拓扑学中的经典例子和直觉主义学派作为本节的引言部分,目的是想让读者们了解数学的概念是不断的抽象的过程,例如我们可以用连续函数去描述搅拌咖啡机的起始点到终点。拓扑学就是这样一个学科,本质上是研究多维空间下的“捏娃娃”或者说“搅拌”对于一个对象中的点的影响。图就是一个拓扑对象。\\以及,有关数学的哲学和思想也是不断的与时俱进的。以往的数学家通常都是以“直觉”作为证明的基础,这一思想对现代数学家也有很深远的影响,往往数学家认为idea重要而详细而繁琐的证明不重要。然而我们不得不承认的是,现代数学必须建立在一个“欧几里得”式的严格的公理体系之下,但这个体系每个人往往都有不同的看法,例如直觉主义的观点。\\
有的人也会认为选择公理(我们可以从无穷个非空集合中分别选择出一个元素组成一个集合)是不成立的。这听上去很奇怪,但似乎你也没有一个理由去解释它,特别是当罗素提出罗素悖论后,人们对集合的态度往往很谨慎。\\
我们唯一能做的,就是严谨的研究每一个证明过程,这样认同或不认同的人都能找到适合自己的一个定理。而严谨体现在方方面面,例如“连续”的定义,我们必须对“搅拌”或者“捏娃娃”或者说”连续“的东西的概念作出数学上的定义。而这样的定义也是帮助我们理解这个对象的,例如Brower定理就是这样一个例子,只有严格的定义我们才能更自然地去研究。\\
引言到这里基本结束了,接下来我会引导读者研究数学的方方面面,以材料的形式探讨数学家的思维方式。
\begin{figure}[htb]
    \centering
    \includegraphics[width=0.5\textwidth]{麒麟.jpg}
    \caption{三维空间中一个外表面封闭的形状内部不一定与外界不连通}
    \label{麒麟图}
\end{figure}
\newpage
\section{微分几何导论}
\subsection{引言}
我们看一个有趣的定理
\begin{theorem}[毛球定理]
    如果我们有一只猫,猫表面每一个点都有一根毛。那么我们不可能将所有毛“捋顺”且全部贴紧表皮。
\end{theorem}
这是一个很生活化的命题,然而有一个非常重要的问题是:“捋顺”和“贴紧”的定义是什么?\\
这就需要数学的描述。一个很重要的思想是将“毛”视作vector,即向量。那么贴紧表皮的意思就是与猫的表面相切。
当然这个定义仍不完善,我们仍然面临着什么是相切的问题。当然这个事情在数学科普中没必要讲,大家都有对于相切的直观理解。\\
下一个问题则是,什么是“捋顺”,这个也好直观理解,就是没有“断层”,读者可以理解为,每根毛的顶端必须连成一个光滑的平面。\\
于是这个问题可以变成这样的数学语言:
\begin{theorem}[毛球定理]
    对于一个三维空间中的一个球,不存在一个没有地方为零(nowhere vanishing)的光滑切向量场(smooth tangent vector field)。
\end{theorem}
这里光滑切向量场的意思就是球上每一个点都有一个切向量(tangent vector), 这些切向量总体来看是光滑的。\\
但是依然存在一个问题,凭什么一只猫可以等价于一个毛球?这件事情并不平凡。
从上一节的连续变换的拓扑学内容我们大概可以推测:一只猫和一个球是同胚的,即存在一个连续双射,其逆也连续。
然而连续变化未必保持“光滑结构”不变,例如,我们可以想象一个正方形和一个圆是同胚的,但是从直观上来看,正方形不是光滑的(有棱角),而圆是光滑的。
这要求我们创造新的结构:\\
事实上,我们其实只要在连续映射的定义下加以修正就行:
\begin{definition}[光滑映射]
    我们说一个函数 $ f:\mathbb{R}^n\rightarrow \mathbb{R}^m $ 是\textbf{光滑映射}, 如果它无限阶可导。 
\end{definition}
多元函数的导数我们按下不表,但是可导函数必然连续,因此我们确实是在连续函数上加以修正。值得注意的是,我们将“光滑”表述为导数,这本身就是一种进步,是数学表述战胜自然描述的胜利。
\begin{definition}
    称空间 $ X,Y $ 是\textbf{微分同胚},如果存在双射 $ f:X\rightarrow Y $ 使得 $ f,f^{-1} $ 是光滑函数。   
\end{definition}
这样我们就简单的理解了为什么猫和球在这个题目里是等价的这件事,实际上是我们总是能找到一个微分同胚使得猫和球可以互相转化,同时,微分同胚的定义又要求我们保持光滑结构不变。\\
这个引言简单的介绍了一下微分几何的研究对象,也让我们对于现代数学的定义理解更深。现代数学对于“结构”非常强调。在拓扑学中,结构表现为在连续函数变化下不变的结构。在微分几何中,结构表现为在光滑函数变化下不变的结构。而在传统的微积分中,可能并没有一个特殊的结构,但是微积分本身由欧氏空间 $ \mathbb{R}^n $ 所生,也能通过约束条件转化为拓扑学和微分几何的问题。比如著名的隐函数定理和反函数定理。
这样的“特殊映射保特定结构”进而延伸出一个新的数学学科的现象有很多。现代数学的很多方向都是在抽象出一个不变的结构,然后研究特定的对象,再应用到特定的学科中。事实上,比较“抽象”的数学家就会想,我们能不能找出这些数学学科的通性,毕竟所有的学科定义都遵循这样的顺序:
给出结构,对应映射,然后研究性质。那么这样的性质会不会有相通呢?答案是有的。例如拓扑学就有很多代数的群论知识。那么这样研究学科结构的抽象化的学科叫做\textbf{范畴学}。
\\
其实我们到目前还没有明显的感受到“结构”的存在,只知道有这样的光滑函数,因而总有一些东西是不变的。一种自然的想法是我们没必要非要点出结构来,所谓结构就是若干定理的集合。这一点我们在数学的定义里颇有感触,我们没必要仔仔细细的描述一个自然对象。我们只要对于一个自然对象的特定性质进行数学的描述,然后说它的一些其他性质也可以用这个性质推导出来就行了。\\

这样做自然是非常好,而且我们总是在做这样的事,毕竟谁也说不好椭圆是什么,于是用一个方程来描述它。这也是数学抽象的来源。\\
但是我们能否找到一个直观的结构也是数学家所追求的。数学家追求本质,而一个对象特定的结构如果能用数学结构来描述的话,那几乎就说清楚这个对象的本质了。\\
这样的事情在微分几何里几乎是做到了,但之所以是几乎是因为我们还没有解决微分几何的所有问题,所以不够本质(笑)。我们在接下来一小节简单介绍一下:
\subsection{微分结构}
\end{document}